% \documentclass[12pt]{article}
\documentclass[a4paper,]{book}

  \usepackage{lmodern}


\usepackage[a4paper]{geometry}

\usepackage{graphicx}

\usepackage{longtable}
\usepackage{booktabs}
\usepackage[htt]{hyphenat}

\usepackage{amssymb}

% use upquote if available, for straight quotes in verbatim environments
\IfFileExists{upquote.sty}{\usepackage{upquote}}{}
% use microtype if available
\IfFileExists{microtype.sty}{\usepackage{microtype}}{}

\ifxetex
  \usepackage[setpagesize=false, % page size defined by xetex
              unicode=false, % unicode breaks when used with xetex
              xetex]{hyperref}
\else
  \usepackage[unicode=true]{hyperref}
\fi

% Break URLs in a few more places. (For breaking lines, not functionality!)
% cf. http://tex.stackexchange.com/questions/3033/forcing-linebreaks-in-url
\expandafter\def\expandafter\UrlBreaks\expandafter{\UrlBreaks%  save the current one
  \do\*\do\-\do\~\do\'\do\"\do\-}

\hypersetup{breaklinks=true,
            bookmarks=true,
            pdfauthor={The Rust Team},
            pdftitle={The Rust Programming Language},
            colorlinks=false,
            citecolor=blue,
            urlcolor=blue,
            linkcolor=magenta,
            pdfborder={0 0 0}}
\urlstyle{same}  % don't use monospace font for urls

\renewcommand*{\hyperref}[2][\ar]{%
  \def\ar{#2}%
  #2 (\autoref{#1}, page~\pageref{#1})}

\ifnum 0\ifxetex 1\fi\ifluatex 1\fi=0 % if pdftex
  \usepackage[T1]{fontenc}
  \usepackage[utf8]{inputenc}
  \else % if luatex or xelatex
  \ifxetex
    \usepackage{mathspec}
    \usepackage{xltxtra,xunicode}
    \usepackage{xeCJK}
    \setCJKmainfont{IPAexMincho}
  \else
    \usepackage{fontspec}
  \fi
  \defaultfontfeatures{Mapping=tex-text,Scale=MatchLowercase}
  \newcommand{\euro}{€}

  
  
      \setmonofont[Mapping=tex-ansi]{DejaVu Sans Mono}
  
  \fi

\setlength{\emergencystretch}{3em}  % prevent overfull lines

% use upquote if available, for straight quotes in verbatim environments
\IfFileExists{upquote.sty}{\usepackage{upquote}}{}
% use microtype if available
\IfFileExists{microtype.sty}{\usepackage{microtype}}{}


\usepackage{xcolor}

  \usepackage{color}
  \usepackage{fancyvrb}
  \newcommand{\VerbBar}{|}
  \newcommand{\VERB}{\Verb[commandchars=\\\{\}]}
  \DefineVerbatimEnvironment{Highlighting}{Verbatim}{commandchars=\\\{\}}
  % Add ',fontsize=\small' for more characters per line
  \usepackage{framed}
  \definecolor{shadecolor}{RGB}{248,248,248}
  \newenvironment{Shaded}{\begin{snugshade}}{\end{snugshade}}
  \newcommand{\KeywordTok}[1]{\textcolor[rgb]{0.13,0.29,0.53}{\textbf{{#1}}}}
  \newcommand{\DataTypeTok}[1]{\textcolor[rgb]{0.13,0.29,0.53}{{#1}}}
  \newcommand{\DecValTok}[1]{\textcolor[rgb]{0.00,0.00,0.81}{{#1}}}
  \newcommand{\BaseNTok}[1]{\textcolor[rgb]{0.00,0.00,0.81}{{#1}}}
  \newcommand{\FloatTok}[1]{\textcolor[rgb]{0.00,0.00,0.81}{{#1}}}
  \newcommand{\ConstantTok}[1]{\textcolor[rgb]{0.00,0.00,0.00}{{#1}}}
  \newcommand{\CharTok}[1]{\textcolor[rgb]{0.31,0.60,0.02}{{#1}}}
  \newcommand{\SpecialCharTok}[1]{\textcolor[rgb]{0.00,0.00,0.00}{{#1}}}
  \newcommand{\StringTok}[1]{\textcolor[rgb]{0.31,0.60,0.02}{{#1}}}
  \newcommand{\VerbatimStringTok}[1]{\textcolor[rgb]{0.31,0.60,0.02}{{#1}}}
  \newcommand{\SpecialStringTok}[1]{\textcolor[rgb]{0.31,0.60,0.02}{{#1}}}
  \newcommand{\ImportTok}[1]{{#1}}
  \newcommand{\CommentTok}[1]{\textcolor[rgb]{0.56,0.35,0.01}{\textit{{#1}}}}
  \newcommand{\DocumentationTok}[1]{\textcolor[rgb]{0.56,0.35,0.01}{\textbf{\textit{{#1}}}}}
  \newcommand{\AnnotationTok}[1]{\textcolor[rgb]{0.56,0.35,0.01}{\textbf{\textit{{#1}}}}}
  \newcommand{\CommentVarTok}[1]{\textcolor[rgb]{0.56,0.35,0.01}{\textbf{\textit{{#1}}}}}
  \newcommand{\OtherTok}[1]{\textcolor[rgb]{0.56,0.35,0.01}{{#1}}}
  \newcommand{\FunctionTok}[1]{\textcolor[rgb]{0.00,0.00,0.00}{{#1}}}
  \newcommand{\VariableTok}[1]{\textcolor[rgb]{0.00,0.00,0.00}{{#1}}}
  \newcommand{\ControlFlowTok}[1]{\textcolor[rgb]{0.13,0.29,0.53}{\textbf{{#1}}}}
  \newcommand{\OperatorTok}[1]{\textcolor[rgb]{0.81,0.36,0.00}{\textbf{{#1}}}}
  \newcommand{\BuiltInTok}[1]{{#1}}
  \newcommand{\ExtensionTok}[1]{{#1}}
  \newcommand{\PreprocessorTok}[1]{\textcolor[rgb]{0.56,0.35,0.01}{\textit{{#1}}}}
  \newcommand{\AttributeTok}[1]{\textcolor[rgb]{0.77,0.63,0.00}{{#1}}}
  \newcommand{\RegionMarkerTok}[1]{{#1}}
  \newcommand{\InformationTok}[1]{\textcolor[rgb]{0.56,0.35,0.01}{\textbf{\textit{{#1}}}}}
  \newcommand{\WarningTok}[1]{\textcolor[rgb]{0.56,0.35,0.01}{\textbf{\textit{{#1}}}}}
  \newcommand{\AlertTok}[1]{\textcolor[rgb]{0.94,0.16,0.16}{{#1}}}
  \newcommand{\ErrorTok}[1]{\textcolor[rgb]{0.64,0.00,0.00}{\textbf{{#1}}}}
  \newcommand{\NormalTok}[1]{{#1}}

  \usepackage{fancyvrb}

  % Make links footnotes instead of hotlinks:
  \renewcommand{\href}[2]{#2\footnote{\url{#1}}}

  \VerbatimFootnotes % allows verbatim text in footnotes

\setlength{\parindent}{0pt}
\setlength{\parskip}{6pt plus 2pt minus 1pt}
\setlength{\emergencystretch}{3em}  % prevent overfull lines

\providecommand{\tightlist}{%
  \setlength{\itemsep}{0pt}\setlength{\parskip}{0pt}}
\setcounter{secnumdepth}{0}

\usepackage[font={footnotesize,sf}]{caption}

% Redefines (sub)paragraphs to behave more like sections
\ifx\paragraph\undefined\else
\let\oldparagraph\paragraph
\renewcommand{\paragraph}[1]{\oldparagraph{#1}\mbox{}}
\fi
\ifx\subparagraph\undefined\else
\let\oldsubparagraph\subparagraph
\renewcommand{\subparagraph}[1]{\oldsubparagraph{#1}\mbox{}}
\fi

% - - -
% Make it fancy

% Headers and page numbering
\usepackage{fancyhdr}

% Set figure legends and captions to be smaller sized sans serif font
\usepackage[font={footnotesize,sf}]{caption}

% Chapter styling
\usepackage[grey]{quotchap}
\makeatletter
\renewcommand*{\chapnumfont}{%
  \thispagestyle{empty}
  \usefont{T1}{\@defaultcnfont}{b}{n}\fontsize{80}{100}\selectfont% Default: 100/130
  \color{chaptergrey}%
}
\makeatother

% - - -

  \title{The Rust Programming Language}
  \author{The Rust Team}
  \date{2016-05-19}

\begin{document}

  \maketitle



{
  \hypersetup{linkcolor=black}
  \setcounter{tocdepth}{2}
  \tableofcontents
  \newpage
}

\chapter{Introduction}\label{introduction}

Welcome! This book will teach you about the
\href{https://www.rust-lang.org}{Rust Programming Language}. Rust is a
systems programming language focused on three goals: safety, speed, and
concurrency. It maintains these goals without having a garbage
collector, making it a useful language for a number of use cases other
languages aren't good at: embedding in other languages, programs with
specific space and time requirements, and writing low-level code, like
device drivers and operating systems. It improves on current languages
targeting this space by having a number of compile-time safety checks
that produce no runtime overhead, while eliminating all data races. Rust
also aims to achieve `zero-cost abstractions' even though some of these
abstractions feel like those of a high-level language. Even then, Rust
still allows precise control like a low-level language would.

``The Rust Programming Language'' is split into chapters. This
introduction is the first. After this:

\begin{itemize}
\tightlist
\item
  \protect\hyperlink{sec--getting-started}{Getting started} - Set up
  your computer for Rust development.
\item
  \protect\hyperlink{sec--guessing-game}{Tutorial: Guessing Game} -
  Learn some Rust with a small project.
\item
  \protect\hyperlink{sec--syntax-and-semantics}{Syntax and Semantics} -
  Each bit of Rust, broken down into small chunks.
\item
  \protect\hyperlink{sec--effective-rust}{Effective Rust} - Higher-level
  concepts for writing excellent Rust code.
\item
  \protect\hyperlink{sec--nightly-rust}{Nightly Rust} - Cutting-edge
  features that aren't in stable builds yet.
\item
  \protect\hyperlink{sec--glossary}{Glossary} - A reference of terms
  used in the book.
\item
  \protect\hyperlink{sec--bibliography}{Bibliography} - Background on
  Rust's influences, papers about Rust.
\end{itemize}

\subsection{Contributing}\label{contributing}

The source files from which this book is generated can be found on
\href{https://github.com/rust-lang/rust/tree/master/src/doc/book}{GitHub}.

\hypertarget{sec--getting-started}{\chapter{Getting
Started}\label{sec--getting-started}}

This first chapter of the book will get us going with Rust and its
tooling. First, we'll install Rust. Then, the classic `Hello World'
program. Finally, we'll talk about Cargo, Rust's build system and
package manager.

\subsection{Installing Rust}\label{installing-rust}

The first step to using Rust is to install it. Generally speaking,
you'll need an Internet connection to run the commands in this section,
as we'll be downloading Rust from the Internet.

We'll be showing off a number of commands using a terminal, and those
lines all start with \texttt{\$}. We don't need to type in the
\texttt{\$}s, they are there to indicate the start of each command.
We'll see many tutorials and examples around the web that follow this
convention: \texttt{\$} for commands run as our regular user, and
\texttt{\#} for commands we should be running as an administrator.

\subsubsection{Platform support}\label{platform-support}

The Rust compiler runs on, and compiles to, a great number of platforms,
though not all platforms are equally supported. Rust's support levels
are organized into three tiers, each with a different set of guarantees.

Platforms are identified by their ``target triple'' which is the string
to inform the compiler what kind of output should be produced. The
columns below indicate whether the corresponding component works on the
specified platform.

\paragraph{Tier 1}\label{tier-1}

Tier 1 platforms can be thought of as ``guaranteed to build and work''.
Specifically they will each satisfy the following requirements:

\begin{itemize}
\tightlist
\item
  Automated testing is set up to run tests for the platform.
\item
  Landing changes to the \texttt{rust-lang/rust} repository's master
  branch is gated on tests passing.
\item
  Official release artifacts are provided for the platform.
\item
  Documentation for how to use and how to build the platform is
  available.
\end{itemize}

\begin{longtable}[c]{@{}lllll@{}}
\toprule
Target & std & rustc & cargo & notes\tabularnewline
\midrule
\endhead
\texttt{i686-apple-darwin} & \checkmark   & \checkmark   & \checkmark  
& 32-bit OSX (10.7+, Lion+)\tabularnewline
\texttt{i686-pc-windows-gnu} & \checkmark   & \checkmark   &
\checkmark   & 32-bit MinGW (Windows 7+)\tabularnewline
\texttt{i686-pc-windows-msvc} & \checkmark   & \checkmark   &
\checkmark   & 32-bit MSVC (Windows 7+)\tabularnewline
\texttt{i686-unknown-linux-gnu} & \checkmark   & \checkmark   &
\checkmark   & 32-bit Linux (2.6.18+)\tabularnewline
\texttt{x86\_64-apple-darwin} & \checkmark   & \checkmark   &
\checkmark   & 64-bit OSX (10.7+, Lion+)\tabularnewline
\texttt{x86\_64-pc-windows-gnu} & \checkmark   & \checkmark   &
\checkmark   & 64-bit MinGW (Windows 7+)\tabularnewline
\texttt{x86\_64-pc-windows-msvc} & \checkmark   & \checkmark   &
\checkmark   & 64-bit MSVC (Windows 7+)\tabularnewline
\texttt{x86\_64-unknown-linux-gnu} & \checkmark   & \checkmark   &
\checkmark   & 64-bit Linux (2.6.18+)\tabularnewline
\bottomrule
\end{longtable}

\paragraph{Tier 2}\label{tier-2}

Tier 2 platforms can be thought of as ``guaranteed to build''. Automated
tests are not run so it's not guaranteed to produce a working build, but
platforms often work to quite a good degree and patches are always
welcome! Specifically, these platforms are required to have each of the
following:

\begin{itemize}
\tightlist
\item
  Automated building is set up, but may not be running tests.
\item
  Landing changes to the \texttt{rust-lang/rust} repository's master
  branch is gated on platforms \textbf{building}. Note that this means
  for some platforms only the standard library is compiled, but for
  others the full bootstrap is run.
\item
  Official release artifacts are provided for the platform.
\end{itemize}

\begin{longtable}[c]{@{}lllll@{}}
\toprule
Target & std & rustc & cargo & notes\tabularnewline
\midrule
\endhead
\texttt{aarch64-apple-ios} & \checkmark   & & & ARM64 iOS\tabularnewline
\texttt{aarch64-unknown-linux-gnu} & \checkmark   & \checkmark   &
\checkmark   & ARM64 Linux (2.6.18+)\tabularnewline
\texttt{arm-linux-androideabi} & \checkmark   & & & ARM
Android\tabularnewline
\texttt{arm-unknown-linux-gnueabi} & \checkmark   & \checkmark   &
\checkmark   & ARM Linux (2.6.18+)\tabularnewline
\texttt{arm-unknown-linux-gnueabihf} & \checkmark   & \checkmark   &
\checkmark   & ARM Linux (2.6.18+)\tabularnewline
\texttt{armv7-apple-ios} & \checkmark   & & & ARM iOS\tabularnewline
\texttt{armv7-unknown-linux-gnueabihf} & \checkmark   & \checkmark   &
\checkmark   & ARMv7 Linux (2.6.18+)\tabularnewline
\texttt{armv7s-apple-ios} & \checkmark   & & & ARM iOS\tabularnewline
\texttt{i386-apple-ios} & \checkmark   & & & 32-bit x86
iOS\tabularnewline
\texttt{i586-pc-windows-msvc} & \checkmark   & & & 32-bit Windows w/o
SSE\tabularnewline
\texttt{mips-unknown-linux-gnu} & \checkmark   & & & MIPS Linux
(2.6.18+)\tabularnewline
\texttt{mips-unknown-linux-musl} & \checkmark   & & & MIPS Linux with
MUSL\tabularnewline
\texttt{mipsel-unknown-linux-gnu} & \checkmark   & & & MIPS (LE) Linux
(2.6.18+)\tabularnewline
\texttt{mipsel-unknown-linux-musl} & \checkmark   & & & MIPS (LE) Linux
with MUSL\tabularnewline
\texttt{powerpc-unknown-linux-gnu} & \checkmark   & & & PowerPC Linux
(2.6.18+)\tabularnewline
\texttt{powerpc64-unknown-linux-gnu} & \checkmark   & & & PPC64 Linux
(2.6.18+)\tabularnewline
\texttt{powerpc64le-unknown-linux-gnu} & \checkmark   & & & PPC64LE
Linux (2.6.18+)\tabularnewline
\texttt{x86\_64-apple-ios} & \checkmark   & & & 64-bit x86
iOS\tabularnewline
\texttt{x86\_64-rumprun-netbsd} & \checkmark   & & & 64-bit NetBSD Rump
Kernel\tabularnewline
\texttt{x86\_64-unknown-freebsd} & \checkmark   & \checkmark   &
\checkmark   & 64-bit FreeBSD\tabularnewline
\texttt{x86\_64-unknown-linux-musl} & \checkmark   & & & 64-bit Linux
with MUSL\tabularnewline
\texttt{x86\_64-unknown-netbsd} & \checkmark   & \checkmark   &
\checkmark   & 64-bit NetBSD\tabularnewline
\bottomrule
\end{longtable}

\paragraph{Tier 3}\label{tier-3}

Tier 3 platforms are those which Rust has support for, but landing
changes is not gated on the platform either building or passing tests.
Working builds for these platforms may be spotty as their reliability is
often defined in terms of community contributions. Additionally, release
artifacts and installers are not provided, but there may be community
infrastructure producing these in unofficial locations.

\begin{longtable}[c]{@{}lllll@{}}
\toprule
Target & std & rustc & cargo & notes\tabularnewline
\midrule
\endhead
\texttt{aarch64-linux-android} & \checkmark   & & & ARM64
Android\tabularnewline
\texttt{armv7-linux-androideabi} & \checkmark   & & & ARM-v7a
Android\tabularnewline
\texttt{i686-linux-android} & \checkmark   & & & 32-bit x86
Android\tabularnewline
\texttt{i686-pc-windows-msvc} (XP) & \checkmark   & & & Windows XP
support\tabularnewline
\texttt{i686-unknown-freebsd} & \checkmark   & \checkmark   &
\checkmark   & 32-bit FreeBSD\tabularnewline
\texttt{x86\_64-pc-windows-msvc} (XP) & \checkmark   & & & Windows XP
support\tabularnewline
\texttt{x86\_64-sun-solaris} & \checkmark   & \checkmark   & & 64-bit
Solaris/SunOS\tabularnewline
\texttt{x86\_64-unknown-bitrig} & \checkmark   & \checkmark   & & 64-bit
Bitrig\tabularnewline
\texttt{x86\_64-unknown-dragonfly} & \checkmark   & \checkmark   & &
64-bit DragonFlyBSD\tabularnewline
\texttt{x86\_64-unknown-openbsd} & \checkmark   & \checkmark   & &
64-bit OpenBSD\tabularnewline
\bottomrule
\end{longtable}

Note that this table can be expanded over time, this isn't the
exhaustive set of tier 3 platforms that will ever be!

\subsubsection{Installing on Linux or
Mac}\label{installing-on-linux-or-mac}

If we're on Linux or a Mac, all we need to do is open a terminal and
type this:

\begin{Shaded}
\begin{Highlighting}[]
\NormalTok{$ }\KeywordTok{curl} \NormalTok{-sSf https://static.rust-lang.org/rustup.sh }\KeywordTok{|} \KeywordTok{sh}
\end{Highlighting}
\end{Shaded}

This will download a script, and start the installation. If it all goes
well, you'll see this appear:

\begin{verbatim}
Rust is ready to roll.
\end{verbatim}

From here, press \texttt{y} for `yes', and then follow the rest of the
prompts.

\subsubsection{Installing on Windows}\label{installing-on-windows}

If you're on Windows, please download the appropriate
\href{https://www.rust-lang.org/install.html}{installer}.

\subsubsection{Uninstalling}\label{uninstalling}

Uninstalling Rust is as easy as installing it. On Linux or Mac, run the
uninstall script:

\begin{Shaded}
\begin{Highlighting}[]
\NormalTok{$ }\KeywordTok{sudo} \NormalTok{/usr/local/lib/rustlib/uninstall.sh}
\end{Highlighting}
\end{Shaded}

If we used the Windows installer, we can re-run the \texttt{.msi} and it
will give us an uninstall option.

\subsubsection{Troubleshooting}\label{troubleshooting}

If we've got Rust installed, we can open up a shell, and type this:

\begin{Shaded}
\begin{Highlighting}[]
\NormalTok{$ }\KeywordTok{rustc} \NormalTok{--version}
\end{Highlighting}
\end{Shaded}

You should see the version number, commit hash, and commit date.

If you do, Rust has been installed successfully! Congrats!

If you don't and you're on Windows, check that Rust is in your \%PATH\%
system variable. If it isn't, run the installer again, select ``Change''
on the ``Change, repair, or remove installation'' page and ensure ``Add
to PATH'' is installed on the local hard drive.

Rust does not do its own linking, and so you'll need to have a linker
installed. Doing so will depend on your specific system, consult its
documentation for more details.

If not, there are a number of places where we can get help. The easiest
is \href{irc://irc.mozilla.org/\#rust-beginners}{the \#rust-beginners
IRC channel on irc.mozilla.org} and for general discussion
\href{irc://irc.mozilla.org/\#rust}{the \#rust IRC channel on
irc.mozilla.org}, which we can access through
\href{http://chat.mibbit.com/?server=irc.mozilla.org\&channel=\%23rust-beginners,\%23rust}{Mibbit}.
Then we'll be chatting with other Rustaceans (a silly nickname we call
ourselves) who can help us out. Other great resources include
\href{https://users.rust-lang.org/}{the user's forum} and
\href{http://stackoverflow.com/questions/tagged/rust}{Stack Overflow}.

This installer also installs a copy of the documentation locally, so we
can read it offline. On UNIX systems, \texttt{/usr/local/share/doc/rust}
is the location. On Windows, it's in a \texttt{share/doc} directory,
inside the directory to which Rust was installed.

\subsection{Hello, world!}\label{hello-world}

Now that you have Rust installed, we'll help you write your first Rust
program. It's traditional when learning a new language to write a little
program to print the text ``Hello, world!'' to the screen, and in this
section, we'll follow that tradition.

The nice thing about starting with such a simple program is that you can
quickly verify that your compiler is installed, and that it's working
properly. Printing information to the screen is also a pretty common
thing to do, so practicing it early on is good.

\begin{quote}
Note: This book assumes basic familiarity with the command line. Rust
itself makes no specific demands about your editing, tooling, or where
your code lives, so if you prefer an IDE to the command line, that's an
option. You may want to check out {[}SolidOak{]}, which was built
specifically with Rust in mind. There are a number of extensions in
development by the community, and the Rust team ships plugins for
{[}various editors{]}. Configuring your editor or IDE is out of the
scope of this tutorial, so check the documentation for your specific
setup.
\end{quote}

\subsubsection{Creating a Project File}\label{creating-a-project-file}

First, make a file to put your Rust code in. Rust doesn't care where
your code lives, but for this book, I suggest making a \emph{projects}
directory in your home directory, and keeping all your projects there.
Open a terminal and enter the following commands to make a directory for
this particular project:

\begin{Shaded}
\begin{Highlighting}[]
\NormalTok{$ }\KeywordTok{mkdir} \NormalTok{~/projects}
\NormalTok{$ }\KeywordTok{cd} \NormalTok{~/projects}
\NormalTok{$ }\KeywordTok{mkdir} \NormalTok{hello_world}
\NormalTok{$ }\KeywordTok{cd} \NormalTok{hello_world}
\end{Highlighting}
\end{Shaded}

\begin{quote}
Note: If you're on Windows and not using PowerShell, the
\texttt{\textasciitilde{}} may not work. Consult the documentation for
your shell for more details.
\end{quote}

\subsubsection{Writing and Running a Rust
Program}\label{writing-and-running-a-rust-program}

Next, make a new source file and call it \emph{main.rs}. Rust files
always end in a \emph{.rs} extension. If you're using more than one word
in your filename, use an underscore to separate them; for example, you'd
use \emph{hello\_world.rs} rather than \emph{helloworld.rs}.

Now open the \emph{main.rs} file you just created, and type the
following code:

\begin{Shaded}
\begin{Highlighting}[]
\KeywordTok{fn} \NormalTok{main() \{}
    \PreprocessorTok{println!}\NormalTok{(}\StringTok{"Hello, world!"}\NormalTok{);}
\NormalTok{\}}
\end{Highlighting}
\end{Shaded}

Save the file, and go back to your terminal window. On Linux or OSX,
enter the following commands:

\begin{Shaded}
\begin{Highlighting}[]
\NormalTok{$ }\KeywordTok{rustc} \NormalTok{main.rs}
\NormalTok{$ }\KeywordTok{./main}
\KeywordTok{Hello}\NormalTok{, world!}
\end{Highlighting}
\end{Shaded}

In Windows, replace \texttt{main} with \texttt{main.exe}. Regardless of
your operating system, you should see the string \texttt{Hello,\ world!}
print to the terminal. If you did, then congratulations! You've
officially written a Rust program. That makes you a Rust programmer!
Welcome.

\subsubsection{Anatomy of a Rust
Program}\label{anatomy-of-a-rust-program}

Now, let's go over what just happened in your ``Hello, world!'' program
in detail. Here's the first piece of the puzzle:

\begin{Shaded}
\begin{Highlighting}[]
\KeywordTok{fn} \NormalTok{main() \{}

\NormalTok{\}}
\end{Highlighting}
\end{Shaded}

These lines define a \emph{function} in Rust. The \texttt{main} function
is special: it's the beginning of every Rust program. The first line
says, ``I'm declaring a function named \texttt{main} that takes no
arguments and returns nothing.'' If there were arguments, they would go
inside the parentheses (\texttt{(} and \texttt{)}), and because we
aren't returning anything from this function, we can omit the return
type entirely.

Also note that the function body is wrapped in curly braces (\texttt{\{}
and \texttt{\}}). Rust requires these around all function bodies. It's
considered good style to put the opening curly brace on the same line as
the function declaration, with one space in between.

Inside the \texttt{main()} function:

\begin{Shaded}
\begin{Highlighting}[]
    \PreprocessorTok{println!}\NormalTok{(}\StringTok{"Hello, world!"}\NormalTok{);}
\end{Highlighting}
\end{Shaded}

This line does all of the work in this little program: it prints text to
the screen. There are a number of details that are important here. The
first is that it's indented with four spaces, not tabs.

The second important part is the \texttt{println!()} line. This is
calling a Rust \emph{{[}macro{]}}, which is how metaprogramming is done
in Rust. If it were calling a function instead, it would look like this:
\texttt{println()} (without the !). We'll discuss Rust macros in more
detail later, but for now you only need to know that when you see a
\texttt{!} that means that you're calling a macro instead of a normal
function.

Next is \texttt{"Hello,\ world!"} which is a \emph{string}. Strings are
a surprisingly complicated topic in a systems programming language, and
this is a \emph{{[}statically allocated{]}} string. We pass this string
as an argument to \texttt{println!}, which prints the string to the
screen. Easy enough!

The line ends with a semicolon (\texttt{;}). Rust is an
\emph{\protect\hyperlink{expression-oriented-language}{expression-oriented
language}}, which means that most things are expressions, rather than
statements. The \texttt{;} indicates that this expression is over, and
the next one is ready to begin. Most lines of Rust code end with a
\texttt{;}.

\subsubsection{Compiling and Running Are Separate
Steps}\label{compiling-and-running-are-separate-steps}

In ``Writing and Running a Rust Program'', we showed you how to run a
newly created program. We'll break that process down and examine each
step now.

Before running a Rust program, you have to compile it. You can use the
Rust compiler by entering the \texttt{rustc} command and passing it the
name of your source file, like this:

\begin{Shaded}
\begin{Highlighting}[]
\NormalTok{$ }\KeywordTok{rustc} \NormalTok{main.rs}
\end{Highlighting}
\end{Shaded}

If you come from a C or C++ background, you'll notice that this is
similar to \texttt{gcc} or \texttt{clang}. After compiling successfully,
Rust should output a binary executable, which you can see on Linux or
OSX by entering the \texttt{ls} command in your shell as follows:

\begin{Shaded}
\begin{Highlighting}[]
\NormalTok{$ }\KeywordTok{ls}
\KeywordTok{main}  \NormalTok{main.rs}
\end{Highlighting}
\end{Shaded}

On Windows, you'd enter:

\begin{Shaded}
\begin{Highlighting}[]
\NormalTok{$ }\KeywordTok{dir}
\KeywordTok{main.exe}  \NormalTok{main.rs}
\end{Highlighting}
\end{Shaded}

This shows we have two files: the source code, with an \texttt{.rs}
extension, and the executable (\texttt{main.exe} on Windows,
\texttt{main} everywhere else). All that's left to do from here is run
the \texttt{main} or \texttt{main.exe} file, like this:

\begin{Shaded}
\begin{Highlighting}[]
\NormalTok{$ }\KeywordTok{./main}  \CommentTok{# or main.exe on Windows}
\end{Highlighting}
\end{Shaded}

If \emph{main.rs} were your ``Hello, world!'' program, this would print
\texttt{Hello,\ world!} to your terminal.

If you come from a dynamic language like Ruby, Python, or JavaScript,
you may not be used to compiling and running a program being separate
steps. Rust is an \emph{ahead-of-time compiled} language, which means
that you can compile a program, give it to someone else, and they can
run it even without Rust installed. If you give someone a \texttt{.rb}
or \texttt{.py} or \texttt{.js} file, on the other hand, they need to
have a Ruby, Python, or JavaScript implementation installed
(respectively), but you only need one command to both compile and run
your program. Everything is a tradeoff in language design.

Just compiling with \texttt{rustc} is fine for simple programs, but as
your project grows, you'll want to be able to manage all of the options
your project has, and make it easy to share your code with other people
and projects. Next, I'll introduce you to a tool called Cargo, which
will help you write real-world Rust programs.

\hypertarget{hello-cargo}{\subsection{Hello, Cargo!}\label{hello-cargo}}

Cargo is Rust's build system and package manager, and Rustaceans use
Cargo to manage their Rust projects. Cargo manages three things:
building your code, downloading the libraries your code depends on, and
building those libraries. We call libraries your code needs
`dependencies' since your code depends on them.

The simplest Rust programs don't have any dependencies, so right now,
you'd only use the first part of its functionality. As you write more
complex Rust programs, you'll want to add dependencies, and if you start
off using Cargo, that will be a lot easier to do.

As the vast, vast majority of Rust projects use Cargo, we will assume
that you're using it for the rest of the book. Cargo comes installed
with Rust itself, if you used the official installers. If you installed
Rust through some other means, you can check if you have Cargo installed
by typing:

\begin{Shaded}
\begin{Highlighting}[]
\NormalTok{$ }\KeywordTok{cargo} \NormalTok{--version}
\end{Highlighting}
\end{Shaded}

Into a terminal. If you see a version number, great! If you see an error
like `\texttt{command\ not\ found}', then you should look at the
documentation for the system in which you installed Rust, to determine
if Cargo is separate.

\subsubsection{Converting to Cargo}\label{converting-to-cargo}

Let's convert the Hello World program to Cargo. To Cargo-fy a project,
you need to do three things:

\begin{enumerate}
\def\labelenumi{\arabic{enumi}.}
\tightlist
\item
  Put your source file in the right directory.
\item
  Get rid of the old executable (\texttt{main.exe} on Windows,
  \texttt{main} everywhere else).
\item
  Make a Cargo configuration file.
\end{enumerate}

Let's get started!

\paragraph{Creating a Source Directory and Removing the Old
Executable}\label{creating-a-source-directory-and-removing-the-old-executable}

First, go back to your terminal, move to your \emph{hello\_world}
directory, and enter the following commands:

\begin{Shaded}
\begin{Highlighting}[]
\NormalTok{$ }\KeywordTok{mkdir} \NormalTok{src}
\NormalTok{$ }\KeywordTok{mv} \NormalTok{main.rs src/main.rs }\CommentTok{# or 'move main.rs src/main.rs' on Windows}
\NormalTok{$ }\KeywordTok{rm} \NormalTok{main  }\CommentTok{# or 'del main.exe' on Windows}
\end{Highlighting}
\end{Shaded}

Cargo expects your source files to live inside a \emph{src} directory,
so do that first. This leaves the top-level project directory (in this
case, \emph{hello\_world}) for READMEs, license information, and
anything else not related to your code. In this way, using Cargo helps
you keep your projects nice and tidy. There's a place for everything,
and everything is in its place.

Now, move \emph{main.rs} into the \emph{src} directory, and delete the
compiled file you created with \texttt{rustc}. As usual, replace
\texttt{main} with \texttt{main.exe} if you're on Windows.

This example retains \texttt{main.rs} as the source filename because
it's creating an executable. If you wanted to make a library instead,
you'd name the file \texttt{lib.rs}. This convention is used by Cargo to
successfully compile your projects, but it can be overridden if you
wish.

\paragraph{Creating a Configuration
File}\label{creating-a-configuration-file}

Next, create a new file inside your \emph{hello\_world} directory, and
call it \texttt{Cargo.toml}.

Make sure to capitalize the \texttt{C} in \texttt{Cargo.toml}, or Cargo
won't know what to do with the configuration file.

This file is in the \emph{{[}TOML{]}} (Tom's Obvious, Minimal Language)
format. TOML is similar to INI, but has some extra goodies, and is used
as Cargo's configuration format.

Inside this file, type the following information:

\begin{verbatim}
[package]

name = "hello_world"
version = "0.0.1"
authors = [ "Your name <you@example.com>" ]
\end{verbatim}

The first line, \texttt{{[}package{]}}, indicates that the following
statements are configuring a package. As we add more information to this
file, we'll add other sections, but for now, we only have the package
configuration.

The other three lines set the three bits of configuration that Cargo
needs to know to compile your program: its name, what version it is, and
who wrote it.

Once you've added this information to the \emph{Cargo.toml} file, save
it to finish creating the configuration file.

\subsubsection{Building and Running a Cargo
Project}\label{building-and-running-a-cargo-project}

With your \emph{Cargo.toml} file in place in your project's root
directory, you should be ready to build and run your Hello World
program! To do so, enter the following commands:

\begin{Shaded}
\begin{Highlighting}[]
\NormalTok{$ }\KeywordTok{cargo} \NormalTok{build}
   \KeywordTok{Compiling} \NormalTok{hello_world v0.0.1 (file:///home/yourname/projects/hello_world)}
\NormalTok{$ }\KeywordTok{./target/debug/hello_world}
\KeywordTok{Hello}\NormalTok{, world!}
\end{Highlighting}
\end{Shaded}

Bam! If all goes well, \texttt{Hello,\ world!} should print to the
terminal once more.

You just built a project with \texttt{cargo\ build} and ran it with
\texttt{./target/debug/hello\_world}, but you can actually do both in
one step with \texttt{cargo\ run} as follows:

\begin{Shaded}
\begin{Highlighting}[]
\NormalTok{$ }\KeywordTok{cargo} \NormalTok{run}
     \KeywordTok{Running} \KeywordTok{`target/debug/hello_world`}
\KeywordTok{Hello}\NormalTok{, world!}
\end{Highlighting}
\end{Shaded}

Notice that this example didn't re-build the project. Cargo figured out
that the file hasn't changed, and so it just ran the binary. If you'd
modified your source code, Cargo would have rebuilt the project before
running it, and you would have seen something like this:

\begin{Shaded}
\begin{Highlighting}[]
\NormalTok{$ }\KeywordTok{cargo} \NormalTok{run}
   \KeywordTok{Compiling} \NormalTok{hello_world v0.0.1 (file:///home/yourname/projects/hello_world)}
     \KeywordTok{Running} \KeywordTok{`target/debug/hello_world`}
\KeywordTok{Hello}\NormalTok{, world!}
\end{Highlighting}
\end{Shaded}

Cargo checks to see if any of your project's files have been modified,
and only rebuilds your project if they've changed since the last time
you built it.

With simple projects, Cargo doesn't bring a whole lot over just using
\texttt{rustc}, but it will become useful in the future. This is
especially true when you start using crates; these are synonymous with a
`library' or `package' in other programming languages. For complex
projects composed of multiple crates, it's much easier to let Cargo
coordinate the build. Using Cargo, you can run \texttt{cargo\ build},
and it should work the right way.

\paragraph{Building for Release}\label{building-for-release}

When your project is ready for release, you can use
\texttt{cargo\ build\ -\/-release} to compile your project with
optimizations. These optimizations make your Rust code run faster, but
turning them on makes your program take longer to compile. This is why
there are two different profiles, one for development, and one for
building the final program you'll give to a user.

\paragraph{\texorpdfstring{What Is That
\texttt{Cargo.lock}?}{What Is That Cargo.lock?}}\label{what-is-that-cargo.lock}

Running \texttt{cargo\ build} also causes Cargo to create a new file
called \emph{Cargo.lock}, which looks like this:

\begin{verbatim}
[root]
name = "hello_world"
version = "0.0.1"
\end{verbatim}

Cargo uses the \emph{Cargo.lock} file to keep track of dependencies in
your application. This is the Hello World project's \emph{Cargo.lock}
file. This project doesn't have dependencies, so the file is a bit
sparse. Realistically, you won't ever need to touch this file yourself;
just let Cargo handle it.

That's it! If you've been following along, you should have successfully
built \texttt{hello\_world} with Cargo.

Even though the project is simple, it now uses much of the real tooling
you'll use for the rest of your Rust career. In fact, you can expect to
start virtually all Rust projects with some variation on the following
commands:

\begin{Shaded}
\begin{Highlighting}[]
\NormalTok{$ }\KeywordTok{git} \NormalTok{clone someurl.com/foo}
\NormalTok{$ }\KeywordTok{cd} \NormalTok{foo}
\NormalTok{$ }\KeywordTok{cargo} \NormalTok{build}
\end{Highlighting}
\end{Shaded}

\subsubsection{Making A New Cargo Project the Easy
Way}\label{making-a-new-cargo-project-the-easy-way}

You don't have to go through that previous process every time you want
to start a new project! Cargo can quickly make a bare-bones project
directory that you can start developing in right away.

To start a new project with Cargo, enter \texttt{cargo\ new} at the
command line:

\begin{Shaded}
\begin{Highlighting}[]
\NormalTok{$ }\KeywordTok{cargo} \NormalTok{new hello_world --bin}
\end{Highlighting}
\end{Shaded}

This command passes \texttt{-\/-bin} because the goal is to get straight
to making an executable application, as opposed to a library.
Executables are often called \emph{binaries} (as in \texttt{/usr/bin},
if you're on a Unix system).

Cargo has generated two files and one directory for us: a
\texttt{Cargo.toml} and a \emph{src} directory with a \emph{main.rs}
file inside. These should look familiar, they're exactly what we created
by hand, above.

This output is all you need to get started. First, open
\texttt{Cargo.toml}. It should look something like this:

\begin{verbatim}
[package]

name = "hello_world"
version = "0.1.0"
authors = ["Your Name <you@example.com>"]

[dependencies]
\end{verbatim}

Do not worry about the \texttt{{[}dependencies{]}} line, we will come
back to it later.

Cargo has populated \emph{Cargo.toml} with reasonable defaults based on
the arguments you gave it and your \texttt{git} global configuration.
You may notice that Cargo has also initialized the \texttt{hello\_world}
directory as a \texttt{git} repository.

Here's what should be in \texttt{src/main.rs}:

\begin{Shaded}
\begin{Highlighting}[]
\KeywordTok{fn} \NormalTok{main() \{}
    \PreprocessorTok{println!}\NormalTok{(}\StringTok{"Hello, world!"}\NormalTok{);}
\NormalTok{\}}
\end{Highlighting}
\end{Shaded}

Cargo has generated a ``Hello World!'' for you, and you're ready to
start coding!

\begin{quote}
Note: If you want to look at Cargo in more detail, check out the
official {[}Cargo guide{]}, which covers all of its features.
\end{quote}

\subsection{Closing Thoughts}\label{closing-thoughts}

This chapter covered the basics that will serve you well through the
rest of this book, and the rest of your time with Rust. Now that you've
got the tools down, we'll cover more about the Rust language itself.

You have two options: Dive into a project with
`\protect\hyperlink{sec--guessing-game}{Tutorial: Guessing Game}', or
start from the bottom and work your way up with
`\protect\hyperlink{sec--syntax-and-semantics}{Syntax and Semantics}'.
More experienced systems programmers will probably prefer `Tutorial:
Guessing Game', while those from dynamic backgrounds may enjoy either.
Different people learn differently! Choose whatever's right for you.

\hypertarget{sec--guessing-game}{\chapter{Tutorial: Guessing
Game}\label{sec--guessing-game}}

Let's learn some Rust! For our first project, we'll implement a classic
beginner programming problem: the guessing game. Here's how it works:
Our program will generate a random integer between one and a hundred. It
will then prompt us to enter a guess. Upon entering our guess, it will
tell us if we're too low or too high. Once we guess correctly, it will
congratulate us. Sounds good?

Along the way, we'll learn a little bit about Rust. The next chapter,
`Syntax and Semantics', will dive deeper into each part.

\subsection{Set up}\label{set-up}

Let's set up a new project. Go to your projects directory. Remember how
we had to create our directory structure and a \texttt{Cargo.toml} for
\texttt{hello\_world}? Cargo has a command that does that for us. Let's
give it a shot:

\begin{Shaded}
\begin{Highlighting}[]
\NormalTok{$ }\KeywordTok{cd} \NormalTok{~/projects}
\NormalTok{$ }\KeywordTok{cargo} \NormalTok{new guessing_game --bin}
\NormalTok{$ }\KeywordTok{cd} \NormalTok{guessing_game}
\end{Highlighting}
\end{Shaded}

We pass the name of our project to \texttt{cargo\ new}, and then the
\texttt{-\/-bin} flag, since we're making a binary, rather than a
library.

Check out the generated \texttt{Cargo.toml}:

\begin{verbatim}
[package]

name = "guessing_game"
version = "0.1.0"
authors = ["Your Name <you@example.com>"]
\end{verbatim}

Cargo gets this information from your environment. If it's not correct,
go ahead and fix that.

Finally, Cargo generated a `Hello, world!' for us. Check out
\texttt{src/main.rs}:

\begin{Shaded}
\begin{Highlighting}[]
\KeywordTok{fn} \NormalTok{main() \{}
    \PreprocessorTok{println!}\NormalTok{(}\StringTok{"Hello, world!"}\NormalTok{);}
\NormalTok{\}}
\end{Highlighting}
\end{Shaded}

Let's try compiling what Cargo gave us:

\begin{verbatim}
$ cargo build
   Compiling guessing_game v0.1.0 (file:///home/you/projects/guessing_game)
\end{verbatim}

Excellent! Open up your \texttt{src/main.rs} again. We'll be writing all
of our code in this file.

Before we move on, let me show you one more Cargo command: \texttt{run}.
\texttt{cargo\ run} is kind of like \texttt{cargo\ build}, but it also
then runs the produced executable. Try it out:

\begin{Shaded}
\begin{Highlighting}[]
\NormalTok{$ }\KeywordTok{cargo} \NormalTok{run}
   \KeywordTok{Compiling} \NormalTok{guessing_game v0.1.0 (file:///home/you/projects/guessing_game)}
     \KeywordTok{Running} \KeywordTok{`target/debug/guessing_game`}
\KeywordTok{Hello}\NormalTok{, world!}
\end{Highlighting}
\end{Shaded}

Great! The \texttt{run} command comes in handy when you need to rapidly
iterate on a project. Our game is such a project, we need to quickly
test each iteration before moving on to the next one.

\subsection{Processing a Guess}\label{processing-a-guess}

Let's get to it! The first thing we need to do for our guessing game is
allow our player to input a guess. Put this in your
\texttt{src/main.rs}:

\begin{Shaded}
\begin{Highlighting}[]
\KeywordTok{use} \NormalTok{std::io;}

\KeywordTok{fn} \NormalTok{main() \{}
    \PreprocessorTok{println!}\NormalTok{(}\StringTok{"Guess the number!"}\NormalTok{);}

    \PreprocessorTok{println!}\NormalTok{(}\StringTok{"Please input your guess."}\NormalTok{);}

    \KeywordTok{let} \KeywordTok{mut} \NormalTok{guess = }\DataTypeTok{String}\NormalTok{::new();}

    \NormalTok{io::stdin().read_line(&}\KeywordTok{mut} \NormalTok{guess)}
        \NormalTok{.expect(}\StringTok{"Failed to read line"}\NormalTok{);}

    \PreprocessorTok{println!}\NormalTok{(}\StringTok{"You guessed: \{\}"}\NormalTok{, guess);}
\NormalTok{\}}
\end{Highlighting}
\end{Shaded}

There's a lot here! Let's go over it, bit by bit.

\begin{Shaded}
\begin{Highlighting}[]
\KeywordTok{use} \NormalTok{std::io;}
\end{Highlighting}
\end{Shaded}

We'll need to take user input, and then print the result as output. As
such, we need the \texttt{io} library from the standard library. Rust
only imports a few things by default into every program,
\href{http://doc.rust-lang.org/std/prelude/index.html}{the `prelude'}.
If it's not in the prelude, you'll have to \texttt{use} it directly.
There is also a second `prelude', the
\href{http://doc.rust-lang.org/std/io/prelude/index.html}{\texttt{io}
prelude}, which serves a similar function: you import it, and it imports
a number of useful, \texttt{io}-related things.

\begin{Shaded}
\begin{Highlighting}[]
\KeywordTok{fn} \NormalTok{main() \{}
\end{Highlighting}
\end{Shaded}

As you've seen before, the \texttt{main()} function is the entry point
into your program. The \texttt{fn} syntax declares a new function, the
\texttt{()}s indicate that there are no arguments, and \texttt{\{}
starts the body of the function. Because we didn't include a return
type, it's assumed to be \texttt{()}, an empty
\protect\hyperlink{tuples}{tuple}.

\begin{Shaded}
\begin{Highlighting}[]
    \PreprocessorTok{println!}\NormalTok{(}\StringTok{"Guess the number!"}\NormalTok{);}

    \PreprocessorTok{println!}\NormalTok{(}\StringTok{"Please input your guess."}\NormalTok{);}
\end{Highlighting}
\end{Shaded}

We previously learned that \texttt{println!()} is a
\protect\hyperlink{sec--macros}{macro} that prints a
\protect\hyperlink{sec--strings}{string} to the screen.

\begin{Shaded}
\begin{Highlighting}[]
    \KeywordTok{let} \KeywordTok{mut} \NormalTok{guess = }\DataTypeTok{String}\NormalTok{::new();}
\end{Highlighting}
\end{Shaded}

Now we're getting interesting! There's a lot going on in this little
line. The first thing to notice is that this is a
\protect\hyperlink{sec--variable-bindings}{let statement}, which is used
to create `variable bindings'. They take this form:

\begin{Shaded}
\begin{Highlighting}[]
\KeywordTok{let} \NormalTok{foo = bar;}
\end{Highlighting}
\end{Shaded}

This will create a new binding named \texttt{foo}, and bind it to the
value \texttt{bar}. In many languages, this is called a `variable', but
Rust's variable bindings have a few tricks up their sleeves.

For example, they're \protect\hyperlink{sec--mutability}{immutable} by
default. That's why our example uses \texttt{mut}: it makes a binding
mutable, rather than immutable. \texttt{let} doesn't take a name on the
left hand side of the assignment, it actually accepts a
`\protect\hyperlink{sec--patterns}{pattern}'. We'll use patterns later.
It's easy enough to use for now:

\begin{Shaded}
\begin{Highlighting}[]
\KeywordTok{let} \NormalTok{foo = }\DecValTok{5}\NormalTok{; }\CommentTok{// immutable.}
\KeywordTok{let} \KeywordTok{mut} \NormalTok{bar = }\DecValTok{5}\NormalTok{; }\CommentTok{// mutable}
\end{Highlighting}
\end{Shaded}

Oh, and \texttt{//} will start a comment, until the end of the line.
Rust ignores everything in \protect\hyperlink{sec--comments}{comments}.

So now we know that \texttt{let\ mut\ guess} will introduce a mutable
binding named \texttt{guess}, but we have to look at the other side of
the \texttt{=} for what it's bound to: \texttt{String::new()}.

\texttt{String} is a string type, provided by the standard library. A
\href{http://doc.rust-lang.org/std/string/struct.String.html}{\texttt{String}}
is a growable, UTF-8 encoded bit of text.

The \texttt{::new()} syntax uses \texttt{::} because this is an
`associated function' of a particular type. That is to say, it's
associated with \texttt{String} itself, rather than a particular
instance of a \texttt{String}. Some languages call this a `static
method'.

This function is named \texttt{new()}, because it creates a new, empty
\texttt{String}. You'll find a \texttt{new()} function on many types, as
it's a common name for making a new value of some kind.

Let's move forward:

\begin{Shaded}
\begin{Highlighting}[]
    \NormalTok{io::stdin().read_line(&}\KeywordTok{mut} \NormalTok{guess)}
        \NormalTok{.expect(}\StringTok{"Failed to read line"}\NormalTok{);}
\end{Highlighting}
\end{Shaded}

That's a lot more! Let's go bit-by-bit. The first line has two parts.
Here's the first:

\begin{Shaded}
\begin{Highlighting}[]
\NormalTok{io::stdin()}
\end{Highlighting}
\end{Shaded}

Remember how we \texttt{use}d \texttt{std::io} on the first line of the
program? We're now calling an associated function on it. If we didn't
\texttt{use\ std::io}, we could have written this line as
\texttt{std::io::stdin()}.

This particular function returns a handle to the standard input for your
terminal. More specifically, a
\href{http://doc.rust-lang.org/std/io/struct.Stdin.html}{std::io::Stdin}.

The next part will use this handle to get input from the user:

\begin{Shaded}
\begin{Highlighting}[]
\NormalTok{.read_line(&}\KeywordTok{mut} \NormalTok{guess)}
\end{Highlighting}
\end{Shaded}

Here, we call the
\href{http://doc.rust-lang.org/std/io/struct.Stdin.html\#method.read_line}{\texttt{read\_line()}}
method on our handle. \protect\hyperlink{sec--method-syntax}{Methods}
are like associated functions, but are only available on a particular
instance of a type, rather than the type itself. We're also passing one
argument to \texttt{read\_line()}: \texttt{\&mut\ guess}.

Remember how we bound \texttt{guess} above? We said it was mutable.
However, \texttt{read\_line} doesn't take a \texttt{String} as an
argument: it takes a \texttt{\&mut\ String}. Rust has a feature called
`\protect\hyperlink{sec--references-and-borrowing}{references}', which
allows you to have multiple references to one piece of data, which can
reduce copying. References are a complex feature, as one of Rust's major
selling points is how safe and easy it is to use references. We don't
need to know a lot of those details to finish our program right now,
though. For now, all we need to know is that like \texttt{let} bindings,
references are immutable by default. Hence, we need to write
\texttt{\&mut\ guess}, rather than \texttt{\&guess}.

Why does \texttt{read\_line()} take a mutable reference to a string? Its
job is to take what the user types into standard input, and place that
into a string. So it takes that string as an argument, and in order to
add the input, it needs to be mutable.

But we're not quite done with this line of code, though. While it's a
single line of text, it's only the first part of the single logical line
of code:

\begin{Shaded}
\begin{Highlighting}[]
        \NormalTok{.expect(}\StringTok{"Failed to read line"}\NormalTok{);}
\end{Highlighting}
\end{Shaded}

When you call a method with the \texttt{.foo()} syntax, you may
introduce a newline and other whitespace. This helps you split up long
lines. We \emph{could} have done:

\begin{Shaded}
\begin{Highlighting}[]
    \NormalTok{io::stdin().read_line(&}\KeywordTok{mut} \NormalTok{guess).expect(}\StringTok{"failed to read line"}\NormalTok{);}
\end{Highlighting}
\end{Shaded}

But that gets hard to read. So we've split it up, two lines for two
method calls. We already talked about \texttt{read\_line()}, but what
about \texttt{expect()}? Well, we already mentioned that
\texttt{read\_line()} puts what the user types into the
\texttt{\&mut\ String} we pass it. But it also returns a value: in this
case, an
\href{http://doc.rust-lang.org/std/io/type.Result.html}{\texttt{io::Result}}.
Rust has a number of types named \texttt{Result} in its standard
library: a generic
\href{http://doc.rust-lang.org/std/result/enum.Result.html}{\texttt{Result}},
and then specific versions for sub-libraries, like \texttt{io::Result}.

The purpose of these \texttt{Result} types is to encode error handling
information. Values of the \texttt{Result} type, like any type, have
methods defined on them. In this case, \texttt{io::Result} has an
\href{http://doc.rust-lang.org/std/result/enum.Result.html\#method.expect}{\texttt{expect()}
method} that takes a value it's called on, and if it isn't a successful
one, \protect\hyperlink{sec--error-handling}{\texttt{panic!}}s with a
message you passed it. A \texttt{panic!} like this will cause our
program to crash, displaying the message.

If we leave off calling this method, our program will compile, but we'll
get a warning:

\begin{Shaded}
\begin{Highlighting}[]
\NormalTok{$ }\KeywordTok{cargo} \NormalTok{build}
   \KeywordTok{Compiling} \NormalTok{guessing_game v0.1.0 (file:///home/you/projects/guessing_game)}
\KeywordTok{src}\NormalTok{/main.rs:}\KeywordTok{10}\NormalTok{:5: 10:39 warning: unused result which must be used,}
\CommentTok{#[warn(unused_must_use)] on by default}
\KeywordTok{src}\NormalTok{/main.rs:}\KeywordTok{10}     \NormalTok{io::stdin()}\KeywordTok{.read_line}\NormalTok{(}\KeywordTok{&mut} \NormalTok{guess);}
                   \NormalTok{^}\KeywordTok{~~~~~~~~~~~~~~~~~~~~~~~~~~~~~~~~~}
\end{Highlighting}
\end{Shaded}

Rust warns us that we haven't used the \texttt{Result} value. This
warning comes from a special annotation that \texttt{io::Result} has.
Rust is trying to tell you that you haven't handled a possible error.
The right way to suppress the error is to actually write error handling.
Luckily, if we want to crash if there's a problem, we can use
\texttt{expect()}. If we can recover from the error somehow, we'd do
something else, but we'll save that for a future project.

There's only one line of this first example left:

\begin{Shaded}
\begin{Highlighting}[]
    \PreprocessorTok{println!}\NormalTok{(}\StringTok{"You guessed: \{\}"}\NormalTok{, guess);}
\NormalTok{\}}
\end{Highlighting}
\end{Shaded}

This prints out the string we saved our input in. The \texttt{\{\}}s are
a placeholder, and so we pass it \texttt{guess} as an argument. If we
had multiple \texttt{\{\}}s, we would pass multiple arguments:

\begin{Shaded}
\begin{Highlighting}[]
\KeywordTok{let} \NormalTok{x = }\DecValTok{5}\NormalTok{;}
\KeywordTok{let} \NormalTok{y = }\DecValTok{10}\NormalTok{;}

\PreprocessorTok{println!}\NormalTok{(}\StringTok{"x and y: \{\} and \{\}"}\NormalTok{, x, y);}
\end{Highlighting}
\end{Shaded}

Easy.

Anyway, that's the tour. We can run what we have with
\texttt{cargo\ run}:

\begin{Shaded}
\begin{Highlighting}[]
\NormalTok{$ }\KeywordTok{cargo} \NormalTok{run}
   \KeywordTok{Compiling} \NormalTok{guessing_game v0.1.0 (file:///home/you/projects/guessing_game)}
     \KeywordTok{Running} \KeywordTok{`target/debug/guessing_game`}
\KeywordTok{Guess} \NormalTok{the number!}
\KeywordTok{Please} \NormalTok{input your guess.}
\KeywordTok{6}
\KeywordTok{You} \NormalTok{guessed: 6}
\end{Highlighting}
\end{Shaded}

All right! Our first part is done: we can get input from the keyboard,
and then print it back out.

\subsection{Generating a secret
number}\label{generating-a-secret-number}

Next, we need to generate a secret number. Rust does not yet include
random number functionality in its standard library. The Rust team does,
however, provide a \href{https://crates.io/crates/rand}{\texttt{rand}
crate}. A `crate' is a package of Rust code. We've been building a
`binary crate', which is an executable. \texttt{rand} is a `library
crate', which contains code that's intended to be used with other
programs.

Using external crates is where Cargo really shines. Before we can write
the code using \texttt{rand}, we need to modify our \texttt{Cargo.toml}.
Open it up, and add these few lines at the bottom:

\begin{verbatim}
[dependencies]

rand="0.3.0"
\end{verbatim}

The \texttt{{[}dependencies{]}} section of \texttt{Cargo.toml} is like
the \texttt{{[}package{]}} section: everything that follows it is part
of it, until the next section starts. Cargo uses the dependencies
section to know what dependencies on external crates you have, and what
versions you require. In this case, we've specified version
\texttt{0.3.0}, which Cargo understands to be any release that's
compatible with this specific version. Cargo understands
\href{http://semver.org}{Semantic Versioning}, which is a standard for
writing version numbers. A bare number like above is actually shorthand
for \texttt{\^{}0.3.0}, meaning ``anything compatible with 0.3.0''. If
we wanted to use only \texttt{0.3.0} exactly, we could say
\texttt{rand="=0.3.0"} (note the two equal signs). And if we wanted to
use the latest version we could use \texttt{*}. We could also use a
range of versions. \href{http://doc.crates.io/crates-io.html}{Cargo's
documentation} contains more details.

Now, without changing any of our code, let's build our project:

\begin{Shaded}
\begin{Highlighting}[]
\NormalTok{$ }\KeywordTok{cargo} \NormalTok{build}
    \KeywordTok{Updating} \NormalTok{registry }\KeywordTok{`https}\NormalTok{://github.com/rust-lang/crates.io-index}\KeywordTok{`}
 \KeywordTok{Downloading} \NormalTok{rand v0.3.8}
 \KeywordTok{Downloading} \NormalTok{libc v0.1.6}
   \KeywordTok{Compiling} \NormalTok{libc v0.1.6}
   \KeywordTok{Compiling} \NormalTok{rand v0.3.8}
   \KeywordTok{Compiling} \NormalTok{guessing_game v0.1.0 (file:///home/you/projects/guessing_game)}
\end{Highlighting}
\end{Shaded}

(You may see different versions, of course.)

Lots of new output! Now that we have an external dependency, Cargo
fetches the latest versions of everything from the registry, which is a
copy of data from \href{https://crates.io}{Crates.io}. Crates.io is
where people in the Rust ecosystem post their open source Rust projects
for others to use.

After updating the registry, Cargo checks our
\texttt{{[}dependencies{]}} and downloads any we don't have yet. In this
case, while we only said we wanted to depend on \texttt{rand}, we've
also grabbed a copy of \texttt{libc}. This is because \texttt{rand}
depends on \texttt{libc} to work. After downloading them, it compiles
them, and then compiles our project.

If we run \texttt{cargo\ build} again, we'll get different output:

\begin{Shaded}
\begin{Highlighting}[]
\NormalTok{$ }\KeywordTok{cargo} \NormalTok{build}
\end{Highlighting}
\end{Shaded}

That's right, no output! Cargo knows that our project has been built,
and that all of its dependencies are built, and so there's no reason to
do all that stuff. With nothing to do, it simply exits. If we open up
\texttt{src/main.rs} again, make a trivial change, and then save it
again, we'll only see one line:

\begin{Shaded}
\begin{Highlighting}[]
\NormalTok{$ }\KeywordTok{cargo} \NormalTok{build}
   \KeywordTok{Compiling} \NormalTok{guessing_game v0.1.0 (file:///home/you/projects/guessing_game)}
\end{Highlighting}
\end{Shaded}

So, we told Cargo we wanted any \texttt{0.3.x} version of \texttt{rand},
and so it fetched the latest version at the time this was written,
\texttt{v0.3.8}. But what happens when next week, version
\texttt{v0.3.9} comes out, with an important bugfix? While getting
bugfixes is important, what if \texttt{0.3.9} contains a regression that
breaks our code?

The answer to this problem is the \texttt{Cargo.lock} file you'll now
find in your project directory. When you build your project for the
first time, Cargo figures out all of the versions that fit your
criteria, and then writes them to the \texttt{Cargo.lock} file. When you
build your project in the future, Cargo will see that the
\texttt{Cargo.lock} file exists, and then use that specific version
rather than do all the work of figuring out versions again. This lets
you have a repeatable build automatically. In other words, we'll stay at
\texttt{0.3.8} until we explicitly upgrade, and so will anyone who we
share our code with, thanks to the lock file.

What about when we \emph{do} want to use \texttt{v0.3.9}? Cargo has
another command, \texttt{update}, which says `ignore the lock, figure
out all the latest versions that fit what we've specified. If that
works, write those versions out to the lock file'. But, by default,
Cargo will only look for versions larger than \texttt{0.3.0} and smaller
than \texttt{0.4.0}. If we want to move to \texttt{0.4.x}, we'd have to
update the \texttt{Cargo.toml} directly. When we do, the next time we
\texttt{cargo\ build}, Cargo will update the index and re-evaluate our
\texttt{rand} requirements.

There's a lot more to say about \href{http://doc.crates.io}{Cargo} and
\href{http://doc.crates.io/crates-io.html}{its ecosystem}, but for now,
that's all we need to know. Cargo makes it really easy to re-use
libraries, and so Rustaceans tend to write smaller projects which are
assembled out of a number of sub-packages.

Let's get on to actually \emph{using} \texttt{rand}. Here's our next
step:

\begin{Shaded}
\begin{Highlighting}[]
\KeywordTok{extern} \KeywordTok{crate} \NormalTok{rand;}

\KeywordTok{use} \NormalTok{std::io;}
\KeywordTok{use} \NormalTok{rand::Rng;}

\KeywordTok{fn} \NormalTok{main() \{}
    \PreprocessorTok{println!}\NormalTok{(}\StringTok{"Guess the number!"}\NormalTok{);}

    \KeywordTok{let} \NormalTok{secret_number = rand::thread_rng().gen_range(}\DecValTok{1}\NormalTok{, }\DecValTok{101}\NormalTok{);}

    \PreprocessorTok{println!}\NormalTok{(}\StringTok{"The secret number is: \{\}"}\NormalTok{, secret_number);}

    \PreprocessorTok{println!}\NormalTok{(}\StringTok{"Please input your guess."}\NormalTok{);}

    \KeywordTok{let} \KeywordTok{mut} \NormalTok{guess = }\DataTypeTok{String}\NormalTok{::new();}

    \NormalTok{io::stdin().read_line(&}\KeywordTok{mut} \NormalTok{guess)}
        \NormalTok{.expect(}\StringTok{"failed to read line"}\NormalTok{);}

    \PreprocessorTok{println!}\NormalTok{(}\StringTok{"You guessed: \{\}"}\NormalTok{, guess);}
\NormalTok{\}}
\end{Highlighting}
\end{Shaded}

The first thing we've done is change the first line. It now says
\texttt{extern\ crate\ rand}. Because we declared \texttt{rand} in our
\texttt{{[}dependencies{]}}, we can use \texttt{extern\ crate} to let
Rust know we'll be making use of it. This also does the equivalent of a
\texttt{use\ rand;} as well, so we can make use of anything in the
\texttt{rand} crate by prefixing it with \texttt{rand::}.

Next, we added another \texttt{use} line: \texttt{use\ rand::Rng}. We're
going to use a method in a moment, and it requires that \texttt{Rng} be
in scope to work. The basic idea is this: methods are defined on
something called `traits', and for the method to work, it needs the
trait to be in scope. For more about the details, read the
\protect\hyperlink{sec--traits}{traits} section.

There are two other lines we added, in the middle:

\begin{Shaded}
\begin{Highlighting}[]
    \KeywordTok{let} \NormalTok{secret_number = rand::thread_rng().gen_range(}\DecValTok{1}\NormalTok{, }\DecValTok{101}\NormalTok{);}

    \PreprocessorTok{println!}\NormalTok{(}\StringTok{"The secret number is: \{\}"}\NormalTok{, secret_number);}
\end{Highlighting}
\end{Shaded}

We use the \texttt{rand::thread\_rng()} function to get a copy of the
random number generator, which is local to the particular
\protect\hyperlink{sec--concurrency}{thread} of execution we're in.
Because we \texttt{use\ rand::Rng}'d above, it has a
\texttt{gen\_range()} method available. This method takes two arguments,
and generates a number between them. It's inclusive on the lower bound,
but exclusive on the upper bound, so we need \texttt{1} and \texttt{101}
to get a number ranging from one to a hundred.

The second line prints out the secret number. This is useful while we're
developing our program, so we can easily test it out. But we'll be
deleting it for the final version. It's not much of a game if it prints
out the answer when you start it up!

Try running our new program a few times:

\begin{Shaded}
\begin{Highlighting}[]
\NormalTok{$ }\KeywordTok{cargo} \NormalTok{run}
   \KeywordTok{Compiling} \NormalTok{guessing_game v0.1.0 (file:///home/you/projects/guessing_game)}
     \KeywordTok{Running} \KeywordTok{`target/debug/guessing_game`}
\KeywordTok{Guess} \NormalTok{the number!}
\KeywordTok{The} \NormalTok{secret number is: 7}
\KeywordTok{Please} \NormalTok{input your guess.}
\KeywordTok{4}
\KeywordTok{You} \NormalTok{guessed: 4}
\NormalTok{$ }\KeywordTok{cargo} \NormalTok{run}
     \KeywordTok{Running} \KeywordTok{`target/debug/guessing_game`}
\KeywordTok{Guess} \NormalTok{the number!}
\KeywordTok{The} \NormalTok{secret number is: 83}
\KeywordTok{Please} \NormalTok{input your guess.}
\KeywordTok{5}
\KeywordTok{You} \NormalTok{guessed: 5}
\end{Highlighting}
\end{Shaded}

Great! Next up: comparing our guess to the secret number.

\subsection{Comparing guesses}\label{comparing-guesses}

Now that we've got user input, let's compare our guess to the secret
number. Here's our next step, though it doesn't quite compile yet:

\begin{Shaded}
\begin{Highlighting}[]
\KeywordTok{extern} \KeywordTok{crate} \NormalTok{rand;}

\KeywordTok{use} \NormalTok{std::io;}
\KeywordTok{use} \NormalTok{std::cmp::Ordering;}
\KeywordTok{use} \NormalTok{rand::Rng;}

\KeywordTok{fn} \NormalTok{main() \{}
    \PreprocessorTok{println!}\NormalTok{(}\StringTok{"Guess the number!"}\NormalTok{);}

    \KeywordTok{let} \NormalTok{secret_number = rand::thread_rng().gen_range(}\DecValTok{1}\NormalTok{, }\DecValTok{101}\NormalTok{);}

    \PreprocessorTok{println!}\NormalTok{(}\StringTok{"The secret number is: \{\}"}\NormalTok{, secret_number);}

    \PreprocessorTok{println!}\NormalTok{(}\StringTok{"Please input your guess."}\NormalTok{);}

    \KeywordTok{let} \KeywordTok{mut} \NormalTok{guess = }\DataTypeTok{String}\NormalTok{::new();}

    \NormalTok{io::stdin().read_line(&}\KeywordTok{mut} \NormalTok{guess)}
        \NormalTok{.expect(}\StringTok{"failed to read line"}\NormalTok{);}

    \PreprocessorTok{println!}\NormalTok{(}\StringTok{"You guessed: \{\}"}\NormalTok{, guess);}

    \KeywordTok{match} \NormalTok{guess.cmp(&secret_number) \{}
        \NormalTok{Ordering::Less    => }\PreprocessorTok{println!}\NormalTok{(}\StringTok{"Too small!"}\NormalTok{),}
        \NormalTok{Ordering::Greater => }\PreprocessorTok{println!}\NormalTok{(}\StringTok{"Too big!"}\NormalTok{),}
        \NormalTok{Ordering::Equal   => }\PreprocessorTok{println!}\NormalTok{(}\StringTok{"You win!"}\NormalTok{),}
    \NormalTok{\}}
\NormalTok{\}}
\end{Highlighting}
\end{Shaded}

A few new bits here. The first is another \texttt{use}. We bring a type
called \texttt{std::cmp::Ordering} into scope. Then, five new lines at
the bottom that use it:

\begin{Shaded}
\begin{Highlighting}[]
\KeywordTok{match} \NormalTok{guess.cmp(&secret_number) \{}
    \NormalTok{Ordering::Less    => }\PreprocessorTok{println!}\NormalTok{(}\StringTok{"Too small!"}\NormalTok{),}
    \NormalTok{Ordering::Greater => }\PreprocessorTok{println!}\NormalTok{(}\StringTok{"Too big!"}\NormalTok{),}
    \NormalTok{Ordering::Equal   => }\PreprocessorTok{println!}\NormalTok{(}\StringTok{"You win!"}\NormalTok{),}
\NormalTok{\}}
\end{Highlighting}
\end{Shaded}

The \texttt{cmp()} method can be called on anything that can be
compared, and it takes a reference to the thing you want to compare it
to. It returns the \texttt{Ordering} type we \texttt{use}d earlier. We
use a \protect\hyperlink{sec--match}{\texttt{match}} statement to
determine exactly what kind of \texttt{Ordering} it is.
\texttt{Ordering} is an \protect\hyperlink{sec--enums}{\texttt{enum}},
short for `enumeration', which looks like this:

\begin{Shaded}
\begin{Highlighting}[]
\KeywordTok{enum} \NormalTok{Foo \{}
    \NormalTok{Bar,}
    \NormalTok{Baz,}
\NormalTok{\}}
\end{Highlighting}
\end{Shaded}

With this definition, anything of type \texttt{Foo} can be either a
\texttt{Foo::Bar} or a \texttt{Foo::Baz}. We use the \texttt{::} to
indicate the namespace for a particular \texttt{enum} variant.

The
\href{http://doc.rust-lang.org/std/cmp/enum.Ordering.html}{\texttt{Ordering}}
\texttt{enum} has three possible variants: \texttt{Less},
\texttt{Equal}, and \texttt{Greater}. The \texttt{match} statement takes
a value of a type, and lets you create an `arm' for each possible value.
Since we have three types of \texttt{Ordering}, we have three arms:

\begin{Shaded}
\begin{Highlighting}[]
\KeywordTok{match} \NormalTok{guess.cmp(&secret_number) \{}
    \NormalTok{Ordering::Less    => }\PreprocessorTok{println!}\NormalTok{(}\StringTok{"Too small!"}\NormalTok{),}
    \NormalTok{Ordering::Greater => }\PreprocessorTok{println!}\NormalTok{(}\StringTok{"Too big!"}\NormalTok{),}
    \NormalTok{Ordering::Equal   => }\PreprocessorTok{println!}\NormalTok{(}\StringTok{"You win!"}\NormalTok{),}
\NormalTok{\}}
\end{Highlighting}
\end{Shaded}

If it's \texttt{Less}, we print \texttt{Too\ small!}, if it's
\texttt{Greater}, \texttt{Too\ big!}, and if \texttt{Equal},
\texttt{You\ win!}. \texttt{match} is really useful, and is used often
in Rust.

I did mention that this won't quite compile yet, though. Let's try it:

\begin{Shaded}
\begin{Highlighting}[]
\NormalTok{$ }\KeywordTok{cargo} \NormalTok{build}
   \KeywordTok{Compiling} \NormalTok{guessing_game v0.1.0 (file:///home/you/projects/guessing_game)}
\KeywordTok{src}\NormalTok{/main.rs:}\KeywordTok{28}\NormalTok{:21: 28:35 error: mismatched types:}
 \KeywordTok{expected} \KeywordTok{`&collections}\NormalTok{::string::String}\KeywordTok{`}\NormalTok{,}
    \KeywordTok{found} \KeywordTok{`&_`}
\KeywordTok{(expected} \NormalTok{struct }\KeywordTok{`collections}\NormalTok{::string::String}\KeywordTok{`}\NormalTok{,}
    \KeywordTok{found} \NormalTok{integral variable}\KeywordTok{)} \NormalTok{[}\KeywordTok{E0308}\NormalTok{]}
\KeywordTok{src}\NormalTok{/main.rs:}\KeywordTok{28}     \NormalTok{match guess.cmp(}\KeywordTok{&secret_number}\NormalTok{) }\KeywordTok{\{}
                                   \NormalTok{^}\KeywordTok{~~~~~~~~~~~~~}
\KeywordTok{error}\NormalTok{: aborting due to previous error}
\KeywordTok{Could} \NormalTok{not compile }\KeywordTok{`guessing_game`}\NormalTok{.}
\end{Highlighting}
\end{Shaded}

Whew! This is a big error. The core of it is that we have `mismatched
types'. Rust has a strong, static type system. However, it also has type
inference. When we wrote \texttt{let\ guess\ =\ String::new()}, Rust was
able to infer that \texttt{guess} should be a \texttt{String}, and so it
doesn't make us write out the type. And with our
\texttt{secret\_number}, there are a number of types which can have a
value between one and a hundred: \texttt{i32}, a thirty-two-bit number,
or \texttt{u32}, an unsigned thirty-two-bit number, or \texttt{i64}, a
sixty-four-bit number or others. So far, that hasn't mattered, and so
Rust defaults to an \texttt{i32}. However, here, Rust doesn't know how
to compare the \texttt{guess} and the \texttt{secret\_number}. They need
to be the same type. Ultimately, we want to convert the \texttt{String}
we read as input into a real number type, for comparison. We can do that
with two more lines. Here's our new program:

\begin{Shaded}
\begin{Highlighting}[]
\KeywordTok{extern} \KeywordTok{crate} \NormalTok{rand;}

\KeywordTok{use} \NormalTok{std::io;}
\KeywordTok{use} \NormalTok{std::cmp::Ordering;}
\KeywordTok{use} \NormalTok{rand::Rng;}

\KeywordTok{fn} \NormalTok{main() \{}
    \PreprocessorTok{println!}\NormalTok{(}\StringTok{"Guess the number!"}\NormalTok{);}

    \KeywordTok{let} \NormalTok{secret_number = rand::thread_rng().gen_range(}\DecValTok{1}\NormalTok{, }\DecValTok{101}\NormalTok{);}

    \PreprocessorTok{println!}\NormalTok{(}\StringTok{"The secret number is: \{\}"}\NormalTok{, secret_number);}

    \PreprocessorTok{println!}\NormalTok{(}\StringTok{"Please input your guess."}\NormalTok{);}

    \KeywordTok{let} \KeywordTok{mut} \NormalTok{guess = }\DataTypeTok{String}\NormalTok{::new();}

    \NormalTok{io::stdin().read_line(&}\KeywordTok{mut} \NormalTok{guess)}
        \NormalTok{.expect(}\StringTok{"failed to read line"}\NormalTok{);}

    \KeywordTok{let} \NormalTok{guess: }\DataTypeTok{u32} \NormalTok{= guess.trim().parse()}
        \NormalTok{.expect(}\StringTok{"Please type a number!"}\NormalTok{);}

    \PreprocessorTok{println!}\NormalTok{(}\StringTok{"You guessed: \{\}"}\NormalTok{, guess);}

    \KeywordTok{match} \NormalTok{guess.cmp(&secret_number) \{}
        \NormalTok{Ordering::Less    => }\PreprocessorTok{println!}\NormalTok{(}\StringTok{"Too small!"}\NormalTok{),}
        \NormalTok{Ordering::Greater => }\PreprocessorTok{println!}\NormalTok{(}\StringTok{"Too big!"}\NormalTok{),}
        \NormalTok{Ordering::Equal   => }\PreprocessorTok{println!}\NormalTok{(}\StringTok{"You win!"}\NormalTok{),}
    \NormalTok{\}}
\NormalTok{\}}
\end{Highlighting}
\end{Shaded}

The new two lines:

\begin{Shaded}
\begin{Highlighting}[]
    \KeywordTok{let} \NormalTok{guess: }\DataTypeTok{u32} \NormalTok{= guess.trim().parse()}
        \NormalTok{.expect(}\StringTok{"Please type a number!"}\NormalTok{);}
\end{Highlighting}
\end{Shaded}

Wait a minute, I thought we already had a \texttt{guess}? We do, but
Rust allows us to `shadow' the previous \texttt{guess} with a new one.
This is often used in this exact situation, where \texttt{guess} starts
as a \texttt{String}, but we want to convert it to an \texttt{u32}.
Shadowing lets us re-use the \texttt{guess} name, rather than forcing us
to come up with two unique names like \texttt{guess\_str} and
\texttt{guess}, or something else.

We bind \texttt{guess} to an expression that looks like something we
wrote earlier:

\begin{Shaded}
\begin{Highlighting}[]
\NormalTok{guess.trim().parse()}
\end{Highlighting}
\end{Shaded}

Here, \texttt{guess} refers to the old \texttt{guess}, the one that was
a \texttt{String} with our input in it. The \texttt{trim()} method on
\texttt{String}s will eliminate any white space at the beginning and end
of our string. This is important, as we had to press the `return' key to
satisfy \texttt{read\_line()}. This means that if we type \texttt{5} and
hit return, \texttt{guess} looks like this: \texttt{5\textbackslash{}n}.
The \texttt{\textbackslash{}n} represents `newline', the enter key.
\texttt{trim()} gets rid of this, leaving our string with only the
\texttt{5}. The
\href{http://doc.rust-lang.org/std/primitive.str.html\#method.parse}{\texttt{parse()}
method on strings} parses a string into some kind of number. Since it
can parse a variety of numbers, we need to give Rust a hint as to the
exact type of number we want. Hence, \texttt{let\ guess:\ u32}. The
colon (\texttt{:}) after \texttt{guess} tells Rust we're going to
annotate its type. \texttt{u32} is an unsigned, thirty-two bit integer.
Rust has \protect\hyperlink{numeric-types}{a number of built-in number
types}, but we've chosen \texttt{u32}. It's a good default choice for a
small positive number.

Just like \texttt{read\_line()}, our call to \texttt{parse()} could
cause an error. What if our string contained \texttt{A👍\%}? There'd be
no way to convert that to a number. As such, we'll do the same thing we
did with \texttt{read\_line()}: use the \texttt{expect()} method to
crash if there's an error.

Let's try our program out!

\begin{Shaded}
\begin{Highlighting}[]
\NormalTok{$ }\KeywordTok{cargo} \NormalTok{run}
   \KeywordTok{Compiling} \NormalTok{guessing_game v0.1.0 (file:///home/you/projects/guessing_game)}
     \KeywordTok{Running} \KeywordTok{`target/guessing_game`}
\KeywordTok{Guess} \NormalTok{the number!}
\KeywordTok{The} \NormalTok{secret number is: 58}
\KeywordTok{Please} \NormalTok{input your guess.}
  \KeywordTok{76}
\KeywordTok{You} \NormalTok{guessed: 76}
\KeywordTok{Too} \NormalTok{big!}
\end{Highlighting}
\end{Shaded}

Nice! You can see I even added spaces before my guess, and it still
figured out that I guessed 76. Run the program a few times, and verify
that guessing the number works, as well as guessing a number too small.

Now we've got most of the game working, but we can only make one guess.
Let's change that by adding loops!

\subsection{Looping}\label{looping}

The \texttt{loop} keyword gives us an infinite loop. Let's add that in:

\begin{Shaded}
\begin{Highlighting}[]
\KeywordTok{extern} \KeywordTok{crate} \NormalTok{rand;}

\KeywordTok{use} \NormalTok{std::io;}
\KeywordTok{use} \NormalTok{std::cmp::Ordering;}
\KeywordTok{use} \NormalTok{rand::Rng;}

\KeywordTok{fn} \NormalTok{main() \{}
    \PreprocessorTok{println!}\NormalTok{(}\StringTok{"Guess the number!"}\NormalTok{);}

    \KeywordTok{let} \NormalTok{secret_number = rand::thread_rng().gen_range(}\DecValTok{1}\NormalTok{, }\DecValTok{101}\NormalTok{);}

    \PreprocessorTok{println!}\NormalTok{(}\StringTok{"The secret number is: \{\}"}\NormalTok{, secret_number);}

    \KeywordTok{loop} \NormalTok{\{}
        \PreprocessorTok{println!}\NormalTok{(}\StringTok{"Please input your guess."}\NormalTok{);}

        \KeywordTok{let} \KeywordTok{mut} \NormalTok{guess = }\DataTypeTok{String}\NormalTok{::new();}

        \NormalTok{io::stdin().read_line(&}\KeywordTok{mut} \NormalTok{guess)}
            \NormalTok{.expect(}\StringTok{"failed to read line"}\NormalTok{);}

        \KeywordTok{let} \NormalTok{guess: }\DataTypeTok{u32} \NormalTok{= guess.trim().parse()}
            \NormalTok{.expect(}\StringTok{"Please type a number!"}\NormalTok{);}

        \PreprocessorTok{println!}\NormalTok{(}\StringTok{"You guessed: \{\}"}\NormalTok{, guess);}

        \KeywordTok{match} \NormalTok{guess.cmp(&secret_number) \{}
            \NormalTok{Ordering::Less    => }\PreprocessorTok{println!}\NormalTok{(}\StringTok{"Too small!"}\NormalTok{),}
            \NormalTok{Ordering::Greater => }\PreprocessorTok{println!}\NormalTok{(}\StringTok{"Too big!"}\NormalTok{),}
            \NormalTok{Ordering::Equal   => }\PreprocessorTok{println!}\NormalTok{(}\StringTok{"You win!"}\NormalTok{),}
        \NormalTok{\}}
    \NormalTok{\}}
\NormalTok{\}}
\end{Highlighting}
\end{Shaded}

And try it out. But wait, didn't we just add an infinite loop? Yup.
Remember our discussion about \texttt{parse()}? If we give a non-number
answer, we'll \texttt{panic!} and quit. Observe:

\begin{Shaded}
\begin{Highlighting}[]
\NormalTok{$ }\KeywordTok{cargo} \NormalTok{run}
   \KeywordTok{Compiling} \NormalTok{guessing_game v0.1.0 (file:///home/you/projects/guessing_game)}
     \KeywordTok{Running} \KeywordTok{`target/guessing_game`}
\KeywordTok{Guess} \NormalTok{the number!}
\KeywordTok{The} \NormalTok{secret number is: 59}
\KeywordTok{Please} \NormalTok{input your guess.}
\KeywordTok{45}
\KeywordTok{You} \NormalTok{guessed: 45}
\KeywordTok{Too} \NormalTok{small!}
\KeywordTok{Please} \NormalTok{input your guess.}
\KeywordTok{60}
\KeywordTok{You} \NormalTok{guessed: 60}
\KeywordTok{Too} \NormalTok{big!}
\KeywordTok{Please} \NormalTok{input your guess.}
\KeywordTok{59}
\KeywordTok{You} \NormalTok{guessed: 59}
\KeywordTok{You} \NormalTok{win!}
\KeywordTok{Please} \NormalTok{input your guess.}
\KeywordTok{quit}
\KeywordTok{thread} \StringTok{'<main>'} \NormalTok{panicked at }\StringTok{'Please type a number!'}
\end{Highlighting}
\end{Shaded}

Ha! \texttt{quit} actually quits. As does any other non-number input.
Well, this is suboptimal to say the least. First, let's actually quit
when you win the game:

\begin{Shaded}
\begin{Highlighting}[]
\KeywordTok{extern} \KeywordTok{crate} \NormalTok{rand;}

\KeywordTok{use} \NormalTok{std::io;}
\KeywordTok{use} \NormalTok{std::cmp::Ordering;}
\KeywordTok{use} \NormalTok{rand::Rng;}

\KeywordTok{fn} \NormalTok{main() \{}
    \PreprocessorTok{println!}\NormalTok{(}\StringTok{"Guess the number!"}\NormalTok{);}

    \KeywordTok{let} \NormalTok{secret_number = rand::thread_rng().gen_range(}\DecValTok{1}\NormalTok{, }\DecValTok{101}\NormalTok{);}

    \PreprocessorTok{println!}\NormalTok{(}\StringTok{"The secret number is: \{\}"}\NormalTok{, secret_number);}

    \KeywordTok{loop} \NormalTok{\{}
        \PreprocessorTok{println!}\NormalTok{(}\StringTok{"Please input your guess."}\NormalTok{);}

        \KeywordTok{let} \KeywordTok{mut} \NormalTok{guess = }\DataTypeTok{String}\NormalTok{::new();}

        \NormalTok{io::stdin().read_line(&}\KeywordTok{mut} \NormalTok{guess)}
            \NormalTok{.expect(}\StringTok{"failed to read line"}\NormalTok{);}

        \KeywordTok{let} \NormalTok{guess: }\DataTypeTok{u32} \NormalTok{= guess.trim().parse()}
            \NormalTok{.expect(}\StringTok{"Please type a number!"}\NormalTok{);}

        \PreprocessorTok{println!}\NormalTok{(}\StringTok{"You guessed: \{\}"}\NormalTok{, guess);}

        \KeywordTok{match} \NormalTok{guess.cmp(&secret_number) \{}
            \NormalTok{Ordering::Less    => }\PreprocessorTok{println!}\NormalTok{(}\StringTok{"Too small!"}\NormalTok{),}
            \NormalTok{Ordering::Greater => }\PreprocessorTok{println!}\NormalTok{(}\StringTok{"Too big!"}\NormalTok{),}
            \NormalTok{Ordering::Equal   => \{}
                \PreprocessorTok{println!}\NormalTok{(}\StringTok{"You win!"}\NormalTok{);}
                \KeywordTok{break}\NormalTok{;}
            \NormalTok{\}}
        \NormalTok{\}}
    \NormalTok{\}}
\NormalTok{\}}
\end{Highlighting}
\end{Shaded}

By adding the \texttt{break} line after the \texttt{You\ win!}, we'll
exit the loop when we win. Exiting the loop also means exiting the
program, since it's the last thing in \texttt{main()}. We have only one
more tweak to make: when someone inputs a non-number, we don't want to
quit, we want to ignore it. We can do that like this:

\begin{Shaded}
\begin{Highlighting}[]
\KeywordTok{extern} \KeywordTok{crate} \NormalTok{rand;}

\KeywordTok{use} \NormalTok{std::io;}
\KeywordTok{use} \NormalTok{std::cmp::Ordering;}
\KeywordTok{use} \NormalTok{rand::Rng;}

\KeywordTok{fn} \NormalTok{main() \{}
    \PreprocessorTok{println!}\NormalTok{(}\StringTok{"Guess the number!"}\NormalTok{);}

    \KeywordTok{let} \NormalTok{secret_number = rand::thread_rng().gen_range(}\DecValTok{1}\NormalTok{, }\DecValTok{101}\NormalTok{);}

    \PreprocessorTok{println!}\NormalTok{(}\StringTok{"The secret number is: \{\}"}\NormalTok{, secret_number);}

    \KeywordTok{loop} \NormalTok{\{}
        \PreprocessorTok{println!}\NormalTok{(}\StringTok{"Please input your guess."}\NormalTok{);}

        \KeywordTok{let} \KeywordTok{mut} \NormalTok{guess = }\DataTypeTok{String}\NormalTok{::new();}

        \NormalTok{io::stdin().read_line(&}\KeywordTok{mut} \NormalTok{guess)}
            \NormalTok{.expect(}\StringTok{"failed to read line"}\NormalTok{);}

        \KeywordTok{let} \NormalTok{guess: }\DataTypeTok{u32} \NormalTok{= }\KeywordTok{match} \NormalTok{guess.trim().parse() \{}
            \ConstantTok{Ok}\NormalTok{(num) => num,}
            \ConstantTok{Err}\NormalTok{(_) => }\KeywordTok{continue}\NormalTok{,}
        \NormalTok{\};}

        \PreprocessorTok{println!}\NormalTok{(}\StringTok{"You guessed: \{\}"}\NormalTok{, guess);}

        \KeywordTok{match} \NormalTok{guess.cmp(&secret_number) \{}
            \NormalTok{Ordering::Less    => }\PreprocessorTok{println!}\NormalTok{(}\StringTok{"Too small!"}\NormalTok{),}
            \NormalTok{Ordering::Greater => }\PreprocessorTok{println!}\NormalTok{(}\StringTok{"Too big!"}\NormalTok{),}
            \NormalTok{Ordering::Equal   => \{}
                \PreprocessorTok{println!}\NormalTok{(}\StringTok{"You win!"}\NormalTok{);}
                \KeywordTok{break}\NormalTok{;}
            \NormalTok{\}}
        \NormalTok{\}}
    \NormalTok{\}}
\NormalTok{\}}
\end{Highlighting}
\end{Shaded}

These are the lines that changed:

\begin{Shaded}
\begin{Highlighting}[]
\KeywordTok{let} \NormalTok{guess: }\DataTypeTok{u32} \NormalTok{= }\KeywordTok{match} \NormalTok{guess.trim().parse() \{}
    \ConstantTok{Ok}\NormalTok{(num) => num,}
    \ConstantTok{Err}\NormalTok{(_) => }\KeywordTok{continue}\NormalTok{,}
\NormalTok{\};}
\end{Highlighting}
\end{Shaded}

This is how you generally move from `crash on error' to `actually handle
the error', by switching from \texttt{expect()} to a \texttt{match}
statement. A \texttt{Result} is returned by \texttt{parse()}, this is an
\texttt{enum} like \texttt{Ordering}, but in this case, each variant has
some data associated with it: \texttt{Ok} is a success, and \texttt{Err}
is a failure. Each contains more information: the successfully parsed
integer, or an error type. In this case, we \texttt{match} on
\texttt{Ok(num)}, which sets the name \texttt{num} to the unwrapped
\texttt{Ok} value (the integer), and then we return it on the right-hand
side. In the \texttt{Err} case, we don't care what kind of error it is,
so we just use the catch all \texttt{\_} instead of a name. This catches
everything that isn't \texttt{Ok}, and \texttt{continue} lets us move to
the next iteration of the loop; in effect, this enables us to ignore all
errors and continue with our program.

Now we should be good! Let's try:

\begin{Shaded}
\begin{Highlighting}[]
\NormalTok{$ }\KeywordTok{cargo} \NormalTok{run}
   \KeywordTok{Compiling} \NormalTok{guessing_game v0.1.0 (file:///home/you/projects/guessing_game)}
     \KeywordTok{Running} \KeywordTok{`target/guessing_game`}
\KeywordTok{Guess} \NormalTok{the number!}
\KeywordTok{The} \NormalTok{secret number is: 61}
\KeywordTok{Please} \NormalTok{input your guess.}
\KeywordTok{10}
\KeywordTok{You} \NormalTok{guessed: 10}
\KeywordTok{Too} \NormalTok{small!}
\KeywordTok{Please} \NormalTok{input your guess.}
\KeywordTok{99}
\KeywordTok{You} \NormalTok{guessed: 99}
\KeywordTok{Too} \NormalTok{big!}
\KeywordTok{Please} \NormalTok{input your guess.}
\KeywordTok{foo}
\KeywordTok{Please} \NormalTok{input your guess.}
\KeywordTok{61}
\KeywordTok{You} \NormalTok{guessed: 61}
\KeywordTok{You} \NormalTok{win!}
\end{Highlighting}
\end{Shaded}

Awesome! With one tiny last tweak, we have finished the guessing game.
Can you think of what it is? That's right, we don't want to print out
the secret number. It was good for testing, but it kind of ruins the
game. Here's our final source:

\begin{Shaded}
\begin{Highlighting}[]
\KeywordTok{extern} \KeywordTok{crate} \NormalTok{rand;}

\KeywordTok{use} \NormalTok{std::io;}
\KeywordTok{use} \NormalTok{std::cmp::Ordering;}
\KeywordTok{use} \NormalTok{rand::Rng;}

\KeywordTok{fn} \NormalTok{main() \{}
    \PreprocessorTok{println!}\NormalTok{(}\StringTok{"Guess the number!"}\NormalTok{);}

    \KeywordTok{let} \NormalTok{secret_number = rand::thread_rng().gen_range(}\DecValTok{1}\NormalTok{, }\DecValTok{101}\NormalTok{);}

    \KeywordTok{loop} \NormalTok{\{}
        \PreprocessorTok{println!}\NormalTok{(}\StringTok{"Please input your guess."}\NormalTok{);}

        \KeywordTok{let} \KeywordTok{mut} \NormalTok{guess = }\DataTypeTok{String}\NormalTok{::new();}

        \NormalTok{io::stdin().read_line(&}\KeywordTok{mut} \NormalTok{guess)}
            \NormalTok{.expect(}\StringTok{"failed to read line"}\NormalTok{);}

        \KeywordTok{let} \NormalTok{guess: }\DataTypeTok{u32} \NormalTok{= }\KeywordTok{match} \NormalTok{guess.trim().parse() \{}
            \ConstantTok{Ok}\NormalTok{(num) => num,}
            \ConstantTok{Err}\NormalTok{(_) => }\KeywordTok{continue}\NormalTok{,}
        \NormalTok{\};}

        \PreprocessorTok{println!}\NormalTok{(}\StringTok{"You guessed: \{\}"}\NormalTok{, guess);}

        \KeywordTok{match} \NormalTok{guess.cmp(&secret_number) \{}
            \NormalTok{Ordering::Less    => }\PreprocessorTok{println!}\NormalTok{(}\StringTok{"Too small!"}\NormalTok{),}
            \NormalTok{Ordering::Greater => }\PreprocessorTok{println!}\NormalTok{(}\StringTok{"Too big!"}\NormalTok{),}
            \NormalTok{Ordering::Equal   => \{}
                \PreprocessorTok{println!}\NormalTok{(}\StringTok{"You win!"}\NormalTok{);}
                \KeywordTok{break}\NormalTok{;}
            \NormalTok{\}}
        \NormalTok{\}}
    \NormalTok{\}}
\NormalTok{\}}
\end{Highlighting}
\end{Shaded}

\subsection{Complete!}\label{complete}

This project showed you a lot: \texttt{let}, \texttt{match}, methods,
associated functions, using external crates, and more.

At this point, you have successfully built the Guessing Game!
Congratulations!

\hypertarget{sec--syntax-and-semantics}{\chapter{Syntax and
Semantics}\label{sec--syntax-and-semantics}}

This chapter breaks Rust down into small chunks, one for each concept.

If you'd like to learn Rust from the bottom up, reading this in order is
a great way to do that.

These sections also form a reference for each concept, so if you're
reading another tutorial and find something confusing, you can find it
explained somewhere in here.

\hypertarget{sec--variable-bindings}{\section{Variable
Bindings}\label{sec--variable-bindings}}

Virtually every non-`Hello World' Rust program uses \emph{variable
bindings}. They bind some value to a name, so it can be used later.
\texttt{let} is used to introduce a binding, like this:

\begin{Shaded}
\begin{Highlighting}[]
\KeywordTok{fn} \NormalTok{main() \{}
    \KeywordTok{let} \NormalTok{x = }\DecValTok{5}\NormalTok{;}
\NormalTok{\}}
\end{Highlighting}
\end{Shaded}

Putting \texttt{fn\ main()\ \{} in each example is a bit tedious, so
we'll leave that out in the future. If you're following along, make sure
to edit your \texttt{main()} function, rather than leaving it off.
Otherwise, you'll get an error.

\subsection{Patterns}\label{patterns}

In many languages, a variable binding would be called a \emph{variable},
but Rust's variable bindings have a few tricks up their sleeves. For
example the left-hand side of a \texttt{let} statement is a
`\protect\hyperlink{sec--patterns}{pattern}', not a variable name. This
means we can do things like:

\begin{Shaded}
\begin{Highlighting}[]
\KeywordTok{let} \NormalTok{(x, y) = (}\DecValTok{1}\NormalTok{, }\DecValTok{2}\NormalTok{);}
\end{Highlighting}
\end{Shaded}

After this statement is evaluated, \texttt{x} will be one, and
\texttt{y} will be two. Patterns are really powerful, and have
\protect\hyperlink{sec--patterns}{their own section} in the book. We
don't need those features for now, so we'll keep this in the back of our
minds as we go forward.

\subsection{Type annotations}\label{type-annotations}

Rust is a statically typed language, which means that we specify our
types up front, and they're checked at compile time. So why does our
first example compile? Well, Rust has this thing called `type
inference'. If it can figure out what the type of something is, Rust
doesn't require you to actually type it out.

We can add the type if we want to, though. Types come after a colon
(\texttt{:}):

\begin{Shaded}
\begin{Highlighting}[]
\KeywordTok{let} \NormalTok{x: }\DataTypeTok{i32} \NormalTok{= }\DecValTok{5}\NormalTok{;}
\end{Highlighting}
\end{Shaded}

If I asked you to read this out loud to the rest of the class, you'd say
``\texttt{x} is a binding with the type \texttt{i32} and the value
\texttt{five}.''

In this case we chose to represent \texttt{x} as a 32-bit signed
integer. Rust has many different primitive integer types. They begin
with \texttt{i} for signed integers and \texttt{u} for unsigned
integers. The possible integer sizes are 8, 16, 32, and 64 bits.

In future examples, we may annotate the type in a comment. The examples
will look like this:

\begin{Shaded}
\begin{Highlighting}[]
\KeywordTok{fn} \NormalTok{main() \{}
    \KeywordTok{let} \NormalTok{x = }\DecValTok{5}\NormalTok{; }\CommentTok{// x: i32}
\NormalTok{\}}
\end{Highlighting}
\end{Shaded}

Note the similarities between this annotation and the syntax you use
with \texttt{let}. Including these kinds of comments is not idiomatic
Rust, but we'll occasionally include them to help you understand what
the types that Rust infers are.

\subsection{Mutability}\label{mutability}

By default, bindings are \emph{immutable}. This code will not compile:

\begin{Shaded}
\begin{Highlighting}[]
\KeywordTok{let} \NormalTok{x = }\DecValTok{5}\NormalTok{;}
\NormalTok{x = }\DecValTok{10}\NormalTok{;}
\end{Highlighting}
\end{Shaded}

It will give you this error:

\begin{verbatim}
error: re-assignment of immutable variable `x`
     x = 10;
     ^~~~~~~
\end{verbatim}

If you want a binding to be mutable, you can use \texttt{mut}:

\begin{Shaded}
\begin{Highlighting}[]
\KeywordTok{let} \KeywordTok{mut} \NormalTok{x = }\DecValTok{5}\NormalTok{; }\CommentTok{// mut x: i32}
\NormalTok{x = }\DecValTok{10}\NormalTok{;}
\end{Highlighting}
\end{Shaded}

There is no single reason that bindings are immutable by default, but we
can think about it through one of Rust's primary focuses: safety. If you
forget to say \texttt{mut}, the compiler will catch it, and let you know
that you have mutated something you may not have intended to mutate. If
bindings were mutable by default, the compiler would not be able to tell
you this. If you \emph{did} intend mutation, then the solution is quite
easy: add \texttt{mut}.

There are other good reasons to avoid mutable state when possible, but
they're out of the scope of this guide. In general, you can often avoid
explicit mutation, and so it is preferable in Rust. That said,
sometimes, mutation is what you need, so it's not verboten.

\subsection{Initializing bindings}\label{initializing-bindings}

Rust variable bindings have one more aspect that differs from other
languages: bindings are required to be initialized with a value before
you're allowed to use them.

Let's try it out. Change your \texttt{src/main.rs} file to look like
this:

\begin{Shaded}
\begin{Highlighting}[]
\KeywordTok{fn} \NormalTok{main() \{}
    \KeywordTok{let} \NormalTok{x: }\DataTypeTok{i32}\NormalTok{;}

    \PreprocessorTok{println!}\NormalTok{(}\StringTok{"Hello world!"}\NormalTok{);}
\NormalTok{\}}
\end{Highlighting}
\end{Shaded}

You can use \texttt{cargo\ build} on the command line to build it.
You'll get a warning, but it will still print ``Hello, world!'':

\begin{verbatim}
   Compiling hello_world v0.0.1 (file:///home/you/projects/hello_world)
src/main.rs:2:9: 2:10 warning: unused variable: `x`, #[warn(unused_variable)]
   on by default
src/main.rs:2     let x: i32;
                      ^
\end{verbatim}

Rust warns us that we never use the variable binding, but since we never
use it, no harm, no foul. Things change if we try to actually use this
\texttt{x}, however. Let's do that. Change your program to look like
this:

\begin{Shaded}
\begin{Highlighting}[]
\KeywordTok{fn} \NormalTok{main() \{}
    \KeywordTok{let} \NormalTok{x: }\DataTypeTok{i32}\NormalTok{;}

    \PreprocessorTok{println!}\NormalTok{(}\StringTok{"The value of x is: \{\}"}\NormalTok{, x);}
\NormalTok{\}}
\end{Highlighting}
\end{Shaded}

And try to build it. You'll get an error:

\begin{Shaded}
\begin{Highlighting}[]
\NormalTok{$ }\KeywordTok{cargo} \NormalTok{build}
   \KeywordTok{Compiling} \NormalTok{hello_world v0.0.1 (file:///home/you/projects/hello_world)}
\KeywordTok{src}\NormalTok{/main.rs:}\KeywordTok{4}\NormalTok{:39: 4:40 error: use of possibly uninitialized variable: }\KeywordTok{`x`}
\KeywordTok{src}\NormalTok{/main.rs:}\KeywordTok{4}     \NormalTok{println!(}\StringTok{"The value of x is: \{\}"}\NormalTok{, x);}
                                                    \NormalTok{^}
\KeywordTok{note}\NormalTok{: in expansion of format_args!}
\KeywordTok{<std} \NormalTok{macros}\KeywordTok{>}\NormalTok{:2:23: 2:77 note: expansion site}
\KeywordTok{<std} \NormalTok{macros}\KeywordTok{>}\NormalTok{:1:1: 3:2 note: in expansion of println!}
\KeywordTok{src}\NormalTok{/main.rs:}\KeywordTok{4}\NormalTok{:5: 4:42 note: expansion site}
\KeywordTok{error}\NormalTok{: aborting due to previous error}
\KeywordTok{Could} \NormalTok{not compile }\KeywordTok{`hello_world`}\NormalTok{.}
\end{Highlighting}
\end{Shaded}

Rust will not let us use a value that has not been initialized. Next,
let's talk about this stuff we've added to \texttt{println!}.

If you include two curly braces (\texttt{\{\}}, some call them
moustaches\ldots{}) in your string to print, Rust will interpret this as
a request to interpolate some sort of value. \emph{String interpolation}
is a computer science term that means ``stick in the middle of a
string.'' We add a comma, and then \texttt{x}, to indicate that we want
\texttt{x} to be the value we're interpolating. The comma is used to
separate arguments we pass to functions and macros, if you're passing
more than one.

When you use the curly braces, Rust will attempt to display the value in
a meaningful way by checking out its type. If you want to specify the
format in a more detailed manner, there are a
\href{http://doc.rust-lang.org/std/fmt/index.html}{wide number of
options available}. For now, we'll stick to the default: integers aren't
very complicated to print.

\subsection{Scope and shadowing}\label{scope-and-shadowing}

Let's get back to bindings. Variable bindings have a scope - they are
constrained to live in a block they were defined in. A block is a
collection of statements enclosed by \texttt{\{} and \texttt{\}}.
Function definitions are also blocks! In the following example we define
two variable bindings, \texttt{x} and \texttt{y}, which live in
different blocks. \texttt{x} can be accessed from inside the
\texttt{fn\ main()\ \{\}} block, while \texttt{y} can be accessed only
from inside the inner block:

\begin{Shaded}
\begin{Highlighting}[]
\KeywordTok{fn} \NormalTok{main() \{}
    \KeywordTok{let} \NormalTok{x: }\DataTypeTok{i32} \NormalTok{= }\DecValTok{17}\NormalTok{;}
    \NormalTok{\{}
        \KeywordTok{let} \NormalTok{y: }\DataTypeTok{i32} \NormalTok{= }\DecValTok{3}\NormalTok{;}
        \PreprocessorTok{println!}\NormalTok{(}\StringTok{"The value of x is \{\} and value of y is \{\}"}\NormalTok{, x, y);}
    \NormalTok{\}}
    \PreprocessorTok{println!}\NormalTok{(}\StringTok{"The value of x is \{\} and value of y is \{\}"}\NormalTok{, x, y); }\CommentTok{// This won't work}
\NormalTok{\}}
\end{Highlighting}
\end{Shaded}

The first \texttt{println!} would print ``The value of x is 17 and the
value of y is 3'', but this example cannot be compiled successfully,
because the second \texttt{println!} cannot access the value of
\texttt{y}, since it is not in scope anymore. Instead we get this error:

\begin{Shaded}
\begin{Highlighting}[]
\NormalTok{$ }\KeywordTok{cargo} \NormalTok{build}
   \KeywordTok{Compiling} \NormalTok{hello v0.1.0 (file:///home/you/projects/hello_world)}
\KeywordTok{main.rs}\NormalTok{:7:62: 7:63 error: unresolved name }\KeywordTok{`y`}\NormalTok{. Did you mean }\KeywordTok{`x`}\NormalTok{? [E0425]}
\KeywordTok{main.rs}\NormalTok{:7     println!(}\StringTok{"The value of x is \{\} and value of y is \{\}"}\NormalTok{, x, y); }\KeywordTok{//} \NormalTok{This won}
\NormalTok{↳ }\StringTok{'t work}
\StringTok{                                                                       ^}
\StringTok{note: in expansion of format_args!}
\StringTok{<std macros>:2:25: 2:56 note: expansion site}
\StringTok{<std macros>:1:1: 2:62 note: in expansion of print!}
\StringTok{<std macros>:3:1: 3:54 note: expansion site}
\StringTok{<std macros>:1:1: 3:58 note: in expansion of println!}
\StringTok{main.rs:7:5: 7:65 note: expansion site}
\StringTok{main.rs:7:62: 7:63 help: run `rustc --explain E0425` to see a detailed explanation}
\StringTok{error: aborting due to previous error}
\StringTok{Could not compile `hello`.}

\StringTok{To learn more, run the command again with --verbose.}
\end{Highlighting}
\end{Shaded}

Additionally, variable bindings can be shadowed. This means that a later
variable binding with the same name as another binding, that's currently
in scope, will override the previous binding.

\begin{Shaded}
\begin{Highlighting}[]
\KeywordTok{let} \NormalTok{x: }\DataTypeTok{i32} \NormalTok{= }\DecValTok{8}\NormalTok{;}
\NormalTok{\{}
    \PreprocessorTok{println!}\NormalTok{(}\StringTok{"\{\}"}\NormalTok{, x); }\CommentTok{// Prints "8"}
    \KeywordTok{let} \NormalTok{x = }\DecValTok{12}\NormalTok{;}
    \PreprocessorTok{println!}\NormalTok{(}\StringTok{"\{\}"}\NormalTok{, x); }\CommentTok{// Prints "12"}
\NormalTok{\}}
\PreprocessorTok{println!}\NormalTok{(}\StringTok{"\{\}"}\NormalTok{, x); }\CommentTok{// Prints "8"}
\KeywordTok{let} \NormalTok{x =  }\DecValTok{42}\NormalTok{;}
\PreprocessorTok{println!}\NormalTok{(}\StringTok{"\{\}"}\NormalTok{, x); }\CommentTok{// Prints "42"}
\end{Highlighting}
\end{Shaded}

Shadowing and mutable bindings may appear as two sides of the same coin,
but they are two distinct concepts that can't always be used
interchangeably. For one, shadowing enables us to rebind a name to a
value of a different type. It is also possible to change the mutability
of a binding.

\begin{Shaded}
\begin{Highlighting}[]
\KeywordTok{let} \KeywordTok{mut} \NormalTok{x: }\DataTypeTok{i32} \NormalTok{= }\DecValTok{1}\NormalTok{;}
\NormalTok{x = }\DecValTok{7}\NormalTok{;}
\KeywordTok{let} \NormalTok{x = x; }\CommentTok{// x is now immutable and is bound to 7}

\KeywordTok{let} \NormalTok{y = }\DecValTok{4}\NormalTok{;}
\KeywordTok{let} \NormalTok{y = }\StringTok{"I can also be bound to text!"}\NormalTok{; }\CommentTok{// y is now of a different type}
\end{Highlighting}
\end{Shaded}

\hypertarget{sec--functions}{\section{Functions}\label{sec--functions}}

Every Rust program has at least one function, the \texttt{main}
function:

\begin{Shaded}
\begin{Highlighting}[]
\KeywordTok{fn} \NormalTok{main() \{}
\NormalTok{\}}
\end{Highlighting}
\end{Shaded}

This is the simplest possible function declaration. As we mentioned
before, \texttt{fn} says `this is a function', followed by the name,
some parentheses because this function takes no arguments, and then some
curly braces to indicate the body. Here's a function named \texttt{foo}:

\begin{Shaded}
\begin{Highlighting}[]
\KeywordTok{fn} \NormalTok{foo() \{}
\NormalTok{\}}
\end{Highlighting}
\end{Shaded}

So, what about taking arguments? Here's a function that prints a number:

\begin{Shaded}
\begin{Highlighting}[]
\KeywordTok{fn} \NormalTok{print_number(x: }\DataTypeTok{i32}\NormalTok{) \{}
    \PreprocessorTok{println!}\NormalTok{(}\StringTok{"x is: \{\}"}\NormalTok{, x);}
\NormalTok{\}}
\end{Highlighting}
\end{Shaded}

Here's a complete program that uses \texttt{print\_number}:

\begin{Shaded}
\begin{Highlighting}[]
\KeywordTok{fn} \NormalTok{main() \{}
    \NormalTok{print_number(}\DecValTok{5}\NormalTok{);}
\NormalTok{\}}

\KeywordTok{fn} \NormalTok{print_number(x: }\DataTypeTok{i32}\NormalTok{) \{}
    \PreprocessorTok{println!}\NormalTok{(}\StringTok{"x is: \{\}"}\NormalTok{, x);}
\NormalTok{\}}
\end{Highlighting}
\end{Shaded}

As you can see, function arguments work very similar to \texttt{let}
declarations: you add a type to the argument name, after a colon.

Here's a complete program that adds two numbers together and prints
them:

\begin{Shaded}
\begin{Highlighting}[]
\KeywordTok{fn} \NormalTok{main() \{}
    \NormalTok{print_sum(}\DecValTok{5}\NormalTok{, }\DecValTok{6}\NormalTok{);}
\NormalTok{\}}

\KeywordTok{fn} \NormalTok{print_sum(x: }\DataTypeTok{i32}\NormalTok{, y: }\DataTypeTok{i32}\NormalTok{) \{}
    \PreprocessorTok{println!}\NormalTok{(}\StringTok{"sum is: \{\}"}\NormalTok{, x + y);}
\NormalTok{\}}
\end{Highlighting}
\end{Shaded}

You separate arguments with a comma, both when you call the function, as
well as when you declare it.

Unlike \texttt{let}, you \emph{must} declare the types of function
arguments. This does not work:

\begin{Shaded}
\begin{Highlighting}[]
\KeywordTok{fn} \NormalTok{print_sum(x, y) \{}
    \PreprocessorTok{println!}\NormalTok{(}\StringTok{"sum is: \{\}"}\NormalTok{, x + y);}
\NormalTok{\}}
\end{Highlighting}
\end{Shaded}

You get this error:

\begin{verbatim}
expected one of `!`, `:`, or `@`, found `)`
fn print_sum(x, y) {
\end{verbatim}

This is a deliberate design decision. While full-program inference is
possible, languages which have it, like Haskell, often suggest that
documenting your types explicitly is a best-practice. We agree that
forcing functions to declare types while allowing for inference inside
of function bodies is a wonderful sweet spot between full inference and
no inference.

What about returning a value? Here's a function that adds one to an
integer:

\begin{Shaded}
\begin{Highlighting}[]
\KeywordTok{fn} \NormalTok{add_one(x: }\DataTypeTok{i32}\NormalTok{) -> }\DataTypeTok{i32} \NormalTok{\{}
    \NormalTok{x + }\DecValTok{1}
\NormalTok{\}}
\end{Highlighting}
\end{Shaded}

Rust functions return exactly one value, and you declare the type after
an `arrow', which is a dash (\texttt{-}) followed by a greater-than sign
(\texttt{\textgreater{}}). The last line of a function determines what
it returns. You'll note the lack of a semicolon here. If we added it in:

\begin{Shaded}
\begin{Highlighting}[]
\KeywordTok{fn} \NormalTok{add_one(x: }\DataTypeTok{i32}\NormalTok{) -> }\DataTypeTok{i32} \NormalTok{\{}
    \NormalTok{x + }\DecValTok{1}\NormalTok{;}
\NormalTok{\}}
\end{Highlighting}
\end{Shaded}

We would get an error:

\begin{verbatim}
error: not all control paths return a value
fn add_one(x: i32) -> i32 {
     x + 1;
}

help: consider removing this semicolon:
     x + 1;
          ^
\end{verbatim}

This reveals two interesting things about Rust: it is an
expression-based language, and semicolons are different from semicolons
in other `curly brace and semicolon'-based languages. These two things
are related.

\subsubsection{Expressions
vs.~Statements}\label{expressions-vs.statements}

Rust is primarily an expression-based language. There are only two kinds
of statements, and everything else is an expression.

So what's the difference? Expressions return a value, and statements do
not. That's why we end up with `not all control paths return a value'
here: the statement \texttt{x\ +\ 1;} doesn't return a value. There are
two kinds of statements in Rust: `declaration statements' and
`expression statements'. Everything else is an expression. Let's talk
about declaration statements first.

In some languages, variable bindings can be written as expressions, not
statements. Like Ruby:

\begin{Shaded}
\begin{Highlighting}[]
\NormalTok{x = y = }\DecValTok{5}
\end{Highlighting}
\end{Shaded}

In Rust, however, using \texttt{let} to introduce a binding is
\emph{not} an expression. The following will produce a compile-time
error:

\begin{Shaded}
\begin{Highlighting}[]
\KeywordTok{let} \NormalTok{x = (}\KeywordTok{let} \NormalTok{y = }\DecValTok{5}\NormalTok{); }\CommentTok{// expected identifier, found keyword `let`}
\end{Highlighting}
\end{Shaded}

The compiler is telling us here that it was expecting to see the
beginning of an expression, and a \texttt{let} can only begin a
statement, not an expression.

Note that assigning to an already-bound variable (e.g. \texttt{y\ =\ 5})
is still an expression, although its value is not particularly useful.
Unlike other languages where an assignment evaluates to the assigned
value (e.g. \texttt{5} in the previous example), in Rust the value of an
assignment is an empty tuple \texttt{()} because the assigned value can
have \protect\hyperlink{sec--ownership}{only one owner}, and any other
returned value would be too surprising:

\begin{Shaded}
\begin{Highlighting}[]
\KeywordTok{let} \KeywordTok{mut} \NormalTok{y = }\DecValTok{5}\NormalTok{;}

\KeywordTok{let} \NormalTok{x = (y = }\DecValTok{6}\NormalTok{);  }\CommentTok{// x has the value `()`, not `6`}
\end{Highlighting}
\end{Shaded}

The second kind of statement in Rust is the \emph{expression statement}.
Its purpose is to turn any expression into a statement. In practical
terms, Rust's grammar expects statements to follow other statements.
This means that you use semicolons to separate expressions from each
other. This means that Rust looks a lot like most other languages that
require you to use semicolons at the end of every line, and you will see
semicolons at the end of almost every line of Rust code you see.

What is this exception that makes us say ``almost''? You saw it already,
in this code:

\begin{Shaded}
\begin{Highlighting}[]
\KeywordTok{fn} \NormalTok{add_one(x: }\DataTypeTok{i32}\NormalTok{) -> }\DataTypeTok{i32} \NormalTok{\{}
    \NormalTok{x + }\DecValTok{1}
\NormalTok{\}}
\end{Highlighting}
\end{Shaded}

Our function claims to return an \texttt{i32}, but with a semicolon, it
would return \texttt{()} instead. Rust realizes this probably isn't what
we want, and suggests removing the semicolon in the error we saw before.

\hypertarget{early-returns}{\subsubsection{Early
returns}\label{early-returns}}

But what about early returns? Rust does have a keyword for that,
\texttt{return}:

\begin{Shaded}
\begin{Highlighting}[]
\KeywordTok{fn} \NormalTok{foo(x: }\DataTypeTok{i32}\NormalTok{) -> }\DataTypeTok{i32} \NormalTok{\{}
    \KeywordTok{return} \NormalTok{x;}

    \CommentTok{// we never run this code!}
    \NormalTok{x + }\DecValTok{1}
\NormalTok{\}}
\end{Highlighting}
\end{Shaded}

Using a \texttt{return} as the last line of a function works, but is
considered poor style:

\begin{Shaded}
\begin{Highlighting}[]
\KeywordTok{fn} \NormalTok{foo(x: }\DataTypeTok{i32}\NormalTok{) -> }\DataTypeTok{i32} \NormalTok{\{}
    \KeywordTok{return} \NormalTok{x + }\DecValTok{1}\NormalTok{;}
\NormalTok{\}}
\end{Highlighting}
\end{Shaded}

The previous definition without \texttt{return} may look a bit strange
if you haven't worked in an expression-based language before, but it
becomes intuitive over time.

\hypertarget{diverging-functions}{\subsubsection{Diverging
functions}\label{diverging-functions}}

Rust has some special syntax for `diverging functions', which are
functions that do not return:

\begin{Shaded}
\begin{Highlighting}[]
\KeywordTok{fn} \NormalTok{diverges() -> ! \{}
    \PreprocessorTok{panic!}\NormalTok{(}\StringTok{"This function never returns!"}\NormalTok{);}
\NormalTok{\}}
\end{Highlighting}
\end{Shaded}

\texttt{panic!} is a macro, similar to \texttt{println!()} that we've
already seen. Unlike \texttt{println!()}, \texttt{panic!()} causes the
current thread of execution to crash with the given message. Because
this function will cause a crash, it will never return, and so it has
the type `\texttt{!}', which is read `diverges'.

If you add a main function that calls \texttt{diverges()} and run it,
you'll get some output that looks like this:

\begin{verbatim}
thread ‘<main>’ panicked at ‘This function never returns!’, hello.rs:2
\end{verbatim}

If you want more information, you can get a backtrace by setting the
\texttt{RUST\_BACKTRACE} environment variable:

\begin{verbatim}
$ RUST_BACKTRACE=1 ./diverges
thread '<main>' panicked at 'This function never returns!', hello.rs:2
stack backtrace:
   1:     0x7f402773a829 - sys::backtrace::write::h0942de78b6c02817K8r
   2:     0x7f402773d7fc - panicking::on_panic::h3f23f9d0b5f4c91bu9w
   3:     0x7f402773960e - rt::unwind::begin_unwind_inner::h2844b8c5e81e79558Bw
   4:     0x7f4027738893 - rt::unwind::begin_unwind::h4375279447423903650
   5:     0x7f4027738809 - diverges::h2266b4c4b850236beaa
   6:     0x7f40277389e5 - main::h19bb1149c2f00ecfBaa
   7:     0x7f402773f514 - rt::unwind::try::try_fn::h13186883479104382231
   8:     0x7f402773d1d8 - __rust_try
   9:     0x7f402773f201 - rt::lang_start::ha172a3ce74bb453aK5w
  10:     0x7f4027738a19 - main
  11:     0x7f402694ab44 - __libc_start_main
  12:     0x7f40277386c8 - <unknown>
  13:                0x0 - <unknown>
\end{verbatim}

If you need to override an already set \texttt{RUST\_BACKTRACE}, in
cases when you cannot just unset the variable, then set it to \texttt{0}
to avoid getting a backtrace. Any other value(even no value at all)
turns on backtrace.

\begin{verbatim}
$ export RUST_BACKTRACE=1
...
$ RUST_BACKTRACE=0 ./diverges 
thread '<main>' panicked at 'This function never returns!', hello.rs:2
note: Run with `RUST_BACKTRACE=1` for a backtrace.
\end{verbatim}

\texttt{RUST\_BACKTRACE} also works with Cargo's \texttt{run} command:

\begin{verbatim}
$ RUST_BACKTRACE=1 cargo run
     Running `target/debug/diverges`
thread '<main>' panicked at 'This function never returns!', hello.rs:2
stack backtrace:
   1:     0x7f402773a829 - sys::backtrace::write::h0942de78b6c02817K8r
   2:     0x7f402773d7fc - panicking::on_panic::h3f23f9d0b5f4c91bu9w
   3:     0x7f402773960e - rt::unwind::begin_unwind_inner::h2844b8c5e81e79558Bw
   4:     0x7f4027738893 - rt::unwind::begin_unwind::h4375279447423903650
   5:     0x7f4027738809 - diverges::h2266b4c4b850236beaa
   6:     0x7f40277389e5 - main::h19bb1149c2f00ecfBaa
   7:     0x7f402773f514 - rt::unwind::try::try_fn::h13186883479104382231
   8:     0x7f402773d1d8 - __rust_try
   9:     0x7f402773f201 - rt::lang_start::ha172a3ce74bb453aK5w
  10:     0x7f4027738a19 - main
  11:     0x7f402694ab44 - __libc_start_main
  12:     0x7f40277386c8 - <unknown>
  13:                0x0 - <unknown>
\end{verbatim}

A diverging function can be used as any type:

\begin{Shaded}
\begin{Highlighting}[]
\KeywordTok{let} \NormalTok{x: }\DataTypeTok{i32} \NormalTok{= diverges();}
\KeywordTok{let} \NormalTok{x: }\DataTypeTok{String} \NormalTok{= diverges();}
\end{Highlighting}
\end{Shaded}

\subsubsection{Function pointers}\label{function-pointers}

We can also create variable bindings which point to functions:

\begin{Shaded}
\begin{Highlighting}[]
\KeywordTok{let} \NormalTok{f: }\KeywordTok{fn}\NormalTok{(}\DataTypeTok{i32}\NormalTok{) -> }\DataTypeTok{i32}\NormalTok{;}
\end{Highlighting}
\end{Shaded}

\texttt{f} is a variable binding which points to a function that takes
an \texttt{i32} as an argument and returns an \texttt{i32}. For example:

\begin{Shaded}
\begin{Highlighting}[]
\KeywordTok{fn} \NormalTok{plus_one(i: }\DataTypeTok{i32}\NormalTok{) -> }\DataTypeTok{i32} \NormalTok{\{}
    \NormalTok{i + }\DecValTok{1}
\NormalTok{\}}

\CommentTok{// without type inference}
\KeywordTok{let} \NormalTok{f: }\KeywordTok{fn}\NormalTok{(}\DataTypeTok{i32}\NormalTok{) -> }\DataTypeTok{i32} \NormalTok{= plus_one;}

\CommentTok{// with type inference}
\KeywordTok{let} \NormalTok{f = plus_one;}
\end{Highlighting}
\end{Shaded}

We can then use \texttt{f} to call the function:

\begin{Shaded}
\begin{Highlighting}[]
\KeywordTok{let} \NormalTok{six = f(}\DecValTok{5}\NormalTok{);}
\end{Highlighting}
\end{Shaded}

\section{Primitive Types}\label{sec--primitive-types}

The Rust language has a number of types that are considered `primitive'.
This means that they're built-in to the language. Rust is structured in
such a way that the standard library also provides a number of useful
types built on top of these ones, as well, but these are the most
primitive.

\subsection{Booleans}\label{booleans}

Rust has a built-in boolean type, named \texttt{bool}. It has two
values, \texttt{true} and \texttt{false}:

\begin{Shaded}
\begin{Highlighting}[]
\KeywordTok{let} \NormalTok{x = }\ConstantTok{true}\NormalTok{;}

\KeywordTok{let} \NormalTok{y: }\DataTypeTok{bool} \NormalTok{= }\ConstantTok{false}\NormalTok{;}
\end{Highlighting}
\end{Shaded}

A common use of booleans is in \protect\hyperlink{sec--if}{\texttt{if}
conditionals}.

You can find more documentation for \texttt{bool}s
\href{http://doc.rust-lang.org/std/primitive.bool.html}{in the standard
library documentation}.

\subsection{\texorpdfstring{\texttt{char}}{char}}\label{char}

The \texttt{char} type represents a single Unicode scalar value. You can
create \texttt{char}s with a single tick: (\texttt{\textquotesingle{}})

\begin{Shaded}
\begin{Highlighting}[]
\KeywordTok{let} \NormalTok{x = }\CharTok{'x'}\NormalTok{;}
\KeywordTok{let} \NormalTok{two_hearts = }\CharTok{'💕'}\NormalTok{;}
\end{Highlighting}
\end{Shaded}

Unlike some other languages, this means that Rust's \texttt{char} is not
a single byte, but four.

You can find more documentation for \texttt{char}s
\href{http://doc.rust-lang.org/std/primitive.char.html}{in the standard
library documentation}.

\hypertarget{numeric-types}{\subsection{Numeric
types}\label{numeric-types}}

Rust has a variety of numeric types in a few categories: signed and
unsigned, fixed and variable, floating-point and integer.

These types consist of two parts: the category, and the size. For
example, \texttt{u16} is an unsigned type with sixteen bits of size.
More bits lets you have bigger numbers.

If a number literal has nothing to cause its type to be inferred, it
defaults:

\begin{Shaded}
\begin{Highlighting}[]
\KeywordTok{let} \NormalTok{x = }\DecValTok{42}\NormalTok{; }\CommentTok{// x has type i32}

\KeywordTok{let} \NormalTok{y = }\DecValTok{1.0}\NormalTok{; }\CommentTok{// y has type f64}
\end{Highlighting}
\end{Shaded}

Here's a list of the different numeric types, with links to their
documentation in the standard library:

\begin{itemize}
\tightlist
\item
  \href{http://doc.rust-lang.org/std/primitive.i8.html}{i8}
\item
  \href{http://doc.rust-lang.org/std/primitive.i16.html}{i16}
\item
  \href{http://doc.rust-lang.org/std/primitive.i32.html}{i32}
\item
  \href{http://doc.rust-lang.org/std/primitive.i64.html}{i64}
\item
  \href{http://doc.rust-lang.org/std/primitive.u8.html}{u8}
\item
  \href{http://doc.rust-lang.org/std/primitive.u16.html}{u16}
\item
  \href{http://doc.rust-lang.org/std/primitive.u32.html}{u32}
\item
  \href{http://doc.rust-lang.org/std/primitive.u64.html}{u64}
\item
  \href{http://doc.rust-lang.org/std/primitive.isize.html}{isize}
\item
  \href{http://doc.rust-lang.org/std/primitive.usize.html}{usize}
\item
  \href{http://doc.rust-lang.org/std/primitive.f32.html}{f32}
\item
  \href{http://doc.rust-lang.org/std/primitive.f64.html}{f64}
\end{itemize}

Let's go over them by category:

\subsubsection{Signed and Unsigned}\label{signed-and-unsigned}

Integer types come in two varieties: signed and unsigned. To understand
the difference, let's consider a number with four bits of size. A
signed, four-bit number would let you store numbers from \texttt{-8} to
\texttt{+7}. Signed numbers use ``two's complement representation''. An
unsigned four bit number, since it does not need to store negatives, can
store values from \texttt{0} to \texttt{+15}.

Unsigned types use a \texttt{u} for their category, and signed types use
\texttt{i}. The \texttt{i} is for `integer'. So \texttt{u8} is an
eight-bit unsigned number, and \texttt{i8} is an eight-bit signed
number.

\subsubsection{Fixed-size types}\label{fixed-size-types}

Fixed-size types have a specific number of bits in their representation.
Valid bit sizes are \texttt{8}, \texttt{16}, \texttt{32}, and
\texttt{64}. So, \texttt{u32} is an unsigned, 32-bit integer, and
\texttt{i64} is a signed, 64-bit integer.

\subsubsection{Variable-size types}\label{variable-size-types}

Rust also provides types whose particular size depends on the underlying
machine architecture. Their range is sufficient to express the size of
any collection, so these types have `size' as the category. They come in
signed and unsigned varieties which account for two types:
\texttt{isize} and \texttt{usize}.

\subsubsection{Floating-point types}\label{floating-point-types}

Rust also has two floating point types: \texttt{f32} and \texttt{f64}.
These correspond to IEEE-754 single and double precision numbers.

\hypertarget{arrays}{\subsection{Arrays}\label{arrays}}

Like many programming languages, Rust has list types to represent a
sequence of things. The most basic is the \emph{array}, a fixed-size
list of elements of the same type. By default, arrays are immutable.

\begin{Shaded}
\begin{Highlighting}[]
\KeywordTok{let} \NormalTok{a = [}\DecValTok{1}\NormalTok{, }\DecValTok{2}\NormalTok{, }\DecValTok{3}\NormalTok{]; }\CommentTok{// a: [i32; 3]}
\KeywordTok{let} \KeywordTok{mut} \NormalTok{m = [}\DecValTok{1}\NormalTok{, }\DecValTok{2}\NormalTok{, }\DecValTok{3}\NormalTok{]; }\CommentTok{// m: [i32; 3]}
\end{Highlighting}
\end{Shaded}

Arrays have type \texttt{{[}T;\ N{]}}. We'll talk about this \texttt{T}
notation \protect\hyperlink{sec--generics}{in the generics section}. The
\texttt{N} is a compile-time constant, for the length of the array.

There's a shorthand for initializing each element of an array to the
same value. In this example, each element of \texttt{a} will be
initialized to \texttt{0}:

\begin{Shaded}
\begin{Highlighting}[]
\KeywordTok{let} \NormalTok{a = [}\DecValTok{0}\NormalTok{; }\DecValTok{20}\NormalTok{]; }\CommentTok{// a: [i32; 20]}
\end{Highlighting}
\end{Shaded}

You can get the number of elements in an array \texttt{a} with
\texttt{a.len()}:

\begin{Shaded}
\begin{Highlighting}[]
\KeywordTok{let} \NormalTok{a = [}\DecValTok{1}\NormalTok{, }\DecValTok{2}\NormalTok{, }\DecValTok{3}\NormalTok{];}

\PreprocessorTok{println!}\NormalTok{(}\StringTok{"a has \{\} elements"}\NormalTok{, a.len());}
\end{Highlighting}
\end{Shaded}

You can access a particular element of an array with \emph{subscript
notation}:

\begin{Shaded}
\begin{Highlighting}[]
\KeywordTok{let} \NormalTok{names = [}\StringTok{"Graydon"}\NormalTok{, }\StringTok{"Brian"}\NormalTok{, }\StringTok{"Niko"}\NormalTok{]; }\CommentTok{// names: [&str; 3]}

\PreprocessorTok{println!}\NormalTok{(}\StringTok{"The second name is: \{\}"}\NormalTok{, names[}\DecValTok{1}\NormalTok{]);}
\end{Highlighting}
\end{Shaded}

Subscripts start at zero, like in most programming languages, so the
first name is \texttt{names{[}0{]}} and the second name is
\texttt{names{[}1{]}}. The above example prints
\texttt{The\ second\ name\ is:\ Brian}. If you try to use a subscript
that is not in the array, you will get an error: array access is
bounds-checked at run-time. Such errant access is the source of many
bugs in other systems programming languages.

You can find more documentation for \texttt{array}s
\href{http://doc.rust-lang.org/std/primitive.array.html}{in the standard
library documentation}.

\subsection{Slices}\label{slices}

A `slice' is a reference to (or ``view'' into) another data structure.
They are useful for allowing safe, efficient access to a portion of an
array without copying. For example, you might want to reference only one
line of a file read into memory. By nature, a slice is not created
directly, but from an existing variable binding. Slices have a defined
length, can be mutable or immutable.

Internally, slices are represented as a pointer to the beginning of the
data and a length.

\subsubsection{Slicing syntax}\label{slicing-syntax}

You can use a combo of \texttt{\&} and \texttt{{[}{]}} to create a slice
from various things. The \texttt{\&} indicates that slices are similar
to {[}references{]}, which we will cover in detail later in this
section. The \texttt{{[}{]}}s, with a range, let you define the length
of the slice:

\begin{Shaded}
\begin{Highlighting}[]
\KeywordTok{let} \NormalTok{a = [}\DecValTok{0}\NormalTok{, }\DecValTok{1}\NormalTok{, }\DecValTok{2}\NormalTok{, }\DecValTok{3}\NormalTok{, }\DecValTok{4}\NormalTok{];}
\KeywordTok{let} \NormalTok{complete = &a[..]; }\CommentTok{// A slice containing all of the elements in a}
\KeywordTok{let} \NormalTok{middle = &a[}\DecValTok{1.}\NormalTok{.}\DecValTok{4}\NormalTok{]; }\CommentTok{// A slice of a: only the elements 1, 2, and 3}
\end{Highlighting}
\end{Shaded}

Slices have type \texttt{\&{[}T{]}}. We'll talk about that \texttt{T}
when we cover \protect\hyperlink{sec--generics}{generics}.

You can find more documentation for slices
\href{http://doc.rust-lang.org/std/primitive.slice.html}{in the standard
library documentation}.

\subsection{\texorpdfstring{\texttt{str}}{str}}\label{str}

Rust's \texttt{str} type is the most primitive string type. As an
\protect\hyperlink{sec--unsized-types}{unsized type}, it's not very
useful by itself, but becomes useful when placed behind a reference,
like \texttt{\&str}. We'll elaborate further when we cover
\protect\hyperlink{sec--strings}{Strings} and {[}references{]}.

You can find more documentation for \texttt{str}
\href{http://doc.rust-lang.org/std/primitive.str.html}{in the standard
library documentation}.

\hypertarget{tuples}{\subsection{Tuples}\label{tuples}}

A tuple is an ordered list of fixed size. Like this:

\begin{Shaded}
\begin{Highlighting}[]
\KeywordTok{let} \NormalTok{x = (}\DecValTok{1}\NormalTok{, }\StringTok{"hello"}\NormalTok{);}
\end{Highlighting}
\end{Shaded}

The parentheses and commas form this two-length tuple. Here's the same
code, but with the type annotated:

\begin{Shaded}
\begin{Highlighting}[]
\KeywordTok{let} \NormalTok{x: (}\DataTypeTok{i32}\NormalTok{, &}\DataTypeTok{str}\NormalTok{) = (}\DecValTok{1}\NormalTok{, }\StringTok{"hello"}\NormalTok{);}
\end{Highlighting}
\end{Shaded}

As you can see, the type of a tuple looks like the tuple, but with each
position having a type name rather than the value. Careful readers will
also note that tuples are heterogeneous: we have an \texttt{i32} and a
\texttt{\&str} in this tuple. In systems programming languages, strings
are a bit more complex than in other languages. For now, read
\texttt{\&str} as a \emph{string slice}, and we'll learn more soon.

You can assign one tuple into another, if they have the same contained
types and \protect\hyperlink{arity}{arity}. Tuples have the same arity
when they have the same length.

\begin{Shaded}
\begin{Highlighting}[]
\KeywordTok{let} \KeywordTok{mut} \NormalTok{x = (}\DecValTok{1}\NormalTok{, }\DecValTok{2}\NormalTok{); }\CommentTok{// x: (i32, i32)}
\KeywordTok{let} \NormalTok{y = (}\DecValTok{2}\NormalTok{, }\DecValTok{3}\NormalTok{); }\CommentTok{// y: (i32, i32)}

\NormalTok{x = y;}
\end{Highlighting}
\end{Shaded}

You can access the fields in a tuple through a \emph{destructuring let}.
Here's an example:

\begin{Shaded}
\begin{Highlighting}[]
\KeywordTok{let} \NormalTok{(x, y, z) = (}\DecValTok{1}\NormalTok{, }\DecValTok{2}\NormalTok{, }\DecValTok{3}\NormalTok{);}

\PreprocessorTok{println!}\NormalTok{(}\StringTok{"x is \{\}"}\NormalTok{, x);}
\end{Highlighting}
\end{Shaded}

Remember \protect\hyperlink{sec--variable-bindings}{before} when I said
the left-hand side of a \texttt{let} statement was more powerful than
assigning a binding? Here we are. We can put a pattern on the left-hand
side of the \texttt{let}, and if it matches up to the right-hand side,
we can assign multiple bindings at once. In this case, \texttt{let}
``destructures'' or ``breaks up'' the tuple, and assigns the bits to
three bindings.

This pattern is very powerful, and we'll see it repeated more later.

You can disambiguate a single-element tuple from a value in parentheses
with a comma:

\begin{Shaded}
\begin{Highlighting}[]
\NormalTok{(}\DecValTok{0}\NormalTok{,); }\CommentTok{// single-element tuple}
\NormalTok{(}\DecValTok{0}\NormalTok{); }\CommentTok{// zero in parentheses}
\end{Highlighting}
\end{Shaded}

\subsubsection{Tuple Indexing}\label{tuple-indexing}

You can also access fields of a tuple with indexing syntax:

\begin{Shaded}
\begin{Highlighting}[]
\KeywordTok{let} \NormalTok{tuple = (}\DecValTok{1}\NormalTok{, }\DecValTok{2}\NormalTok{, }\DecValTok{3}\NormalTok{);}

\KeywordTok{let} \NormalTok{x = tuple.}\DecValTok{0}\NormalTok{;}
\KeywordTok{let} \NormalTok{y = tuple.}\DecValTok{1}\NormalTok{;}
\KeywordTok{let} \NormalTok{z = tuple.}\DecValTok{2}\NormalTok{;}

\PreprocessorTok{println!}\NormalTok{(}\StringTok{"x is \{\}"}\NormalTok{, x);}
\end{Highlighting}
\end{Shaded}

Like array indexing, it starts at zero, but unlike array indexing, it
uses a \texttt{.}, rather than \texttt{{[}{]}}s.

You can find more documentation for tuples
\href{http://doc.rust-lang.org/std/primitive.tuple.html}{in the standard
library documentation}.

\hypertarget{functions}{\subsection{Functions}\label{functions}}

Functions also have a type! They look like this:

\begin{Shaded}
\begin{Highlighting}[]
\KeywordTok{fn} \NormalTok{foo(x: }\DataTypeTok{i32}\NormalTok{) -> }\DataTypeTok{i32} \NormalTok{\{ x \}}

\KeywordTok{let} \NormalTok{x: }\KeywordTok{fn}\NormalTok{(}\DataTypeTok{i32}\NormalTok{) -> }\DataTypeTok{i32} \NormalTok{= foo;}
\end{Highlighting}
\end{Shaded}

In this case, \texttt{x} is a `function pointer' to a function that
takes an \texttt{i32} and returns an \texttt{i32}.

\hypertarget{sec--comments}{\section{Comments}\label{sec--comments}}

Now that we have some functions, it's a good idea to learn about
comments. Comments are notes that you leave to other programmers to help
explain things about your code. The compiler mostly ignores them.

Rust has two kinds of comments that you should care about: \emph{line
comments} and \emph{doc comments}.

\begin{Shaded}
\begin{Highlighting}[]
\CommentTok{// Line comments are anything after ‘//’ and extend to the end of the line.}

\KeywordTok{let} \NormalTok{x = }\DecValTok{5}\NormalTok{; }\CommentTok{// this is also a line comment.}

\CommentTok{// If you have a long explanation for something, you can put line comments next}
\CommentTok{// to each other. Put a space between the // and your comment so that it’s}
\CommentTok{// more readable.}
\end{Highlighting}
\end{Shaded}

The other kind of comment is a doc comment. Doc comments use
\texttt{///} instead of \texttt{//}, and support Markdown notation
inside:

\begin{Shaded}
\begin{Highlighting}[]
\CommentTok{/// Adds one to the number given.}
\CommentTok{///}
\CommentTok{/// # Examples}
\CommentTok{///}
\CommentTok{/// ```}
\CommentTok{/// let five = 5;}
\CommentTok{///}
\CommentTok{/// assert_eq!(6, add_one(5));}
\CommentTok{/// # fn add_one(x: i32) -> i32 \{}
\CommentTok{/// #     x + 1}
\CommentTok{/// # \}}
\CommentTok{/// ```}
\KeywordTok{fn} \NormalTok{add_one(x: }\DataTypeTok{i32}\NormalTok{) -> }\DataTypeTok{i32} \NormalTok{\{}
    \NormalTok{x + }\DecValTok{1}
\NormalTok{\}}
\end{Highlighting}
\end{Shaded}

There is another style of doc comment, \texttt{//!}, to comment
containing items (e.g.~crates, modules or functions), instead of the
items following it. Commonly used inside crates root (lib.rs) or modules
root (mod.rs):

\begin{verbatim}
//! # The Rust Standard Library
//!
//! The Rust Standard Library provides the essential runtime
//! functionality for building portable Rust software.
\end{verbatim}

When writing doc comments, providing some examples of usage is very,
very helpful. You'll notice we've used a new macro here:
\texttt{assert\_eq!}. This compares two values, and \texttt{panic!}s if
they're not equal to each other. It's very helpful in documentation.
There's another macro, \texttt{assert!}, which \texttt{panic!}s if the
value passed to it is \texttt{false}.

You can use the \protect\hyperlink{sec--documentation}{\texttt{rustdoc}}
tool to generate HTML documentation from these doc comments, and also to
run the code examples as tests!

\hypertarget{sec--if}{\section{if}\label{sec--if}}

Rust's take on \texttt{if} is not particularly complex, but it's much
more like the \texttt{if} you'll find in a dynamically typed language
than in a more traditional systems language. So let's talk about it, to
make sure you grasp the nuances.

\texttt{if} is a specific form of a more general concept, the `branch',
whose name comes from a branch in a tree: a decision point, where
depending on a choice, multiple paths can be taken.

In the case of \texttt{if}, there is one choice that leads down two
paths:

\begin{Shaded}
\begin{Highlighting}[]
\KeywordTok{let} \NormalTok{x = }\DecValTok{5}\NormalTok{;}

\KeywordTok{if} \NormalTok{x == }\DecValTok{5} \NormalTok{\{}
    \PreprocessorTok{println!}\NormalTok{(}\StringTok{"x is five!"}\NormalTok{);}
\NormalTok{\}}
\end{Highlighting}
\end{Shaded}

If we changed the value of \texttt{x} to something else, this line would
not print. More specifically, if the expression after the \texttt{if}
evaluates to \texttt{true}, then the block is executed. If it's
\texttt{false}, then it is not.

If you want something to happen in the \texttt{false} case, use an
\texttt{else}:

\begin{Shaded}
\begin{Highlighting}[]
\KeywordTok{let} \NormalTok{x = }\DecValTok{5}\NormalTok{;}

\KeywordTok{if} \NormalTok{x == }\DecValTok{5} \NormalTok{\{}
    \PreprocessorTok{println!}\NormalTok{(}\StringTok{"x is five!"}\NormalTok{);}
\NormalTok{\} }\KeywordTok{else} \NormalTok{\{}
    \PreprocessorTok{println!}\NormalTok{(}\StringTok{"x is not five :("}\NormalTok{);}
\NormalTok{\}}
\end{Highlighting}
\end{Shaded}

If there is more than one case, use an \texttt{else\ if}:

\begin{Shaded}
\begin{Highlighting}[]
\KeywordTok{let} \NormalTok{x = }\DecValTok{5}\NormalTok{;}

\KeywordTok{if} \NormalTok{x == }\DecValTok{5} \NormalTok{\{}
    \PreprocessorTok{println!}\NormalTok{(}\StringTok{"x is five!"}\NormalTok{);}
\NormalTok{\} }\KeywordTok{else} \KeywordTok{if} \NormalTok{x == }\DecValTok{6} \NormalTok{\{}
    \PreprocessorTok{println!}\NormalTok{(}\StringTok{"x is six!"}\NormalTok{);}
\NormalTok{\} }\KeywordTok{else} \NormalTok{\{}
    \PreprocessorTok{println!}\NormalTok{(}\StringTok{"x is not five or six :("}\NormalTok{);}
\NormalTok{\}}
\end{Highlighting}
\end{Shaded}

This is all pretty standard. However, you can also do this:

\begin{Shaded}
\begin{Highlighting}[]
\KeywordTok{let} \NormalTok{x = }\DecValTok{5}\NormalTok{;}

\KeywordTok{let} \NormalTok{y = }\KeywordTok{if} \NormalTok{x == }\DecValTok{5} \NormalTok{\{}
    \DecValTok{10}
\NormalTok{\} }\KeywordTok{else} \NormalTok{\{}
    \DecValTok{15}
\NormalTok{\}; }\CommentTok{// y: i32}
\end{Highlighting}
\end{Shaded}

Which we can (and probably should) write like this:

\begin{Shaded}
\begin{Highlighting}[]
\KeywordTok{let} \NormalTok{x = }\DecValTok{5}\NormalTok{;}

\KeywordTok{let} \NormalTok{y = }\KeywordTok{if} \NormalTok{x == }\DecValTok{5} \NormalTok{\{ }\DecValTok{10} \NormalTok{\} }\KeywordTok{else} \NormalTok{\{ }\DecValTok{15} \NormalTok{\}; }\CommentTok{// y: i32}
\end{Highlighting}
\end{Shaded}

This works because \texttt{if} is an expression. The value of the
expression is the value of the last expression in whichever branch was
chosen. An \texttt{if} without an \texttt{else} always results in
\texttt{()} as the value.

\section{Loops}\label{sec--loops}

Rust currently provides three approaches to performing some kind of
iterative activity. They are: \texttt{loop}, \texttt{while} and
\texttt{for}. Each approach has its own set of uses.

\subsubsection{loop}\label{loop}

The infinite \texttt{loop} is the simplest form of loop available in
Rust. Using the keyword \texttt{loop}, Rust provides a way to loop
indefinitely until some terminating statement is reached. Rust's
infinite \texttt{loop}s look like this:

\begin{Shaded}
\begin{Highlighting}[]
\KeywordTok{loop} \NormalTok{\{}
    \PreprocessorTok{println!}\NormalTok{(}\StringTok{"Loop forever!"}\NormalTok{);}
\NormalTok{\}}
\end{Highlighting}
\end{Shaded}

\subsubsection{while}\label{while}

Rust also has a \texttt{while} loop. It looks like this:

\begin{Shaded}
\begin{Highlighting}[]
\KeywordTok{let} \KeywordTok{mut} \NormalTok{x = }\DecValTok{5}\NormalTok{; }\CommentTok{// mut x: i32}
\KeywordTok{let} \KeywordTok{mut} \NormalTok{done = }\ConstantTok{false}\NormalTok{; }\CommentTok{// mut done: bool}

\KeywordTok{while} \NormalTok{!done \{}
    \NormalTok{x += x - }\DecValTok{3}\NormalTok{;}

    \PreprocessorTok{println!}\NormalTok{(}\StringTok{"\{\}"}\NormalTok{, x);}

    \KeywordTok{if} \NormalTok{x % }\DecValTok{5} \NormalTok{== }\DecValTok{0} \NormalTok{\{}
        \NormalTok{done = }\ConstantTok{true}\NormalTok{;}
    \NormalTok{\}}
\NormalTok{\}}
\end{Highlighting}
\end{Shaded}

\texttt{while} loops are the correct choice when you're not sure how
many times you need to loop.

If you need an infinite loop, you may be tempted to write this:

\begin{Shaded}
\begin{Highlighting}[]
\KeywordTok{while} \ConstantTok{true} \NormalTok{\{}
\end{Highlighting}
\end{Shaded}

However, \texttt{loop} is far better suited to handle this case:

\begin{Shaded}
\begin{Highlighting}[]
\KeywordTok{loop} \NormalTok{\{}
\end{Highlighting}
\end{Shaded}

Rust's control-flow analysis treats this construct differently than a
\texttt{while\ true}, since we know that it will always loop. In
general, the more information we can give to the compiler, the better it
can do with safety and code generation, so you should always prefer
\texttt{loop} when you plan to loop infinitely.

\subsubsection{for}\label{for}

The \texttt{for} loop is used to loop a particular number of times.
Rust's \texttt{for} loops work a bit differently than in other systems
languages, however. Rust's \texttt{for} loop doesn't look like this
``C-style'' \texttt{for} loop:

\begin{Shaded}
\begin{Highlighting}[]
\KeywordTok{for} \NormalTok{(x = }\DecValTok{0}\NormalTok{; x < }\DecValTok{10}\NormalTok{; x++) \{}
    \NormalTok{printf( }\StringTok{"%d}\CharTok{\textbackslash{}n}\StringTok{"}\NormalTok{, x );}
\NormalTok{\}}
\end{Highlighting}
\end{Shaded}

Instead, it looks like this:

\begin{Shaded}
\begin{Highlighting}[]
\KeywordTok{for} \NormalTok{x }\KeywordTok{in} \DecValTok{0.}\NormalTok{.}\DecValTok{10} \NormalTok{\{}
    \PreprocessorTok{println!}\NormalTok{(}\StringTok{"\{\}"}\NormalTok{, x); }\CommentTok{// x: i32}
\NormalTok{\}}
\end{Highlighting}
\end{Shaded}

In slightly more abstract terms,

\begin{Shaded}
\begin{Highlighting}[]
\KeywordTok{for} \NormalTok{var }\KeywordTok{in} \NormalTok{expression \{}
    \NormalTok{code}
\NormalTok{\}}
\end{Highlighting}
\end{Shaded}

The expression is an item that can be converted into an {[}iterator{]}
using {[}\texttt{IntoIterator}{]}. The iterator gives back a series of
elements. Each element is one iteration of the loop. That value is then
bound to the name \texttt{var}, which is valid for the loop body. Once
the body is over, the next value is fetched from the iterator, and we
loop another time. When there are no more values, the \texttt{for} loop
is over.

In our example, \texttt{0..10} is an expression that takes a start and
an end position, and gives an iterator over those values. The upper
bound is exclusive, though, so our loop will print \texttt{0} through
\texttt{9}, not \texttt{10}.

Rust does not have the ``C-style'' \texttt{for} loop on purpose.
Manually controlling each element of the loop is complicated and error
prone, even for experienced C developers.

\paragraph{Enumerate}\label{enumerate}

When you need to keep track of how many times you already looped, you
can use the \texttt{.enumerate()} function.

\subparagraph{On ranges:}\label{on-ranges}

\begin{Shaded}
\begin{Highlighting}[]
\KeywordTok{for} \NormalTok{(i,j) }\KeywordTok{in} \NormalTok{(}\DecValTok{5.}\NormalTok{.}\DecValTok{10}\NormalTok{).enumerate() \{}
    \PreprocessorTok{println!}\NormalTok{(}\StringTok{"i = \{\} and j = \{\}"}\NormalTok{, i, j);}
\NormalTok{\}}
\end{Highlighting}
\end{Shaded}

Outputs:

\begin{verbatim}
i = 0 and j = 5
i = 1 and j = 6
i = 2 and j = 7
i = 3 and j = 8
i = 4 and j = 9
\end{verbatim}

Don't forget to add the parentheses around the range.

\subparagraph{On iterators:}\label{on-iterators}

\begin{Shaded}
\begin{Highlighting}[]
\KeywordTok{let} \NormalTok{lines = }\StringTok{"hello}\SpecialCharTok{\textbackslash{}n}\StringTok{world"}\NormalTok{.lines();}

\KeywordTok{for} \NormalTok{(linenumber, line) }\KeywordTok{in} \NormalTok{lines.enumerate() \{}
    \PreprocessorTok{println!}\NormalTok{(}\StringTok{"\{\}: \{\}"}\NormalTok{, linenumber, line);}
\NormalTok{\}}
\end{Highlighting}
\end{Shaded}

Outputs:

\begin{verbatim}
0: hello
1: world
\end{verbatim}

\subsubsection{Ending iteration early}\label{ending-iteration-early}

Let's take a look at that \texttt{while} loop we had earlier:

\begin{Shaded}
\begin{Highlighting}[]
\KeywordTok{let} \KeywordTok{mut} \NormalTok{x = }\DecValTok{5}\NormalTok{;}
\KeywordTok{let} \KeywordTok{mut} \NormalTok{done = }\ConstantTok{false}\NormalTok{;}

\KeywordTok{while} \NormalTok{!done \{}
    \NormalTok{x += x - }\DecValTok{3}\NormalTok{;}

    \PreprocessorTok{println!}\NormalTok{(}\StringTok{"\{\}"}\NormalTok{, x);}

    \KeywordTok{if} \NormalTok{x % }\DecValTok{5} \NormalTok{== }\DecValTok{0} \NormalTok{\{}
        \NormalTok{done = }\ConstantTok{true}\NormalTok{;}
    \NormalTok{\}}
\NormalTok{\}}
\end{Highlighting}
\end{Shaded}

We had to keep a dedicated \texttt{mut} boolean variable binding,
\texttt{done}, to know when we should exit out of the loop. Rust has two
keywords to help us with modifying iteration: \texttt{break} and
\texttt{continue}.

In this case, we can write the loop in a better way with \texttt{break}:

\begin{Shaded}
\begin{Highlighting}[]
\KeywordTok{let} \KeywordTok{mut} \NormalTok{x = }\DecValTok{5}\NormalTok{;}

\KeywordTok{loop} \NormalTok{\{}
    \NormalTok{x += x - }\DecValTok{3}\NormalTok{;}

    \PreprocessorTok{println!}\NormalTok{(}\StringTok{"\{\}"}\NormalTok{, x);}

    \KeywordTok{if} \NormalTok{x % }\DecValTok{5} \NormalTok{== }\DecValTok{0} \NormalTok{\{ }\KeywordTok{break}\NormalTok{; \}}
\NormalTok{\}}
\end{Highlighting}
\end{Shaded}

We now loop forever with \texttt{loop} and use \texttt{break} to break
out early. Issuing an explicit \texttt{return} statement will also serve
to terminate the loop early.

\texttt{continue} is similar, but instead of ending the loop, goes to
the next iteration. This will only print the odd numbers:

\begin{Shaded}
\begin{Highlighting}[]
\KeywordTok{for} \NormalTok{x }\KeywordTok{in} \DecValTok{0.}\NormalTok{.}\DecValTok{10} \NormalTok{\{}
    \KeywordTok{if} \NormalTok{x % }\DecValTok{2} \NormalTok{== }\DecValTok{0} \NormalTok{\{ }\KeywordTok{continue}\NormalTok{; \}}

    \PreprocessorTok{println!}\NormalTok{(}\StringTok{"\{\}"}\NormalTok{, x);}
\NormalTok{\}}
\end{Highlighting}
\end{Shaded}

\subsubsection{Loop labels}\label{loop-labels}

You may also encounter situations where you have nested loops and need
to specify which one your \texttt{break} or \texttt{continue} statement
is for. Like most other languages, by default a \texttt{break} or
\texttt{continue} will apply to innermost loop. In a situation where you
would like to \texttt{break} or \texttt{continue} for one of the outer
loops, you can use labels to specify which loop the \texttt{break} or
\texttt{continue} statement applies to. This will only print when both
\texttt{x} and \texttt{y} are odd:

\begin{Shaded}
\begin{Highlighting}[]
\OtherTok{'outer}\NormalTok{: }\KeywordTok{for} \NormalTok{x }\KeywordTok{in} \DecValTok{0.}\NormalTok{.}\DecValTok{10} \NormalTok{\{}
    \OtherTok{'inner}\NormalTok{: }\KeywordTok{for} \NormalTok{y }\KeywordTok{in} \DecValTok{0.}\NormalTok{.}\DecValTok{10} \NormalTok{\{}
        \KeywordTok{if} \NormalTok{x % }\DecValTok{2} \NormalTok{== }\DecValTok{0} \NormalTok{\{ }\KeywordTok{continue} \OtherTok{'outer}\NormalTok{; \} }\CommentTok{// continues the loop over x}
        \KeywordTok{if} \NormalTok{y % }\DecValTok{2} \NormalTok{== }\DecValTok{0} \NormalTok{\{ }\KeywordTok{continue} \OtherTok{'inner}\NormalTok{; \} }\CommentTok{// continues the loop over y}
        \PreprocessorTok{println!}\NormalTok{(}\StringTok{"x: \{\}, y: \{\}"}\NormalTok{, x, y);}
    \NormalTok{\}}
\NormalTok{\}}
\end{Highlighting}
\end{Shaded}

\hypertarget{sec--vectors}{\section{Vectors}\label{sec--vectors}}

A `vector' is a dynamic or `growable' array, implemented as the standard
library type
\href{http://doc.rust-lang.org/std/vec/index.html}{\texttt{Vec\textless{}T\textgreater{}}}.
The \texttt{T} means that we can have vectors of any type (see the
chapter on \protect\hyperlink{sec--generics}{generics} for more).
Vectors always allocate their data on the heap. You can create them with
the \texttt{vec!} macro:

\begin{Shaded}
\begin{Highlighting}[]
\KeywordTok{let} \NormalTok{v = }\PreprocessorTok{vec!}\NormalTok{[}\DecValTok{1}\NormalTok{, }\DecValTok{2}\NormalTok{, }\DecValTok{3}\NormalTok{, }\DecValTok{4}\NormalTok{, }\DecValTok{5}\NormalTok{]; }\CommentTok{// v: Vec<i32>}
\end{Highlighting}
\end{Shaded}

(Notice that unlike the \texttt{println!} macro we've used in the past,
we use square brackets \texttt{{[}{]}} with \texttt{vec!} macro. Rust
allows you to use either in either situation, this is just convention.)

There's an alternate form of \texttt{vec!} for repeating an initial
value:

\begin{Shaded}
\begin{Highlighting}[]
\KeywordTok{let} \NormalTok{v = }\PreprocessorTok{vec!}\NormalTok{[}\DecValTok{0}\NormalTok{; }\DecValTok{10}\NormalTok{]; }\CommentTok{// ten zeroes}
\end{Highlighting}
\end{Shaded}

Vectors store their contents as contiguous arrays of \texttt{T} on the
heap. This means that they must be able to know the size of \texttt{T}
at compile time (that is, how many bytes are needed to store a
\texttt{T}?). The size of some things can't be known at compile time.
For these you'll have to store a pointer to that thing: thankfully, the
\href{http://doc.rust-lang.org/std/boxed/index.html}{\texttt{Box}} type
works perfectly for this.

\subsubsection{Accessing elements}\label{accessing-elements}

To get the value at a particular index in the vector, we use
\texttt{{[}{]}}s:

\begin{Shaded}
\begin{Highlighting}[]
\KeywordTok{let} \NormalTok{v = }\PreprocessorTok{vec!}\NormalTok{[}\DecValTok{1}\NormalTok{, }\DecValTok{2}\NormalTok{, }\DecValTok{3}\NormalTok{, }\DecValTok{4}\NormalTok{, }\DecValTok{5}\NormalTok{];}

\PreprocessorTok{println!}\NormalTok{(}\StringTok{"The third element of v is \{\}"}\NormalTok{, v[}\DecValTok{2}\NormalTok{]);}
\end{Highlighting}
\end{Shaded}

The indices count from \texttt{0}, so the third element is
\texttt{v{[}2{]}}.

It's also important to note that you must index with the \texttt{usize}
type:

\begin{Shaded}
\begin{Highlighting}[]
\KeywordTok{let} \NormalTok{v = }\PreprocessorTok{vec!}\NormalTok{[}\DecValTok{1}\NormalTok{, }\DecValTok{2}\NormalTok{, }\DecValTok{3}\NormalTok{, }\DecValTok{4}\NormalTok{, }\DecValTok{5}\NormalTok{];}

\KeywordTok{let} \NormalTok{i: }\DataTypeTok{usize} \NormalTok{= }\DecValTok{0}\NormalTok{;}
\KeywordTok{let} \NormalTok{j: }\DataTypeTok{i32} \NormalTok{= }\DecValTok{0}\NormalTok{;}

\CommentTok{// works}
\NormalTok{v[i];}

\CommentTok{// doesn’t}
\NormalTok{v[j];}
\end{Highlighting}
\end{Shaded}

Indexing with a non-\texttt{usize} type gives an error that looks like
this:

\begin{verbatim}
error: the trait bound `collections::vec::Vec<_> : core::ops::Index<i32>`
is not satisfied [E0277]
v[j];
^~~~
note: the type `collections::vec::Vec<_>` cannot be indexed by `i32`
error: aborting due to previous error
\end{verbatim}

There's a lot of punctuation in that message, but the core of it makes
sense: you cannot index with an \texttt{i32}.

\subsubsection{Out-of-bounds Access}\label{out-of-bounds-access}

If you try to access an index that doesn't exist:

\begin{Shaded}
\begin{Highlighting}[]
\KeywordTok{let} \NormalTok{v = }\PreprocessorTok{vec!}\NormalTok{[}\DecValTok{1}\NormalTok{, }\DecValTok{2}\NormalTok{, }\DecValTok{3}\NormalTok{];}
\PreprocessorTok{println!}\NormalTok{(}\StringTok{"Item 7 is \{\}"}\NormalTok{, v[}\DecValTok{7}\NormalTok{]);}
\end{Highlighting}
\end{Shaded}

then the current thread will {[}panic{]} with a message like this:

\begin{verbatim}
thread '<main>' panicked at 'index out of bounds: the len is 3 but the index is 7'
\end{verbatim}

If you want to handle out-of-bounds errors without panicking, you can
use methods like
\href{http://doc.rust-lang.org/std/vec/struct.Vec.html\#method.get}{\texttt{get}}
or
\href{http://doc.rust-lang.org/std/vec/struct.Vec.html\#method.get_mut}{\texttt{get\_mut}}
that return \texttt{None} when given an invalid index:

\begin{Shaded}
\begin{Highlighting}[]
\KeywordTok{let} \NormalTok{v = }\PreprocessorTok{vec!}\NormalTok{[}\DecValTok{1}\NormalTok{, }\DecValTok{2}\NormalTok{, }\DecValTok{3}\NormalTok{];}
\KeywordTok{match} \NormalTok{v.get(}\DecValTok{7}\NormalTok{) \{}
    \ConstantTok{Some}\NormalTok{(x) => }\PreprocessorTok{println!}\NormalTok{(}\StringTok{"Item 7 is \{\}"}\NormalTok{, x),}
    \ConstantTok{None} \NormalTok{=> }\PreprocessorTok{println!}\NormalTok{(}\StringTok{"Sorry, this vector is too short."}\NormalTok{)}
\NormalTok{\}}
\end{Highlighting}
\end{Shaded}

\subsubsection{Iterating}\label{iterating}

Once you have a vector, you can iterate through its elements with
\texttt{for}. There are three versions:

\begin{Shaded}
\begin{Highlighting}[]
\KeywordTok{let} \KeywordTok{mut} \NormalTok{v = }\PreprocessorTok{vec!}\NormalTok{[}\DecValTok{1}\NormalTok{, }\DecValTok{2}\NormalTok{, }\DecValTok{3}\NormalTok{, }\DecValTok{4}\NormalTok{, }\DecValTok{5}\NormalTok{];}

\KeywordTok{for} \NormalTok{i }\KeywordTok{in} \NormalTok{&v \{}
    \PreprocessorTok{println!}\NormalTok{(}\StringTok{"A reference to \{\}"}\NormalTok{, i);}
\NormalTok{\}}

\KeywordTok{for} \NormalTok{i }\KeywordTok{in} \NormalTok{&}\KeywordTok{mut} \NormalTok{v \{}
    \PreprocessorTok{println!}\NormalTok{(}\StringTok{"A mutable reference to \{\}"}\NormalTok{, i);}
\NormalTok{\}}

\KeywordTok{for} \NormalTok{i }\KeywordTok{in} \NormalTok{v \{}
    \PreprocessorTok{println!}\NormalTok{(}\StringTok{"Take ownership of the vector and its element \{\}"}\NormalTok{, i);}
\NormalTok{\}}
\end{Highlighting}
\end{Shaded}

Note: You cannot use the vector again once you have iterated by taking
ownership of the vector. You can iterate the vector multiple times by
taking a reference to the vector whilst iterating. For example, the
following code does not compile.

\begin{Shaded}
\begin{Highlighting}[]
\KeywordTok{let} \NormalTok{v = }\PreprocessorTok{vec!}\NormalTok{[}\DecValTok{1}\NormalTok{, }\DecValTok{2}\NormalTok{, }\DecValTok{3}\NormalTok{, }\DecValTok{4}\NormalTok{, }\DecValTok{5}\NormalTok{];}

\KeywordTok{for} \NormalTok{i }\KeywordTok{in} \NormalTok{v \{}
    \PreprocessorTok{println!}\NormalTok{(}\StringTok{"Take ownership of the vector and its element \{\}"}\NormalTok{, i);}
\NormalTok{\}}

\KeywordTok{for} \NormalTok{i }\KeywordTok{in} \NormalTok{v \{}
    \PreprocessorTok{println!}\NormalTok{(}\StringTok{"Take ownership of the vector and its element \{\}"}\NormalTok{, i);}
\NormalTok{\}}
\end{Highlighting}
\end{Shaded}

Whereas the following works perfectly,

\begin{Shaded}
\begin{Highlighting}[]
\KeywordTok{let} \NormalTok{v = }\PreprocessorTok{vec!}\NormalTok{[}\DecValTok{1}\NormalTok{, }\DecValTok{2}\NormalTok{, }\DecValTok{3}\NormalTok{, }\DecValTok{4}\NormalTok{, }\DecValTok{5}\NormalTok{];}

\KeywordTok{for} \NormalTok{i }\KeywordTok{in} \NormalTok{&v \{}
    \PreprocessorTok{println!}\NormalTok{(}\StringTok{"This is a reference to \{\}"}\NormalTok{, i);}
\NormalTok{\}}

\KeywordTok{for} \NormalTok{i }\KeywordTok{in} \NormalTok{&v \{}
    \PreprocessorTok{println!}\NormalTok{(}\StringTok{"This is a reference to \{\}"}\NormalTok{, i);}
\NormalTok{\}}
\end{Highlighting}
\end{Shaded}

Vectors have many more useful methods, which you can read about in
\href{http://doc.rust-lang.org/std/vec/index.html}{their API
documentation}.

\hypertarget{sec--ownership}{\section{Ownership}\label{sec--ownership}}

This is the first of three sections presenting Rust's ownership system.
This is one of Rust's most distinct and compelling features, with which
Rust developers should become quite acquainted. Ownership is how Rust
achieves its largest goal, memory safety. There are a few distinct
concepts, each with its own chapter:

\begin{itemize}
\tightlist
\item
  ownership, which you're reading now
\item
  \protect\hyperlink{sec--references-and-borrowing}{borrowing}, and
  their associated feature `references'
\item
  \protect\hyperlink{sec--lifetimes}{lifetimes}, an advanced concept of
  borrowing
\end{itemize}

These three chapters are related, and in order. You'll need all three to
fully understand the ownership system.

\subsection{Meta}\label{meta}

Before we get to the details, two important notes about the ownership
system.

Rust has a focus on safety and speed. It accomplishes these goals
through many `zero-cost abstractions', which means that in Rust,
abstractions cost as little as possible in order to make them work. The
ownership system is a prime example of a zero-cost abstraction. All of
the analysis we'll talk about in this guide is \emph{done at compile
time}. You do not pay any run-time cost for any of these features.

However, this system does have a certain cost: learning curve. Many new
users to Rust experience something we like to call `fighting with the
borrow checker', where the Rust compiler refuses to compile a program
that the author thinks is valid. This often happens because the
programmer's mental model of how ownership should work doesn't match the
actual rules that Rust implements. You probably will experience similar
things at first. There is good news, however: more experienced Rust
developers report that once they work with the rules of the ownership
system for a period of time, they fight the borrow checker less and
less.

With that in mind, let's learn about ownership.

\subsection{Ownership}\label{ownership}

\protect\hyperlink{sec--variable-bindings}{Variable bindings} have a
property in Rust: they `have ownership' of what they're bound to. This
means that when a binding goes out of scope, Rust will free the bound
resources. For example:

\begin{Shaded}
\begin{Highlighting}[]
\KeywordTok{fn} \NormalTok{foo() \{}
    \KeywordTok{let} \NormalTok{v = }\PreprocessorTok{vec!}\NormalTok{[}\DecValTok{1}\NormalTok{, }\DecValTok{2}\NormalTok{, }\DecValTok{3}\NormalTok{];}
\NormalTok{\}}
\end{Highlighting}
\end{Shaded}

When \texttt{v} comes into scope, a new
\protect\hyperlink{sec--vectors}{vector} is created on
\protect\hyperlink{the-stack}{the stack}, and it allocates space on
\protect\hyperlink{sec--the-stack-and-the-heap}{the heap} for its
elements. When \texttt{v} goes out of scope at the end of
\texttt{foo()}, Rust will clean up everything related to the vector,
even the heap-allocated memory. This happens deterministically, at the
end of the scope.

We'll cover \protect\hyperlink{sec--vectors}{vectors} in detail later in
this chapter; we only use them here as an example of a type that
allocates space on the heap at runtime. They behave like
\protect\hyperlink{arrays}{arrays}, except their size may change by
\texttt{push()}ing more elements onto them.

Vectors have a \protect\hyperlink{sec--generics}{generic type}
\texttt{Vec\textless{}T\textgreater{}}, so in this example \texttt{v}
will have type \texttt{Vec\textless{}i32\textgreater{}}. We'll cover
generics in detail later in this chapter.

\subsection{Move semantics}\label{move-semantics}

There's some more subtlety here, though: Rust ensures that there is
\emph{exactly one} binding to any given resource. For example, if we
have a vector, we can assign it to another binding:

\begin{Shaded}
\begin{Highlighting}[]
\KeywordTok{let} \NormalTok{v = }\PreprocessorTok{vec!}\NormalTok{[}\DecValTok{1}\NormalTok{, }\DecValTok{2}\NormalTok{, }\DecValTok{3}\NormalTok{];}

\KeywordTok{let} \NormalTok{v2 = v;}
\end{Highlighting}
\end{Shaded}

But, if we try to use \texttt{v} afterwards, we get an error:

\begin{Shaded}
\begin{Highlighting}[]
\KeywordTok{let} \NormalTok{v = }\PreprocessorTok{vec!}\NormalTok{[}\DecValTok{1}\NormalTok{, }\DecValTok{2}\NormalTok{, }\DecValTok{3}\NormalTok{];}

\KeywordTok{let} \NormalTok{v2 = v;}

\PreprocessorTok{println!}\NormalTok{(}\StringTok{"v[0] is: \{\}"}\NormalTok{, v[}\DecValTok{0}\NormalTok{]);}
\end{Highlighting}
\end{Shaded}

It looks like this:

\begin{verbatim}
error: use of moved value: `v`
println!("v[0] is: {}", v[0]);
                        ^
\end{verbatim}

A similar thing happens if we define a function which takes ownership,
and try to use something after we've passed it as an argument:

\begin{Shaded}
\begin{Highlighting}[]
\KeywordTok{fn} \NormalTok{take(v: }\DataTypeTok{Vec}\NormalTok{<}\DataTypeTok{i32}\NormalTok{>) \{}
    \CommentTok{// what happens here isn’t important.}
\NormalTok{\}}

\KeywordTok{let} \NormalTok{v = }\PreprocessorTok{vec!}\NormalTok{[}\DecValTok{1}\NormalTok{, }\DecValTok{2}\NormalTok{, }\DecValTok{3}\NormalTok{];}

\NormalTok{take(v);}

\PreprocessorTok{println!}\NormalTok{(}\StringTok{"v[0] is: \{\}"}\NormalTok{, v[}\DecValTok{0}\NormalTok{]);}
\end{Highlighting}
\end{Shaded}

Same error: `use of moved value'. When we transfer ownership to
something else, we say that we've `moved' the thing we refer to. You
don't need some sort of special annotation here, it's the default thing
that Rust does.

\subsubsection{The details}\label{the-details}

The reason that we cannot use a binding after we've moved it is subtle,
but important.

When we write code like this:

\begin{Shaded}
\begin{Highlighting}[]
\KeywordTok{let} \NormalTok{x = }\DecValTok{10}\NormalTok{;}
\end{Highlighting}
\end{Shaded}

Rust allocates memory for an integer {[}i32{]} on the
\protect\hyperlink{sec--the-stack-and-the-heap}{stack}, copies the bit
pattern representing the value of 10 to the allocated memory and binds
the variable name x to this memory region for future reference.

Now consider the following code fragment:

\begin{Shaded}
\begin{Highlighting}[]
\KeywordTok{let} \NormalTok{v = }\PreprocessorTok{vec!}\NormalTok{[}\DecValTok{1}\NormalTok{, }\DecValTok{2}\NormalTok{, }\DecValTok{3}\NormalTok{];}

\KeywordTok{let} \KeywordTok{mut} \NormalTok{v2 = v;}
\end{Highlighting}
\end{Shaded}

The first line allocates memory for the vector object \texttt{v} on the
stack like it does for \texttt{x} above. But in addition to that it also
allocates some memory on the
\protect\hyperlink{sec--the-stack-and-the-heap}{heap} for the actual
data (\texttt{{[}1,\ 2,\ 3{]}}). Rust copies the address of this heap
allocation to an internal pointer, which is part of the vector object
placed on the stack (let's call it the data pointer).

It is worth pointing out (even at the risk of stating the obvious) that
the vector object and its data live in separate memory regions instead
of being a single contiguous memory allocation (due to reasons we will
not go into at this point of time). These two parts of the vector (the
one on the stack and one on the heap) must agree with each other at all
times with regards to things like the length, capacity etc.

When we move \texttt{v} to \texttt{v2}, Rust actually does a bitwise
copy of the vector object \texttt{v} into the stack allocation
represented by \texttt{v2}. This shallow copy does not create a copy of
the heap allocation containing the actual data. Which means that there
would be two pointers to the contents of the vector both pointing to the
same memory allocation on the heap. It would violate Rust's safety
guarantees by introducing a data race if one could access both
\texttt{v} and \texttt{v2} at the same time.

For example if we truncated the vector to just two elements through
\texttt{v2}:

\begin{Shaded}
\begin{Highlighting}[]
\NormalTok{v2.truncate(}\DecValTok{2}\NormalTok{);}
\end{Highlighting}
\end{Shaded}

and \texttt{v} were still accessible we'd end up with an invalid vector
since \texttt{v} would not know that the heap data has been truncated.
Now, the part of the vector \texttt{v} on the stack does not agree with
the corresponding part on the heap. \texttt{v} still thinks there are
three elements in the vector and will happily let us access the non
existent element \texttt{v{[}2{]}} but as you might already know this is
a recipe for disaster. Especially because it might lead to a
segmentation fault or worse allow an unauthorized user to read from
memory to which they don't have access.

This is why Rust forbids using \texttt{v} after we've done the move.

It's also important to note that optimizations may remove the actual
copy of the bytes on the stack, depending on circumstances. So it may
not be as inefficient as it initially seems.

\subsubsection{\texorpdfstring{\texttt{Copy}
types}{Copy types}}\label{copy-types}

We've established that when ownership is transferred to another binding,
you cannot use the original binding. However, there's a
\protect\hyperlink{sec--traits}{trait} that changes this behavior, and
it's called \texttt{Copy}. We haven't discussed traits yet, but for now,
you can think of them as an annotation to a particular type that adds
extra behavior. For example:

\begin{Shaded}
\begin{Highlighting}[]
\KeywordTok{let} \NormalTok{v = }\DecValTok{1}\NormalTok{;}

\KeywordTok{let} \NormalTok{v2 = v;}

\PreprocessorTok{println!}\NormalTok{(}\StringTok{"v is: \{\}"}\NormalTok{, v);}
\end{Highlighting}
\end{Shaded}

In this case, \texttt{v} is an \texttt{i32}, which implements the
\texttt{Copy} trait. This means that, just like a move, when we assign
\texttt{v} to \texttt{v2}, a copy of the data is made. But, unlike a
move, we can still use \texttt{v} afterward. This is because an
\texttt{i32} has no pointers to data somewhere else, copying it is a
full copy.

All primitive types implement the \texttt{Copy} trait and their
ownership is therefore not moved like one would assume, following the
´ownership rules´. To give an example, the two following snippets of
code only compile because the \texttt{i32} and \texttt{bool} types
implement the \texttt{Copy} trait.

\begin{Shaded}
\begin{Highlighting}[]
\KeywordTok{fn} \NormalTok{main() \{}
    \KeywordTok{let} \NormalTok{a = }\DecValTok{5}\NormalTok{;}

    \KeywordTok{let} \NormalTok{_y = double(a);}
    \PreprocessorTok{println!}\NormalTok{(}\StringTok{"\{\}"}\NormalTok{, a);}
\NormalTok{\}}

\KeywordTok{fn} \NormalTok{double(x: }\DataTypeTok{i32}\NormalTok{) -> }\DataTypeTok{i32} \NormalTok{\{}
    \NormalTok{x * }\DecValTok{2}
\NormalTok{\}}
\end{Highlighting}
\end{Shaded}

\begin{Shaded}
\begin{Highlighting}[]
\KeywordTok{fn} \NormalTok{main() \{}
    \KeywordTok{let} \NormalTok{a = }\ConstantTok{true}\NormalTok{;}

    \KeywordTok{let} \NormalTok{_y = change_truth(a);}
    \PreprocessorTok{println!}\NormalTok{(}\StringTok{"\{\}"}\NormalTok{, a);}
\NormalTok{\}}

\KeywordTok{fn} \NormalTok{change_truth(x: }\DataTypeTok{bool}\NormalTok{) -> }\DataTypeTok{bool} \NormalTok{\{}
    \NormalTok{!x}
\NormalTok{\}}
\end{Highlighting}
\end{Shaded}

If we had used types that do not implement the \texttt{Copy} trait, we
would have gotten a compile error because we tried to use a moved value.

\begin{verbatim}
error: use of moved value: `a`
println!("{}", a);
               ^
\end{verbatim}

We will discuss how to make your own types \texttt{Copy} in the
\protect\hyperlink{sec--traits}{traits} section.

\subsection{More than ownership}\label{more-than-ownership}

Of course, if we had to hand ownership back with every function we
wrote:

\begin{Shaded}
\begin{Highlighting}[]
\KeywordTok{fn} \NormalTok{foo(v: }\DataTypeTok{Vec}\NormalTok{<}\DataTypeTok{i32}\NormalTok{>) -> }\DataTypeTok{Vec}\NormalTok{<}\DataTypeTok{i32}\NormalTok{> \{}
    \CommentTok{// do stuff with v}

    \CommentTok{// hand back ownership}
    \NormalTok{v}
\NormalTok{\}}
\end{Highlighting}
\end{Shaded}

This would get very tedious. It gets worse the more things we want to
take ownership of:

\begin{Shaded}
\begin{Highlighting}[]
\KeywordTok{fn} \NormalTok{foo(v1: }\DataTypeTok{Vec}\NormalTok{<}\DataTypeTok{i32}\NormalTok{>, v2: }\DataTypeTok{Vec}\NormalTok{<}\DataTypeTok{i32}\NormalTok{>) -> (}\DataTypeTok{Vec}\NormalTok{<}\DataTypeTok{i32}\NormalTok{>, }\DataTypeTok{Vec}\NormalTok{<}\DataTypeTok{i32}\NormalTok{>, }\DataTypeTok{i32}\NormalTok{) \{}
    \CommentTok{// do stuff with v1 and v2}

    \CommentTok{// hand back ownership, and the result of our function}
    \NormalTok{(v1, v2, }\DecValTok{42}\NormalTok{)}
\NormalTok{\}}

\KeywordTok{let} \NormalTok{v1 = }\PreprocessorTok{vec!}\NormalTok{[}\DecValTok{1}\NormalTok{, }\DecValTok{2}\NormalTok{, }\DecValTok{3}\NormalTok{];}
\KeywordTok{let} \NormalTok{v2 = }\PreprocessorTok{vec!}\NormalTok{[}\DecValTok{1}\NormalTok{, }\DecValTok{2}\NormalTok{, }\DecValTok{3}\NormalTok{];}

\KeywordTok{let} \NormalTok{(v1, v2, answer) = foo(v1, v2);}
\end{Highlighting}
\end{Shaded}

Ugh! The return type, return line, and calling the function gets way
more complicated.

Luckily, Rust offers a feature, borrowing, which helps us solve this
problem. It's the topic of the next section!

\hypertarget{sec--references-and-borrowing}{\section{References and
Borrowing}\label{sec--references-and-borrowing}}

This is the second of three sections presenting Rust's ownership system.
This is one of Rust's most distinct and compelling features, with which
Rust developers should become quite acquainted. Ownership is how Rust
achieves its largest goal, memory safety. There are a few distinct
concepts, each with its own chapter:

\begin{itemize}
\tightlist
\item
  \protect\hyperlink{sec--ownership}{ownership}, the key concept
\item
  borrowing, which you're reading now
\item
  \protect\hyperlink{sec--lifetimes}{lifetimes}, an advanced concept of
  borrowing
\end{itemize}

These three chapters are related, and in order. You'll need all three to
fully understand the ownership system.

\subsection{Meta}\label{meta-1}

Before we get to the details, two important notes about the ownership
system.

Rust has a focus on safety and speed. It accomplishes these goals
through many `zero-cost abstractions', which means that in Rust,
abstractions cost as little as possible in order to make them work. The
ownership system is a prime example of a zero-cost abstraction. All of
the analysis we'll talk about in this guide is \emph{done at compile
time}. You do not pay any run-time cost for any of these features.

However, this system does have a certain cost: learning curve. Many new
users to Rust experience something we like to call `fighting with the
borrow checker', where the Rust compiler refuses to compile a program
that the author thinks is valid. This often happens because the
programmer's mental model of how ownership should work doesn't match the
actual rules that Rust implements. You probably will experience similar
things at first. There is good news, however: more experienced Rust
developers report that once they work with the rules of the ownership
system for a period of time, they fight the borrow checker less and
less.

With that in mind, let's learn about borrowing.

\hypertarget{borrowing}{\subsection{Borrowing}\label{borrowing}}

At the end of the \protect\hyperlink{sec--ownership}{ownership} section,
we had a nasty function that looked like this:

\begin{Shaded}
\begin{Highlighting}[]
\KeywordTok{fn} \NormalTok{foo(v1: }\DataTypeTok{Vec}\NormalTok{<}\DataTypeTok{i32}\NormalTok{>, v2: }\DataTypeTok{Vec}\NormalTok{<}\DataTypeTok{i32}\NormalTok{>) -> (}\DataTypeTok{Vec}\NormalTok{<}\DataTypeTok{i32}\NormalTok{>, }\DataTypeTok{Vec}\NormalTok{<}\DataTypeTok{i32}\NormalTok{>, }\DataTypeTok{i32}\NormalTok{) \{}
    \CommentTok{// do stuff with v1 and v2}

    \CommentTok{// hand back ownership, and the result of our function}
    \NormalTok{(v1, v2, }\DecValTok{42}\NormalTok{)}
\NormalTok{\}}

\KeywordTok{let} \NormalTok{v1 = }\PreprocessorTok{vec!}\NormalTok{[}\DecValTok{1}\NormalTok{, }\DecValTok{2}\NormalTok{, }\DecValTok{3}\NormalTok{];}
\KeywordTok{let} \NormalTok{v2 = }\PreprocessorTok{vec!}\NormalTok{[}\DecValTok{1}\NormalTok{, }\DecValTok{2}\NormalTok{, }\DecValTok{3}\NormalTok{];}

\KeywordTok{let} \NormalTok{(v1, v2, answer) = foo(v1, v2);}
\end{Highlighting}
\end{Shaded}

This is not idiomatic Rust, however, as it doesn't take advantage of
borrowing. Here's the first step:

\begin{Shaded}
\begin{Highlighting}[]
\KeywordTok{fn} \NormalTok{foo(v1: &}\DataTypeTok{Vec}\NormalTok{<}\DataTypeTok{i32}\NormalTok{>, v2: &}\DataTypeTok{Vec}\NormalTok{<}\DataTypeTok{i32}\NormalTok{>) -> }\DataTypeTok{i32} \NormalTok{\{}
    \CommentTok{// do stuff with v1 and v2}

    \CommentTok{// return the answer}
    \DecValTok{42}
\NormalTok{\}}

\KeywordTok{let} \NormalTok{v1 = }\PreprocessorTok{vec!}\NormalTok{[}\DecValTok{1}\NormalTok{, }\DecValTok{2}\NormalTok{, }\DecValTok{3}\NormalTok{];}
\KeywordTok{let} \NormalTok{v2 = }\PreprocessorTok{vec!}\NormalTok{[}\DecValTok{1}\NormalTok{, }\DecValTok{2}\NormalTok{, }\DecValTok{3}\NormalTok{];}

\KeywordTok{let} \NormalTok{answer = foo(&v1, &v2);}

\CommentTok{// we can use v1 and v2 here!}
\end{Highlighting}
\end{Shaded}

A more concrete example:

\begin{Shaded}
\begin{Highlighting}[]
\KeywordTok{fn} \NormalTok{main() \{}
    \CommentTok{// Don't worry if you don't understand how `fold` works, the point here is that an}
\NormalTok{↳  immutable reference is borrowed.}
    \KeywordTok{fn} \NormalTok{sum_vec(v: &}\DataTypeTok{Vec}\NormalTok{<}\DataTypeTok{i32}\NormalTok{>) -> }\DataTypeTok{i32} \NormalTok{\{}
        \KeywordTok{return} \NormalTok{v.iter().fold(}\DecValTok{0}\NormalTok{, |a, &b| a + b);}
    \NormalTok{\}}
    \CommentTok{// Borrow two vectors and and sum them.}
    \CommentTok{// This kind of borrowing does not allow mutation to the borrowed.}
    \KeywordTok{fn} \NormalTok{foo(v1: &}\DataTypeTok{Vec}\NormalTok{<}\DataTypeTok{i32}\NormalTok{>, v2: &}\DataTypeTok{Vec}\NormalTok{<}\DataTypeTok{i32}\NormalTok{>) -> }\DataTypeTok{i32} \NormalTok{\{}
        \CommentTok{// do stuff with v1 and v2}
        \KeywordTok{let} \NormalTok{s1 = sum_vec(v1);}
        \KeywordTok{let} \NormalTok{s2 = sum_vec(v2);}
        \CommentTok{// return the answer}
        \NormalTok{s1 + s2}
    \NormalTok{\}}

    \KeywordTok{let} \NormalTok{v1 = }\PreprocessorTok{vec!}\NormalTok{[}\DecValTok{1}\NormalTok{, }\DecValTok{2}\NormalTok{, }\DecValTok{3}\NormalTok{];}
    \KeywordTok{let} \NormalTok{v2 = }\PreprocessorTok{vec!}\NormalTok{[}\DecValTok{4}\NormalTok{, }\DecValTok{5}\NormalTok{, }\DecValTok{6}\NormalTok{];}

    \KeywordTok{let} \NormalTok{answer = foo(&v1, &v2);}
    \PreprocessorTok{println!}\NormalTok{(}\StringTok{"\{\}"}\NormalTok{, answer);}
\NormalTok{\}}
\end{Highlighting}
\end{Shaded}

Instead of taking \texttt{Vec\textless{}i32\textgreater{}}s as our
arguments, we take a reference:
\texttt{\&Vec\textless{}i32\textgreater{}}. And instead of passing
\texttt{v1} and \texttt{v2} directly, we pass \texttt{\&v1} and
\texttt{\&v2}. We call the \texttt{\&T} type a `reference', and rather
than owning the resource, it borrows ownership. A binding that borrows
something does not deallocate the resource when it goes out of scope.
This means that after the call to \texttt{foo()}, we can use our
original bindings again.

References are immutable, like bindings. This means that inside of
\texttt{foo()}, the vectors can't be changed at all:

\begin{Shaded}
\begin{Highlighting}[]
\KeywordTok{fn} \NormalTok{foo(v: &}\DataTypeTok{Vec}\NormalTok{<}\DataTypeTok{i32}\NormalTok{>) \{}
     \NormalTok{v.push(}\DecValTok{5}\NormalTok{);}
\NormalTok{\}}

\KeywordTok{let} \NormalTok{v = }\PreprocessorTok{vec!}\NormalTok{[];}

\NormalTok{foo(&v);}
\end{Highlighting}
\end{Shaded}

errors with:

\begin{verbatim}
error: cannot borrow immutable borrowed content `*v` as mutable
v.push(5);
^
\end{verbatim}

Pushing a value mutates the vector, and so we aren't allowed to do it.

\subsection{\&mut references}\label{mut-references}

There's a second kind of reference: \texttt{\&mut\ T}. A `mutable
reference' allows you to mutate the resource you're borrowing. For
example:

\begin{Shaded}
\begin{Highlighting}[]
\KeywordTok{let} \KeywordTok{mut} \NormalTok{x = }\DecValTok{5}\NormalTok{;}
\NormalTok{\{}
    \KeywordTok{let} \NormalTok{y = &}\KeywordTok{mut} \NormalTok{x;}
    \NormalTok{*y += }\DecValTok{1}\NormalTok{;}
\NormalTok{\}}
\PreprocessorTok{println!}\NormalTok{(}\StringTok{"\{\}"}\NormalTok{, x);}
\end{Highlighting}
\end{Shaded}

This will print \texttt{6}. We make \texttt{y} a mutable reference to
\texttt{x}, then add one to the thing \texttt{y} points at. You'll
notice that \texttt{x} had to be marked \texttt{mut} as well. If it
wasn't, we couldn't take a mutable borrow to an immutable value.

You'll also notice we added an asterisk (\texttt{*}) in front of
\texttt{y}, making it \texttt{*y}, this is because \texttt{y} is a
\texttt{\&mut} reference. You'll also need to use them for accessing the
contents of a reference as well.

Otherwise, \texttt{\&mut} references are like references. There
\emph{is} a large difference between the two, and how they interact,
though. You can tell something is fishy in the above example, because we
need that extra scope, with the \texttt{\{} and \texttt{\}}. If we
remove them, we get an error:

\begin{verbatim}
error: cannot borrow `x` as immutable because it is also borrowed as mutable
    println!("{}", x);
                   ^
note: previous borrow of `x` occurs here; the mutable borrow prevents
subsequent moves, borrows, or modification of `x` until the borrow ends
        let y = &mut x;
                     ^
note: previous borrow ends here
fn main() {

}
^
\end{verbatim}

As it turns out, there are rules.

\subsection{The Rules}\label{the-rules}

Here's the rules about borrowing in Rust:

First, any borrow must last for a scope no greater than that of the
owner. Second, you may have one or the other of these two kinds of
borrows, but not both at the same time:

\begin{itemize}
\tightlist
\item
  one or more references (\texttt{\&T}) to a resource,
\item
  exactly one mutable reference (\texttt{\&mut\ T}).
\end{itemize}

You may notice that this is very similar to, though not exactly the same
as, the definition of a data race:

\begin{quote}
There is a `data race' when two or more pointers access the same memory
location at the same time, where at least one of them is writing, and
the operations are not synchronized.
\end{quote}

With references, you may have as many as you'd like, since none of them
are writing. However, as we can only have one \texttt{\&mut} at a time,
it is impossible to have a data race. This is how Rust prevents data
races at compile time: we'll get errors if we break the rules.

With this in mind, let's consider our example again.

\subsubsection{Thinking in scopes}\label{thinking-in-scopes}

Here's the code:

\begin{Shaded}
\begin{Highlighting}[]
\KeywordTok{let} \KeywordTok{mut} \NormalTok{x = }\DecValTok{5}\NormalTok{;}
\KeywordTok{let} \NormalTok{y = &}\KeywordTok{mut} \NormalTok{x;}

\NormalTok{*y += }\DecValTok{1}\NormalTok{;}

\PreprocessorTok{println!}\NormalTok{(}\StringTok{"\{\}"}\NormalTok{, x);}
\end{Highlighting}
\end{Shaded}

This code gives us this error:

\begin{verbatim}
error: cannot borrow `x` as immutable because it is also borrowed as mutable
    println!("{}", x);
                   ^
\end{verbatim}

This is because we've violated the rules: we have a \texttt{\&mut\ T}
pointing to \texttt{x}, and so we aren't allowed to create any
\texttt{\&T}s. One or the other. The note hints at how to think about
this problem:

\begin{verbatim}
note: previous borrow ends here
fn main() {

}
^
\end{verbatim}

In other words, the mutable borrow is held through the rest of our
example. What we want is for the mutable borrow by \texttt{y} to end so
that the resource can be returned to the owner, \texttt{x}. \texttt{x}
can then provide a immutable borrow to \texttt{println!}. In Rust,
borrowing is tied to the scope that the borrow is valid for. And our
scopes look like this:

\begin{Shaded}
\begin{Highlighting}[]
\KeywordTok{let} \KeywordTok{mut} \NormalTok{x = }\DecValTok{5}\NormalTok{;}

\KeywordTok{let} \NormalTok{y = &}\KeywordTok{mut} \NormalTok{x;    }\CommentTok{// -+ &mut borrow of x starts here}
                   \CommentTok{//  |}
\NormalTok{*y += }\DecValTok{1}\NormalTok{;           }\CommentTok{//  |}
                   \CommentTok{//  |}
\PreprocessorTok{println!}\NormalTok{(}\StringTok{"\{\}"}\NormalTok{, x); }\CommentTok{// -+ - try to borrow x here}
                   \CommentTok{// -+ &mut borrow of x ends here}
\end{Highlighting}
\end{Shaded}

The scopes conflict: we can't make an \texttt{\&x} while \texttt{y} is
in scope.

So when we add the curly braces:

\begin{Shaded}
\begin{Highlighting}[]
\KeywordTok{let} \KeywordTok{mut} \NormalTok{x = }\DecValTok{5}\NormalTok{;}

\NormalTok{\{}
    \KeywordTok{let} \NormalTok{y = &}\KeywordTok{mut} \NormalTok{x; }\CommentTok{// -+ &mut borrow starts here}
    \NormalTok{*y += }\DecValTok{1}\NormalTok{;        }\CommentTok{//  |}
\NormalTok{\}                   }\CommentTok{// -+ ... and ends here}

\PreprocessorTok{println!}\NormalTok{(}\StringTok{"\{\}"}\NormalTok{, x);  }\CommentTok{// <- try to borrow x here}
\end{Highlighting}
\end{Shaded}

There's no problem. Our mutable borrow goes out of scope before we
create an immutable one. But scope is the key to seeing how long a
borrow lasts for.

\subsubsection{Issues borrowing
prevents}\label{issues-borrowing-prevents}

Why have these restrictive rules? Well, as we noted, these rules prevent
data races. What kinds of issues do data races cause? Here's a few.

\paragraph{Iterator invalidation}\label{iterator-invalidation}

One example is `iterator invalidation', which happens when you try to
mutate a collection that you're iterating over. Rust's borrow checker
prevents this from happening:

\begin{Shaded}
\begin{Highlighting}[]
\KeywordTok{let} \KeywordTok{mut} \NormalTok{v = }\PreprocessorTok{vec!}\NormalTok{[}\DecValTok{1}\NormalTok{, }\DecValTok{2}\NormalTok{, }\DecValTok{3}\NormalTok{];}

\KeywordTok{for} \NormalTok{i }\KeywordTok{in} \NormalTok{&v \{}
    \PreprocessorTok{println!}\NormalTok{(}\StringTok{"\{\}"}\NormalTok{, i);}
\NormalTok{\}}
\end{Highlighting}
\end{Shaded}

This prints out one through three. As we iterate through the vector,
we're only given references to the elements. And \texttt{v} is itself
borrowed as immutable, which means we can't change it while we're
iterating:

\begin{Shaded}
\begin{Highlighting}[]
\KeywordTok{let} \KeywordTok{mut} \NormalTok{v = }\PreprocessorTok{vec!}\NormalTok{[}\DecValTok{1}\NormalTok{, }\DecValTok{2}\NormalTok{, }\DecValTok{3}\NormalTok{];}

\KeywordTok{for} \NormalTok{i }\KeywordTok{in} \NormalTok{&v \{}
    \PreprocessorTok{println!}\NormalTok{(}\StringTok{"\{\}"}\NormalTok{, i);}
    \NormalTok{v.push(}\DecValTok{34}\NormalTok{);}
\NormalTok{\}}
\end{Highlighting}
\end{Shaded}

Here's the error:

\begin{verbatim}
error: cannot borrow `v` as mutable because it is also borrowed as immutable
    v.push(34);
    ^
note: previous borrow of `v` occurs here; the immutable borrow prevents
subsequent moves or mutable borrows of `v` until the borrow ends
for i in &v {
          ^
note: previous borrow ends here
for i in &v {
    println!(“{}”, i);
    v.push(34);
}
^
\end{verbatim}

We can't modify \texttt{v} because it's borrowed by the loop.

\paragraph{use after free}\label{use-after-free}

References must not live longer than the resource they refer to. Rust
will check the scopes of your references to ensure that this is true.

If Rust didn't check this property, we could accidentally use a
reference which was invalid. For example:

\begin{Shaded}
\begin{Highlighting}[]
\KeywordTok{let} \NormalTok{y: &}\DataTypeTok{i32}\NormalTok{;}
\NormalTok{\{}
    \KeywordTok{let} \NormalTok{x = }\DecValTok{5}\NormalTok{;}
    \NormalTok{y = &x;}
\NormalTok{\}}

\PreprocessorTok{println!}\NormalTok{(}\StringTok{"\{\}"}\NormalTok{, y);}
\end{Highlighting}
\end{Shaded}

We get this error:

\begin{verbatim}
error: `x` does not live long enough
    y = &x;
         ^
note: reference must be valid for the block suffix following statement 0 at
2:16...
let y: &i32;
{
    let x = 5;
    y = &x;
}

note: ...but borrowed value is only valid for the block suffix following
statement 0 at 4:18
    let x = 5;
    y = &x;
}
\end{verbatim}

In other words, \texttt{y} is only valid for the scope where \texttt{x}
exists. As soon as \texttt{x} goes away, it becomes invalid to refer to
it. As such, the error says that the borrow `doesn't live long enough'
because it's not valid for the right amount of time.

The same problem occurs when the reference is declared \emph{before} the
variable it refers to. This is because resources within the same scope
are freed in the opposite order they were declared:

\begin{Shaded}
\begin{Highlighting}[]
\KeywordTok{let} \NormalTok{y: &}\DataTypeTok{i32}\NormalTok{;}
\KeywordTok{let} \NormalTok{x = }\DecValTok{5}\NormalTok{;}
\NormalTok{y = &x;}

\PreprocessorTok{println!}\NormalTok{(}\StringTok{"\{\}"}\NormalTok{, y);}
\end{Highlighting}
\end{Shaded}

We get this error:

\begin{verbatim}
error: `x` does not live long enough
y = &x;
     ^
note: reference must be valid for the block suffix following statement 0 at
2:16...
    let y: &i32;
    let x = 5;
    y = &x;

    println!("{}", y);
}

note: ...but borrowed value is only valid for the block suffix following
statement 1 at 3:14
    let x = 5;
    y = &x;

    println!("{}", y);
}
\end{verbatim}

In the above example, \texttt{y} is declared before \texttt{x}, meaning
that \texttt{y} lives longer than \texttt{x}, which is not allowed.

\hypertarget{sec--lifetimes}{\section{Lifetimes}\label{sec--lifetimes}}

This is the last of three sections presenting Rust's ownership system.
This is one of Rust's most distinct and compelling features, with which
Rust developers should become quite acquainted. Ownership is how Rust
achieves its largest goal, memory safety. There are a few distinct
concepts, each with its own chapter:

\begin{itemize}
\tightlist
\item
  \protect\hyperlink{sec--ownership}{ownership}, the key concept
\item
  \protect\hyperlink{sec--references-and-borrowing}{borrowing}, and
  their associated feature `references'
\item
  lifetimes, which you're reading now
\end{itemize}

These three chapters are related, and in order. You'll need all three to
fully understand the ownership system.

\subsection{Meta}\label{meta-2}

Before we get to the details, two important notes about the ownership
system.

Rust has a focus on safety and speed. It accomplishes these goals
through many `zero-cost abstractions', which means that in Rust,
abstractions cost as little as possible in order to make them work. The
ownership system is a prime example of a zero-cost abstraction. All of
the analysis we'll talk about in this guide is \emph{done at compile
time}. You do not pay any run-time cost for any of these features.

However, this system does have a certain cost: learning curve. Many new
users to Rust experience something we like to call `fighting with the
borrow checker', where the Rust compiler refuses to compile a program
that the author thinks is valid. This often happens because the
programmer's mental model of how ownership should work doesn't match the
actual rules that Rust implements. You probably will experience similar
things at first. There is good news, however: more experienced Rust
developers report that once they work with the rules of the ownership
system for a period of time, they fight the borrow checker less and
less.

With that in mind, let's learn about lifetimes.

\hypertarget{lifetimes}{\subsection{Lifetimes}\label{lifetimes}}

Lending out a reference to a resource that someone else owns can be
complicated. For example, imagine this set of operations:

\begin{enumerate}
\def\labelenumi{\arabic{enumi}.}
\tightlist
\item
  I acquire a handle to some kind of resource.
\item
  I lend you a reference to the resource.
\item
  I decide I'm done with the resource, and deallocate it, while you
  still have your reference.
\item
  You decide to use the resource.
\end{enumerate}

Uh oh! Your reference is pointing to an invalid resource. This is called
a dangling pointer or `use after free', when the resource is memory.

To fix this, we have to make sure that step four never happens after
step three. The ownership system in Rust does this through a concept
called lifetimes, which describe the scope that a reference is valid
for.

When we have a function that takes an argument by reference, we can be
implicit or explicit about the lifetime of the reference:

\begin{Shaded}
\begin{Highlighting}[]
\CommentTok{// implicit}
\KeywordTok{fn} \NormalTok{foo(x: &}\DataTypeTok{i32}\NormalTok{) \{}
\NormalTok{\}}

\CommentTok{// explicit}
\KeywordTok{fn} \NormalTok{bar<}\OtherTok{'a}\NormalTok{>(x: &}\OtherTok{'a} \DataTypeTok{i32}\NormalTok{) \{}
\NormalTok{\}}
\end{Highlighting}
\end{Shaded}

The \texttt{\textquotesingle{}a} reads `the lifetime a'. Technically,
every reference has some lifetime associated with it, but the compiler
lets you elide (i.e.~omit, see
\protect\hyperlink{lifetime-elision}{``Lifetime Elision''} below) them
in common cases. Before we get to that, though, let's break the explicit
example down:

\begin{Shaded}
\begin{Highlighting}[]
\KeywordTok{fn} \NormalTok{bar<}\OtherTok{'a}\NormalTok{>(...)}
\end{Highlighting}
\end{Shaded}

We previously talked a little about
\protect\hyperlink{sec--functions}{function syntax}, but we didn't
discuss the \texttt{\textless{}\textgreater{}}s after a function's name.
A function can have `generic parameters' between the
\texttt{\textless{}\textgreater{}}s, of which lifetimes are one kind.
We'll discuss other kinds of generics
\protect\hyperlink{sec--generics}{later in the book}, but for now, let's
focus on the lifetimes aspect.

We use \texttt{\textless{}\textgreater{}} to declare our lifetimes. This
says that \texttt{bar} has one lifetime, \texttt{\textquotesingle{}a}.
If we had two reference parameters, it would look like this:

\begin{Shaded}
\begin{Highlighting}[]
\KeywordTok{fn} \NormalTok{bar<}\OtherTok{'a}\NormalTok{, }\OtherTok{'b}\NormalTok{>(...)}
\end{Highlighting}
\end{Shaded}

Then in our parameter list, we use the lifetimes we've named:

\begin{Shaded}
\begin{Highlighting}[]
\NormalTok{...(x: &}\OtherTok{'a} \DataTypeTok{i32}\NormalTok{)}
\end{Highlighting}
\end{Shaded}

If we wanted a \texttt{\&mut} reference, we'd do this:

\begin{Shaded}
\begin{Highlighting}[]
\NormalTok{...(x: &}\OtherTok{'a} \KeywordTok{mut} \DataTypeTok{i32}\NormalTok{)}
\end{Highlighting}
\end{Shaded}

If you compare \texttt{\&mut\ i32} to
\texttt{\&\textquotesingle{}a\ mut\ i32}, they're the same, it's that
the lifetime \texttt{\textquotesingle{}a} has snuck in between the
\texttt{\&} and the \texttt{mut\ i32}. We read \texttt{\&mut\ i32} as `a
mutable reference to an \texttt{i32}' and
\texttt{\&\textquotesingle{}a\ mut\ i32} as `a mutable reference to an
\texttt{i32} with the lifetime \texttt{\textquotesingle{}a}'.

\subsection{\texorpdfstring{In
\texttt{struct}s}{In structs}}\label{in-structs}

You'll also need explicit lifetimes when working with
\protect\hyperlink{sec--structs}{\texttt{struct}}s that contain
references:

\begin{Shaded}
\begin{Highlighting}[]
\KeywordTok{struct} \NormalTok{Foo<}\OtherTok{'a}\NormalTok{> \{}
    \NormalTok{x: &}\OtherTok{'a} \DataTypeTok{i32}\NormalTok{,}
\NormalTok{\}}

\KeywordTok{fn} \NormalTok{main() \{}
    \KeywordTok{let} \NormalTok{y = &}\DecValTok{5}\NormalTok{; }\CommentTok{// this is the same as `let _y = 5; let y = &_y;`}
    \KeywordTok{let} \NormalTok{f = Foo \{ x: y \};}

    \PreprocessorTok{println!}\NormalTok{(}\StringTok{"\{\}"}\NormalTok{, f.x);}
\NormalTok{\}}
\end{Highlighting}
\end{Shaded}

As you can see, \texttt{struct}s can also have lifetimes. In a similar
way to functions,

\begin{Shaded}
\begin{Highlighting}[]
\KeywordTok{struct} \NormalTok{Foo<}\OtherTok{'a}\NormalTok{> \{}
\end{Highlighting}
\end{Shaded}

declares a lifetime, and

\begin{Shaded}
\begin{Highlighting}[]
\NormalTok{x: &}\OtherTok{'a} \DataTypeTok{i32}\NormalTok{,}
\end{Highlighting}
\end{Shaded}

uses it. So why do we need a lifetime here? We need to ensure that any
reference to a \texttt{Foo} cannot outlive the reference to an
\texttt{i32} it contains.

\subsubsection{\texorpdfstring{\texttt{impl}
blocks}{impl blocks}}\label{impl-blocks}

Let's implement a method on \texttt{Foo}:

\begin{Shaded}
\begin{Highlighting}[]
\KeywordTok{struct} \NormalTok{Foo<}\OtherTok{'a}\NormalTok{> \{}
    \NormalTok{x: &}\OtherTok{'a} \DataTypeTok{i32}\NormalTok{,}
\NormalTok{\}}

\KeywordTok{impl}\NormalTok{<}\OtherTok{'a}\NormalTok{> Foo<}\OtherTok{'a}\NormalTok{> \{}
    \KeywordTok{fn} \NormalTok{x(&}\KeywordTok{self}\NormalTok{) -> &}\OtherTok{'a} \DataTypeTok{i32} \NormalTok{\{ }\KeywordTok{self}\NormalTok{.x \}}
\NormalTok{\}}

\KeywordTok{fn} \NormalTok{main() \{}
    \KeywordTok{let} \NormalTok{y = &}\DecValTok{5}\NormalTok{; }\CommentTok{// this is the same as `let _y = 5; let y = &_y;`}
    \KeywordTok{let} \NormalTok{f = Foo \{ x: y \};}

    \PreprocessorTok{println!}\NormalTok{(}\StringTok{"x is: \{\}"}\NormalTok{, f.x());}
\NormalTok{\}}
\end{Highlighting}
\end{Shaded}

As you can see, we need to declare a lifetime for \texttt{Foo} in the
\texttt{impl} line. We repeat \texttt{\textquotesingle{}a} twice, like
on functions: \texttt{impl\textless{}\textquotesingle{}a\textgreater{}}
defines a lifetime \texttt{\textquotesingle{}a}, and
\texttt{Foo\textless{}\textquotesingle{}a\textgreater{}} uses it.

\subsubsection{Multiple lifetimes}\label{multiple-lifetimes}

If you have multiple references, you can use the same lifetime multiple
times:

\begin{Shaded}
\begin{Highlighting}[]
\KeywordTok{fn} \NormalTok{x_or_y<}\OtherTok{'a}\NormalTok{>(x: &}\OtherTok{'a} \DataTypeTok{str}\NormalTok{, y: &}\OtherTok{'a} \DataTypeTok{str}\NormalTok{) -> &}\OtherTok{'a} \DataTypeTok{str} \NormalTok{\{}
\end{Highlighting}
\end{Shaded}

This says that \texttt{x} and \texttt{y} both are alive for the same
scope, and that the return value is also alive for that scope. If you
wanted \texttt{x} and \texttt{y} to have different lifetimes, you can
use multiple lifetime parameters:

\begin{Shaded}
\begin{Highlighting}[]
\KeywordTok{fn} \NormalTok{x_or_y<}\OtherTok{'a}\NormalTok{, }\OtherTok{'b}\NormalTok{>(x: &}\OtherTok{'a} \DataTypeTok{str}\NormalTok{, y: &}\OtherTok{'b} \DataTypeTok{str}\NormalTok{) -> &}\OtherTok{'a} \DataTypeTok{str} \NormalTok{\{}
\end{Highlighting}
\end{Shaded}

In this example, \texttt{x} and \texttt{y} have different valid scopes,
but the return value has the same lifetime as \texttt{x}.

\subsubsection{Thinking in scopes}\label{thinking-in-scopes-1}

A way to think about lifetimes is to visualize the scope that a
reference is valid for. For example:

\begin{Shaded}
\begin{Highlighting}[]
\KeywordTok{fn} \NormalTok{main() \{}
    \KeywordTok{let} \NormalTok{y = &}\DecValTok{5}\NormalTok{;     }\CommentTok{// -+ y goes into scope}
                    \CommentTok{//  |}
    \CommentTok{// stuff        //  |}
                    \CommentTok{//  |}
\NormalTok{\}                   }\CommentTok{// -+ y goes out of scope}
\end{Highlighting}
\end{Shaded}

Adding in our \texttt{Foo}:

\begin{Shaded}
\begin{Highlighting}[]
\KeywordTok{struct} \NormalTok{Foo<}\OtherTok{'a}\NormalTok{> \{}
    \NormalTok{x: &}\OtherTok{'a} \DataTypeTok{i32}\NormalTok{,}
\NormalTok{\}}

\KeywordTok{fn} \NormalTok{main() \{}
    \KeywordTok{let} \NormalTok{y = &}\DecValTok{5}\NormalTok{;           }\CommentTok{// -+ y goes into scope}
    \KeywordTok{let} \NormalTok{f = Foo \{ x: y \}; }\CommentTok{// -+ f goes into scope}
    \CommentTok{// stuff              //  |}
                          \CommentTok{//  |}
\NormalTok{\}                         }\CommentTok{// -+ f and y go out of scope}
\end{Highlighting}
\end{Shaded}

Our \texttt{f} lives within the scope of \texttt{y}, so everything
works. What if it didn't? This code won't work:

\begin{Shaded}
\begin{Highlighting}[]
\KeywordTok{struct} \NormalTok{Foo<}\OtherTok{'a}\NormalTok{> \{}
    \NormalTok{x: &}\OtherTok{'a} \DataTypeTok{i32}\NormalTok{,}
\NormalTok{\}}

\KeywordTok{fn} \NormalTok{main() \{}
    \KeywordTok{let} \NormalTok{x;                    }\CommentTok{// -+ x goes into scope}
                              \CommentTok{//  |}
    \NormalTok{\{                         }\CommentTok{//  |}
        \KeywordTok{let} \NormalTok{y = &}\DecValTok{5}\NormalTok{;           }\CommentTok{// ---+ y goes into scope}
        \KeywordTok{let} \NormalTok{f = Foo \{ x: y \}; }\CommentTok{// ---+ f goes into scope}
        \NormalTok{x = &f.x;             }\CommentTok{//  | | error here}
    \NormalTok{\}                         }\CommentTok{// ---+ f and y go out of scope}
                              \CommentTok{//  |}
    \PreprocessorTok{println!}\NormalTok{(}\StringTok{"\{\}"}\NormalTok{, x);        }\CommentTok{//  |}
\NormalTok{\}                             }\CommentTok{// -+ x goes out of scope}
\end{Highlighting}
\end{Shaded}

Whew! As you can see here, the scopes of \texttt{f} and \texttt{y} are
smaller than the scope of \texttt{x}. But when we do
\texttt{x\ =\ \&f.x}, we make \texttt{x} a reference to something that's
about to go out of scope.

Named lifetimes are a way of giving these scopes a name. Giving
something a name is the first step towards being able to talk about it.

\hypertarget{static}{\subsubsection{'static}\label{static}}

The lifetime named `static' is a special lifetime. It signals that
something has the lifetime of the entire program. Most Rust programmers
first come across \texttt{\textquotesingle{}static} when dealing with
strings:

\begin{Shaded}
\begin{Highlighting}[]
\KeywordTok{let} \NormalTok{x: &}\OtherTok{'static} \DataTypeTok{str} \NormalTok{= }\StringTok{"Hello, world."}\NormalTok{;}
\end{Highlighting}
\end{Shaded}

String literals have the type \texttt{\&\textquotesingle{}static\ str}
because the reference is always alive: they are baked into the data
segment of the final binary. Another example are globals:

\begin{Shaded}
\begin{Highlighting}[]
\KeywordTok{static} \NormalTok{FOO: }\DataTypeTok{i32} \NormalTok{= }\DecValTok{5}\NormalTok{;}
\KeywordTok{let} \NormalTok{x: &}\OtherTok{'static} \DataTypeTok{i32} \NormalTok{= &FOO;}
\end{Highlighting}
\end{Shaded}

This adds an \texttt{i32} to the data segment of the binary, and
\texttt{x} is a reference to it.

\hypertarget{lifetime-elision}{\subsubsection{Lifetime
Elision}\label{lifetime-elision}}

Rust supports powerful local type inference in the bodies of functions
but not in their item signatures. It's forbidden to allow reasoning
about types based on the item signature alone. However, for ergonomic
reasons, a very restricted secondary inference algorithm called
``lifetime elision'' does apply when judging lifetimes. Lifetime elision
is concerned solely to infer lifetime parameters using three easily
memorizable and unambiguous rules. This means lifetime elision acts as a
shorthand for writing an item signature, while not hiding away the
actual types involved as full local inference would if applied to it.

When talking about lifetime elision, we use the term \emph{input
lifetime} and \emph{output lifetime}. An \emph{input lifetime} is a
lifetime associated with a parameter of a function, and an \emph{output
lifetime} is a lifetime associated with the return value of a function.
For example, this function has an input lifetime:

\begin{Shaded}
\begin{Highlighting}[]
\KeywordTok{fn} \NormalTok{foo<}\OtherTok{'a}\NormalTok{>(bar: &}\OtherTok{'a} \DataTypeTok{str}\NormalTok{)}
\end{Highlighting}
\end{Shaded}

This one has an output lifetime:

\begin{Shaded}
\begin{Highlighting}[]
\KeywordTok{fn} \NormalTok{foo<}\OtherTok{'a}\NormalTok{>() -> &}\OtherTok{'a} \DataTypeTok{str}
\end{Highlighting}
\end{Shaded}

This one has a lifetime in both positions:

\begin{Shaded}
\begin{Highlighting}[]
\KeywordTok{fn} \NormalTok{foo<}\OtherTok{'a}\NormalTok{>(bar: &}\OtherTok{'a} \DataTypeTok{str}\NormalTok{) -> &}\OtherTok{'a} \DataTypeTok{str}
\end{Highlighting}
\end{Shaded}

Here are the three rules:

\begin{itemize}
\item
  Each elided lifetime in a function's arguments becomes a distinct
  lifetime parameter.
\item
  If there is exactly one input lifetime, elided or not, that lifetime
  is assigned to all elided lifetimes in the return values of that
  function.
\item
  If there are multiple input lifetimes, but one of them is
  \texttt{\&self} or \texttt{\&mut\ \ \ self}, the lifetime of
  \texttt{self} is assigned to all elided output lifetimes.
\end{itemize}

Otherwise, it is an error to elide an output lifetime.

\paragraph{Examples}\label{examples}

Here are some examples of functions with elided lifetimes. We've paired
each example of an elided lifetime with its expanded form.

\begin{Shaded}
\begin{Highlighting}[]
\KeywordTok{fn} \NormalTok{print(s: &}\DataTypeTok{str}\NormalTok{); }\CommentTok{// elided}
\KeywordTok{fn} \NormalTok{print<}\OtherTok{'a}\NormalTok{>(s: &}\OtherTok{'a} \DataTypeTok{str}\NormalTok{); }\CommentTok{// expanded}

\KeywordTok{fn} \NormalTok{debug(lvl: }\DataTypeTok{u32}\NormalTok{, s: &}\DataTypeTok{str}\NormalTok{); }\CommentTok{// elided}
\KeywordTok{fn} \NormalTok{debug<}\OtherTok{'a}\NormalTok{>(lvl: }\DataTypeTok{u32}\NormalTok{, s: &}\OtherTok{'a} \DataTypeTok{str}\NormalTok{); }\CommentTok{// expanded}

\CommentTok{// In the preceding example, `lvl` doesn’t need a lifetime because it’s not a}
\CommentTok{// reference (`&`). Only things relating to references (such as a `struct`}
\CommentTok{// which contains a reference) need lifetimes.}

\KeywordTok{fn} \NormalTok{substr(s: &}\DataTypeTok{str}\NormalTok{, until: }\DataTypeTok{u32}\NormalTok{) -> &}\DataTypeTok{str}\NormalTok{; }\CommentTok{// elided}
\KeywordTok{fn} \NormalTok{substr<}\OtherTok{'a}\NormalTok{>(s: &}\OtherTok{'a} \DataTypeTok{str}\NormalTok{, until: }\DataTypeTok{u32}\NormalTok{) -> &}\OtherTok{'a} \DataTypeTok{str}\NormalTok{; }\CommentTok{// expanded}

\KeywordTok{fn} \NormalTok{get_str() -> &}\DataTypeTok{str}\NormalTok{; }\CommentTok{// ILLEGAL, no inputs}

\KeywordTok{fn} \NormalTok{frob(s: &}\DataTypeTok{str}\NormalTok{, t: &}\DataTypeTok{str}\NormalTok{) -> &}\DataTypeTok{str}\NormalTok{; }\CommentTok{// ILLEGAL, two inputs}
\KeywordTok{fn} \NormalTok{frob<}\OtherTok{'a}\NormalTok{, }\OtherTok{'b}\NormalTok{>(s: &}\OtherTok{'a} \DataTypeTok{str}\NormalTok{, t: &}\OtherTok{'b} \DataTypeTok{str}\NormalTok{) -> &}\DataTypeTok{str}\NormalTok{; }\CommentTok{// Expanded: Output lifetime is ambig}
\NormalTok{↳ uous}

\KeywordTok{fn} \NormalTok{get_mut(&}\KeywordTok{mut} \KeywordTok{self}\NormalTok{) -> &}\KeywordTok{mut} \NormalTok{T; }\CommentTok{// elided}
\KeywordTok{fn} \NormalTok{get_mut<}\OtherTok{'a}\NormalTok{>(&}\OtherTok{'a} \KeywordTok{mut} \KeywordTok{self}\NormalTok{) -> &}\OtherTok{'a} \KeywordTok{mut} \NormalTok{T; }\CommentTok{// expanded}

\KeywordTok{fn} \NormalTok{args<T: ToCStr>(&}\KeywordTok{mut} \KeywordTok{self}\NormalTok{, args: &[T]) -> &}\KeywordTok{mut} \NormalTok{Command; }\CommentTok{// elided}
\KeywordTok{fn} \NormalTok{args<}\OtherTok{'a}\NormalTok{, }\OtherTok{'b}\NormalTok{, T: ToCStr>(&}\OtherTok{'a} \KeywordTok{mut} \KeywordTok{self}\NormalTok{, args: &}\OtherTok{'b} \NormalTok{[T]) -> &}\OtherTok{'a} \KeywordTok{mut} \NormalTok{Command; }\CommentTok{// expande}
\NormalTok{↳ d}

\KeywordTok{fn} \NormalTok{new(buf: &}\KeywordTok{mut} \NormalTok{[}\DataTypeTok{u8}\NormalTok{]) -> BufWriter; }\CommentTok{// elided}
\KeywordTok{fn} \NormalTok{new<}\OtherTok{'a}\NormalTok{>(buf: &}\OtherTok{'a} \KeywordTok{mut} \NormalTok{[}\DataTypeTok{u8}\NormalTok{]) -> BufWriter<}\OtherTok{'a}\NormalTok{>; }\CommentTok{// expanded}
\end{Highlighting}
\end{Shaded}

\hypertarget{sec--mutability}{\section{Mutability}\label{sec--mutability}}

Mutability, the ability to change something, works a bit differently in
Rust than in other languages. The first aspect of mutability is its
non-default status:

\begin{Shaded}
\begin{Highlighting}[]
\KeywordTok{let} \NormalTok{x = }\DecValTok{5}\NormalTok{;}
\NormalTok{x = }\DecValTok{6}\NormalTok{; }\CommentTok{// error!}
\end{Highlighting}
\end{Shaded}

We can introduce mutability with the \texttt{mut} keyword:

\begin{Shaded}
\begin{Highlighting}[]
\KeywordTok{let} \KeywordTok{mut} \NormalTok{x = }\DecValTok{5}\NormalTok{;}

\NormalTok{x = }\DecValTok{6}\NormalTok{; }\CommentTok{// no problem!}
\end{Highlighting}
\end{Shaded}

This is a mutable \protect\hyperlink{sec--variable-bindings}{variable
binding}. When a binding is mutable, it means you're allowed to change
what the binding points to. So in the above example, it's not so much
that the value at \texttt{x} is changing, but that the binding changed
from one \texttt{i32} to another.

You can also create a
\protect\hyperlink{sec--references-and-borrowing}{reference} to it,
using \texttt{\&x}, but if you want to use the reference to change it,
you will need a mutable reference:

\begin{Shaded}
\begin{Highlighting}[]
\KeywordTok{let} \KeywordTok{mut} \NormalTok{x = }\DecValTok{5}\NormalTok{;}
\KeywordTok{let} \NormalTok{y = &}\KeywordTok{mut} \NormalTok{x;}
\end{Highlighting}
\end{Shaded}

\texttt{y} is an immutable binding to a mutable reference, which means
that you can't bind `y' to something else (\texttt{y\ =\ \&mut\ z}), but
\texttt{y} can be used to bind \texttt{x} to something else
(\texttt{*y\ =\ 5}). A subtle distinction.

Of course, if you need both:

\begin{Shaded}
\begin{Highlighting}[]
\KeywordTok{let} \KeywordTok{mut} \NormalTok{x = }\DecValTok{5}\NormalTok{;}
\KeywordTok{let} \KeywordTok{mut} \NormalTok{y = &}\KeywordTok{mut} \NormalTok{x;}
\end{Highlighting}
\end{Shaded}

Now \texttt{y} can be bound to another value, and the value it's
referencing can be changed.

It's important to note that \texttt{mut} is part of a
\protect\hyperlink{sec--patterns}{pattern}, so you can do things like
this:

\begin{Shaded}
\begin{Highlighting}[]
\KeywordTok{let} \NormalTok{(}\KeywordTok{mut} \NormalTok{x, y) = (}\DecValTok{5}\NormalTok{, }\DecValTok{6}\NormalTok{);}

\KeywordTok{fn} \NormalTok{foo(}\KeywordTok{mut} \NormalTok{x: }\DataTypeTok{i32}\NormalTok{) \{}
\end{Highlighting}
\end{Shaded}

\subsection{Interior vs.~Exterior
Mutability}\label{interior-vs.exterior-mutability}

However, when we say something is `immutable' in Rust, that doesn't mean
that it's not able to be changed: we mean something has `exterior
mutability'. Consider, for example,
\href{http://doc.rust-lang.org/std/sync/struct.Arc.html}{\texttt{Arc\textless{}T\textgreater{}}}:

\begin{Shaded}
\begin{Highlighting}[]
\KeywordTok{use} \NormalTok{std::sync::Arc;}

\KeywordTok{let} \NormalTok{x = Arc::new(}\DecValTok{5}\NormalTok{);}
\KeywordTok{let} \NormalTok{y = x.clone();}
\end{Highlighting}
\end{Shaded}

When we call \texttt{clone()}, the
\texttt{Arc\textless{}T\textgreater{}} needs to update the reference
count. Yet we've not used any \texttt{mut}s here, \texttt{x} is an
immutable binding, and we didn't take \texttt{\&mut\ 5} or anything. So
what gives?

To understand this, we have to go back to the core of Rust's guiding
philosophy, memory safety, and the mechanism by which Rust guarantees
it, the \protect\hyperlink{sec--ownership}{ownership} system, and more
specifically, \protect\hyperlink{borrowing}{borrowing}:

\begin{quote}
You may have one or the other of these two kinds of borrows, but not
both at the same time:

\begin{itemize}
\tightlist
\item
  one or more references (\texttt{\&T}) to a resource,
\item
  exactly one mutable reference (\texttt{\&mut\ T}).
\end{itemize}
\end{quote}

So, that's the real definition of `immutability': is this safe to have
two pointers to? In \texttt{Arc\textless{}T\textgreater{}}'s case, yes:
the mutation is entirely contained inside the structure itself. It's not
user facing. For this reason, it hands out \texttt{\&T} with
\texttt{clone()}. If it handed out \texttt{\&mut\ T}s, though, that
would be a problem.

Other types, like the ones in the
\href{http://doc.rust-lang.org/std/cell/index.html}{\texttt{std::cell}}
module, have the opposite: interior mutability. For example:

\begin{Shaded}
\begin{Highlighting}[]
\KeywordTok{use} \NormalTok{std::cell::RefCell;}

\KeywordTok{let} \NormalTok{x = RefCell::new(}\DecValTok{42}\NormalTok{);}

\KeywordTok{let} \NormalTok{y = x.borrow_mut();}
\end{Highlighting}
\end{Shaded}

RefCell hands out \texttt{\&mut} references to what's inside of it with
the \texttt{borrow\_mut()} method. Isn't that dangerous? What if we do:

\begin{Shaded}
\begin{Highlighting}[]
\KeywordTok{use} \NormalTok{std::cell::RefCell;}

\KeywordTok{let} \NormalTok{x = RefCell::new(}\DecValTok{42}\NormalTok{);}

\KeywordTok{let} \NormalTok{y = x.borrow_mut();}
\KeywordTok{let} \NormalTok{z = x.borrow_mut();}
\end{Highlighting}
\end{Shaded}

This will in fact panic, at runtime. This is what \texttt{RefCell} does:
it enforces Rust's borrowing rules at runtime, and \texttt{panic!}s if
they're violated. This allows us to get around another aspect of Rust's
mutability rules. Let's talk about it first.

\subsubsection{Field-level mutability}\label{field-level-mutability}

Mutability is a property of either a borrow (\texttt{\&mut}) or a
binding (\texttt{let\ mut}). This means that, for example, you cannot
have a \protect\hyperlink{sec--structs}{\texttt{struct}} with some
fields mutable and some immutable:

\begin{Shaded}
\begin{Highlighting}[]
\KeywordTok{struct} \NormalTok{Point \{}
    \NormalTok{x: }\DataTypeTok{i32}\NormalTok{,}
    \KeywordTok{mut} \NormalTok{y: }\DataTypeTok{i32}\NormalTok{, }\CommentTok{// nope}
\NormalTok{\}}
\end{Highlighting}
\end{Shaded}

The mutability of a struct is in its binding:

\begin{Shaded}
\begin{Highlighting}[]
\KeywordTok{struct} \NormalTok{Point \{}
    \NormalTok{x: }\DataTypeTok{i32}\NormalTok{,}
    \NormalTok{y: }\DataTypeTok{i32}\NormalTok{,}
\NormalTok{\}}

\KeywordTok{let} \KeywordTok{mut} \NormalTok{a = Point \{ x: }\DecValTok{5}\NormalTok{, y: }\DecValTok{6} \NormalTok{\};}

\NormalTok{a.x = }\DecValTok{10}\NormalTok{;}

\KeywordTok{let} \NormalTok{b = Point \{ x: }\DecValTok{5}\NormalTok{, y: }\DecValTok{6}\NormalTok{\};}

\NormalTok{b.x = }\DecValTok{10}\NormalTok{; }\CommentTok{// error: cannot assign to immutable field `b.x`}
\end{Highlighting}
\end{Shaded}

However, by using
\href{http://doc.rust-lang.org/std/cell/struct.Cell.html}{\texttt{Cell\textless{}T\textgreater{}}},
you can emulate field-level mutability:

\begin{Shaded}
\begin{Highlighting}[]
\KeywordTok{use} \NormalTok{std::cell::Cell;}

\KeywordTok{struct} \NormalTok{Point \{}
    \NormalTok{x: }\DataTypeTok{i32}\NormalTok{,}
    \NormalTok{y: Cell<}\DataTypeTok{i32}\NormalTok{>,}
\NormalTok{\}}

\KeywordTok{let} \NormalTok{point = Point \{ x: }\DecValTok{5}\NormalTok{, y: Cell::new(}\DecValTok{6}\NormalTok{) \};}

\NormalTok{point.y.set(}\DecValTok{7}\NormalTok{);}

\PreprocessorTok{println!}\NormalTok{(}\StringTok{"y: \{:?\}"}\NormalTok{, point.y);}
\end{Highlighting}
\end{Shaded}

This will print \texttt{y:\ Cell\ \{\ value:\ 7\ \}}. We've successfully
updated \texttt{y}.

\hypertarget{sec--structs}{\section{Structs}\label{sec--structs}}

\texttt{struct}s are a way of creating more complex data types. For
example, if we were doing calculations involving coordinates in 2D
space, we would need both an \texttt{x} and a \texttt{y} value:

\begin{Shaded}
\begin{Highlighting}[]
\KeywordTok{let} \NormalTok{origin_x = }\DecValTok{0}\NormalTok{;}
\KeywordTok{let} \NormalTok{origin_y = }\DecValTok{0}\NormalTok{;}
\end{Highlighting}
\end{Shaded}

A \texttt{struct} lets us combine these two into a single, unified
datatype with \texttt{x} and \texttt{y} as field labels:

\begin{Shaded}
\begin{Highlighting}[]
\KeywordTok{struct} \NormalTok{Point \{}
    \NormalTok{x: }\DataTypeTok{i32}\NormalTok{,}
    \NormalTok{y: }\DataTypeTok{i32}\NormalTok{,}
\NormalTok{\}}

\KeywordTok{fn} \NormalTok{main() \{}
    \KeywordTok{let} \NormalTok{origin = Point \{ x: }\DecValTok{0}\NormalTok{, y: }\DecValTok{0} \NormalTok{\}; }\CommentTok{// origin: Point}

    \PreprocessorTok{println!}\NormalTok{(}\StringTok{"The origin is at (\{\}, \{\})"}\NormalTok{, origin.x, origin.y);}
\NormalTok{\}}
\end{Highlighting}
\end{Shaded}

There's a lot going on here, so let's break it down. We declare a
\texttt{struct} with the \texttt{struct} keyword, and then with a name.
By convention, \texttt{struct}s begin with a capital letter and are
camel cased: \texttt{PointInSpace}, not \texttt{Point\_In\_Space}.

We can create an instance of our \texttt{struct} via \texttt{let}, as
usual, but we use a \texttt{key:\ value} style syntax to set each field.
The order doesn't need to be the same as in the original declaration.

Finally, because fields have names, we can access them through dot
notation: \texttt{origin.x}.

The values in \texttt{struct}s are immutable by default, like other
bindings in Rust. Use \texttt{mut} to make them mutable:

\begin{Shaded}
\begin{Highlighting}[]
\KeywordTok{struct} \NormalTok{Point \{}
    \NormalTok{x: }\DataTypeTok{i32}\NormalTok{,}
    \NormalTok{y: }\DataTypeTok{i32}\NormalTok{,}
\NormalTok{\}}

\KeywordTok{fn} \NormalTok{main() \{}
    \KeywordTok{let} \KeywordTok{mut} \NormalTok{point = Point \{ x: }\DecValTok{0}\NormalTok{, y: }\DecValTok{0} \NormalTok{\};}

    \NormalTok{point.x = }\DecValTok{5}\NormalTok{;}

    \PreprocessorTok{println!}\NormalTok{(}\StringTok{"The point is at (\{\}, \{\})"}\NormalTok{, point.x, point.y);}
\NormalTok{\}}
\end{Highlighting}
\end{Shaded}

This will print \texttt{The\ point\ is\ at\ (5,\ 0)}.

Rust does not support field mutability at the language level, so you
cannot write something like this:

\begin{Shaded}
\begin{Highlighting}[]
\KeywordTok{struct} \NormalTok{Point \{}
    \KeywordTok{mut} \NormalTok{x: }\DataTypeTok{i32}\NormalTok{,}
    \NormalTok{y: }\DataTypeTok{i32}\NormalTok{,}
\NormalTok{\}}
\end{Highlighting}
\end{Shaded}

Mutability is a property of the binding, not of the structure itself. If
you're used to field-level mutability, this may seem strange at first,
but it significantly simplifies things. It even lets you make things
mutable on a temporary basis:

\begin{Shaded}
\begin{Highlighting}[]
\KeywordTok{struct} \NormalTok{Point \{}
    \NormalTok{x: }\DataTypeTok{i32}\NormalTok{,}
    \NormalTok{y: }\DataTypeTok{i32}\NormalTok{,}
\NormalTok{\}}

\KeywordTok{fn} \NormalTok{main() \{}
    \KeywordTok{let} \KeywordTok{mut} \NormalTok{point = Point \{ x: }\DecValTok{0}\NormalTok{, y: }\DecValTok{0} \NormalTok{\};}

    \NormalTok{point.x = }\DecValTok{5}\NormalTok{;}

    \KeywordTok{let} \NormalTok{point = point; }\CommentTok{// now immutable}

    \NormalTok{point.y = }\DecValTok{6}\NormalTok{; }\CommentTok{// this causes an error}
\NormalTok{\}}
\end{Highlighting}
\end{Shaded}

Your structure can still contain \texttt{\&mut} pointers, which will let
you do some kinds of mutation:

\begin{Shaded}
\begin{Highlighting}[]
\KeywordTok{struct} \NormalTok{Point \{}
    \NormalTok{x: }\DataTypeTok{i32}\NormalTok{,}
    \NormalTok{y: }\DataTypeTok{i32}\NormalTok{,}
\NormalTok{\}}

\KeywordTok{struct} \NormalTok{PointRef<}\OtherTok{'a}\NormalTok{> \{}
    \NormalTok{x: &}\OtherTok{'a} \KeywordTok{mut} \DataTypeTok{i32}\NormalTok{,}
    \NormalTok{y: &}\OtherTok{'a} \KeywordTok{mut} \DataTypeTok{i32}\NormalTok{,}
\NormalTok{\}}

\KeywordTok{fn} \NormalTok{main() \{}
    \KeywordTok{let} \KeywordTok{mut} \NormalTok{point = Point \{ x: }\DecValTok{0}\NormalTok{, y: }\DecValTok{0} \NormalTok{\};}

    \NormalTok{\{}
        \KeywordTok{let} \NormalTok{r = PointRef \{ x: &}\KeywordTok{mut} \NormalTok{point.x, y: &}\KeywordTok{mut} \NormalTok{point.y \};}

        \NormalTok{*r.x = }\DecValTok{5}\NormalTok{;}
        \NormalTok{*r.y = }\DecValTok{6}\NormalTok{;}
    \NormalTok{\}}

    \PreprocessorTok{assert_eq!}\NormalTok{(}\DecValTok{5}\NormalTok{, point.x);}
    \PreprocessorTok{assert_eq!}\NormalTok{(}\DecValTok{6}\NormalTok{, point.y);}
\NormalTok{\}}
\end{Highlighting}
\end{Shaded}

\subsection{Update syntax}\label{update-syntax}

A \texttt{struct} can include \texttt{..} to indicate that you want to
use a copy of some other \texttt{struct} for some of the values. For
example:

\begin{Shaded}
\begin{Highlighting}[]
\KeywordTok{struct} \NormalTok{Point3d \{}
    \NormalTok{x: }\DataTypeTok{i32}\NormalTok{,}
    \NormalTok{y: }\DataTypeTok{i32}\NormalTok{,}
    \NormalTok{z: }\DataTypeTok{i32}\NormalTok{,}
\NormalTok{\}}

\KeywordTok{let} \KeywordTok{mut} \NormalTok{point = Point3d \{ x: }\DecValTok{0}\NormalTok{, y: }\DecValTok{0}\NormalTok{, z: }\DecValTok{0} \NormalTok{\};}
\NormalTok{point = Point3d \{ y: }\DecValTok{1}\NormalTok{, .. point \};}
\end{Highlighting}
\end{Shaded}

This gives \texttt{point} a new \texttt{y}, but keeps the old \texttt{x}
and \texttt{z} values. It doesn't have to be the same \texttt{struct}
either, you can use this syntax when making new ones, and it will copy
the values you don't specify:

\begin{Shaded}
\begin{Highlighting}[]
\KeywordTok{let} \NormalTok{origin = Point3d \{ x: }\DecValTok{0}\NormalTok{, y: }\DecValTok{0}\NormalTok{, z: }\DecValTok{0} \NormalTok{\};}
\KeywordTok{let} \NormalTok{point = Point3d \{ z: }\DecValTok{1}\NormalTok{, x: }\DecValTok{2}\NormalTok{, .. origin \};}
\end{Highlighting}
\end{Shaded}

\subsection{Tuple structs}\label{tuple-structs}

Rust has another data type that's like a hybrid between a
\protect\hyperlink{tuples}{tuple} and a \texttt{struct}, called a `tuple
struct'. Tuple structs have a name, but their fields don't. They are
declared with the \texttt{struct} keyword, and then with a name followed
by a tuple:

\begin{Shaded}
\begin{Highlighting}[]
\KeywordTok{struct} \NormalTok{Color(}\DataTypeTok{i32}\NormalTok{, }\DataTypeTok{i32}\NormalTok{, }\DataTypeTok{i32}\NormalTok{);}
\KeywordTok{struct} \NormalTok{Point(}\DataTypeTok{i32}\NormalTok{, }\DataTypeTok{i32}\NormalTok{, }\DataTypeTok{i32}\NormalTok{);}

\KeywordTok{let} \NormalTok{black = Color(}\DecValTok{0}\NormalTok{, }\DecValTok{0}\NormalTok{, }\DecValTok{0}\NormalTok{);}
\KeywordTok{let} \NormalTok{origin = Point(}\DecValTok{0}\NormalTok{, }\DecValTok{0}\NormalTok{, }\DecValTok{0}\NormalTok{);}
\end{Highlighting}
\end{Shaded}

Here, \texttt{black} and \texttt{origin} are not equal, even though they
contain the same values.

It is almost always better to use a \texttt{struct} than a tuple struct.
We would write \texttt{Color} and \texttt{Point} like this instead:

\begin{Shaded}
\begin{Highlighting}[]
\KeywordTok{struct} \NormalTok{Color \{}
    \NormalTok{red: }\DataTypeTok{i32}\NormalTok{,}
    \NormalTok{blue: }\DataTypeTok{i32}\NormalTok{,}
    \NormalTok{green: }\DataTypeTok{i32}\NormalTok{,}
\NormalTok{\}}

\KeywordTok{struct} \NormalTok{Point \{}
    \NormalTok{x: }\DataTypeTok{i32}\NormalTok{,}
    \NormalTok{y: }\DataTypeTok{i32}\NormalTok{,}
    \NormalTok{z: }\DataTypeTok{i32}\NormalTok{,}
\NormalTok{\}}
\end{Highlighting}
\end{Shaded}

Good names are important, and while values in a tuple struct can be
referenced with dot notation as well, a \texttt{struct} gives us actual
names, rather than positions.

There \emph{is} one case when a tuple struct is very useful, though, and
that is when it has only one element. We call this the `newtype'
pattern, because it allows you to create a new type that is distinct
from its contained value and also expresses its own semantic meaning:

\begin{Shaded}
\begin{Highlighting}[]
\KeywordTok{struct} \NormalTok{Inches(}\DataTypeTok{i32}\NormalTok{);}

\KeywordTok{let} \NormalTok{length = Inches(}\DecValTok{10}\NormalTok{);}

\KeywordTok{let} \NormalTok{Inches(integer_length) = length;}
\PreprocessorTok{println!}\NormalTok{(}\StringTok{"length is \{\} inches"}\NormalTok{, integer_length);}
\end{Highlighting}
\end{Shaded}

As you can see here, you can extract the inner integer type through a
destructuring \texttt{let}, as with regular tuples. In this case, the
\texttt{let\ Inches(integer\_length)} assigns \texttt{10} to
\texttt{integer\_length}.

\subsection{Unit-like structs}\label{unit-like-structs}

You can define a \texttt{struct} with no members at all:

\begin{Shaded}
\begin{Highlighting}[]
\KeywordTok{struct} \NormalTok{Electron;}

\KeywordTok{let} \NormalTok{x = Electron;}
\end{Highlighting}
\end{Shaded}

Such a \texttt{struct} is called `unit-like' because it resembles the
empty tuple, \texttt{()}, sometimes called `unit'. Like a tuple struct,
it defines a new type.

This is rarely useful on its own (although sometimes it can serve as a
marker type), but in combination with other features, it can become
useful. For instance, a library may ask you to create a structure that
implements a certain \protect\hyperlink{sec--traits}{trait} to handle
events. If you don't have any data you need to store in the structure,
you can create a unit-like \texttt{struct}.

\hypertarget{sec--enums}{\section{Enums}\label{sec--enums}}

An \texttt{enum} in Rust is a type that represents data that is one of
several possible variants. Each variant in the \texttt{enum} can
optionally have data associated with it:

\begin{Shaded}
\begin{Highlighting}[]
\KeywordTok{enum} \NormalTok{Message \{}
    \NormalTok{Quit,}
    \NormalTok{ChangeColor(}\DataTypeTok{i32}\NormalTok{, }\DataTypeTok{i32}\NormalTok{, }\DataTypeTok{i32}\NormalTok{),}
    \NormalTok{Move \{ x: }\DataTypeTok{i32}\NormalTok{, y: }\DataTypeTok{i32} \NormalTok{\},}
    \NormalTok{Write(}\DataTypeTok{String}\NormalTok{),}
\NormalTok{\}}
\end{Highlighting}
\end{Shaded}

The syntax for defining variants resembles the syntaxes used to define
structs: you can have variants with no data (like unit-like structs),
variants with named data, and variants with unnamed data (like tuple
structs). Unlike separate struct definitions, however, an \texttt{enum}
is a single type. A value of the enum can match any of the variants. For
this reason, an enum is sometimes called a `sum type': the set of
possible values of the enum is the sum of the sets of possible values
for each variant.

We use the \texttt{::} syntax to use the name of each variant: they're
scoped by the name of the \texttt{enum} itself. This allows both of
these to work:

\begin{Shaded}
\begin{Highlighting}[]
\KeywordTok{let} \NormalTok{x: Message = Message::Move \{ x: }\DecValTok{3}\NormalTok{, y: }\DecValTok{4} \NormalTok{\};}

\KeywordTok{enum} \NormalTok{BoardGameTurn \{}
    \NormalTok{Move \{ squares: }\DataTypeTok{i32} \NormalTok{\},}
    \NormalTok{Pass,}
\NormalTok{\}}

\KeywordTok{let} \NormalTok{y: BoardGameTurn = BoardGameTurn::Move \{ squares: }\DecValTok{1} \NormalTok{\};}
\end{Highlighting}
\end{Shaded}

Both variants are named \texttt{Move}, but since they're scoped to the
name of the enum, they can both be used without conflict.

A value of an \texttt{enum} type contains information about which
variant it is, in addition to any data associated with that variant.
This is sometimes referred to as a `tagged union', since the data
includes a `tag' indicating what type it is. The compiler uses this
information to enforce that you're accessing the data in the enum
safely. For instance, you can't simply try to destructure a value as if
it were one of the possible variants:

\begin{Shaded}
\begin{Highlighting}[]
\KeywordTok{fn} \NormalTok{process_color_change(msg: Message) \{}
    \KeywordTok{let} \NormalTok{Message::ChangeColor(r, g, b) = msg; }\CommentTok{// compile-time error}
\NormalTok{\}}
\end{Highlighting}
\end{Shaded}

Not supporting these operations may seem rather limiting, but it's a
limitation which we can overcome. There are two ways: by implementing
equality ourselves, or by pattern matching variants with
\protect\hyperlink{sec--match}{\texttt{match}} expressions, which you'll
learn in the next section. We don't know enough about Rust to implement
equality yet, but we'll find out in the
\protect\hyperlink{sec--traits}{\texttt{traits}} section.

\subsection{Constructors as functions}\label{constructors-as-functions}

An \texttt{enum} constructor can also be used like a function. For
example:

\begin{Shaded}
\begin{Highlighting}[]
\KeywordTok{let} \NormalTok{m = Message::Write(}\StringTok{"Hello, world"}\NormalTok{.to_string());}
\end{Highlighting}
\end{Shaded}

is the same as

\begin{Shaded}
\begin{Highlighting}[]
\KeywordTok{fn} \NormalTok{foo(x: }\DataTypeTok{String}\NormalTok{) -> Message \{}
    \NormalTok{Message::Write(x)}
\NormalTok{\}}

\KeywordTok{let} \NormalTok{x = foo(}\StringTok{"Hello, world"}\NormalTok{.to_string());}
\end{Highlighting}
\end{Shaded}

This is not immediately useful to us, but when we get to
\protect\hyperlink{sec--closures}{\texttt{closures}}, we'll talk about
passing functions as arguments to other functions. For example, with
\protect\hyperlink{sec--iterators}{\texttt{iterators}}, we can do this
to convert a vector of \texttt{String}s into a vector of
\texttt{Message::Write}s:

\begin{Shaded}
\begin{Highlighting}[]

\KeywordTok{let} \NormalTok{v = }\PreprocessorTok{vec!}\NormalTok{[}\StringTok{"Hello"}\NormalTok{.to_string(), }\StringTok{"World"}\NormalTok{.to_string()];}

\KeywordTok{let} \NormalTok{v1: }\DataTypeTok{Vec}\NormalTok{<Message> = v.into_iter().map(Message::Write).collect();}
\end{Highlighting}
\end{Shaded}

\hypertarget{sec--match}{\section{Match}\label{sec--match}}

Often, a simple \protect\hyperlink{sec--if}{\texttt{if}}/\texttt{else}
isn't enough, because you have more than two possible options. Also,
conditions can get quite complex. Rust has a keyword, \texttt{match},
that allows you to replace complicated \texttt{if}/\texttt{else}
groupings with something more powerful. Check it out:

\begin{Shaded}
\begin{Highlighting}[]
\KeywordTok{let} \NormalTok{x = }\DecValTok{5}\NormalTok{;}

\KeywordTok{match} \NormalTok{x \{}
    \DecValTok{1} \NormalTok{=> }\PreprocessorTok{println!}\NormalTok{(}\StringTok{"one"}\NormalTok{),}
    \DecValTok{2} \NormalTok{=> }\PreprocessorTok{println!}\NormalTok{(}\StringTok{"two"}\NormalTok{),}
    \DecValTok{3} \NormalTok{=> }\PreprocessorTok{println!}\NormalTok{(}\StringTok{"three"}\NormalTok{),}
    \DecValTok{4} \NormalTok{=> }\PreprocessorTok{println!}\NormalTok{(}\StringTok{"four"}\NormalTok{),}
    \DecValTok{5} \NormalTok{=> }\PreprocessorTok{println!}\NormalTok{(}\StringTok{"five"}\NormalTok{),}
    \NormalTok{_ => }\PreprocessorTok{println!}\NormalTok{(}\StringTok{"something else"}\NormalTok{),}
\NormalTok{\}}
\end{Highlighting}
\end{Shaded}

\texttt{match} takes an expression and then branches based on its value.
Each `arm' of the branch is of the form
\texttt{val\ =\textgreater{}\ expression}. When the value matches, that
arm's expression will be evaluated. It's called \texttt{match} because
of the term `pattern matching', which \texttt{match} is an
implementation of. There's a \protect\hyperlink{sec--patterns}{separate
section on patterns} that covers all the patterns that are possible
here.

One of the many advantages of \texttt{match} is it enforces
`exhaustiveness checking'. For example if we remove the last arm with
the underscore \texttt{\_}, the compiler will give us an error:

\begin{verbatim}
error: non-exhaustive patterns: `_` not covered
\end{verbatim}

Rust is telling us that we forgot some value. The compiler infers from
\texttt{x} that it can have any 32bit integer value; for example
-2,147,483,648 to 2,147,483,647. The \texttt{\_} acts as a `catch-all',
and will catch all possible values that \emph{aren't} specified in an
arm of \texttt{match}. As you can see in the previous example, we
provide \texttt{match} arms for integers 1-5, if \texttt{x} is 6 or any
other value, then it is caught by \texttt{\_}.

\texttt{match} is also an expression, which means we can use it on the
right-hand side of a \texttt{let} binding or directly where an
expression is used:

\begin{Shaded}
\begin{Highlighting}[]
\KeywordTok{let} \NormalTok{x = }\DecValTok{5}\NormalTok{;}

\KeywordTok{let} \NormalTok{number = }\KeywordTok{match} \NormalTok{x \{}
    \DecValTok{1} \NormalTok{=> }\StringTok{"one"}\NormalTok{,}
    \DecValTok{2} \NormalTok{=> }\StringTok{"two"}\NormalTok{,}
    \DecValTok{3} \NormalTok{=> }\StringTok{"three"}\NormalTok{,}
    \DecValTok{4} \NormalTok{=> }\StringTok{"four"}\NormalTok{,}
    \DecValTok{5} \NormalTok{=> }\StringTok{"five"}\NormalTok{,}
    \NormalTok{_ => }\StringTok{"something else"}\NormalTok{,}
\NormalTok{\};}
\end{Highlighting}
\end{Shaded}

Sometimes it's a nice way of converting something from one type to
another; in this example the integers are converted to \texttt{String}.

\subsection{Matching on enums}\label{matching-on-enums}

Another important use of the \texttt{match} keyword is to process the
possible variants of an enum:

\begin{Shaded}
\begin{Highlighting}[]
\KeywordTok{enum} \NormalTok{Message \{}
    \NormalTok{Quit,}
    \NormalTok{ChangeColor(}\DataTypeTok{i32}\NormalTok{, }\DataTypeTok{i32}\NormalTok{, }\DataTypeTok{i32}\NormalTok{),}
    \NormalTok{Move \{ x: }\DataTypeTok{i32}\NormalTok{, y: }\DataTypeTok{i32} \NormalTok{\},}
    \NormalTok{Write(}\DataTypeTok{String}\NormalTok{),}
\NormalTok{\}}

\KeywordTok{fn} \NormalTok{quit() \{ }\CommentTok{/* ... */} \NormalTok{\}}
\KeywordTok{fn} \NormalTok{change_color(r: }\DataTypeTok{i32}\NormalTok{, g: }\DataTypeTok{i32}\NormalTok{, b: }\DataTypeTok{i32}\NormalTok{) \{ }\CommentTok{/* ... */} \NormalTok{\}}
\KeywordTok{fn} \NormalTok{move_cursor(x: }\DataTypeTok{i32}\NormalTok{, y: }\DataTypeTok{i32}\NormalTok{) \{ }\CommentTok{/* ... */} \NormalTok{\}}

\KeywordTok{fn} \NormalTok{process_message(msg: Message) \{}
    \KeywordTok{match} \NormalTok{msg \{}
        \NormalTok{Message::Quit => quit(),}
        \NormalTok{Message::ChangeColor(r, g, b) => change_color(r, g, b),}
        \NormalTok{Message::Move \{ x: x, y: y \} => move_cursor(x, y),}
        \NormalTok{Message::Write(s) => }\PreprocessorTok{println!}\NormalTok{(}\StringTok{"\{\}"}\NormalTok{, s),}
    \NormalTok{\};}
\NormalTok{\}}
\end{Highlighting}
\end{Shaded}

Again, the Rust compiler checks exhaustiveness, so it demands that you
have a match arm for every variant of the enum. If you leave one off, it
will give you a compile-time error unless you use \texttt{\_} or provide
all possible arms.

Unlike the previous uses of \texttt{match}, you can't use the normal
\texttt{if} statement to do this. You can use the
\protect\hyperlink{sec--if-let}{\texttt{if\ let}} statement, which can
be seen as an abbreviated form of \texttt{match}.

\hypertarget{sec--patterns}{\section{Patterns}\label{sec--patterns}}

Patterns are quite common in Rust. We use them in
\protect\hyperlink{sec--variable-bindings}{variable bindings},
\protect\hyperlink{sec--match}{match expressions}, and other places,
too. Let's go on a whirlwind tour of all of the things patterns can do!

A quick refresher: you can match against literals directly, and
\texttt{\_} acts as an `any' case:

\begin{Shaded}
\begin{Highlighting}[]
\KeywordTok{let} \NormalTok{x = }\DecValTok{1}\NormalTok{;}

\KeywordTok{match} \NormalTok{x \{}
    \DecValTok{1} \NormalTok{=> }\PreprocessorTok{println!}\NormalTok{(}\StringTok{"one"}\NormalTok{),}
    \DecValTok{2} \NormalTok{=> }\PreprocessorTok{println!}\NormalTok{(}\StringTok{"two"}\NormalTok{),}
    \DecValTok{3} \NormalTok{=> }\PreprocessorTok{println!}\NormalTok{(}\StringTok{"three"}\NormalTok{),}
    \NormalTok{_ => }\PreprocessorTok{println!}\NormalTok{(}\StringTok{"anything"}\NormalTok{),}
\NormalTok{\}}
\end{Highlighting}
\end{Shaded}

This prints \texttt{one}.

There's one pitfall with patterns: like anything that introduces a new
binding, they introduce shadowing. For example:

\begin{Shaded}
\begin{Highlighting}[]
\KeywordTok{let} \NormalTok{x = }\DecValTok{1}\NormalTok{;}
\KeywordTok{let} \NormalTok{c = }\CharTok{'c'}\NormalTok{;}

\KeywordTok{match} \NormalTok{c \{}
    \NormalTok{x => }\PreprocessorTok{println!}\NormalTok{(}\StringTok{"x: \{\} c: \{\}"}\NormalTok{, x, c),}
\NormalTok{\}}

\PreprocessorTok{println!}\NormalTok{(}\StringTok{"x: \{\}"}\NormalTok{, x)}
\end{Highlighting}
\end{Shaded}

This prints:

\begin{verbatim}
x: c c: c
x: 1
\end{verbatim}

In other words, \texttt{x\ =\textgreater{}} matches the pattern and
introduces a new binding named \texttt{x}. This new binding is in scope
for the match arm and takes on the value of \texttt{c}. Notice that the
value of \texttt{x} outside the scope of the match has no bearing on the
value of \texttt{x} within it. Because we already have a binding named
\texttt{x}, this new \texttt{x} shadows it.

\subsection{Multiple patterns}\label{multiple-patterns}

You can match multiple patterns with \texttt{\textbar{}}:

\begin{Shaded}
\begin{Highlighting}[]
\KeywordTok{let} \NormalTok{x = }\DecValTok{1}\NormalTok{;}

\KeywordTok{match} \NormalTok{x \{}
    \DecValTok{1} \NormalTok{| }\DecValTok{2} \NormalTok{=> }\PreprocessorTok{println!}\NormalTok{(}\StringTok{"one or two"}\NormalTok{),}
    \DecValTok{3} \NormalTok{=> }\PreprocessorTok{println!}\NormalTok{(}\StringTok{"three"}\NormalTok{),}
    \NormalTok{_ => }\PreprocessorTok{println!}\NormalTok{(}\StringTok{"anything"}\NormalTok{),}
\NormalTok{\}}
\end{Highlighting}
\end{Shaded}

This prints \texttt{one\ or\ two}.

\subsection{Destructuring}\label{destructuring}

If you have a compound data type, like a
\protect\hyperlink{sec--structs}{\texttt{struct}}, you can destructure
it inside of a pattern:

\begin{Shaded}
\begin{Highlighting}[]
\KeywordTok{struct} \NormalTok{Point \{}
    \NormalTok{x: }\DataTypeTok{i32}\NormalTok{,}
    \NormalTok{y: }\DataTypeTok{i32}\NormalTok{,}
\NormalTok{\}}

\KeywordTok{let} \NormalTok{origin = Point \{ x: }\DecValTok{0}\NormalTok{, y: }\DecValTok{0} \NormalTok{\};}

\KeywordTok{match} \NormalTok{origin \{}
    \NormalTok{Point \{ x, y \} => }\PreprocessorTok{println!}\NormalTok{(}\StringTok{"(\{\},\{\})"}\NormalTok{, x, y),}
\NormalTok{\}}
\end{Highlighting}
\end{Shaded}

We can use \texttt{:} to give a value a different name.

\begin{Shaded}
\begin{Highlighting}[]
\KeywordTok{struct} \NormalTok{Point \{}
    \NormalTok{x: }\DataTypeTok{i32}\NormalTok{,}
    \NormalTok{y: }\DataTypeTok{i32}\NormalTok{,}
\NormalTok{\}}

\KeywordTok{let} \NormalTok{origin = Point \{ x: }\DecValTok{0}\NormalTok{, y: }\DecValTok{0} \NormalTok{\};}

\KeywordTok{match} \NormalTok{origin \{}
    \NormalTok{Point \{ x: x1, y: y1 \} => }\PreprocessorTok{println!}\NormalTok{(}\StringTok{"(\{\},\{\})"}\NormalTok{, x1, y1),}
\NormalTok{\}}
\end{Highlighting}
\end{Shaded}

If we only care about some of the values, we don't have to give them all
names:

\begin{Shaded}
\begin{Highlighting}[]
\KeywordTok{struct} \NormalTok{Point \{}
    \NormalTok{x: }\DataTypeTok{i32}\NormalTok{,}
    \NormalTok{y: }\DataTypeTok{i32}\NormalTok{,}
\NormalTok{\}}

\KeywordTok{let} \NormalTok{origin = Point \{ x: }\DecValTok{0}\NormalTok{, y: }\DecValTok{0} \NormalTok{\};}

\KeywordTok{match} \NormalTok{origin \{}
    \NormalTok{Point \{ x, .. \} => }\PreprocessorTok{println!}\NormalTok{(}\StringTok{"x is \{\}"}\NormalTok{, x),}
\NormalTok{\}}
\end{Highlighting}
\end{Shaded}

This prints \texttt{x\ is\ 0}.

You can do this kind of match on any member, not only the first:

\begin{Shaded}
\begin{Highlighting}[]
\KeywordTok{struct} \NormalTok{Point \{}
    \NormalTok{x: }\DataTypeTok{i32}\NormalTok{,}
    \NormalTok{y: }\DataTypeTok{i32}\NormalTok{,}
\NormalTok{\}}

\KeywordTok{let} \NormalTok{origin = Point \{ x: }\DecValTok{0}\NormalTok{, y: }\DecValTok{0} \NormalTok{\};}

\KeywordTok{match} \NormalTok{origin \{}
    \NormalTok{Point \{ y, .. \} => }\PreprocessorTok{println!}\NormalTok{(}\StringTok{"y is \{\}"}\NormalTok{, y),}
\NormalTok{\}}
\end{Highlighting}
\end{Shaded}

This prints \texttt{y\ is\ 0}.

This `destructuring' behavior works on any compound data type, like
\protect\hyperlink{tuples}{tuples} or
\protect\hyperlink{sec--enums}{enums}.

\subsection{Ignoring bindings}\label{ignoring-bindings}

You can use \texttt{\_} in a pattern to disregard the type and value.
For example, here's a \texttt{match} against a
\texttt{Result\textless{}T,\ E\textgreater{}}:

\begin{Shaded}
\begin{Highlighting}[]
\KeywordTok{match} \NormalTok{some_value \{}
    \ConstantTok{Ok}\NormalTok{(value) => }\PreprocessorTok{println!}\NormalTok{(}\StringTok{"got a value: \{\}"}\NormalTok{, value),}
    \ConstantTok{Err}\NormalTok{(_) => }\PreprocessorTok{println!}\NormalTok{(}\StringTok{"an error occurred"}\NormalTok{),}
\NormalTok{\}}
\end{Highlighting}
\end{Shaded}

In the first arm, we bind the value inside the \texttt{Ok} variant to
\texttt{value}. But in the \texttt{Err} arm, we use \texttt{\_} to
disregard the specific error, and print a general error message.

\texttt{\_} is valid in any pattern that creates a binding. This can be
useful to ignore parts of a larger structure:

\begin{Shaded}
\begin{Highlighting}[]
\KeywordTok{fn} \NormalTok{coordinate() -> (}\DataTypeTok{i32}\NormalTok{, }\DataTypeTok{i32}\NormalTok{, }\DataTypeTok{i32}\NormalTok{) \{}
    \CommentTok{// generate and return some sort of triple tuple}
\NormalTok{\}}

\KeywordTok{let} \NormalTok{(x, _, z) = coordinate();}
\end{Highlighting}
\end{Shaded}

Here, we bind the first and last element of the tuple to \texttt{x} and
\texttt{z}, but ignore the middle element.

It's worth noting that using \texttt{\_} never binds the value in the
first place, which means a value may not move:

\begin{Shaded}
\begin{Highlighting}[]
\KeywordTok{let} \NormalTok{tuple: (}\DataTypeTok{u32}\NormalTok{, }\DataTypeTok{String}\NormalTok{) = (}\DecValTok{5}\NormalTok{, }\DataTypeTok{String}\NormalTok{::from(}\StringTok{"five"}\NormalTok{));}

\CommentTok{// Here, tuple is moved, because the String moved:}
\KeywordTok{let} \NormalTok{(x, _s) = tuple;}

\CommentTok{// The next line would give "error: use of partially moved value: `tuple`"}
\CommentTok{// println!("Tuple is: \{:?\}", tuple);}

\CommentTok{// However,}

\KeywordTok{let} \NormalTok{tuple = (}\DecValTok{5}\NormalTok{, }\DataTypeTok{String}\NormalTok{::from(}\StringTok{"five"}\NormalTok{));}

\CommentTok{// Here, tuple is _not_ moved, as the String was never moved, and u32 is Copy:}
\KeywordTok{let} \NormalTok{(x, _) = tuple;}

\CommentTok{// That means this works:}
\PreprocessorTok{println!}\NormalTok{(}\StringTok{"Tuple is: \{:?\}"}\NormalTok{, tuple);}
\end{Highlighting}
\end{Shaded}

This also means that any temporary variables will be dropped at the end
of the statement:

\begin{Shaded}
\begin{Highlighting}[]
\CommentTok{// Here, the String created will be dropped immediately, as it’s not bound:}

\KeywordTok{let} \NormalTok{_ = }\DataTypeTok{String}\NormalTok{::from(}\StringTok{"  hello  "}\NormalTok{).trim();}
\end{Highlighting}
\end{Shaded}

You can also use \texttt{..} in a pattern to disregard multiple values:

\begin{Shaded}
\begin{Highlighting}[]
\KeywordTok{enum} \NormalTok{OptionalTuple \{}
    \NormalTok{Value(}\DataTypeTok{i32}\NormalTok{, }\DataTypeTok{i32}\NormalTok{, }\DataTypeTok{i32}\NormalTok{),}
    \NormalTok{Missing,}
\NormalTok{\}}

\KeywordTok{let} \NormalTok{x = OptionalTuple::Value(}\DecValTok{5}\NormalTok{, -}\DecValTok{2}\NormalTok{, }\DecValTok{3}\NormalTok{);}

\KeywordTok{match} \NormalTok{x \{}
    \NormalTok{OptionalTuple::Value(..) => }\PreprocessorTok{println!}\NormalTok{(}\StringTok{"Got a tuple!"}\NormalTok{),}
    \NormalTok{OptionalTuple::Missing => }\PreprocessorTok{println!}\NormalTok{(}\StringTok{"No such luck."}\NormalTok{),}
\NormalTok{\}}
\end{Highlighting}
\end{Shaded}

This prints \texttt{Got\ a\ tuple!}.

\subsection{ref and ref mut}\label{ref-and-ref-mut}

If you want to get a
\protect\hyperlink{sec--references-and-borrowing}{reference}, use the
\texttt{ref} keyword:

\begin{Shaded}
\begin{Highlighting}[]
\KeywordTok{let} \NormalTok{x = }\DecValTok{5}\NormalTok{;}

\KeywordTok{match} \NormalTok{x \{}
    \KeywordTok{ref} \NormalTok{r => }\PreprocessorTok{println!}\NormalTok{(}\StringTok{"Got a reference to \{\}"}\NormalTok{, r),}
\NormalTok{\}}
\end{Highlighting}
\end{Shaded}

This prints \texttt{Got\ a\ reference\ to\ 5}.

Here, the \texttt{r} inside the \texttt{match} has the type
\texttt{\&i32}. In other words, the \texttt{ref} keyword \emph{creates}
a reference, for use in the pattern. If you need a mutable reference,
\texttt{ref\ mut} will work in the same way:

\begin{Shaded}
\begin{Highlighting}[]
\KeywordTok{let} \KeywordTok{mut} \NormalTok{x = }\DecValTok{5}\NormalTok{;}

\KeywordTok{match} \NormalTok{x \{}
    \KeywordTok{ref} \KeywordTok{mut} \NormalTok{mr => }\PreprocessorTok{println!}\NormalTok{(}\StringTok{"Got a mutable reference to \{\}"}\NormalTok{, mr),}
\NormalTok{\}}
\end{Highlighting}
\end{Shaded}

\subsection{Ranges}\label{ranges}

You can match a range of values with \texttt{...}:

\begin{Shaded}
\begin{Highlighting}[]
\KeywordTok{let} \NormalTok{x = }\DecValTok{1}\NormalTok{;}

\KeywordTok{match} \NormalTok{x \{}
    \DecValTok{1} \NormalTok{... }\DecValTok{5} \NormalTok{=> }\PreprocessorTok{println!}\NormalTok{(}\StringTok{"one through five"}\NormalTok{),}
    \NormalTok{_ => }\PreprocessorTok{println!}\NormalTok{(}\StringTok{"anything"}\NormalTok{),}
\NormalTok{\}}
\end{Highlighting}
\end{Shaded}

This prints \texttt{one\ through\ five}.

Ranges are mostly used with integers and \texttt{char}s:

\begin{Shaded}
\begin{Highlighting}[]
\KeywordTok{let} \NormalTok{x = }\CharTok{'💅'}\NormalTok{;}

\KeywordTok{match} \NormalTok{x \{}
    \CharTok{'a'} \NormalTok{... }\CharTok{'j'} \NormalTok{=> }\PreprocessorTok{println!}\NormalTok{(}\StringTok{"early letter"}\NormalTok{),}
    \CharTok{'k'} \NormalTok{... }\CharTok{'z'} \NormalTok{=> }\PreprocessorTok{println!}\NormalTok{(}\StringTok{"late letter"}\NormalTok{),}
    \NormalTok{_ => }\PreprocessorTok{println!}\NormalTok{(}\StringTok{"something else"}\NormalTok{),}
\NormalTok{\}}
\end{Highlighting}
\end{Shaded}

This prints \texttt{something\ else}.

\subsection{Bindings}\label{bindings}

You can bind values to names with \texttt{@}:

\begin{Shaded}
\begin{Highlighting}[]
\KeywordTok{let} \NormalTok{x = }\DecValTok{1}\NormalTok{;}

\KeywordTok{match} \NormalTok{x \{}
    \NormalTok{e @ }\DecValTok{1} \NormalTok{... }\DecValTok{5} \NormalTok{=> }\PreprocessorTok{println!}\NormalTok{(}\StringTok{"got a range element \{\}"}\NormalTok{, e),}
    \NormalTok{_ => }\PreprocessorTok{println!}\NormalTok{(}\StringTok{"anything"}\NormalTok{),}
\NormalTok{\}}
\end{Highlighting}
\end{Shaded}

This prints \texttt{got\ a\ range\ element\ 1}. This is useful when you
want to do a complicated match of part of a data structure:

\begin{Shaded}
\begin{Highlighting}[]
\AttributeTok{#[}\NormalTok{derive}\AttributeTok{(}\BuiltInTok{Debug}\AttributeTok{)]}
\KeywordTok{struct} \NormalTok{Person \{}
    \NormalTok{name: }\DataTypeTok{Option}\NormalTok{<}\DataTypeTok{String}\NormalTok{>,}
\NormalTok{\}}

\KeywordTok{let} \NormalTok{name = }\StringTok{"Steve"}\NormalTok{.to_string();}
\KeywordTok{let} \NormalTok{x: }\DataTypeTok{Option}\NormalTok{<Person> = }\ConstantTok{Some}\NormalTok{(Person \{ name: }\ConstantTok{Some}\NormalTok{(name) \});}
\KeywordTok{match} \NormalTok{x \{}
    \ConstantTok{Some}\NormalTok{(Person \{ name: }\KeywordTok{ref} \NormalTok{a @ }\ConstantTok{Some}\NormalTok{(_), .. \}) => }\PreprocessorTok{println!}\NormalTok{(}\StringTok{"\{:?\}"}\NormalTok{, a),}
    \NormalTok{_ => \{\}}
\NormalTok{\}}
\end{Highlighting}
\end{Shaded}

This prints \texttt{Some("Steve")}: we've bound the inner \texttt{name}
to \texttt{a}.

If you use \texttt{@} with \texttt{\textbar{}}, you need to make sure
the name is bound in each part of the pattern:

\begin{Shaded}
\begin{Highlighting}[]
\KeywordTok{let} \NormalTok{x = }\DecValTok{5}\NormalTok{;}

\KeywordTok{match} \NormalTok{x \{}
    \NormalTok{e @ }\DecValTok{1} \NormalTok{... }\DecValTok{5} \NormalTok{| e @ }\DecValTok{8} \NormalTok{... }\DecValTok{10} \NormalTok{=> }\PreprocessorTok{println!}\NormalTok{(}\StringTok{"got a range element \{\}"}\NormalTok{, e),}
    \NormalTok{_ => }\PreprocessorTok{println!}\NormalTok{(}\StringTok{"anything"}\NormalTok{),}
\NormalTok{\}}
\end{Highlighting}
\end{Shaded}

\subsection{Guards}\label{guards}

You can introduce `match guards' with \texttt{if}:

\begin{Shaded}
\begin{Highlighting}[]
\KeywordTok{enum} \NormalTok{OptionalInt \{}
    \NormalTok{Value(}\DataTypeTok{i32}\NormalTok{),}
    \NormalTok{Missing,}
\NormalTok{\}}

\KeywordTok{let} \NormalTok{x = OptionalInt::Value(}\DecValTok{5}\NormalTok{);}

\KeywordTok{match} \NormalTok{x \{}
    \NormalTok{OptionalInt::Value(i) }\KeywordTok{if} \NormalTok{i > }\DecValTok{5} \NormalTok{=> }\PreprocessorTok{println!}\NormalTok{(}\StringTok{"Got an int bigger than five!"}\NormalTok{),}
    \NormalTok{OptionalInt::Value(..) => }\PreprocessorTok{println!}\NormalTok{(}\StringTok{"Got an int!"}\NormalTok{),}
    \NormalTok{OptionalInt::Missing => }\PreprocessorTok{println!}\NormalTok{(}\StringTok{"No such luck."}\NormalTok{),}
\NormalTok{\}}
\end{Highlighting}
\end{Shaded}

This prints \texttt{Got\ an\ int!}.

If you're using \texttt{if} with multiple patterns, the \texttt{if}
applies to both sides:

\begin{Shaded}
\begin{Highlighting}[]
\KeywordTok{let} \NormalTok{x = }\DecValTok{4}\NormalTok{;}
\KeywordTok{let} \NormalTok{y = }\ConstantTok{false}\NormalTok{;}

\KeywordTok{match} \NormalTok{x \{}
    \DecValTok{4} \NormalTok{| }\DecValTok{5} \KeywordTok{if} \NormalTok{y => }\PreprocessorTok{println!}\NormalTok{(}\StringTok{"yes"}\NormalTok{),}
    \NormalTok{_ => }\PreprocessorTok{println!}\NormalTok{(}\StringTok{"no"}\NormalTok{),}
\NormalTok{\}}
\end{Highlighting}
\end{Shaded}

This prints \texttt{no}, because the \texttt{if} applies to the whole of
\texttt{4\ \textbar{}\ 5}, and not to only the \texttt{5}. In other
words, the precedence of \texttt{if} behaves like this:

\begin{verbatim}
(4 | 5) if y => ...
\end{verbatim}

not this:

\begin{verbatim}
4 | (5 if y) => ...
\end{verbatim}

\subsection{Mix and Match}\label{mix-and-match}

Whew! That's a lot of different ways to match things, and they can all
be mixed and matched, depending on what you're doing:

\begin{Shaded}
\begin{Highlighting}[]
\KeywordTok{match} \NormalTok{x \{}
    \NormalTok{Foo \{ x: }\ConstantTok{Some}\NormalTok{(}\KeywordTok{ref} \NormalTok{name), y: }\ConstantTok{None} \NormalTok{\} => ...}
\NormalTok{\}}
\end{Highlighting}
\end{Shaded}

Patterns are very powerful. Make good use of them.

\hypertarget{sec--method-syntax}{\section{Method
Syntax}\label{sec--method-syntax}}

Functions are great, but if you want to call a bunch of them on some
data, it can be awkward. Consider this code:

\begin{Shaded}
\begin{Highlighting}[]
\NormalTok{baz(bar(foo));}
\end{Highlighting}
\end{Shaded}

We would read this left-to-right, and so we see `baz bar foo'. But this
isn't the order that the functions would get called in, that's
inside-out: `foo bar baz'. Wouldn't it be nice if we could do this
instead?

\begin{Shaded}
\begin{Highlighting}[]
\NormalTok{foo.bar().baz();}
\end{Highlighting}
\end{Shaded}

Luckily, as you may have guessed with the leading question, you can!
Rust provides the ability to use this `method call syntax' via the
\texttt{impl} keyword.

\subsection{Method calls}\label{method-calls}

Here's how it works:

\begin{Shaded}
\begin{Highlighting}[]
\KeywordTok{struct} \NormalTok{Circle \{}
    \NormalTok{x: }\DataTypeTok{f64}\NormalTok{,}
    \NormalTok{y: }\DataTypeTok{f64}\NormalTok{,}
    \NormalTok{radius: }\DataTypeTok{f64}\NormalTok{,}
\NormalTok{\}}

\KeywordTok{impl} \NormalTok{Circle \{}
    \KeywordTok{fn} \NormalTok{area(&}\KeywordTok{self}\NormalTok{) -> }\DataTypeTok{f64} \NormalTok{\{}
        \NormalTok{std::}\DataTypeTok{f64}\NormalTok{::consts::PI * (}\KeywordTok{self}\NormalTok{.radius * }\KeywordTok{self}\NormalTok{.radius)}
    \NormalTok{\}}
\NormalTok{\}}

\KeywordTok{fn} \NormalTok{main() \{}
    \KeywordTok{let} \NormalTok{c = Circle \{ x: }\DecValTok{0.0}\NormalTok{, y: }\DecValTok{0.0}\NormalTok{, radius: }\DecValTok{2.0} \NormalTok{\};}
    \PreprocessorTok{println!}\NormalTok{(}\StringTok{"\{\}"}\NormalTok{, c.area());}
\NormalTok{\}}
\end{Highlighting}
\end{Shaded}

This will print \texttt{12.566371}.

We've made a \texttt{struct} that represents a circle. We then write an
\texttt{impl} block, and inside it, define a method, \texttt{area}.

Methods take a special first parameter, of which there are three
variants: \texttt{self}, \texttt{\&self}, and \texttt{\&mut\ self}. You
can think of this first parameter as being the \texttt{foo} in
\texttt{foo.bar()}. The three variants correspond to the three kinds of
things \texttt{foo} could be: \texttt{self} if it's a value on the
stack, \texttt{\&self} if it's a reference, and \texttt{\&mut\ self} if
it's a mutable reference. Because we took the \texttt{\&self} parameter
to \texttt{area}, we can use it like any other parameter. Because we
know it's a \texttt{Circle}, we can access the \texttt{radius} like we
would with any other \texttt{struct}.

We should default to using \texttt{\&self}, as you should prefer
borrowing over taking ownership, as well as taking immutable references
over mutable ones. Here's an example of all three variants:

\begin{Shaded}
\begin{Highlighting}[]
\KeywordTok{struct} \NormalTok{Circle \{}
    \NormalTok{x: }\DataTypeTok{f64}\NormalTok{,}
    \NormalTok{y: }\DataTypeTok{f64}\NormalTok{,}
    \NormalTok{radius: }\DataTypeTok{f64}\NormalTok{,}
\NormalTok{\}}

\KeywordTok{impl} \NormalTok{Circle \{}
    \KeywordTok{fn} \NormalTok{reference(&}\KeywordTok{self}\NormalTok{) \{}
       \PreprocessorTok{println!}\NormalTok{(}\StringTok{"taking self by reference!"}\NormalTok{);}
    \NormalTok{\}}

    \KeywordTok{fn} \NormalTok{mutable_reference(&}\KeywordTok{mut} \KeywordTok{self}\NormalTok{) \{}
       \PreprocessorTok{println!}\NormalTok{(}\StringTok{"taking self by mutable reference!"}\NormalTok{);}
    \NormalTok{\}}

    \KeywordTok{fn} \NormalTok{takes_ownership(}\KeywordTok{self}\NormalTok{) \{}
       \PreprocessorTok{println!}\NormalTok{(}\StringTok{"taking ownership of self!"}\NormalTok{);}
    \NormalTok{\}}
\NormalTok{\}}
\end{Highlighting}
\end{Shaded}

You can use as many \texttt{impl} blocks as you'd like. The previous
example could have also been written like this:

\begin{Shaded}
\begin{Highlighting}[]
\KeywordTok{struct} \NormalTok{Circle \{}
    \NormalTok{x: }\DataTypeTok{f64}\NormalTok{,}
    \NormalTok{y: }\DataTypeTok{f64}\NormalTok{,}
    \NormalTok{radius: }\DataTypeTok{f64}\NormalTok{,}
\NormalTok{\}}

\KeywordTok{impl} \NormalTok{Circle \{}
    \KeywordTok{fn} \NormalTok{reference(&}\KeywordTok{self}\NormalTok{) \{}
       \PreprocessorTok{println!}\NormalTok{(}\StringTok{"taking self by reference!"}\NormalTok{);}
    \NormalTok{\}}
\NormalTok{\}}

\KeywordTok{impl} \NormalTok{Circle \{}
    \KeywordTok{fn} \NormalTok{mutable_reference(&}\KeywordTok{mut} \KeywordTok{self}\NormalTok{) \{}
       \PreprocessorTok{println!}\NormalTok{(}\StringTok{"taking self by mutable reference!"}\NormalTok{);}
    \NormalTok{\}}
\NormalTok{\}}

\KeywordTok{impl} \NormalTok{Circle \{}
    \KeywordTok{fn} \NormalTok{takes_ownership(}\KeywordTok{self}\NormalTok{) \{}
       \PreprocessorTok{println!}\NormalTok{(}\StringTok{"taking ownership of self!"}\NormalTok{);}
    \NormalTok{\}}
\NormalTok{\}}
\end{Highlighting}
\end{Shaded}

\subsection{Chaining method calls}\label{chaining-method-calls}

So, now we know how to call a method, such as \texttt{foo.bar()}. But
what about our original example, \texttt{foo.bar().baz()}? This is
called `method chaining'. Let's look at an example:

\begin{Shaded}
\begin{Highlighting}[]
\KeywordTok{struct} \NormalTok{Circle \{}
    \NormalTok{x: }\DataTypeTok{f64}\NormalTok{,}
    \NormalTok{y: }\DataTypeTok{f64}\NormalTok{,}
    \NormalTok{radius: }\DataTypeTok{f64}\NormalTok{,}
\NormalTok{\}}

\KeywordTok{impl} \NormalTok{Circle \{}
    \KeywordTok{fn} \NormalTok{area(&}\KeywordTok{self}\NormalTok{) -> }\DataTypeTok{f64} \NormalTok{\{}
        \NormalTok{std::}\DataTypeTok{f64}\NormalTok{::consts::PI * (}\KeywordTok{self}\NormalTok{.radius * }\KeywordTok{self}\NormalTok{.radius)}
    \NormalTok{\}}

    \KeywordTok{fn} \NormalTok{grow(&}\KeywordTok{self}\NormalTok{, increment: }\DataTypeTok{f64}\NormalTok{) -> Circle \{}
        \NormalTok{Circle \{ x: }\KeywordTok{self}\NormalTok{.x, y: }\KeywordTok{self}\NormalTok{.y, radius: }\KeywordTok{self}\NormalTok{.radius + increment \}}
    \NormalTok{\}}
\NormalTok{\}}

\KeywordTok{fn} \NormalTok{main() \{}
    \KeywordTok{let} \NormalTok{c = Circle \{ x: }\DecValTok{0.0}\NormalTok{, y: }\DecValTok{0.0}\NormalTok{, radius: }\DecValTok{2.0} \NormalTok{\};}
    \PreprocessorTok{println!}\NormalTok{(}\StringTok{"\{\}"}\NormalTok{, c.area());}

    \KeywordTok{let} \NormalTok{d = c.grow(}\DecValTok{2.0}\NormalTok{).area();}
    \PreprocessorTok{println!}\NormalTok{(}\StringTok{"\{\}"}\NormalTok{, d);}
\NormalTok{\}}
\end{Highlighting}
\end{Shaded}

Check the return type:

\begin{Shaded}
\begin{Highlighting}[]
\KeywordTok{fn} \NormalTok{grow(&}\KeywordTok{self}\NormalTok{, increment: }\DataTypeTok{f64}\NormalTok{) -> Circle \{}
\end{Highlighting}
\end{Shaded}

We say we're returning a \texttt{Circle}. With this method, we can grow
a new \texttt{Circle} to any arbitrary size.

\subsection{Associated functions}\label{associated-functions}

You can also define associated functions that do not take a
\texttt{self} parameter. Here's a pattern that's very common in Rust
code:

\begin{Shaded}
\begin{Highlighting}[]
\KeywordTok{struct} \NormalTok{Circle \{}
    \NormalTok{x: }\DataTypeTok{f64}\NormalTok{,}
    \NormalTok{y: }\DataTypeTok{f64}\NormalTok{,}
    \NormalTok{radius: }\DataTypeTok{f64}\NormalTok{,}
\NormalTok{\}}

\KeywordTok{impl} \NormalTok{Circle \{}
    \KeywordTok{fn} \NormalTok{new(x: }\DataTypeTok{f64}\NormalTok{, y: }\DataTypeTok{f64}\NormalTok{, radius: }\DataTypeTok{f64}\NormalTok{) -> Circle \{}
        \NormalTok{Circle \{}
            \NormalTok{x: x,}
            \NormalTok{y: y,}
            \NormalTok{radius: radius,}
        \NormalTok{\}}
    \NormalTok{\}}
\NormalTok{\}}

\KeywordTok{fn} \NormalTok{main() \{}
    \KeywordTok{let} \NormalTok{c = Circle::new(}\DecValTok{0.0}\NormalTok{, }\DecValTok{0.0}\NormalTok{, }\DecValTok{2.0}\NormalTok{);}
\NormalTok{\}}
\end{Highlighting}
\end{Shaded}

This `associated function' builds a new \texttt{Circle} for us. Note
that associated functions are called with the
\texttt{Struct::function()} syntax, rather than the
\texttt{ref.method()} syntax. Some other languages call associated
functions `static methods'.

\subsection{Builder Pattern}\label{builder-pattern}

Let's say that we want our users to be able to create \texttt{Circle}s,
but we will allow them to only set the properties they care about.
Otherwise, the \texttt{x} and \texttt{y} attributes will be
\texttt{0.0}, and the \texttt{radius} will be \texttt{1.0}. Rust doesn't
have method overloading, named arguments, or variable arguments. We
employ the builder pattern instead. It looks like this:

\begin{Shaded}
\begin{Highlighting}[]
\KeywordTok{struct} \NormalTok{Circle \{}
    \NormalTok{x: }\DataTypeTok{f64}\NormalTok{,}
    \NormalTok{y: }\DataTypeTok{f64}\NormalTok{,}
    \NormalTok{radius: }\DataTypeTok{f64}\NormalTok{,}
\NormalTok{\}}

\KeywordTok{impl} \NormalTok{Circle \{}
    \KeywordTok{fn} \NormalTok{area(&}\KeywordTok{self}\NormalTok{) -> }\DataTypeTok{f64} \NormalTok{\{}
        \NormalTok{std::}\DataTypeTok{f64}\NormalTok{::consts::PI * (}\KeywordTok{self}\NormalTok{.radius * }\KeywordTok{self}\NormalTok{.radius)}
    \NormalTok{\}}
\NormalTok{\}}

\KeywordTok{struct} \NormalTok{CircleBuilder \{}
    \NormalTok{x: }\DataTypeTok{f64}\NormalTok{,}
    \NormalTok{y: }\DataTypeTok{f64}\NormalTok{,}
    \NormalTok{radius: }\DataTypeTok{f64}\NormalTok{,}
\NormalTok{\}}

\KeywordTok{impl} \NormalTok{CircleBuilder \{}
    \KeywordTok{fn} \NormalTok{new() -> CircleBuilder \{}
        \NormalTok{CircleBuilder \{ x: }\DecValTok{0.0}\NormalTok{, y: }\DecValTok{0.0}\NormalTok{, radius: }\DecValTok{1.0}\NormalTok{, \}}
    \NormalTok{\}}

    \KeywordTok{fn} \NormalTok{x(&}\KeywordTok{mut} \KeywordTok{self}\NormalTok{, coordinate: }\DataTypeTok{f64}\NormalTok{) -> &}\KeywordTok{mut} \NormalTok{CircleBuilder \{}
        \KeywordTok{self}\NormalTok{.x = coordinate;}
        \KeywordTok{self}
    \NormalTok{\}}

    \KeywordTok{fn} \NormalTok{y(&}\KeywordTok{mut} \KeywordTok{self}\NormalTok{, coordinate: }\DataTypeTok{f64}\NormalTok{) -> &}\KeywordTok{mut} \NormalTok{CircleBuilder \{}
        \KeywordTok{self}\NormalTok{.y = coordinate;}
        \KeywordTok{self}
    \NormalTok{\}}

    \KeywordTok{fn} \NormalTok{radius(&}\KeywordTok{mut} \KeywordTok{self}\NormalTok{, radius: }\DataTypeTok{f64}\NormalTok{) -> &}\KeywordTok{mut} \NormalTok{CircleBuilder \{}
        \KeywordTok{self}\NormalTok{.radius = radius;}
        \KeywordTok{self}
    \NormalTok{\}}

    \KeywordTok{fn} \NormalTok{finalize(&}\KeywordTok{self}\NormalTok{) -> Circle \{}
        \NormalTok{Circle \{ x: }\KeywordTok{self}\NormalTok{.x, y: }\KeywordTok{self}\NormalTok{.y, radius: }\KeywordTok{self}\NormalTok{.radius \}}
    \NormalTok{\}}
\NormalTok{\}}

\KeywordTok{fn} \NormalTok{main() \{}
    \KeywordTok{let} \NormalTok{c = CircleBuilder::new()}
                \NormalTok{.x(}\DecValTok{1.0}\NormalTok{)}
                \NormalTok{.y(}\DecValTok{2.0}\NormalTok{)}
                \NormalTok{.radius(}\DecValTok{2.0}\NormalTok{)}
                \NormalTok{.finalize();}

    \PreprocessorTok{println!}\NormalTok{(}\StringTok{"area: \{\}"}\NormalTok{, c.area());}
    \PreprocessorTok{println!}\NormalTok{(}\StringTok{"x: \{\}"}\NormalTok{, c.x);}
    \PreprocessorTok{println!}\NormalTok{(}\StringTok{"y: \{\}"}\NormalTok{, c.y);}
\NormalTok{\}}
\end{Highlighting}
\end{Shaded}

What we've done here is make another \texttt{struct},
\texttt{CircleBuilder}. We've defined our builder methods on it. We've
also defined our \texttt{area()} method on \texttt{Circle}. We also made
one more method on \texttt{CircleBuilder}: \texttt{finalize()}. This
method creates our final \texttt{Circle} from the builder. Now, we've
used the type system to enforce our concerns: we can use the methods on
\texttt{CircleBuilder} to constrain making \texttt{Circle}s in any way
we choose.

\hypertarget{sec--strings}{\section{Strings}\label{sec--strings}}

Strings are an important concept for any programmer to master. Rust's
string handling system is a bit different from other languages, due to
its systems focus. Any time you have a data structure of variable size,
things can get tricky, and strings are a re-sizable data structure. That
being said, Rust's strings also work differently than in some other
systems languages, such as C.

Let's dig into the details. A `string' is a sequence of Unicode scalar
values encoded as a stream of UTF-8 bytes. All strings are guaranteed to
be a valid encoding of UTF-8 sequences. Additionally, unlike some
systems languages, strings are not null-terminated and can contain null
bytes.

Rust has two main types of strings: \texttt{\&str} and \texttt{String}.
Let's talk about \texttt{\&str} first. These are called `string slices'.
A string slice has a fixed size, and cannot be mutated. It is a
reference to a sequence of UTF-8 bytes.

\begin{Shaded}
\begin{Highlighting}[]
\KeywordTok{let} \NormalTok{greeting = }\StringTok{"Hello there."}\NormalTok{; }\CommentTok{// greeting: &'static str}
\end{Highlighting}
\end{Shaded}

\texttt{"Hello\ there."} is a string literal and its type is
\texttt{\&\textquotesingle{}static\ str}. A string literal is a string
slice that is statically allocated, meaning that it's saved inside our
compiled program, and exists for the entire duration it runs. The
\texttt{greeting} binding is a reference to this statically allocated
string. Any function expecting a string slice will also accept a string
literal.

String literals can span multiple lines. There are two forms. The first
will include the newline and the leading spaces:

\begin{Shaded}
\begin{Highlighting}[]
\KeywordTok{let} \NormalTok{s = }\StringTok{"foo}
\StringTok{    bar"}\NormalTok{;}

\PreprocessorTok{assert_eq!}\NormalTok{(}\StringTok{"foo}\SpecialCharTok{\textbackslash{}n}\StringTok{    bar"}\NormalTok{, s);}
\end{Highlighting}
\end{Shaded}

The second, with a \texttt{\textbackslash{}}, trims the spaces and the
newline:

\begin{Shaded}
\begin{Highlighting}[]
\KeywordTok{let} \NormalTok{s = }\StringTok{"foo}\SpecialCharTok{\textbackslash{}}
\StringTok{    bar"}\NormalTok{;}

\PreprocessorTok{assert_eq!}\NormalTok{(}\StringTok{"foobar"}\NormalTok{, s);}
\end{Highlighting}
\end{Shaded}

Note that you normally cannot access a \texttt{str} directly, but only
through a \texttt{\&str} reference. This is because \texttt{str} is an
unsized type which requires additional runtime information to be usable.
For more information see the chapter on
\protect\hyperlink{sec--unsized-types}{unsized types}.

Rust has more than only \texttt{\&str}s though. A \texttt{String} is a
heap-allocated string. This string is growable, and is also guaranteed
to be UTF-8. \texttt{String}s are commonly created by converting from a
string slice using the \texttt{to\_string} method.

\begin{Shaded}
\begin{Highlighting}[]
\KeywordTok{let} \KeywordTok{mut} \NormalTok{s = }\StringTok{"Hello"}\NormalTok{.to_string(); }\CommentTok{// mut s: String}
\PreprocessorTok{println!}\NormalTok{(}\StringTok{"\{\}"}\NormalTok{, s);}

\NormalTok{s.push_str(}\StringTok{", world."}\NormalTok{);}
\PreprocessorTok{println!}\NormalTok{(}\StringTok{"\{\}"}\NormalTok{, s);}
\end{Highlighting}
\end{Shaded}

\texttt{String}s will coerce into \texttt{\&str} with an \texttt{\&}:

\begin{Shaded}
\begin{Highlighting}[]
\KeywordTok{fn} \NormalTok{takes_slice(slice: &}\DataTypeTok{str}\NormalTok{) \{}
    \PreprocessorTok{println!}\NormalTok{(}\StringTok{"Got: \{\}"}\NormalTok{, slice);}
\NormalTok{\}}

\KeywordTok{fn} \NormalTok{main() \{}
    \KeywordTok{let} \NormalTok{s = }\StringTok{"Hello"}\NormalTok{.to_string();}
    \NormalTok{takes_slice(&s);}
\NormalTok{\}}
\end{Highlighting}
\end{Shaded}

This coercion does not happen for functions that accept one of
\texttt{\&str}'s traits instead of \texttt{\&str}. For example,
\href{http://doc.rust-lang.org/std/net/struct.TcpStream.html\#method.connect}{\texttt{TcpStream::connect}}
has a parameter of type \texttt{ToSocketAddrs}. A \texttt{\&str} is okay
but a \texttt{String} must be explicitly converted using \texttt{\&*}.

\begin{Shaded}
\begin{Highlighting}[]
\KeywordTok{use} \NormalTok{std::net::TcpStream;}

\NormalTok{TcpStream::connect(}\StringTok{"192.168.0.1:3000"}\NormalTok{); }\CommentTok{// &str parameter}

\KeywordTok{let} \NormalTok{addr_string = }\StringTok{"192.168.0.1:3000"}\NormalTok{.to_string();}
\NormalTok{TcpStream::connect(&*addr_string); }\CommentTok{// convert addr_string to &str}
\end{Highlighting}
\end{Shaded}

Viewing a \texttt{String} as a \texttt{\&str} is cheap, but converting
the \texttt{\&str} to a \texttt{String} involves allocating memory. No
reason to do that unless you have to!

\subsubsection{Indexing}\label{indexing}

Because strings are valid UTF-8, they do not support indexing:

\begin{Shaded}
\begin{Highlighting}[]
\KeywordTok{let} \NormalTok{s = }\StringTok{"hello"}\NormalTok{;}

\PreprocessorTok{println!}\NormalTok{(}\StringTok{"The first letter of s is \{\}"}\NormalTok{, s[}\DecValTok{0}\NormalTok{]); }\CommentTok{// ERROR!!!}
\end{Highlighting}
\end{Shaded}

Usually, access to a vector with \texttt{{[}{]}} is very fast. But,
because each character in a UTF-8 encoded string can be multiple bytes,
you have to walk over the string to find the nᵗʰ letter of a string.
This is a significantly more expensive operation, and we don't want to
be misleading. Furthermore, `letter' isn't something defined in Unicode,
exactly. We can choose to look at a string as individual bytes, or as
codepoints:

\begin{Shaded}
\begin{Highlighting}[]
\KeywordTok{let} \NormalTok{hachiko = }\StringTok{"忠犬ハチ公"}\NormalTok{;}

\KeywordTok{for} \NormalTok{b }\KeywordTok{in} \NormalTok{hachiko.as_bytes() \{}
    \PreprocessorTok{print!}\NormalTok{(}\StringTok{"\{\}, "}\NormalTok{, b);}
\NormalTok{\}}

\PreprocessorTok{println!}\NormalTok{(}\StringTok{""}\NormalTok{);}

\KeywordTok{for} \NormalTok{c }\KeywordTok{in} \NormalTok{hachiko.chars() \{}
    \PreprocessorTok{print!}\NormalTok{(}\StringTok{"\{\}, "}\NormalTok{, c);}
\NormalTok{\}}

\PreprocessorTok{println!}\NormalTok{(}\StringTok{""}\NormalTok{);}
\end{Highlighting}
\end{Shaded}

This prints:

\begin{verbatim}
229, 191, 160, 231, 138, 172, 227, 131, 143, 227, 131, 129, 229, 133, 172,
忠, 犬, ハ, チ, 公,
\end{verbatim}

As you can see, there are more bytes than \texttt{char}s.

You can get something similar to an index like this:

\begin{Shaded}
\begin{Highlighting}[]
\KeywordTok{let} \NormalTok{dog = hachiko.chars().nth(}\DecValTok{1}\NormalTok{); }\CommentTok{// kinda like hachiko[1]}
\end{Highlighting}
\end{Shaded}

This emphasizes that we have to walk from the beginning of the list of
\texttt{chars}.

\subsubsection{Slicing}\label{slicing}

You can get a slice of a string with slicing syntax:

\begin{Shaded}
\begin{Highlighting}[]
\KeywordTok{let} \NormalTok{dog = }\StringTok{"hachiko"}\NormalTok{;}
\KeywordTok{let} \NormalTok{hachi = &dog[}\DecValTok{0.}\NormalTok{.}\DecValTok{5}\NormalTok{];}
\end{Highlighting}
\end{Shaded}

But note that these are \emph{byte} offsets, not \emph{character}
offsets. So this will fail at runtime:

\begin{Shaded}
\begin{Highlighting}[]
\KeywordTok{let} \NormalTok{dog = }\StringTok{"忠犬ハチ公"}\NormalTok{;}
\KeywordTok{let} \NormalTok{hachi = &dog[}\DecValTok{0.}\NormalTok{.}\DecValTok{2}\NormalTok{];}
\end{Highlighting}
\end{Shaded}

with this error:

\begin{verbatim}
thread '<main>' panicked at 'index 0 and/or 2 in `忠犬ハチ公` do not lie on
character boundary'
\end{verbatim}

\subsubsection{Concatenation}\label{concatenation}

If you have a \texttt{String}, you can concatenate a \texttt{\&str} to
the end of it:

\begin{Shaded}
\begin{Highlighting}[]
\KeywordTok{let} \NormalTok{hello = }\StringTok{"Hello "}\NormalTok{.to_string();}
\KeywordTok{let} \NormalTok{world = }\StringTok{"world!"}\NormalTok{;}

\KeywordTok{let} \NormalTok{hello_world = hello + world;}
\end{Highlighting}
\end{Shaded}

But if you have two \texttt{String}s, you need an \texttt{\&}:

\begin{Shaded}
\begin{Highlighting}[]
\KeywordTok{let} \NormalTok{hello = }\StringTok{"Hello "}\NormalTok{.to_string();}
\KeywordTok{let} \NormalTok{world = }\StringTok{"world!"}\NormalTok{.to_string();}

\KeywordTok{let} \NormalTok{hello_world = hello + &world;}
\end{Highlighting}
\end{Shaded}

This is because \texttt{\&String} can automatically coerce to a
\texttt{\&str}. This is a feature called
`\protect\hyperlink{sec--deref-coercions}{\texttt{Deref} coercions}'.

\hypertarget{sec--generics}{\section{Generics}\label{sec--generics}}

Sometimes, when writing a function or data type, we may want it to work
for multiple types of arguments. In Rust, we can do this with generics.
Generics are called `parametric polymorphism' in type theory, which
means that they are types or functions that have multiple forms (`poly'
is multiple, `morph' is form) over a given parameter (`parametric').

Anyway, enough type theory, let's check out some generic code. Rust's
standard library provides a type,
\texttt{Option\textless{}T\textgreater{}}, that's generic:

\begin{Shaded}
\begin{Highlighting}[]
\KeywordTok{enum} \DataTypeTok{Option}\NormalTok{<T> \{}
    \ConstantTok{Some}\NormalTok{(T),}
    \ConstantTok{None}\NormalTok{,}
\NormalTok{\}}
\end{Highlighting}
\end{Shaded}

The \texttt{\textless{}T\textgreater{}} part, which you've seen a few
times before, indicates that this is a generic data type. Inside the
declaration of our \texttt{enum}, wherever we see a \texttt{T}, we
substitute that type for the same type used in the generic. Here's an
example of using \texttt{Option\textless{}T\textgreater{}}, with some
extra type annotations:

\begin{Shaded}
\begin{Highlighting}[]
\KeywordTok{let} \NormalTok{x: }\DataTypeTok{Option}\NormalTok{<}\DataTypeTok{i32}\NormalTok{> = }\ConstantTok{Some}\NormalTok{(}\DecValTok{5}\NormalTok{);}
\end{Highlighting}
\end{Shaded}

In the type declaration, we say
\texttt{Option\textless{}i32\textgreater{}}. Note how similar this looks
to \texttt{Option\textless{}T\textgreater{}}. So, in this particular
\texttt{Option}, \texttt{T} has the value of \texttt{i32}. On the
right-hand side of the binding, we make a \texttt{Some(T)}, where
\texttt{T} is \texttt{5}. Since that's an \texttt{i32}, the two sides
match, and Rust is happy. If they didn't match, we'd get an error:

\begin{Shaded}
\begin{Highlighting}[]
\KeywordTok{let} \NormalTok{x: }\DataTypeTok{Option}\NormalTok{<}\DataTypeTok{f64}\NormalTok{> = }\ConstantTok{Some}\NormalTok{(}\DecValTok{5}\NormalTok{);}
\CommentTok{// error: mismatched types: expected `core::option::Option<f64>`,}
\CommentTok{// found `core::option::Option<_>` (expected f64 but found integral variable)}
\end{Highlighting}
\end{Shaded}

That doesn't mean we can't make
\texttt{Option\textless{}T\textgreater{}}s that hold an \texttt{f64}!
They have to match up:

\begin{Shaded}
\begin{Highlighting}[]
\KeywordTok{let} \NormalTok{x: }\DataTypeTok{Option}\NormalTok{<}\DataTypeTok{i32}\NormalTok{> = }\ConstantTok{Some}\NormalTok{(}\DecValTok{5}\NormalTok{);}
\KeywordTok{let} \NormalTok{y: }\DataTypeTok{Option}\NormalTok{<}\DataTypeTok{f64}\NormalTok{> = }\ConstantTok{Some}\NormalTok{(}\DecValTok{5.0f64}\NormalTok{);}
\end{Highlighting}
\end{Shaded}

This is just fine. One definition, multiple uses.

Generics don't have to only be generic over one type. Consider another
type from Rust's standard library that's similar,
\texttt{Result\textless{}T,\ E\textgreater{}}:

\begin{Shaded}
\begin{Highlighting}[]
\KeywordTok{enum} \DataTypeTok{Result}\NormalTok{<T, E> \{}
    \ConstantTok{Ok}\NormalTok{(T),}
    \ConstantTok{Err}\NormalTok{(E),}
\NormalTok{\}}
\end{Highlighting}
\end{Shaded}

This type is generic over \emph{two} types: \texttt{T} and \texttt{E}.
By the way, the capital letters can be any letter you'd like. We could
define \texttt{Result\textless{}T,\ E\textgreater{}} as:

\begin{Shaded}
\begin{Highlighting}[]
\KeywordTok{enum} \DataTypeTok{Result}\NormalTok{<A, Z> \{}
    \ConstantTok{Ok}\NormalTok{(A),}
    \ConstantTok{Err}\NormalTok{(Z),}
\NormalTok{\}}
\end{Highlighting}
\end{Shaded}

if we wanted to. Convention says that the first generic parameter should
be \texttt{T}, for `type', and that we use \texttt{E} for `error'. Rust
doesn't care, however.

The \texttt{Result\textless{}T,\ E\textgreater{}} type is intended to be
used to return the result of a computation, and to have the ability to
return an error if it didn't work out.

\subsubsection{Generic functions}\label{generic-functions}

We can write functions that take generic types with a similar syntax:

\begin{Shaded}
\begin{Highlighting}[]
\KeywordTok{fn} \NormalTok{takes_anything<T>(x: T) \{}
    \CommentTok{// do something with x}
\NormalTok{\}}
\end{Highlighting}
\end{Shaded}

The syntax has two parts: the \texttt{\textless{}T\textgreater{}} says
``this function is generic over one type, \texttt{T}'', and the
\texttt{x:\ T} says ``x has the type \texttt{T}.''

Multiple arguments can have the same generic type:

\begin{Shaded}
\begin{Highlighting}[]
\KeywordTok{fn} \NormalTok{takes_two_of_the_same_things<T>(x: T, y: T) \{}
    \CommentTok{// ...}
\NormalTok{\}}
\end{Highlighting}
\end{Shaded}

We could write a version that takes multiple types:

\begin{Shaded}
\begin{Highlighting}[]
\KeywordTok{fn} \NormalTok{takes_two_things<T, U>(x: T, y: U) \{}
    \CommentTok{// ...}
\NormalTok{\}}
\end{Highlighting}
\end{Shaded}

\subsubsection{Generic structs}\label{generic-structs}

You can store a generic type in a \texttt{struct} as well:

\begin{Shaded}
\begin{Highlighting}[]
\KeywordTok{struct} \NormalTok{Point<T> \{}
    \NormalTok{x: T,}
    \NormalTok{y: T,}
\NormalTok{\}}

\KeywordTok{let} \NormalTok{int_origin = Point \{ x: }\DecValTok{0}\NormalTok{, y: }\DecValTok{0} \NormalTok{\};}
\KeywordTok{let} \NormalTok{float_origin = Point \{ x: }\DecValTok{0.0}\NormalTok{, y: }\DecValTok{0.0} \NormalTok{\};}
\end{Highlighting}
\end{Shaded}

Similar to functions, the \texttt{\textless{}T\textgreater{}} is where
we declare the generic parameters, and we then use \texttt{x:\ T} in the
type declaration, too.

When you want to add an implementation for the generic \texttt{struct},
you declare the type parameter after the \texttt{impl}:

\begin{Shaded}
\begin{Highlighting}[]
\KeywordTok{impl}\NormalTok{<T> Point<T> \{}
    \KeywordTok{fn} \NormalTok{swap(&}\KeywordTok{mut} \KeywordTok{self}\NormalTok{) \{}
        \NormalTok{std::mem::swap(&}\KeywordTok{mut} \KeywordTok{self}\NormalTok{.x, &}\KeywordTok{mut} \KeywordTok{self}\NormalTok{.y);}
    \NormalTok{\}}
\NormalTok{\}}
\end{Highlighting}
\end{Shaded}

So far you've seen generics that take absolutely any type. These are
useful in many cases: you've already seen
\texttt{Option\textless{}T\textgreater{}}, and later you'll meet
universal container types like
\href{http://doc.rust-lang.org/std/vec/struct.Vec.html}{\texttt{Vec\textless{}T\textgreater{}}}.
On the other hand, often you want to trade that flexibility for
increased expressive power. Read about
\protect\hyperlink{sec--traits}{trait bounds} to see why and how.

\hypertarget{sec--traits}{\section{Traits}\label{sec--traits}}

A trait is a language feature that tells the Rust compiler about
functionality a type must provide.

Recall the \texttt{impl} keyword, used to call a function with
\protect\hyperlink{sec--method-syntax}{method syntax}:

\begin{Shaded}
\begin{Highlighting}[]
\KeywordTok{struct} \NormalTok{Circle \{}
    \NormalTok{x: }\DataTypeTok{f64}\NormalTok{,}
    \NormalTok{y: }\DataTypeTok{f64}\NormalTok{,}
    \NormalTok{radius: }\DataTypeTok{f64}\NormalTok{,}
\NormalTok{\}}

\KeywordTok{impl} \NormalTok{Circle \{}
    \KeywordTok{fn} \NormalTok{area(&}\KeywordTok{self}\NormalTok{) -> }\DataTypeTok{f64} \NormalTok{\{}
        \NormalTok{std::}\DataTypeTok{f64}\NormalTok{::consts::PI * (}\KeywordTok{self}\NormalTok{.radius * }\KeywordTok{self}\NormalTok{.radius)}
    \NormalTok{\}}
\NormalTok{\}}
\end{Highlighting}
\end{Shaded}

Traits are similar, except that we first define a trait with a method
signature, then implement the trait for a type. In this example, we
implement the trait \texttt{HasArea} for \texttt{Circle}:

\begin{Shaded}
\begin{Highlighting}[]
\KeywordTok{struct} \NormalTok{Circle \{}
    \NormalTok{x: }\DataTypeTok{f64}\NormalTok{,}
    \NormalTok{y: }\DataTypeTok{f64}\NormalTok{,}
    \NormalTok{radius: }\DataTypeTok{f64}\NormalTok{,}
\NormalTok{\}}

\KeywordTok{trait} \NormalTok{HasArea \{}
    \KeywordTok{fn} \NormalTok{area(&}\KeywordTok{self}\NormalTok{) -> }\DataTypeTok{f64}\NormalTok{;}
\NormalTok{\}}

\KeywordTok{impl} \NormalTok{HasArea }\KeywordTok{for} \NormalTok{Circle \{}
    \KeywordTok{fn} \NormalTok{area(&}\KeywordTok{self}\NormalTok{) -> }\DataTypeTok{f64} \NormalTok{\{}
        \NormalTok{std::}\DataTypeTok{f64}\NormalTok{::consts::PI * (}\KeywordTok{self}\NormalTok{.radius * }\KeywordTok{self}\NormalTok{.radius)}
    \NormalTok{\}}
\NormalTok{\}}
\end{Highlighting}
\end{Shaded}

As you can see, the \texttt{trait} block looks very similar to the
\texttt{impl} block, but we don't define a body, only a type signature.
When we \texttt{impl} a trait, we use \texttt{impl\ Trait\ for\ Item},
rather than only \texttt{impl\ Item}.

\subsubsection{Trait bounds on generic
functions}\label{trait-bounds-on-generic-functions}

Traits are useful because they allow a type to make certain promises
about its behavior. Generic functions can exploit this to constrain, or
\protect\hyperlink{bounds}{bound}, the types they accept. Consider this
function, which does not compile:

\begin{Shaded}
\begin{Highlighting}[]
\KeywordTok{fn} \NormalTok{print_area<T>(shape: T) \{}
    \PreprocessorTok{println!}\NormalTok{(}\StringTok{"This shape has an area of \{\}"}\NormalTok{, shape.area());}
\NormalTok{\}}
\end{Highlighting}
\end{Shaded}

Rust complains:

\begin{verbatim}
error: no method named `area` found for type `T` in the current scope
\end{verbatim}

Because \texttt{T} can be any type, we can't be sure that it implements
the \texttt{area} method. But we can add a trait bound to our generic
\texttt{T}, ensuring that it does:

\begin{Shaded}
\begin{Highlighting}[]
\KeywordTok{fn} \NormalTok{print_area<T: HasArea>(shape: T) \{}
    \PreprocessorTok{println!}\NormalTok{(}\StringTok{"This shape has an area of \{\}"}\NormalTok{, shape.area());}
\NormalTok{\}}
\end{Highlighting}
\end{Shaded}

The syntax \texttt{\textless{}T:\ HasArea\textgreater{}} means ``any
type that implements the \texttt{HasArea} trait.'' Because traits define
function type signatures, we can be sure that any type which implements
\texttt{HasArea} will have an \texttt{.area()} method.

Here's an extended example of how this works:

\begin{Shaded}
\begin{Highlighting}[]
\KeywordTok{trait} \NormalTok{HasArea \{}
    \KeywordTok{fn} \NormalTok{area(&}\KeywordTok{self}\NormalTok{) -> }\DataTypeTok{f64}\NormalTok{;}
\NormalTok{\}}

\KeywordTok{struct} \NormalTok{Circle \{}
    \NormalTok{x: }\DataTypeTok{f64}\NormalTok{,}
    \NormalTok{y: }\DataTypeTok{f64}\NormalTok{,}
    \NormalTok{radius: }\DataTypeTok{f64}\NormalTok{,}
\NormalTok{\}}

\KeywordTok{impl} \NormalTok{HasArea }\KeywordTok{for} \NormalTok{Circle \{}
    \KeywordTok{fn} \NormalTok{area(&}\KeywordTok{self}\NormalTok{) -> }\DataTypeTok{f64} \NormalTok{\{}
        \NormalTok{std::}\DataTypeTok{f64}\NormalTok{::consts::PI * (}\KeywordTok{self}\NormalTok{.radius * }\KeywordTok{self}\NormalTok{.radius)}
    \NormalTok{\}}
\NormalTok{\}}

\KeywordTok{struct} \NormalTok{Square \{}
    \NormalTok{x: }\DataTypeTok{f64}\NormalTok{,}
    \NormalTok{y: }\DataTypeTok{f64}\NormalTok{,}
    \NormalTok{side: }\DataTypeTok{f64}\NormalTok{,}
\NormalTok{\}}

\KeywordTok{impl} \NormalTok{HasArea }\KeywordTok{for} \NormalTok{Square \{}
    \KeywordTok{fn} \NormalTok{area(&}\KeywordTok{self}\NormalTok{) -> }\DataTypeTok{f64} \NormalTok{\{}
        \KeywordTok{self}\NormalTok{.side * }\KeywordTok{self}\NormalTok{.side}
    \NormalTok{\}}
\NormalTok{\}}

\KeywordTok{fn} \NormalTok{print_area<T: HasArea>(shape: T) \{}
    \PreprocessorTok{println!}\NormalTok{(}\StringTok{"This shape has an area of \{\}"}\NormalTok{, shape.area());}
\NormalTok{\}}

\KeywordTok{fn} \NormalTok{main() \{}
    \KeywordTok{let} \NormalTok{c = Circle \{}
        \NormalTok{x: }\DecValTok{0.0f64}\NormalTok{,}
        \NormalTok{y: }\DecValTok{0.0f64}\NormalTok{,}
        \NormalTok{radius: }\DecValTok{1.0f64}\NormalTok{,}
    \NormalTok{\};}

    \KeywordTok{let} \NormalTok{s = Square \{}
        \NormalTok{x: }\DecValTok{0.0f64}\NormalTok{,}
        \NormalTok{y: }\DecValTok{0.0f64}\NormalTok{,}
        \NormalTok{side: }\DecValTok{1.0f64}\NormalTok{,}
    \NormalTok{\};}

    \NormalTok{print_area(c);}
    \NormalTok{print_area(s);}
\NormalTok{\}}
\end{Highlighting}
\end{Shaded}

This program outputs:

\begin{verbatim}
This shape has an area of 3.141593
This shape has an area of 1
\end{verbatim}

As you can see, \texttt{print\_area} is now generic, but also ensures
that we have passed in the correct types. If we pass in an incorrect
type:

\begin{Shaded}
\begin{Highlighting}[]
\NormalTok{print_area(}\DecValTok{5}\NormalTok{);}
\end{Highlighting}
\end{Shaded}

We get a compile-time error:

\begin{verbatim}
error: the trait bound `_ : HasArea` is not satisfied [E0277]
\end{verbatim}

\subsubsection{Trait bounds on generic
structs}\label{trait-bounds-on-generic-structs}

Your generic structs can also benefit from trait bounds. All you need to
do is append the bound when you declare type parameters. Here is a new
type \texttt{Rectangle\textless{}T\textgreater{}} and its operation
\texttt{is\_square()}:

\begin{Shaded}
\begin{Highlighting}[]
\KeywordTok{struct} \NormalTok{Rectangle<T> \{}
    \NormalTok{x: T,}
    \NormalTok{y: T,}
    \NormalTok{width: T,}
    \NormalTok{height: T,}
\NormalTok{\}}

\KeywordTok{impl}\NormalTok{<T: }\BuiltInTok{PartialEq}\NormalTok{> Rectangle<T> \{}
    \KeywordTok{fn} \NormalTok{is_square(&}\KeywordTok{self}\NormalTok{) -> }\DataTypeTok{bool} \NormalTok{\{}
        \KeywordTok{self}\NormalTok{.width == }\KeywordTok{self}\NormalTok{.height}
    \NormalTok{\}}
\NormalTok{\}}

\KeywordTok{fn} \NormalTok{main() \{}
    \KeywordTok{let} \KeywordTok{mut} \NormalTok{r = Rectangle \{}
        \NormalTok{x: }\DecValTok{0}\NormalTok{,}
        \NormalTok{y: }\DecValTok{0}\NormalTok{,}
        \NormalTok{width: }\DecValTok{47}\NormalTok{,}
        \NormalTok{height: }\DecValTok{47}\NormalTok{,}
    \NormalTok{\};}

    \PreprocessorTok{assert!}\NormalTok{(r.is_square());}

    \NormalTok{r.height = }\DecValTok{42}\NormalTok{;}
    \PreprocessorTok{assert!}\NormalTok{(!r.is_square());}
\NormalTok{\}}
\end{Highlighting}
\end{Shaded}

\texttt{is\_square()} needs to check that the sides are equal, so the
sides must be of a type that implements the
\href{http://doc.rust-lang.org/core/cmp/trait.PartialEq.html}{\texttt{core::cmp::PartialEq}}
trait:

\begin{Shaded}
\begin{Highlighting}[]
\KeywordTok{impl}\NormalTok{<T: }\BuiltInTok{PartialEq}\NormalTok{> Rectangle<T> \{ ... \}}
\end{Highlighting}
\end{Shaded}

Now, a rectangle can be defined in terms of any type that can be
compared for equality.

Here we defined a new struct \texttt{Rectangle} that accepts numbers of
any precision---really, objects of pretty much any type---as long as
they can be compared for equality. Could we do the same for our
\texttt{HasArea} structs, \texttt{Square} and \texttt{Circle}? Yes, but
they need multiplication, and to work with that we need to know more
about \protect\hyperlink{sec--operators-and-overloading}{operator
traits}.

\subsection{Rules for implementing
traits}\label{rules-for-implementing-traits}

So far, we've only added trait implementations to structs, but you can
implement a trait for any type. So technically, we \emph{could}
implement \texttt{HasArea} for \texttt{i32}:

\begin{Shaded}
\begin{Highlighting}[]
\KeywordTok{trait} \NormalTok{HasArea \{}
    \KeywordTok{fn} \NormalTok{area(&}\KeywordTok{self}\NormalTok{) -> }\DataTypeTok{f64}\NormalTok{;}
\NormalTok{\}}

\KeywordTok{impl} \NormalTok{HasArea }\KeywordTok{for} \DataTypeTok{i32} \NormalTok{\{}
    \KeywordTok{fn} \NormalTok{area(&}\KeywordTok{self}\NormalTok{) -> }\DataTypeTok{f64} \NormalTok{\{}
        \PreprocessorTok{println!}\NormalTok{(}\StringTok{"this is silly"}\NormalTok{);}

        \NormalTok{*}\KeywordTok{self} \KeywordTok{as} \DataTypeTok{f64}
    \NormalTok{\}}
\NormalTok{\}}

\DecValTok{5.}\NormalTok{area();}
\end{Highlighting}
\end{Shaded}

It is considered poor style to implement methods on such primitive
types, even though it is possible.

This may seem like the Wild West, but there are two restrictions around
implementing traits that prevent this from getting out of hand. The
first is that if the trait isn't defined in your scope, it doesn't
apply. Here's an example: the standard library provides a
\href{http://doc.rust-lang.org/std/io/trait.Write.html}{\texttt{Write}}
trait which adds extra functionality to \texttt{File}s, for doing file
I/O. By default, a \texttt{File} won't have its methods:

\begin{Shaded}
\begin{Highlighting}[]
\KeywordTok{let} \KeywordTok{mut} \NormalTok{f = std::fs::File::open(}\StringTok{"foo.txt"}\NormalTok{).expect(}\StringTok{"Couldn’t open foo.txt"}\NormalTok{);}
\KeywordTok{let} \NormalTok{buf = b}\StringTok{"whatever"}\NormalTok{; }\CommentTok{// byte string literal. buf: &[u8; 8]}
\KeywordTok{let} \NormalTok{result = f.write(buf);}
\end{Highlighting}
\end{Shaded}

Here's the error:

\begin{verbatim}
error: type `std::fs::File` does not implement any method in scope named `write`
let result = f.write(buf);
               ^~~~~~~~~~
\end{verbatim}

We need to \texttt{use} the \texttt{Write} trait first:

\begin{Shaded}
\begin{Highlighting}[]
\KeywordTok{use} \NormalTok{std::io::Write;}

\KeywordTok{let} \KeywordTok{mut} \NormalTok{f = std::fs::File::open(}\StringTok{"foo.txt"}\NormalTok{).expect(}\StringTok{"Couldn’t open foo.txt"}\NormalTok{);}
\KeywordTok{let} \NormalTok{buf = b}\StringTok{"whatever"}\NormalTok{;}
\KeywordTok{let} \NormalTok{result = f.write(buf);}
\end{Highlighting}
\end{Shaded}

This will compile without error.

This means that even if someone does something bad like add methods to
\texttt{i32}, it won't affect you, unless you \texttt{use} that trait.

There's one more restriction on implementing traits: either the trait or
the type you're implementing it for must be defined by you. Or more
precisely, one of them must be defined in the same crate as the
\texttt{impl} you're writing. For more on Rust's module and package
system, see the chapter on
\protect\hyperlink{sec--crates-and-modules}{crates and modules}.

So, we could implement the \texttt{HasArea} type for \texttt{i32},
because we defined \texttt{HasArea} in our code. But if we tried to
implement \texttt{ToString}, a trait provided by Rust, for \texttt{i32},
we could not, because neither the trait nor the type are defined in our
crate.

One last thing about traits: generic functions with a trait bound use
`monomorphization' (mono: one, morph: form), so they are statically
dispatched. What's that mean? Check out the chapter on
\protect\hyperlink{sec--trait-objects}{trait objects} for more details.

\subsection{Multiple trait bounds}\label{multiple-trait-bounds}

You've seen that you can bound a generic type parameter with a trait:

\begin{Shaded}
\begin{Highlighting}[]
\KeywordTok{fn} \NormalTok{foo<T: }\BuiltInTok{Clone}\NormalTok{>(x: T) \{}
    \NormalTok{x.clone();}
\NormalTok{\}}
\end{Highlighting}
\end{Shaded}

If you need more than one bound, you can use \texttt{+}:

\begin{Shaded}
\begin{Highlighting}[]
\KeywordTok{use} \NormalTok{std::fmt::}\BuiltInTok{Debug}\NormalTok{;}

\KeywordTok{fn} \NormalTok{foo<T: }\BuiltInTok{Clone} \NormalTok{+ }\BuiltInTok{Debug}\NormalTok{>(x: T) \{}
    \NormalTok{x.clone();}
    \PreprocessorTok{println!}\NormalTok{(}\StringTok{"\{:?\}"}\NormalTok{, x);}
\NormalTok{\}}
\end{Highlighting}
\end{Shaded}

\texttt{T} now needs to be both \texttt{Clone} as well as
\texttt{Debug}.

\subsection{Where clause}\label{where-clause}

Writing functions with only a few generic types and a small number of
trait bounds isn't too bad, but as the number increases, the syntax gets
increasingly awkward:

\begin{Shaded}
\begin{Highlighting}[]
\KeywordTok{use} \NormalTok{std::fmt::}\BuiltInTok{Debug}\NormalTok{;}

\KeywordTok{fn} \NormalTok{foo<T: }\BuiltInTok{Clone}\NormalTok{, K: }\BuiltInTok{Clone} \NormalTok{+ }\BuiltInTok{Debug}\NormalTok{>(x: T, y: K) \{}
    \NormalTok{x.clone();}
    \NormalTok{y.clone();}
    \PreprocessorTok{println!}\NormalTok{(}\StringTok{"\{:?\}"}\NormalTok{, y);}
\NormalTok{\}}
\end{Highlighting}
\end{Shaded}

The name of the function is on the far left, and the parameter list is
on the far right. The bounds are getting in the way.

Rust has a solution, and it's called a `\texttt{where} clause':

\begin{Shaded}
\begin{Highlighting}[]
\KeywordTok{use} \NormalTok{std::fmt::}\BuiltInTok{Debug}\NormalTok{;}

\KeywordTok{fn} \NormalTok{foo<T: }\BuiltInTok{Clone}\NormalTok{, K: }\BuiltInTok{Clone} \NormalTok{+ }\BuiltInTok{Debug}\NormalTok{>(x: T, y: K) \{}
    \NormalTok{x.clone();}
    \NormalTok{y.clone();}
    \PreprocessorTok{println!}\NormalTok{(}\StringTok{"\{:?\}"}\NormalTok{, y);}
\NormalTok{\}}

\KeywordTok{fn} \NormalTok{bar<T, K>(x: T, y: K) }\KeywordTok{where} \NormalTok{T: }\BuiltInTok{Clone}\NormalTok{, K: }\BuiltInTok{Clone} \NormalTok{+ }\BuiltInTok{Debug} \NormalTok{\{}
    \NormalTok{x.clone();}
    \NormalTok{y.clone();}
    \PreprocessorTok{println!}\NormalTok{(}\StringTok{"\{:?\}"}\NormalTok{, y);}
\NormalTok{\}}

\KeywordTok{fn} \NormalTok{main() \{}
    \NormalTok{foo(}\StringTok{"Hello"}\NormalTok{, }\StringTok{"world"}\NormalTok{);}
    \NormalTok{bar(}\StringTok{"Hello"}\NormalTok{, }\StringTok{"world"}\NormalTok{);}
\NormalTok{\}}
\end{Highlighting}
\end{Shaded}

\texttt{foo()} uses the syntax we showed earlier, and \texttt{bar()}
uses a \texttt{where} clause. All you need to do is leave off the bounds
when defining your type parameters, and then add \texttt{where} after
the parameter list. For longer lists, whitespace can be added:

\begin{Shaded}
\begin{Highlighting}[]
\KeywordTok{use} \NormalTok{std::fmt::}\BuiltInTok{Debug}\NormalTok{;}

\KeywordTok{fn} \NormalTok{bar<T, K>(x: T, y: K)}
    \KeywordTok{where} \NormalTok{T: }\BuiltInTok{Clone}\NormalTok{,}
          \NormalTok{K: }\BuiltInTok{Clone} \NormalTok{+ }\BuiltInTok{Debug} \NormalTok{\{}

    \NormalTok{x.clone();}
    \NormalTok{y.clone();}
    \PreprocessorTok{println!}\NormalTok{(}\StringTok{"\{:?\}"}\NormalTok{, y);}
\NormalTok{\}}
\end{Highlighting}
\end{Shaded}

This flexibility can add clarity in complex situations.

\texttt{where} is also more powerful than the simpler syntax. For
example:

\begin{Shaded}
\begin{Highlighting}[]
\KeywordTok{trait} \NormalTok{ConvertTo<Output> \{}
    \KeywordTok{fn} \NormalTok{convert(&}\KeywordTok{self}\NormalTok{) -> Output;}
\NormalTok{\}}

\KeywordTok{impl} \NormalTok{ConvertTo<}\DataTypeTok{i64}\NormalTok{> }\KeywordTok{for} \DataTypeTok{i32} \NormalTok{\{}
    \KeywordTok{fn} \NormalTok{convert(&}\KeywordTok{self}\NormalTok{) -> }\DataTypeTok{i64} \NormalTok{\{ *}\KeywordTok{self} \KeywordTok{as} \DataTypeTok{i64} \NormalTok{\}}
\NormalTok{\}}

\CommentTok{// can be called with T == i32}
\KeywordTok{fn} \NormalTok{normal<T: ConvertTo<}\DataTypeTok{i64}\NormalTok{>>(x: &T) -> }\DataTypeTok{i64} \NormalTok{\{}
    \NormalTok{x.convert()}
\NormalTok{\}}

\CommentTok{// can be called with T == i64}
\KeywordTok{fn} \NormalTok{inverse<T>() -> T}
        \CommentTok{// this is using ConvertTo as if it were "ConvertTo<i64>"}
        \KeywordTok{where} \DataTypeTok{i32}\NormalTok{: ConvertTo<T> \{}
    \DecValTok{42.}\NormalTok{convert()}
\NormalTok{\}}
\end{Highlighting}
\end{Shaded}

This shows off the additional feature of \texttt{where} clauses: they
allow bounds on the left-hand side not only of type parameters
\texttt{T}, but also of types (\texttt{i32} in this case). In this
example, \texttt{i32} must implement
\texttt{ConvertTo\textless{}T\textgreater{}}. Rather than defining what
\texttt{i32} is (since that's obvious), the \texttt{where} clause here
constrains \texttt{T}.

\subsection{Default methods}\label{default-methods}

A default method can be added to a trait definition if it is already
known how a typical implementor will define a method. For example,
\texttt{is\_invalid()} is defined as the opposite of
\texttt{is\_valid()}:

\begin{Shaded}
\begin{Highlighting}[]
\KeywordTok{trait} \NormalTok{Foo \{}
    \KeywordTok{fn} \NormalTok{is_valid(&}\KeywordTok{self}\NormalTok{) -> }\DataTypeTok{bool}\NormalTok{;}

    \KeywordTok{fn} \NormalTok{is_invalid(&}\KeywordTok{self}\NormalTok{) -> }\DataTypeTok{bool} \NormalTok{\{ !}\KeywordTok{self}\NormalTok{.is_valid() \}}
\NormalTok{\}}
\end{Highlighting}
\end{Shaded}

Implementors of the \texttt{Foo} trait need to implement
\texttt{is\_valid()} but not \texttt{is\_invalid()} due to the added
default behavior. This default behavior can still be overridden as in:

\begin{Shaded}
\begin{Highlighting}[]
\KeywordTok{struct} \NormalTok{UseDefault;}

\KeywordTok{impl} \NormalTok{Foo }\KeywordTok{for} \NormalTok{UseDefault \{}
    \KeywordTok{fn} \NormalTok{is_valid(&}\KeywordTok{self}\NormalTok{) -> }\DataTypeTok{bool} \NormalTok{\{}
        \PreprocessorTok{println!}\NormalTok{(}\StringTok{"Called UseDefault.is_valid."}\NormalTok{);}
        \ConstantTok{true}
    \NormalTok{\}}
\NormalTok{\}}

\KeywordTok{struct} \NormalTok{OverrideDefault;}

\KeywordTok{impl} \NormalTok{Foo }\KeywordTok{for} \NormalTok{OverrideDefault \{}
    \KeywordTok{fn} \NormalTok{is_valid(&}\KeywordTok{self}\NormalTok{) -> }\DataTypeTok{bool} \NormalTok{\{}
        \PreprocessorTok{println!}\NormalTok{(}\StringTok{"Called OverrideDefault.is_valid."}\NormalTok{);}
        \ConstantTok{true}
    \NormalTok{\}}

    \KeywordTok{fn} \NormalTok{is_invalid(&}\KeywordTok{self}\NormalTok{) -> }\DataTypeTok{bool} \NormalTok{\{}
        \PreprocessorTok{println!}\NormalTok{(}\StringTok{"Called OverrideDefault.is_invalid!"}\NormalTok{);}
        \ConstantTok{true} \CommentTok{// overrides the expected value of is_invalid()}
    \NormalTok{\}}
\NormalTok{\}}

\KeywordTok{let} \NormalTok{default = UseDefault;}
\PreprocessorTok{assert!}\NormalTok{(!default.is_invalid()); }\CommentTok{// prints "Called UseDefault.is_valid."}

\KeywordTok{let} \NormalTok{over = OverrideDefault;}
\PreprocessorTok{assert!}\NormalTok{(over.is_invalid()); }\CommentTok{// prints "Called OverrideDefault.is_invalid!"}
\end{Highlighting}
\end{Shaded}

\subsection{Inheritance}\label{inheritance}

Sometimes, implementing a trait requires implementing another trait:

\begin{Shaded}
\begin{Highlighting}[]
\KeywordTok{trait} \NormalTok{Foo \{}
    \KeywordTok{fn} \NormalTok{foo(&}\KeywordTok{self}\NormalTok{);}
\NormalTok{\}}

\KeywordTok{trait} \NormalTok{FooBar : Foo \{}
    \KeywordTok{fn} \NormalTok{foobar(&}\KeywordTok{self}\NormalTok{);}
\NormalTok{\}}
\end{Highlighting}
\end{Shaded}

Implementors of \texttt{FooBar} must also implement \texttt{Foo}, like
this:

\begin{Shaded}
\begin{Highlighting}[]
\KeywordTok{struct} \NormalTok{Baz;}

\KeywordTok{impl} \NormalTok{Foo }\KeywordTok{for} \NormalTok{Baz \{}
    \KeywordTok{fn} \NormalTok{foo(&}\KeywordTok{self}\NormalTok{) \{ }\PreprocessorTok{println!}\NormalTok{(}\StringTok{"foo"}\NormalTok{); \}}
\NormalTok{\}}

\KeywordTok{impl} \NormalTok{FooBar }\KeywordTok{for} \NormalTok{Baz \{}
    \KeywordTok{fn} \NormalTok{foobar(&}\KeywordTok{self}\NormalTok{) \{ }\PreprocessorTok{println!}\NormalTok{(}\StringTok{"foobar"}\NormalTok{); \}}
\NormalTok{\}}
\end{Highlighting}
\end{Shaded}

If we forget to implement \texttt{Foo}, Rust will tell us:

\begin{verbatim}
error: the trait bound `main::Baz : main::Foo` is not satisfied [E0277]
\end{verbatim}

\subsection{Deriving}\label{deriving}

Implementing traits like \texttt{Debug} and \texttt{Default} repeatedly
can become quite tedious. For that reason, Rust provides an
\protect\hyperlink{sec--attributes}{attribute} that allows you to let
Rust automatically implement traits for you:

\begin{Shaded}
\begin{Highlighting}[]
\AttributeTok{#[}\NormalTok{derive}\AttributeTok{(}\BuiltInTok{Debug}\AttributeTok{)]}
\KeywordTok{struct} \NormalTok{Foo;}

\KeywordTok{fn} \NormalTok{main() \{}
    \PreprocessorTok{println!}\NormalTok{(}\StringTok{"\{:?\}"}\NormalTok{, Foo);}
\NormalTok{\}}
\end{Highlighting}
\end{Shaded}

However, deriving is limited to a certain set of traits:

\begin{itemize}
\tightlist
\item
  \href{http://doc.rust-lang.org/core/clone/trait.Clone.html}{\texttt{Clone}}
\item
  \href{http://doc.rust-lang.org/core/marker/trait.Copy.html}{\texttt{Copy}}
\item
  \href{http://doc.rust-lang.org/core/fmt/trait.Debug.html}{\texttt{Debug}}
\item
  \href{http://doc.rust-lang.org/core/default/trait.Default.html}{\texttt{Default}}
\item
  \href{http://doc.rust-lang.org/core/cmp/trait.Eq.html}{\texttt{Eq}}
\item
  \href{http://doc.rust-lang.org/core/hash/trait.Hash.html}{\texttt{Hash}}
\item
  \href{http://doc.rust-lang.org/core/cmp/trait.Ord.html}{\texttt{Ord}}
\item
  \href{http://doc.rust-lang.org/core/cmp/trait.PartialEq.html}{\texttt{PartialEq}}
\item
  \href{http://doc.rust-lang.org/core/cmp/trait.PartialOrd.html}{\texttt{PartialOrd}}
\end{itemize}

\hypertarget{sec--drop}{\section{Drop}\label{sec--drop}}

Now that we've discussed traits, let's talk about a particular trait
provided by the Rust standard library,
\href{http://doc.rust-lang.org/std/ops/trait.Drop.html}{\texttt{Drop}}.
The \texttt{Drop} trait provides a way to run some code when a value
goes out of scope. For example:

\begin{Shaded}
\begin{Highlighting}[]
\KeywordTok{struct} \NormalTok{HasDrop;}

\KeywordTok{impl} \BuiltInTok{Drop} \KeywordTok{for} \NormalTok{HasDrop \{}
    \KeywordTok{fn} \NormalTok{drop(&}\KeywordTok{mut} \KeywordTok{self}\NormalTok{) \{}
        \PreprocessorTok{println!}\NormalTok{(}\StringTok{"Dropping!"}\NormalTok{);}
    \NormalTok{\}}
\NormalTok{\}}

\KeywordTok{fn} \NormalTok{main() \{}
    \KeywordTok{let} \NormalTok{x = HasDrop;}

    \CommentTok{// do stuff}

\NormalTok{\} }\CommentTok{// x goes out of scope here}
\end{Highlighting}
\end{Shaded}

When \texttt{x} goes out of scope at the end of \texttt{main()}, the
code for \texttt{Drop} will run. \texttt{Drop} has one method, which is
also called \texttt{drop()}. It takes a mutable reference to
\texttt{self}.

That's it! The mechanics of \texttt{Drop} are very simple, but there are
some subtleties. For example, values are dropped in the opposite order
they are declared. Here's another example:

\begin{Shaded}
\begin{Highlighting}[]
\KeywordTok{struct} \NormalTok{Firework \{}
    \NormalTok{strength: }\DataTypeTok{i32}\NormalTok{,}
\NormalTok{\}}

\KeywordTok{impl} \BuiltInTok{Drop} \KeywordTok{for} \NormalTok{Firework \{}
    \KeywordTok{fn} \NormalTok{drop(&}\KeywordTok{mut} \KeywordTok{self}\NormalTok{) \{}
        \PreprocessorTok{println!}\NormalTok{(}\StringTok{"BOOM times \{\}!!!"}\NormalTok{, }\KeywordTok{self}\NormalTok{.strength);}
    \NormalTok{\}}
\NormalTok{\}}

\KeywordTok{fn} \NormalTok{main() \{}
    \KeywordTok{let} \NormalTok{firecracker = Firework \{ strength: }\DecValTok{1} \NormalTok{\};}
    \KeywordTok{let} \NormalTok{tnt = Firework \{ strength: }\DecValTok{100} \NormalTok{\};}
\NormalTok{\}}
\end{Highlighting}
\end{Shaded}

This will output:

\begin{verbatim}
BOOM times 100!!!
BOOM times 1!!!
\end{verbatim}

The \texttt{tnt} goes off before the \texttt{firecracker} does, because
it was declared afterwards. Last in, first out.

So what is \texttt{Drop} good for? Generally, \texttt{Drop} is used to
clean up any resources associated with a \texttt{struct}. For example,
the
\href{http://doc.rust-lang.org/std/sync/struct.Arc.html}{\texttt{Arc\textless{}T\textgreater{}}
type} is a reference-counted type. When \texttt{Drop} is called, it will
decrement the reference count, and if the total number of references is
zero, will clean up the underlying value.

\hypertarget{sec--if-let}{\section{if let}\label{sec--if-let}}

\texttt{if\ let} allows you to combine \texttt{if} and \texttt{let}
together to reduce the overhead of certain kinds of pattern matches.

For example, let's say we have some sort of
\texttt{Option\textless{}T\textgreater{}}. We want to call a function on
it if it's \texttt{Some\textless{}T\textgreater{}}, but do nothing if
it's \texttt{None}. That looks like this:

\begin{Shaded}
\begin{Highlighting}[]
\KeywordTok{match} \NormalTok{option \{}
    \ConstantTok{Some}\NormalTok{(x) => \{ foo(x) \},}
    \ConstantTok{None} \NormalTok{=> \{\},}
\NormalTok{\}}
\end{Highlighting}
\end{Shaded}

We don't have to use \texttt{match} here, for example, we could use
\texttt{if}:

\begin{Shaded}
\begin{Highlighting}[]
\KeywordTok{if} \NormalTok{option.is_some() \{}
    \KeywordTok{let} \NormalTok{x = option.unwrap();}
    \NormalTok{foo(x);}
\NormalTok{\}}
\end{Highlighting}
\end{Shaded}

Neither of these options is particularly appealing. We can use
\texttt{if\ let} to do the same thing in a nicer way:

\begin{Shaded}
\begin{Highlighting}[]
\KeywordTok{if} \KeywordTok{let} \ConstantTok{Some}\NormalTok{(x) = option \{}
    \NormalTok{foo(x);}
\NormalTok{\}}
\end{Highlighting}
\end{Shaded}

If a \protect\hyperlink{sec--patterns}{pattern} matches successfully, it
binds any appropriate parts of the value to the identifiers in the
pattern, then evaluates the expression. If the pattern doesn't match,
nothing happens.

If you want to do something else when the pattern does not match, you
can use \texttt{else}:

\begin{Shaded}
\begin{Highlighting}[]
\KeywordTok{if} \KeywordTok{let} \ConstantTok{Some}\NormalTok{(x) = option \{}
    \NormalTok{foo(x);}
\NormalTok{\} }\KeywordTok{else} \NormalTok{\{}
    \NormalTok{bar();}
\NormalTok{\}}
\end{Highlighting}
\end{Shaded}

\subsubsection{\texorpdfstring{\texttt{while\ let}}{while let}}\label{while-let}

In a similar fashion, \texttt{while\ let} can be used when you want to
conditionally loop as long as a value matches a certain pattern. It
turns code like this:

\begin{Shaded}
\begin{Highlighting}[]
\KeywordTok{let} \KeywordTok{mut} \NormalTok{v = }\PreprocessorTok{vec!}\NormalTok{[}\DecValTok{1}\NormalTok{, }\DecValTok{3}\NormalTok{, }\DecValTok{5}\NormalTok{, }\DecValTok{7}\NormalTok{, }\DecValTok{11}\NormalTok{];}
\KeywordTok{loop} \NormalTok{\{}
    \KeywordTok{match} \NormalTok{v.pop() \{}
        \ConstantTok{Some}\NormalTok{(x) =>  }\PreprocessorTok{println!}\NormalTok{(}\StringTok{"\{\}"}\NormalTok{, x),}
        \ConstantTok{None} \NormalTok{=> }\KeywordTok{break}\NormalTok{,}
    \NormalTok{\}}
\NormalTok{\}}
\end{Highlighting}
\end{Shaded}

Into code like this:

\begin{Shaded}
\begin{Highlighting}[]
\KeywordTok{let} \KeywordTok{mut} \NormalTok{v = }\PreprocessorTok{vec!}\NormalTok{[}\DecValTok{1}\NormalTok{, }\DecValTok{3}\NormalTok{, }\DecValTok{5}\NormalTok{, }\DecValTok{7}\NormalTok{, }\DecValTok{11}\NormalTok{];}
\KeywordTok{while} \KeywordTok{let} \ConstantTok{Some}\NormalTok{(x) = v.pop() \{}
    \PreprocessorTok{println!}\NormalTok{(}\StringTok{"\{\}"}\NormalTok{, x);}
\NormalTok{\}}
\end{Highlighting}
\end{Shaded}

\hypertarget{sec--trait-objects}{\section{Trait
Objects}\label{sec--trait-objects}}

When code involves polymorphism, there needs to be a mechanism to
determine which specific version is actually run. This is called
`dispatch'. There are two major forms of dispatch: static dispatch and
dynamic dispatch. While Rust favors static dispatch, it also supports
dynamic dispatch through a mechanism called `trait objects'.

\subsubsection{Background}\label{background}

For the rest of this chapter, we'll need a trait and some
implementations. Let's make a simple one, \texttt{Foo}. It has one
method that is expected to return a \texttt{String}.

\begin{Shaded}
\begin{Highlighting}[]
\KeywordTok{trait} \NormalTok{Foo \{}
    \KeywordTok{fn} \NormalTok{method(&}\KeywordTok{self}\NormalTok{) -> }\DataTypeTok{String}\NormalTok{;}
\NormalTok{\}}
\end{Highlighting}
\end{Shaded}

We'll also implement this trait for \texttt{u8} and \texttt{String}:

\begin{Shaded}
\begin{Highlighting}[]
\KeywordTok{impl} \NormalTok{Foo }\KeywordTok{for} \DataTypeTok{u8} \NormalTok{\{}
    \KeywordTok{fn} \NormalTok{method(&}\KeywordTok{self}\NormalTok{) -> }\DataTypeTok{String} \NormalTok{\{ }\PreprocessorTok{format!}\NormalTok{(}\StringTok{"u8: \{\}"}\NormalTok{, *}\KeywordTok{self}\NormalTok{) \}}
\NormalTok{\}}

\KeywordTok{impl} \NormalTok{Foo }\KeywordTok{for} \DataTypeTok{String} \NormalTok{\{}
    \KeywordTok{fn} \NormalTok{method(&}\KeywordTok{self}\NormalTok{) -> }\DataTypeTok{String} \NormalTok{\{ }\PreprocessorTok{format!}\NormalTok{(}\StringTok{"string: \{\}"}\NormalTok{, *}\KeywordTok{self}\NormalTok{) \}}
\NormalTok{\}}
\end{Highlighting}
\end{Shaded}

\subsubsection{Static dispatch}\label{static-dispatch}

We can use this trait to perform static dispatch with trait bounds:

\begin{Shaded}
\begin{Highlighting}[]
\KeywordTok{fn} \NormalTok{do_something<T: Foo>(x: T) \{}
    \NormalTok{x.method();}
\NormalTok{\}}

\KeywordTok{fn} \NormalTok{main() \{}
    \KeywordTok{let} \NormalTok{x = }\DecValTok{5u8}\NormalTok{;}
    \KeywordTok{let} \NormalTok{y = }\StringTok{"Hello"}\NormalTok{.to_string();}

    \NormalTok{do_something(x);}
    \NormalTok{do_something(y);}
\NormalTok{\}}
\end{Highlighting}
\end{Shaded}

Rust uses `monomorphization' to perform static dispatch here. This means
that Rust will create a special version of \texttt{do\_something()} for
both \texttt{u8} and \texttt{String}, and then replace the call sites
with calls to these specialized functions. In other words, Rust
generates something like this:

\begin{Shaded}
\begin{Highlighting}[]
\KeywordTok{fn} \NormalTok{do_something_u8(x: }\DataTypeTok{u8}\NormalTok{) \{}
    \NormalTok{x.method();}
\NormalTok{\}}

\KeywordTok{fn} \NormalTok{do_something_string(x: }\DataTypeTok{String}\NormalTok{) \{}
    \NormalTok{x.method();}
\NormalTok{\}}

\KeywordTok{fn} \NormalTok{main() \{}
    \KeywordTok{let} \NormalTok{x = }\DecValTok{5u8}\NormalTok{;}
    \KeywordTok{let} \NormalTok{y = }\StringTok{"Hello"}\NormalTok{.to_string();}

    \NormalTok{do_something_u8(x);}
    \NormalTok{do_something_string(y);}
\NormalTok{\}}
\end{Highlighting}
\end{Shaded}

This has a great upside: static dispatch allows function calls to be
inlined because the callee is known at compile time, and inlining is the
key to good optimization. Static dispatch is fast, but it comes at a
tradeoff: `code bloat', due to many copies of the same function existing
in the binary, one for each type.

Furthermore, compilers aren't perfect and may ``optimize'' code to
become slower. For example, functions inlined too eagerly will bloat the
instruction cache (cache rules everything around us). This is part of
the reason that \texttt{\#{[}inline{]}} and
\texttt{\#{[}inline(always){]}} should be used carefully, and one reason
why using a dynamic dispatch is sometimes more efficient.

However, the common case is that it is more efficient to use static
dispatch, and one can always have a thin statically-dispatched wrapper
function that does a dynamic dispatch, but not vice versa, meaning
static calls are more flexible. The standard library tries to be
statically dispatched where possible for this reason.

\subsubsection{Dynamic dispatch}\label{dynamic-dispatch}

Rust provides dynamic dispatch through a feature called `trait objects'.
Trait objects, like \texttt{\&Foo} or
\texttt{Box\textless{}Foo\textgreater{}}, are normal values that store a
value of \emph{any} type that implements the given trait, where the
precise type can only be known at runtime.

A trait object can be obtained from a pointer to a concrete type that
implements the trait by \emph{casting} it (e.g. \texttt{\&x\ as\ \&Foo})
or \emph{coercing} it (e.g.~using \texttt{\&x} as an argument to a
function that takes \texttt{\&Foo}).

These trait object coercions and casts also work for pointers like
\texttt{\&mut\ T} to \texttt{\&mut\ Foo} and
\texttt{Box\textless{}T\textgreater{}} to
\texttt{Box\textless{}Foo\textgreater{}}, but that's all at the moment.
Coercions and casts are identical.

This operation can be seen as `erasing' the compiler's knowledge about
the specific type of the pointer, and hence trait objects are sometimes
referred to as `type erasure'.

Coming back to the example above, we can use the same trait to perform
dynamic dispatch with trait objects by casting:

\begin{Shaded}
\begin{Highlighting}[]

\KeywordTok{fn} \NormalTok{do_something(x: &Foo) \{}
    \NormalTok{x.method();}
\NormalTok{\}}

\KeywordTok{fn} \NormalTok{main() \{}
    \KeywordTok{let} \NormalTok{x = }\DecValTok{5u8}\NormalTok{;}
    \NormalTok{do_something(&x }\KeywordTok{as} \NormalTok{&Foo);}
\NormalTok{\}}
\end{Highlighting}
\end{Shaded}

or by coercing:

\begin{Shaded}
\begin{Highlighting}[]

\KeywordTok{fn} \NormalTok{do_something(x: &Foo) \{}
    \NormalTok{x.method();}
\NormalTok{\}}

\KeywordTok{fn} \NormalTok{main() \{}
    \KeywordTok{let} \NormalTok{x = }\StringTok{"Hello"}\NormalTok{.to_string();}
    \NormalTok{do_something(&x);}
\NormalTok{\}}
\end{Highlighting}
\end{Shaded}

A function that takes a trait object is not specialized to each of the
types that implements \texttt{Foo}: only one copy is generated, often
(but not always) resulting in less code bloat. However, this comes at
the cost of requiring slower virtual function calls, and effectively
inhibiting any chance of inlining and related optimizations from
occurring.

\paragraph{Why pointers?}\label{why-pointers}

Rust does not put things behind a pointer by default, unlike many
managed languages, so types can have different sizes. Knowing the size
of the value at compile time is important for things like passing it as
an argument to a function, moving it about on the stack and allocating
(and deallocating) space on the heap to store it.

For \texttt{Foo}, we would need to have a value that could be at least
either a \texttt{String} (24 bytes) or a \texttt{u8} (1 byte), as well
as any other type for which dependent crates may implement \texttt{Foo}
(any number of bytes at all). There's no way to guarantee that this last
point can work if the values are stored without a pointer, because those
other types can be arbitrarily large.

Putting the value behind a pointer means the size of the value is not
relevant when we are tossing a trait object around, only the size of the
pointer itself.

\paragraph{Representation}\label{representation}

The methods of the trait can be called on a trait object via a special
record of function pointers traditionally called a `vtable' (created and
managed by the compiler).

Trait objects are both simple and complicated: their core representation
and layout is quite straight-forward, but there are some curly error
messages and surprising behaviors to discover.

Let's start simple, with the runtime representation of a trait object.
The \texttt{std::raw} module contains structs with layouts that are the
same as the complicated built-in types,
\href{http://doc.rust-lang.org/std/raw/struct.TraitObject.html}{including
trait objects}:

\begin{Shaded}
\begin{Highlighting}[]
\KeywordTok{pub} \KeywordTok{struct} \NormalTok{TraitObject \{}
    \KeywordTok{pub} \NormalTok{data: *}\KeywordTok{mut} \NormalTok{(),}
    \KeywordTok{pub} \NormalTok{vtable: *}\KeywordTok{mut} \NormalTok{(),}
\NormalTok{\}}
\end{Highlighting}
\end{Shaded}

That is, a trait object like \texttt{\&Foo} consists of a `data' pointer
and a `vtable' pointer.

The data pointer addresses the data (of some unknown type \texttt{T})
that the trait object is storing, and the vtable pointer points to the
vtable (`virtual method table') corresponding to the implementation of
\texttt{Foo} for \texttt{T}.

A vtable is essentially a struct of function pointers, pointing to the
concrete piece of machine code for each method in the implementation. A
method call like \texttt{trait\_object.method()} will retrieve the
correct pointer out of the vtable and then do a dynamic call of it. For
example:

\begin{Shaded}
\begin{Highlighting}[]
\KeywordTok{struct} \NormalTok{FooVtable \{}
    \NormalTok{destructor: }\KeywordTok{fn}\NormalTok{(*}\KeywordTok{mut} \NormalTok{()),}
    \NormalTok{size: }\DataTypeTok{usize}\NormalTok{,}
    \NormalTok{align: }\DataTypeTok{usize}\NormalTok{,}
    \NormalTok{method: }\KeywordTok{fn}\NormalTok{(*}\KeywordTok{const} \NormalTok{()) -> }\DataTypeTok{String}\NormalTok{,}
\NormalTok{\}}

\CommentTok{// u8:}

\KeywordTok{fn} \NormalTok{call_method_on_u8(x: *}\KeywordTok{const} \NormalTok{()) -> }\DataTypeTok{String} \NormalTok{\{}
    \CommentTok{// the compiler guarantees that this function is only called}
    \CommentTok{// with `x` pointing to a u8}
    \KeywordTok{let} \NormalTok{byte: &}\DataTypeTok{u8} \NormalTok{= }\KeywordTok{unsafe} \NormalTok{\{ &*(x }\KeywordTok{as} \NormalTok{*}\KeywordTok{const} \DataTypeTok{u8}\NormalTok{) \};}

    \NormalTok{byte.method()}
\NormalTok{\}}

\KeywordTok{static} \NormalTok{Foo_for_u8_vtable: FooVtable = FooVtable \{}
    \NormalTok{destructor: }\CommentTok{/* compiler magic */}\NormalTok{,}
    \NormalTok{size: }\DecValTok{1}\NormalTok{,}
    \NormalTok{align: }\DecValTok{1}\NormalTok{,}

    \CommentTok{// cast to a function pointer}
    \NormalTok{method: call_method_on_u8 }\KeywordTok{as} \KeywordTok{fn}\NormalTok{(*}\KeywordTok{const} \NormalTok{()) -> }\DataTypeTok{String}\NormalTok{,}
\NormalTok{\};}


\CommentTok{// String:}

\KeywordTok{fn} \NormalTok{call_method_on_String(x: *}\KeywordTok{const} \NormalTok{()) -> }\DataTypeTok{String} \NormalTok{\{}
    \CommentTok{// the compiler guarantees that this function is only called}
    \CommentTok{// with `x` pointing to a String}
    \KeywordTok{let} \NormalTok{string: &}\DataTypeTok{String} \NormalTok{= }\KeywordTok{unsafe} \NormalTok{\{ &*(x }\KeywordTok{as} \NormalTok{*}\KeywordTok{const} \DataTypeTok{String}\NormalTok{) \};}

    \NormalTok{string.method()}
\NormalTok{\}}

\KeywordTok{static} \NormalTok{Foo_for_String_vtable: FooVtable = FooVtable \{}
    \NormalTok{destructor: }\CommentTok{/* compiler magic */}\NormalTok{,}
    \CommentTok{// values for a 64-bit computer, halve them for 32-bit ones}
    \NormalTok{size: }\DecValTok{24}\NormalTok{,}
    \NormalTok{align: }\DecValTok{8}\NormalTok{,}

    \NormalTok{method: call_method_on_String }\KeywordTok{as} \KeywordTok{fn}\NormalTok{(*}\KeywordTok{const} \NormalTok{()) -> }\DataTypeTok{String}\NormalTok{,}
\NormalTok{\};}
\end{Highlighting}
\end{Shaded}

The \texttt{destructor} field in each vtable points to a function that
will clean up any resources of the vtable's type: for \texttt{u8} it is
trivial, but for \texttt{String} it will free the memory. This is
necessary for owning trait objects like
\texttt{Box\textless{}Foo\textgreater{}}, which need to clean-up both
the \texttt{Box} allocation as well as the internal type when they go
out of scope. The \texttt{size} and \texttt{align} fields store the size
of the erased type, and its alignment requirements; these are
essentially unused at the moment since the information is embedded in
the destructor, but will be used in the future, as trait objects are
progressively made more flexible.

Suppose we've got some values that implement \texttt{Foo}. The explicit
form of construction and use of \texttt{Foo} trait objects might look a
bit like (ignoring the type mismatches: they're all pointers anyway):

\begin{Shaded}
\begin{Highlighting}[]
\KeywordTok{let} \NormalTok{a: }\DataTypeTok{String} \NormalTok{= }\StringTok{"foo"}\NormalTok{.to_string();}
\KeywordTok{let} \NormalTok{x: }\DataTypeTok{u8} \NormalTok{= }\DecValTok{1}\NormalTok{;}

\CommentTok{// let b: &Foo = &a;}
\KeywordTok{let} \NormalTok{b = TraitObject \{}
    \CommentTok{// store the data}
    \NormalTok{data: &a,}
    \CommentTok{// store the methods}
    \NormalTok{vtable: &Foo_for_String_vtable}
\NormalTok{\};}

\CommentTok{// let y: &Foo = x;}
\KeywordTok{let} \NormalTok{y = TraitObject \{}
    \CommentTok{// store the data}
    \NormalTok{data: &x,}
    \CommentTok{// store the methods}
    \NormalTok{vtable: &Foo_for_u8_vtable}
\NormalTok{\};}

\CommentTok{// b.method();}
\NormalTok{(b.vtable.method)(b.data);}

\CommentTok{// y.method();}
\NormalTok{(y.vtable.method)(y.data);}
\end{Highlighting}
\end{Shaded}

\subsubsection{Object Safety}\label{object-safety}

Not every trait can be used to make a trait object. For example, vectors
implement \texttt{Clone}, but if we try to make a trait object:

\begin{Shaded}
\begin{Highlighting}[]
\KeywordTok{let} \NormalTok{v = }\PreprocessorTok{vec!}\NormalTok{[}\DecValTok{1}\NormalTok{, }\DecValTok{2}\NormalTok{, }\DecValTok{3}\NormalTok{];}
\KeywordTok{let} \NormalTok{o = &v }\KeywordTok{as} \NormalTok{&}\BuiltInTok{Clone}\NormalTok{;}
\end{Highlighting}
\end{Shaded}

We get an error:

\begin{verbatim}
error: cannot convert to a trait object because trait `core::clone::Clone` is not obje
↳ ct-safe [E0038]
let o = &v as &Clone;
        ^~
note: the trait cannot require that `Self : Sized`
let o = &v as &Clone;
        ^~
\end{verbatim}

The error says that \texttt{Clone} is not `object-safe'. Only traits
that are object-safe can be made into trait objects. A trait is
object-safe if both of these are true:

\begin{itemize}
\tightlist
\item
  the trait does not require that \texttt{Self:\ Sized}
\item
  all of its methods are object-safe
\end{itemize}

So what makes a method object-safe? Each method must require that
\texttt{Self:\ Sized} or all of the following:

\begin{itemize}
\tightlist
\item
  must not have any type parameters
\item
  must not use \texttt{Self}
\end{itemize}

Whew! As we can see, almost all of these rules talk about \texttt{Self}.
A good intuition is ``except in special circumstances, if your trait's
method uses \texttt{Self}, it is not object-safe.''

\hypertarget{sec--closures}{\section{Closures}\label{sec--closures}}

Sometimes it is useful to wrap up a function and \emph{free variables}
for better clarity and reuse. The free variables that can be used come
from the enclosing scope and are `closed over' when used in the
function. From this, we get the name `closures' and Rust provides a
really great implementation of them, as we'll see.

\subsection{Syntax}\label{syntax}

Closures look like this:

\begin{Shaded}
\begin{Highlighting}[]
\KeywordTok{let} \NormalTok{plus_one = |x: }\DataTypeTok{i32}\NormalTok{| x + }\DecValTok{1}\NormalTok{;}

\PreprocessorTok{assert_eq!}\NormalTok{(}\DecValTok{2}\NormalTok{, plus_one(}\DecValTok{1}\NormalTok{));}
\end{Highlighting}
\end{Shaded}

We create a binding, \texttt{plus\_one}, and assign it to a closure. The
closure's arguments go between the pipes (\texttt{\textbar{}}), and the
body is an expression, in this case, \texttt{x\ +\ 1}. Remember that
\texttt{\{\ \}} is an expression, so we can have multi-line closures
too:

\begin{Shaded}
\begin{Highlighting}[]
\KeywordTok{let} \NormalTok{plus_two = |x| \{}
    \KeywordTok{let} \KeywordTok{mut} \NormalTok{result: }\DataTypeTok{i32} \NormalTok{= x;}

    \NormalTok{result += }\DecValTok{1}\NormalTok{;}
    \NormalTok{result += }\DecValTok{1}\NormalTok{;}

    \NormalTok{result}
\NormalTok{\};}

\PreprocessorTok{assert_eq!}\NormalTok{(}\DecValTok{4}\NormalTok{, plus_two(}\DecValTok{2}\NormalTok{));}
\end{Highlighting}
\end{Shaded}

You'll notice a few things about closures that are a bit different from
regular named functions defined with \texttt{fn}. The first is that we
did not need to annotate the types of arguments the closure takes or the
values it returns. We can:

\begin{Shaded}
\begin{Highlighting}[]
\KeywordTok{let} \NormalTok{plus_one = |x: }\DataTypeTok{i32}\NormalTok{| -> }\DataTypeTok{i32} \NormalTok{\{ x + }\DecValTok{1} \NormalTok{\};}

\PreprocessorTok{assert_eq!}\NormalTok{(}\DecValTok{2}\NormalTok{, plus_one(}\DecValTok{1}\NormalTok{));}
\end{Highlighting}
\end{Shaded}

But we don't have to. Why is this? Basically, it was chosen for
ergonomic reasons. While specifying the full type for named functions is
helpful with things like documentation and type inference, the full type
signatures of closures are rarely documented since they're anonymous,
and they don't cause the kinds of error-at-a-distance problems that
inferring named function types can.

The second is that the syntax is similar, but a bit different. I've
added spaces here for easier comparison:

\begin{Shaded}
\begin{Highlighting}[]
\KeywordTok{fn}  \NormalTok{plus_one_v1   (x: }\DataTypeTok{i32}\NormalTok{) -> }\DataTypeTok{i32} \NormalTok{\{ x + }\DecValTok{1} \NormalTok{\}}
\KeywordTok{let} \NormalTok{plus_one_v2 = |x: }\DataTypeTok{i32}\NormalTok{| -> }\DataTypeTok{i32} \NormalTok{\{ x + }\DecValTok{1} \NormalTok{\};}
\KeywordTok{let} \NormalTok{plus_one_v3 = |x: }\DataTypeTok{i32}\NormalTok{|          x + }\DecValTok{1}  \NormalTok{;}
\end{Highlighting}
\end{Shaded}

Small differences, but they're similar.

\subsection{Closures and their
environment}\label{closures-and-their-environment}

The environment for a closure can include bindings from its enclosing
scope in addition to parameters and local bindings. It looks like this:

\begin{Shaded}
\begin{Highlighting}[]
\KeywordTok{let} \NormalTok{num = }\DecValTok{5}\NormalTok{;}
\KeywordTok{let} \NormalTok{plus_num = |x: }\DataTypeTok{i32}\NormalTok{| x + num;}

\PreprocessorTok{assert_eq!}\NormalTok{(}\DecValTok{10}\NormalTok{, plus_num(}\DecValTok{5}\NormalTok{));}
\end{Highlighting}
\end{Shaded}

This closure, \texttt{plus\_num}, refers to a \texttt{let} binding in
its scope: \texttt{num}. More specifically, it borrows the binding. If
we do something that would conflict with that binding, we get an error.
Like this one:

\begin{Shaded}
\begin{Highlighting}[]
\KeywordTok{let} \KeywordTok{mut} \NormalTok{num = }\DecValTok{5}\NormalTok{;}
\KeywordTok{let} \NormalTok{plus_num = |x: }\DataTypeTok{i32}\NormalTok{| x + num;}

\KeywordTok{let} \NormalTok{y = &}\KeywordTok{mut} \NormalTok{num;}
\end{Highlighting}
\end{Shaded}

Which errors with:

\begin{verbatim}
error: cannot borrow `num` as mutable because it is also borrowed as immutable
    let y = &mut num;
                 ^~~
note: previous borrow of `num` occurs here due to use in closure; the immutable
  borrow prevents subsequent moves or mutable borrows of `num` until the borrow
  ends
    let plus_num = |x| x + num;
                   ^~~~~~~~~~~
note: previous borrow ends here
fn main() {
    let mut num = 5;
    let plus_num = |x| x + num;

    let y = &mut num;
}
^
\end{verbatim}

A verbose yet helpful error message! As it says, we can't take a mutable
borrow on \texttt{num} because the closure is already borrowing it. If
we let the closure go out of scope, we can:

\begin{Shaded}
\begin{Highlighting}[]
\KeywordTok{let} \KeywordTok{mut} \NormalTok{num = }\DecValTok{5}\NormalTok{;}
\NormalTok{\{}
    \KeywordTok{let} \NormalTok{plus_num = |x: }\DataTypeTok{i32}\NormalTok{| x + num;}

\NormalTok{\} }\CommentTok{// plus_num goes out of scope, borrow of num ends}

\KeywordTok{let} \NormalTok{y = &}\KeywordTok{mut} \NormalTok{num;}
\end{Highlighting}
\end{Shaded}

If your closure requires it, however, Rust will take ownership and move
the environment instead. This doesn't work:

\begin{Shaded}
\begin{Highlighting}[]
\KeywordTok{let} \NormalTok{nums = }\PreprocessorTok{vec!}\NormalTok{[}\DecValTok{1}\NormalTok{, }\DecValTok{2}\NormalTok{, }\DecValTok{3}\NormalTok{];}

\KeywordTok{let} \NormalTok{takes_nums = || nums;}

\PreprocessorTok{println!}\NormalTok{(}\StringTok{"\{:?\}"}\NormalTok{, nums);}
\end{Highlighting}
\end{Shaded}

We get this error:

\begin{verbatim}
note: `nums` moved into closure environment here because it has type
  `[closure(()) -> collections::vec::Vec<i32>]`, which is non-copyable
let takes_nums = || nums;
                 ^~~~~~~
\end{verbatim}

\texttt{Vec\textless{}T\textgreater{}} has ownership over its contents,
and therefore, when we refer to it in our closure, we have to take
ownership of \texttt{nums}. It's the same as if we'd passed
\texttt{nums} to a function that took ownership of it.

\hypertarget{move-closures}{\subsubsection{\texorpdfstring{\texttt{move}
closures}{move closures}}\label{move-closures}}

We can force our closure to take ownership of its environment with the
\texttt{move} keyword:

\begin{Shaded}
\begin{Highlighting}[]
\KeywordTok{let} \NormalTok{num = }\DecValTok{5}\NormalTok{;}

\KeywordTok{let} \NormalTok{owns_num = }\KeywordTok{move} \NormalTok{|x: }\DataTypeTok{i32}\NormalTok{| x + num;}
\end{Highlighting}
\end{Shaded}

Now, even though the keyword is \texttt{move}, the variables follow
normal move semantics. In this case, \texttt{5} implements
\texttt{Copy}, and so \texttt{owns\_num} takes ownership of a copy of
\texttt{num}. So what's the difference?

\begin{Shaded}
\begin{Highlighting}[]
\KeywordTok{let} \KeywordTok{mut} \NormalTok{num = }\DecValTok{5}\NormalTok{;}

\NormalTok{\{}
    \KeywordTok{let} \KeywordTok{mut} \NormalTok{add_num = |x: }\DataTypeTok{i32}\NormalTok{| num += x;}

    \NormalTok{add_num(}\DecValTok{5}\NormalTok{);}
\NormalTok{\}}

\PreprocessorTok{assert_eq!}\NormalTok{(}\DecValTok{10}\NormalTok{, num);}
\end{Highlighting}
\end{Shaded}

So in this case, our closure took a mutable reference to \texttt{num},
and then when we called \texttt{add\_num}, it mutated the underlying
value, as we'd expect. We also needed to declare \texttt{add\_num} as
\texttt{mut} too, because we're mutating its environment.

If we change to a \texttt{move} closure, it's different:

\begin{Shaded}
\begin{Highlighting}[]
\KeywordTok{let} \KeywordTok{mut} \NormalTok{num = }\DecValTok{5}\NormalTok{;}

\NormalTok{\{}
    \KeywordTok{let} \KeywordTok{mut} \NormalTok{add_num = }\KeywordTok{move} \NormalTok{|x: }\DataTypeTok{i32}\NormalTok{| num += x;}

    \NormalTok{add_num(}\DecValTok{5}\NormalTok{);}
\NormalTok{\}}

\PreprocessorTok{assert_eq!}\NormalTok{(}\DecValTok{5}\NormalTok{, num);}
\end{Highlighting}
\end{Shaded}

We only get \texttt{5}. Rather than taking a mutable borrow out on our
\texttt{num}, we took ownership of a copy.

Another way to think about \texttt{move} closures: they give a closure
its own stack frame. Without \texttt{move}, a closure may be tied to the
stack frame that created it, while a \texttt{move} closure is
self-contained. This means that you cannot generally return a
non-\texttt{move} closure from a function, for example.

But before we talk about taking and returning closures, we should talk
some more about the way that closures are implemented. As a systems
language, Rust gives you tons of control over what your code does, and
closures are no different.

\subsection{Closure implementation}\label{closure-implementation}

Rust's implementation of closures is a bit different than other
languages. They are effectively syntax sugar for traits. You'll want to
make sure to have read the \protect\hyperlink{sec--traits}{traits}
section before this one, as well as the section on
\protect\hyperlink{sec--trait-objects}{trait objects}.

Got all that? Good.

The key to understanding how closures work under the hood is something a
bit strange: Using \texttt{()} to call a function, like \texttt{foo()},
is an overloadable operator. From this, everything else clicks into
place. In Rust, we use the trait system to overload operators. Calling
functions is no different. We have three separate traits to overload
with:

\begin{Shaded}
\begin{Highlighting}[]
\KeywordTok{pub} \KeywordTok{trait} \BuiltInTok{Fn}\NormalTok{<Args> : }\BuiltInTok{FnMut}\NormalTok{<Args> \{}
    \KeywordTok{extern} \StringTok{"rust-call"} \KeywordTok{fn} \NormalTok{call(&}\KeywordTok{self}\NormalTok{, args: Args) -> }\KeywordTok{Self}\NormalTok{::Output;}
\NormalTok{\}}

\KeywordTok{pub} \KeywordTok{trait} \BuiltInTok{FnMut}\NormalTok{<Args> : }\BuiltInTok{FnOnce}\NormalTok{<Args> \{}
    \KeywordTok{extern} \StringTok{"rust-call"} \KeywordTok{fn} \NormalTok{call_mut(&}\KeywordTok{mut} \KeywordTok{self}\NormalTok{, args: Args) -> }\KeywordTok{Self}\NormalTok{::Output;}
\NormalTok{\}}

\KeywordTok{pub} \KeywordTok{trait} \BuiltInTok{FnOnce}\NormalTok{<Args> \{}
    \KeywordTok{type} \NormalTok{Output;}

    \KeywordTok{extern} \StringTok{"rust-call"} \KeywordTok{fn} \NormalTok{call_once(}\KeywordTok{self}\NormalTok{, args: Args) -> }\KeywordTok{Self}\NormalTok{::Output;}
\NormalTok{\}}
\end{Highlighting}
\end{Shaded}

You'll notice a few differences between these traits, but a big one is
\texttt{self}: \texttt{Fn} takes \texttt{\&self}, \texttt{FnMut} takes
\texttt{\&mut\ self}, and \texttt{FnOnce} takes \texttt{self}. This
covers all three kinds of \texttt{self} via the usual method call
syntax. But we've split them up into three traits, rather than having a
single one. This gives us a large amount of control over what kind of
closures we can take.

The \texttt{\textbar{}\textbar{}\ \{\}} syntax for closures is sugar for
these three traits. Rust will generate a struct for the environment,
\texttt{impl} the appropriate trait, and then use it.

\subsection{Taking closures as
arguments}\label{taking-closures-as-arguments}

Now that we know that closures are traits, we already know how to accept
and return closures: the same as any other trait!

This also means that we can choose static vs dynamic dispatch as well.
First, let's write a function which takes something callable, calls it,
and returns the result:

\begin{Shaded}
\begin{Highlighting}[]
\KeywordTok{fn} \NormalTok{call_with_one<F>(some_closure: F) -> }\DataTypeTok{i32}
    \KeywordTok{where} \NormalTok{F : }\BuiltInTok{Fn}\NormalTok{(}\DataTypeTok{i32}\NormalTok{) -> }\DataTypeTok{i32} \NormalTok{\{}

    \NormalTok{some_closure(}\DecValTok{1}\NormalTok{)}
\NormalTok{\}}

\KeywordTok{let} \NormalTok{answer = call_with_one(|x| x + }\DecValTok{2}\NormalTok{);}

\PreprocessorTok{assert_eq!}\NormalTok{(}\DecValTok{3}\NormalTok{, answer);}
\end{Highlighting}
\end{Shaded}

We pass our closure, \texttt{\textbar{}x\textbar{}\ x\ +\ 2}, to
\texttt{call\_with\_one}. It does what it suggests: it calls the
closure, giving it \texttt{1} as an argument.

Let's examine the signature of \texttt{call\_with\_one} in more depth:

\begin{Shaded}
\begin{Highlighting}[]
\KeywordTok{fn} \NormalTok{call_with_one<F>(some_closure: F) -> }\DataTypeTok{i32}
\end{Highlighting}
\end{Shaded}

We take one parameter, and it has the type \texttt{F}. We also return a
\texttt{i32}. This part isn't interesting. The next part is:

\begin{Shaded}
\begin{Highlighting}[]
    \KeywordTok{where} \NormalTok{F : }\BuiltInTok{Fn}\NormalTok{(}\DataTypeTok{i32}\NormalTok{) -> }\DataTypeTok{i32} \NormalTok{\{}
\end{Highlighting}
\end{Shaded}

Because \texttt{Fn} is a trait, we can bound our generic with it. In
this case, our closure takes a \texttt{i32} as an argument and returns
an \texttt{i32}, and so the generic bound we use is
\texttt{Fn(i32)\ -\textgreater{}\ i32}.

There's one other key point here: because we're bounding a generic with
a trait, this will get monomorphized, and therefore, we'll be doing
static dispatch into the closure. That's pretty neat. In many languages,
closures are inherently heap allocated, and will always involve dynamic
dispatch. In Rust, we can stack allocate our closure environment, and
statically dispatch the call. This happens quite often with iterators
and their adapters, which often take closures as arguments.

Of course, if we want dynamic dispatch, we can get that too. A trait
object handles this case, as usual:

\begin{Shaded}
\begin{Highlighting}[]
\KeywordTok{fn} \NormalTok{call_with_one(some_closure: &}\BuiltInTok{Fn}\NormalTok{(}\DataTypeTok{i32}\NormalTok{) -> }\DataTypeTok{i32}\NormalTok{) -> }\DataTypeTok{i32} \NormalTok{\{}
    \NormalTok{some_closure(}\DecValTok{1}\NormalTok{)}
\NormalTok{\}}

\KeywordTok{let} \NormalTok{answer = call_with_one(&|x| x + }\DecValTok{2}\NormalTok{);}

\PreprocessorTok{assert_eq!}\NormalTok{(}\DecValTok{3}\NormalTok{, answer);}
\end{Highlighting}
\end{Shaded}

Now we take a trait object, a \texttt{\&Fn}. And we have to make a
reference to our closure when we pass it to \texttt{call\_with\_one}, so
we use \texttt{\&\textbar{}\textbar{}}.

\subsection{Function pointers and
closures}\label{function-pointers-and-closures}

A function pointer is kind of like a closure that has no environment. As
such, you can pass a function pointer to any function expecting a
closure argument, and it will work:

\begin{Shaded}
\begin{Highlighting}[]
\KeywordTok{fn} \NormalTok{call_with_one(some_closure: &}\BuiltInTok{Fn}\NormalTok{(}\DataTypeTok{i32}\NormalTok{) -> }\DataTypeTok{i32}\NormalTok{) -> }\DataTypeTok{i32} \NormalTok{\{}
    \NormalTok{some_closure(}\DecValTok{1}\NormalTok{)}
\NormalTok{\}}

\KeywordTok{fn} \NormalTok{add_one(i: }\DataTypeTok{i32}\NormalTok{) -> }\DataTypeTok{i32} \NormalTok{\{}
    \NormalTok{i + }\DecValTok{1}
\NormalTok{\}}

\KeywordTok{let} \NormalTok{f = add_one;}

\KeywordTok{let} \NormalTok{answer = call_with_one(&f);}

\PreprocessorTok{assert_eq!}\NormalTok{(}\DecValTok{2}\NormalTok{, answer);}
\end{Highlighting}
\end{Shaded}

In this example, we don't strictly need the intermediate variable
\texttt{f}, the name of the function works just fine too:

\begin{Shaded}
\begin{Highlighting}[]
\KeywordTok{let} \NormalTok{answer = call_with_one(&add_one);}
\end{Highlighting}
\end{Shaded}

\subsection{Returning closures}\label{returning-closures}

It's very common for functional-style code to return closures in various
situations. If you try to return a closure, you may run into an error.
At first, it may seem strange, but we'll figure it out. Here's how you'd
probably try to return a closure from a function:

\begin{Shaded}
\begin{Highlighting}[]
\KeywordTok{fn} \NormalTok{factory() -> (}\BuiltInTok{Fn}\NormalTok{(}\DataTypeTok{i32}\NormalTok{) -> }\DataTypeTok{i32}\NormalTok{) \{}
    \KeywordTok{let} \NormalTok{num = }\DecValTok{5}\NormalTok{;}

    \NormalTok{|x| x + num}
\NormalTok{\}}

\KeywordTok{let} \NormalTok{f = factory();}

\KeywordTok{let} \NormalTok{answer = f(}\DecValTok{1}\NormalTok{);}
\PreprocessorTok{assert_eq!}\NormalTok{(}\DecValTok{6}\NormalTok{, answer);}
\end{Highlighting}
\end{Shaded}

This gives us these long, related errors:

\begin{verbatim}
error: the trait bound `core::ops::Fn(i32) -> i32 : core::marker::Sized` is not satisf
↳ ied [E0277]
fn factory() -> (Fn(i32) -> i32) {
                ^~~~~~~~~~~~~~~~
note: `core::ops::Fn(i32) -> i32` does not have a constant size known at compile-time
fn factory() -> (Fn(i32) -> i32) {
                ^~~~~~~~~~~~~~~~
error: the trait bound `core::ops::Fn(i32) -> i32 : core::marker::Sized` is not satisf
↳ ied [E0277]
let f = factory();
    ^
note: `core::ops::Fn(i32) -> i32` does not have a constant size known at compile-time
let f = factory();
    ^
\end{verbatim}

In order to return something from a function, Rust needs to know what
size the return type is. But since \texttt{Fn} is a trait, it could be
various things of various sizes: many different types can implement
\texttt{Fn}. An easy way to give something a size is to take a reference
to it, as references have a known size. So we'd write this:

\begin{Shaded}
\begin{Highlighting}[]
\KeywordTok{fn} \NormalTok{factory() -> &(}\BuiltInTok{Fn}\NormalTok{(}\DataTypeTok{i32}\NormalTok{) -> }\DataTypeTok{i32}\NormalTok{) \{}
    \KeywordTok{let} \NormalTok{num = }\DecValTok{5}\NormalTok{;}

    \NormalTok{|x| x + num}
\NormalTok{\}}

\KeywordTok{let} \NormalTok{f = factory();}

\KeywordTok{let} \NormalTok{answer = f(}\DecValTok{1}\NormalTok{);}
\PreprocessorTok{assert_eq!}\NormalTok{(}\DecValTok{6}\NormalTok{, answer);}
\end{Highlighting}
\end{Shaded}

But we get another error:

\begin{verbatim}
error: missing lifetime specifier [E0106]
fn factory() -> &(Fn(i32) -> i32) {
                ^~~~~~~~~~~~~~~~~
\end{verbatim}

Right. Because we have a reference, we need to give it a lifetime. But
our \texttt{factory()} function takes no arguments, so
\protect\hyperlink{lifetime-elision}{elision} doesn't kick in here. Then
what choices do we have? Try \texttt{\textquotesingle{}static}:

\begin{Shaded}
\begin{Highlighting}[]
\KeywordTok{fn} \NormalTok{factory() -> &}\OtherTok{'static} \NormalTok{(}\BuiltInTok{Fn}\NormalTok{(}\DataTypeTok{i32}\NormalTok{) -> }\DataTypeTok{i32}\NormalTok{) \{}
    \KeywordTok{let} \NormalTok{num = }\DecValTok{5}\NormalTok{;}

    \NormalTok{|x| x + num}
\NormalTok{\}}

\KeywordTok{let} \NormalTok{f = factory();}

\KeywordTok{let} \NormalTok{answer = f(}\DecValTok{1}\NormalTok{);}
\PreprocessorTok{assert_eq!}\NormalTok{(}\DecValTok{6}\NormalTok{, answer);}
\end{Highlighting}
\end{Shaded}

But we get another error:

\begin{verbatim}
error: mismatched types:
 expected `&'static core::ops::Fn(i32) -> i32`,
    found `[closure@<anon>:7:9: 7:20]`
(expected &-ptr,
    found closure) [E0308]
         |x| x + num
         ^~~~~~~~~~~
\end{verbatim}

This error is letting us know that we don't have a
\texttt{\&\textquotesingle{}static\ Fn(i32)\ -\textgreater{}\ i32}, we
have a \texttt{{[}closure@\textless{}anon\textgreater{}:7:9:\ 7:20{]}}.
Wait, what?

Because each closure generates its own environment \texttt{struct} and
implementation of \texttt{Fn} and friends, these types are anonymous.
They exist solely for this closure. So Rust shows them as
\texttt{closure@\textless{}anon\textgreater{}}, rather than some
autogenerated name.

The error also points out that the return type is expected to be a
reference, but what we are trying to return is not. Further, we cannot
directly assign a \texttt{\textquotesingle{}static} lifetime to an
object. So we'll take a different approach and return a `trait object'
by \texttt{Box}ing up the \texttt{Fn}. This \emph{almost} works:

\begin{Shaded}
\begin{Highlighting}[]
\KeywordTok{fn} \NormalTok{factory() -> }\DataTypeTok{Box}\NormalTok{<}\BuiltInTok{Fn}\NormalTok{(}\DataTypeTok{i32}\NormalTok{) -> }\DataTypeTok{i32}\NormalTok{> \{}
    \KeywordTok{let} \NormalTok{num = }\DecValTok{5}\NormalTok{;}

    \DataTypeTok{Box}\NormalTok{::new(|x| x + num)}
\NormalTok{\}}
\KeywordTok{let} \NormalTok{f = factory();}

\KeywordTok{let} \NormalTok{answer = f(}\DecValTok{1}\NormalTok{);}
\PreprocessorTok{assert_eq!}\NormalTok{(}\DecValTok{6}\NormalTok{, answer);}
\end{Highlighting}
\end{Shaded}

There's just one last problem:

\begin{verbatim}
error: closure may outlive the current function, but it borrows `num`,
which is owned by the current function [E0373]
Box::new(|x| x + num)
         ^~~~~~~~~~~
\end{verbatim}

Well, as we discussed before, closures borrow their environment. And in
this case, our environment is based on a stack-allocated \texttt{5}, the
\texttt{num} variable binding. So the borrow has a lifetime of the stack
frame. So if we returned this closure, the function call would be over,
the stack frame would go away, and our closure is capturing an
environment of garbage memory! With one last fix, we can make this work:

\begin{Shaded}
\begin{Highlighting}[]
\KeywordTok{fn} \NormalTok{factory() -> }\DataTypeTok{Box}\NormalTok{<}\BuiltInTok{Fn}\NormalTok{(}\DataTypeTok{i32}\NormalTok{) -> }\DataTypeTok{i32}\NormalTok{> \{}
    \KeywordTok{let} \NormalTok{num = }\DecValTok{5}\NormalTok{;}

    \DataTypeTok{Box}\NormalTok{::new(}\KeywordTok{move} \NormalTok{|x| x + num)}
\NormalTok{\}}
\KeywordTok{fn} \NormalTok{main() \{}
\KeywordTok{let} \NormalTok{f = factory();}

\KeywordTok{let} \NormalTok{answer = f(}\DecValTok{1}\NormalTok{);}
\PreprocessorTok{assert_eq!}\NormalTok{(}\DecValTok{6}\NormalTok{, answer);}
\NormalTok{\}}
\end{Highlighting}
\end{Shaded}

By making the inner closure a \texttt{move\ Fn}, we create a new stack
frame for our closure. By \texttt{Box}ing it up, we've given it a known
size, allowing it to escape our stack frame.

\hypertarget{sec--ufcs}{\section{Universal Function Call
Syntax}\label{sec--ufcs}}

Sometimes, functions can have the same names. Consider this code:

\begin{Shaded}
\begin{Highlighting}[]
\KeywordTok{trait} \NormalTok{Foo \{}
    \KeywordTok{fn} \NormalTok{f(&}\KeywordTok{self}\NormalTok{);}
\NormalTok{\}}

\KeywordTok{trait} \NormalTok{Bar \{}
    \KeywordTok{fn} \NormalTok{f(&}\KeywordTok{self}\NormalTok{);}
\NormalTok{\}}

\KeywordTok{struct} \NormalTok{Baz;}

\KeywordTok{impl} \NormalTok{Foo }\KeywordTok{for} \NormalTok{Baz \{}
    \KeywordTok{fn} \NormalTok{f(&}\KeywordTok{self}\NormalTok{) \{ }\PreprocessorTok{println!}\NormalTok{(}\StringTok{"Baz’s impl of Foo"}\NormalTok{); \}}
\NormalTok{\}}

\KeywordTok{impl} \NormalTok{Bar }\KeywordTok{for} \NormalTok{Baz \{}
    \KeywordTok{fn} \NormalTok{f(&}\KeywordTok{self}\NormalTok{) \{ }\PreprocessorTok{println!}\NormalTok{(}\StringTok{"Baz’s impl of Bar"}\NormalTok{); \}}
\NormalTok{\}}

\KeywordTok{let} \NormalTok{b = Baz;}
\end{Highlighting}
\end{Shaded}

If we were to try to call \texttt{b.f()}, we'd get an error:

\begin{verbatim}
error: multiple applicable methods in scope [E0034]
b.f();
  ^~~
note: candidate #1 is defined in an impl of the trait `main::Foo` for the type
`main::Baz`
    fn f(&self) { println!("Baz’s impl of Foo"); }
    ^~~~~~~~~~~~~~~~~~~~~~~~~~~~~~~~~~~~~~~~~~~~~~
note: candidate #2 is defined in an impl of the trait `main::Bar` for the type
`main::Baz`
    fn f(&self) { println!("Baz’s impl of Bar"); }
    ^~~~~~~~~~~~~~~~~~~~~~~~~~~~~~~~~~~~~~~~~~~~~~
\end{verbatim}

We need a way to disambiguate which method we need. This feature is
called `universal function call syntax', and it looks like this:

\begin{Shaded}
\begin{Highlighting}[]
\NormalTok{Foo::f(&b);}
\NormalTok{Bar::f(&b);}
\end{Highlighting}
\end{Shaded}

Let's break it down.

\begin{Shaded}
\begin{Highlighting}[]
\NormalTok{Foo::}
\NormalTok{Bar::}
\end{Highlighting}
\end{Shaded}

These halves of the invocation are the types of the two traits:
\texttt{Foo} and \texttt{Bar}. This is what ends up actually doing the
disambiguation between the two: Rust calls the one from the trait name
you use.

\begin{Shaded}
\begin{Highlighting}[]
\NormalTok{f(&b)}
\end{Highlighting}
\end{Shaded}

When we call a method like \texttt{b.f()} using
\protect\hyperlink{sec--method-syntax}{method syntax}, Rust will
automatically borrow \texttt{b} if \texttt{f()} takes \texttt{\&self}.
In this case, Rust will not, and so we need to pass an explicit
\texttt{\&b}.

\subsection{Angle-bracket Form}\label{angle-bracket-form}

The form of UFCS we just talked about:

\begin{Shaded}
\begin{Highlighting}[]
\NormalTok{Trait::method(args);}
\end{Highlighting}
\end{Shaded}

Is a short-hand. There's an expanded form of this that's needed in some
situations:

\begin{Shaded}
\begin{Highlighting}[]
\NormalTok{<Type }\KeywordTok{as} \NormalTok{Trait>::method(args);}
\end{Highlighting}
\end{Shaded}

The \texttt{\textless{}\textgreater{}::} syntax is a means of providing
a type hint. The type goes inside the
\texttt{\textless{}\textgreater{}}s. In this case, the type is
\texttt{Type\ as\ Trait}, indicating that we want \texttt{Trait}'s
version of \texttt{method} to be called here. The \texttt{as\ Trait}
part is optional if it's not ambiguous. Same with the angle brackets,
hence the shorter form.

Here's an example of using the longer form.

\begin{Shaded}
\begin{Highlighting}[]
\KeywordTok{trait} \NormalTok{Foo \{}
    \KeywordTok{fn} \NormalTok{foo() -> }\DataTypeTok{i32}\NormalTok{;}
\NormalTok{\}}

\KeywordTok{struct} \NormalTok{Bar;}

\KeywordTok{impl} \NormalTok{Bar \{}
    \KeywordTok{fn} \NormalTok{foo() -> }\DataTypeTok{i32} \NormalTok{\{}
        \DecValTok{20}
    \NormalTok{\}}
\NormalTok{\}}

\KeywordTok{impl} \NormalTok{Foo }\KeywordTok{for} \NormalTok{Bar \{}
    \KeywordTok{fn} \NormalTok{foo() -> }\DataTypeTok{i32} \NormalTok{\{}
        \DecValTok{10}
    \NormalTok{\}}
\NormalTok{\}}

\KeywordTok{fn} \NormalTok{main() \{}
    \PreprocessorTok{assert_eq!}\NormalTok{(}\DecValTok{10}\NormalTok{, <Bar }\KeywordTok{as} \NormalTok{Foo>::foo());}
    \PreprocessorTok{assert_eq!}\NormalTok{(}\DecValTok{20}\NormalTok{, Bar::foo());}
\NormalTok{\}}
\end{Highlighting}
\end{Shaded}

Using the angle bracket syntax lets you call the trait method instead of
the inherent one.

\hypertarget{sec--crates-and-modules}{\section{Crates and
Modules}\label{sec--crates-and-modules}}

When a project starts getting large, it's considered good software
engineering practice to split it up into a bunch of smaller pieces, and
then fit them together. It is also important to have a well-defined
interface, so that some of your functionality is private, and some is
public. To facilitate these kinds of things, Rust has a module system.

\subsection{Basic terminology: Crates and
Modules}\label{basic-terminology-crates-and-modules}

Rust has two distinct terms that relate to the module system: `crate'
and `module'. A crate is synonymous with a `library' or `package' in
other languages. Hence ``Cargo'' as the name of Rust's package
management tool: you ship your crates to others with Cargo. Crates can
produce an executable or a library, depending on the project.

Each crate has an implicit \emph{root module} that contains the code for
that crate. You can then define a tree of sub-modules under that root
module. Modules allow you to partition your code within the crate
itself.

As an example, let's make a \emph{phrases} crate, which will give us
various phrases in different languages. To keep things simple, we'll
stick to `greetings' and `farewells' as two kinds of phrases, and use
English and Japanese (日本語) as two languages for those phrases to be
in. We'll use this module layout:

\begin{verbatim}
                                    +-----------+
                                +---| greetings |
                  +---------+   |   +-----------+
              +---| english |---+
              |   +---------+   |   +-----------+
              |                 +---| farewells |
+---------+   |                     +-----------+
| phrases |---+
+---------+   |                     +-----------+
              |                 +---| greetings |
              |   +----------+  |   +-----------+
              +---| japanese |--+
                  +----------+  |   +-----------+
                                +---| farewells |
                                    +-----------+
\end{verbatim}

In this example, \texttt{phrases} is the name of our crate. All of the
rest are modules. You can see that they form a tree, branching out from
the crate \emph{root}, which is the root of the tree: \texttt{phrases}
itself.

Now that we have a plan, let's define these modules in code. To start,
generate a new crate with Cargo:

\begin{Shaded}
\begin{Highlighting}[]
\NormalTok{$ }\KeywordTok{cargo} \NormalTok{new phrases}
\NormalTok{$ }\KeywordTok{cd} \NormalTok{phrases}
\end{Highlighting}
\end{Shaded}

If you remember, this generates a simple project for us:

\begin{Shaded}
\begin{Highlighting}[]
\NormalTok{$ }\KeywordTok{tree} \NormalTok{.}
\KeywordTok{.}
\NormalTok{├── }\KeywordTok{Cargo.toml}
\NormalTok{└── }\KeywordTok{src}
    \NormalTok{└── }\KeywordTok{lib.rs}

\KeywordTok{1} \NormalTok{directory, 2 files}
\end{Highlighting}
\end{Shaded}

\texttt{src/lib.rs} is our crate root, corresponding to the
\texttt{phrases} in our diagram above.

\subsection{Defining Modules}\label{defining-modules}

To define each of our modules, we use the \texttt{mod} keyword. Let's
make our \texttt{src/lib.rs} look like this:

\begin{Shaded}
\begin{Highlighting}[]
\KeywordTok{mod} \NormalTok{english \{}
    \KeywordTok{mod} \NormalTok{greetings \{}
    \NormalTok{\}}

    \KeywordTok{mod} \NormalTok{farewells \{}
    \NormalTok{\}}
\NormalTok{\}}

\KeywordTok{mod} \NormalTok{japanese \{}
    \KeywordTok{mod} \NormalTok{greetings \{}
    \NormalTok{\}}

    \KeywordTok{mod} \NormalTok{farewells \{}
    \NormalTok{\}}
\NormalTok{\}}
\end{Highlighting}
\end{Shaded}

After the \texttt{mod} keyword, you give the name of the module. Module
names follow the conventions for other Rust identifiers:
\texttt{lower\_snake\_case}. The contents of each module are within
curly braces (\texttt{\{\}}).

Within a given \texttt{mod}, you can declare sub-\texttt{mod}s. We can
refer to sub-modules with double-colon (\texttt{::}) notation: our four
nested modules are \texttt{english::greetings},
\texttt{english::farewells}, \texttt{japanese::greetings}, and
\texttt{japanese::farewells}. Because these sub-modules are namespaced
under their parent module, the names don't conflict:
\texttt{english::greetings} and \texttt{japanese::greetings} are
distinct, even though their names are both \texttt{greetings}.

Because this crate does not have a \texttt{main()} function, and is
called \texttt{lib.rs}, Cargo will build this crate as a library:

\begin{Shaded}
\begin{Highlighting}[]
\NormalTok{$ }\KeywordTok{cargo} \NormalTok{build}
   \KeywordTok{Compiling} \NormalTok{phrases v0.0.1 (file:///home/you/projects/phrases)}
\NormalTok{$ }\KeywordTok{ls} \NormalTok{target/debug}
\KeywordTok{build}  \NormalTok{deps  examples  libphrases-a7448e02a0468eaa.rlib  native}
\end{Highlighting}
\end{Shaded}

\texttt{libphrases-\textless{}hash\textgreater{}.rlib} is the compiled
crate. Before we see how to use this crate from another crate, let's
break it up into multiple files.

\subsection{Multiple File Crates}\label{multiple-file-crates}

If each crate were just one file, these files would get very large. It's
often easier to split up crates into multiple files, and Rust supports
this in two ways.

Instead of declaring a module like this:

\begin{Shaded}
\begin{Highlighting}[]
\KeywordTok{mod} \NormalTok{english \{}
    \CommentTok{// contents of our module go here}
\NormalTok{\}}
\end{Highlighting}
\end{Shaded}

We can instead declare our module like this:

\begin{Shaded}
\begin{Highlighting}[]
\KeywordTok{mod} \NormalTok{english;}
\end{Highlighting}
\end{Shaded}

If we do that, Rust will expect to find either a \texttt{english.rs}
file, or a \texttt{english/mod.rs} file with the contents of our module.

Note that in these files, you don't need to re-declare the module:
that's already been done with the initial \texttt{mod} declaration.

Using these two techniques, we can break up our crate into two
directories and seven files:

\begin{Shaded}
\begin{Highlighting}[]
\NormalTok{$ }\KeywordTok{tree} \NormalTok{.}
\KeywordTok{.}
\NormalTok{├── }\KeywordTok{Cargo.lock}
\NormalTok{├── }\KeywordTok{Cargo.toml}
\NormalTok{├── }\KeywordTok{src}
\NormalTok{│   ├── }\KeywordTok{english}
\NormalTok{│   │   ├── }\KeywordTok{farewells.rs}
\NormalTok{│   │   ├── }\KeywordTok{greetings.rs}
\NormalTok{│   │   └── }\KeywordTok{mod.rs}
\NormalTok{│   ├── }\KeywordTok{japanese}
\NormalTok{│   │   ├── }\KeywordTok{farewells.rs}
\NormalTok{│   │   ├── }\KeywordTok{greetings.rs}
\NormalTok{│   │   └── }\KeywordTok{mod.rs}
\NormalTok{│   └── }\KeywordTok{lib.rs}
\NormalTok{└── }\KeywordTok{target}
    \NormalTok{└── }\KeywordTok{debug}
        \NormalTok{├── }\KeywordTok{build}
        \NormalTok{├── }\KeywordTok{deps}
        \NormalTok{├── }\KeywordTok{examples}
        \NormalTok{├── }\KeywordTok{libphrases-a7448e02a0468eaa.rlib}
        \NormalTok{└── }\KeywordTok{native}
\end{Highlighting}
\end{Shaded}

\texttt{src/lib.rs} is our crate root, and looks like this:

\begin{Shaded}
\begin{Highlighting}[]
\KeywordTok{mod} \NormalTok{english;}
\KeywordTok{mod} \NormalTok{japanese;}
\end{Highlighting}
\end{Shaded}

These two declarations tell Rust to look for either
\texttt{src/english.rs} and \texttt{src/japanese.rs}, or
\texttt{src/english/mod.rs} and \texttt{src/japanese/mod.rs}, depending
on our preference. In this case, because our modules have sub-modules,
we've chosen the second. Both \texttt{src/english/mod.rs} and
\texttt{src/japanese/mod.rs} look like this:

\begin{Shaded}
\begin{Highlighting}[]
\KeywordTok{mod} \NormalTok{greetings;}
\KeywordTok{mod} \NormalTok{farewells;}
\end{Highlighting}
\end{Shaded}

Again, these declarations tell Rust to look for either
\texttt{src/english/greetings.rs}, \texttt{src/english/farewells.rs},
\texttt{src/japanese/greetings.rs} and
\texttt{src/japanese/farewells.rs} or
\texttt{src/english/greetings/mod.rs},
\texttt{src/english/farewells/mod.rs},
\texttt{src/japanese/greetings/mod.rs} and
\texttt{src/japanese/farewells/mod.rs}. Because these sub-modules don't
have their own sub-modules, we've chosen to make them
\texttt{src/english/greetings.rs}, \texttt{src/english/farewells.rs},
\texttt{src/japanese/greetings.rs} and
\texttt{src/japanese/farewells.rs}. Whew!

The contents of \texttt{src/english/greetings.rs},
\texttt{src/english/farewells.rs}, \texttt{src/japanese/greetings.rs}
and \texttt{src/japanese/farewells.rs} are all empty at the moment.
Let's add some functions.

Put this in \texttt{src/english/greetings.rs}:

\begin{Shaded}
\begin{Highlighting}[]
\KeywordTok{fn} \NormalTok{hello() -> }\DataTypeTok{String} \NormalTok{\{}
    \StringTok{"Hello!"}\NormalTok{.to_string()}
\NormalTok{\}}
\end{Highlighting}
\end{Shaded}

Put this in \texttt{src/english/farewells.rs}:

\begin{Shaded}
\begin{Highlighting}[]
\KeywordTok{fn} \NormalTok{goodbye() -> }\DataTypeTok{String} \NormalTok{\{}
    \StringTok{"Goodbye."}\NormalTok{.to_string()}
\NormalTok{\}}
\end{Highlighting}
\end{Shaded}

Put this in \texttt{src/japanese/greetings.rs}:

\begin{Shaded}
\begin{Highlighting}[]
\KeywordTok{fn} \NormalTok{hello() -> }\DataTypeTok{String} \NormalTok{\{}
    \StringTok{"こんにちは"}\NormalTok{.to_string()}
\NormalTok{\}}
\end{Highlighting}
\end{Shaded}

Of course, you can copy and paste this from this web page, or type
something else. It's not important that you actually put `konnichiwa' to
learn about the module system.

Put this in \texttt{src/japanese/farewells.rs}:

\begin{Shaded}
\begin{Highlighting}[]
\KeywordTok{fn} \NormalTok{goodbye() -> }\DataTypeTok{String} \NormalTok{\{}
    \StringTok{"さようなら"}\NormalTok{.to_string()}
\NormalTok{\}}
\end{Highlighting}
\end{Shaded}

(This is `Sayōnara', if you're curious.)

Now that we have some functionality in our crate, let's try to use it
from another crate.

\subsection{Importing External Crates}\label{importing-external-crates}

We have a library crate. Let's make an executable crate that imports and
uses our library.

Make a \texttt{src/main.rs} and put this in it (it won't quite compile
yet):

\begin{Shaded}
\begin{Highlighting}[]
\KeywordTok{extern} \KeywordTok{crate} \NormalTok{phrases;}

\KeywordTok{fn} \NormalTok{main() \{}
    \PreprocessorTok{println!}\NormalTok{(}\StringTok{"Hello in English: \{\}"}\NormalTok{, phrases::english::greetings::hello());}
    \PreprocessorTok{println!}\NormalTok{(}\StringTok{"Goodbye in English: \{\}"}\NormalTok{, phrases::english::farewells::goodbye());}

    \PreprocessorTok{println!}\NormalTok{(}\StringTok{"Hello in Japanese: \{\}"}\NormalTok{, phrases::japanese::greetings::hello());}
    \PreprocessorTok{println!}\NormalTok{(}\StringTok{"Goodbye in Japanese: \{\}"}\NormalTok{, phrases::japanese::farewells::goodbye());}
\NormalTok{\}}
\end{Highlighting}
\end{Shaded}

The \texttt{extern\ crate} declaration tells Rust that we need to
compile and link to the \texttt{phrases} crate. We can then use
\texttt{phrases}' modules in this one. As we mentioned earlier, you can
use double colons to refer to sub-modules and the functions inside of
them.

(Note: when importing a crate that has dashes in its name ``like-this'',
which is not a valid Rust identifier, it will be converted by changing
the dashes to underscores, so you would write
\texttt{extern\ crate\ like\_this;}.)

Also, Cargo assumes that \texttt{src/main.rs} is the crate root of a
binary crate, rather than a library crate. Our package now has two
crates: \texttt{src/lib.rs} and \texttt{src/main.rs}. This pattern is
quite common for executable crates: most functionality is in a library
crate, and the executable crate uses that library. This way, other
programs can also use the library crate, and it's also a nice separation
of concerns.

This doesn't quite work yet, though. We get four errors that look
similar to this:

\begin{Shaded}
\begin{Highlighting}[]
\NormalTok{$ }\KeywordTok{cargo} \NormalTok{build}
   \KeywordTok{Compiling} \NormalTok{phrases v0.0.1 (file:///home/you/projects/phrases)}
\KeywordTok{src}\NormalTok{/main.rs:}\KeywordTok{4}\NormalTok{:38: 4:72 error: function }\KeywordTok{`hello`} \NormalTok{is private}
\KeywordTok{src}\NormalTok{/main.rs:}\KeywordTok{4}     \NormalTok{println!(}\StringTok{"Hello in English: \{\}"}\NormalTok{, phrases::english::greetings::hello(}
\NormalTok{↳ ));}
                                                   \NormalTok{^}\KeywordTok{~~~~~~~~~~~~~~~~~~~~~~~~~~~~~~~~~}
\KeywordTok{note}\NormalTok{: in expansion of format_args!}
\KeywordTok{<std} \NormalTok{macros}\KeywordTok{>}\NormalTok{:2:25: 2:58 note: expansion site}
\KeywordTok{<std} \NormalTok{macros}\KeywordTok{>}\NormalTok{:1:1: 2:62 note: in expansion of print!}
\KeywordTok{<std} \NormalTok{macros}\KeywordTok{>}\NormalTok{:3:1: 3:54 note: expansion site}
\KeywordTok{<std} \NormalTok{macros}\KeywordTok{>}\NormalTok{:1:1: 3:58 note: in expansion of println!}
\KeywordTok{phrases/src}\NormalTok{/main.rs:}\KeywordTok{4}\NormalTok{:5: 4:76 note: expansion site}
\end{Highlighting}
\end{Shaded}

By default, everything is private in Rust. Let's talk about this in some
more depth.

\subsection{Exporting a Public
Interface}\label{exporting-a-public-interface}

Rust allows you to precisely control which aspects of your interface are
public, and so private is the default. To make things public, you use
the \texttt{pub} keyword. Let's focus on the \texttt{english} module
first, so let's reduce our \texttt{src/main.rs} to only this:

\begin{Shaded}
\begin{Highlighting}[]
\KeywordTok{extern} \KeywordTok{crate} \NormalTok{phrases;}

\KeywordTok{fn} \NormalTok{main() \{}
    \PreprocessorTok{println!}\NormalTok{(}\StringTok{"Hello in English: \{\}"}\NormalTok{, phrases::english::greetings::hello());}
    \PreprocessorTok{println!}\NormalTok{(}\StringTok{"Goodbye in English: \{\}"}\NormalTok{, phrases::english::farewells::goodbye());}
\NormalTok{\}}
\end{Highlighting}
\end{Shaded}

In our \texttt{src/lib.rs}, let's add \texttt{pub} to the
\texttt{english} module declaration:

\begin{Shaded}
\begin{Highlighting}[]
\KeywordTok{pub} \KeywordTok{mod} \NormalTok{english;}
\KeywordTok{mod} \NormalTok{japanese;}
\end{Highlighting}
\end{Shaded}

And in our \texttt{src/english/mod.rs}, let's make both \texttt{pub}:

\begin{Shaded}
\begin{Highlighting}[]
\KeywordTok{pub} \KeywordTok{mod} \NormalTok{greetings;}
\KeywordTok{pub} \KeywordTok{mod} \NormalTok{farewells;}
\end{Highlighting}
\end{Shaded}

In our \texttt{src/english/greetings.rs}, let's add \texttt{pub} to our
\texttt{fn} declaration:

\begin{Shaded}
\begin{Highlighting}[]
\KeywordTok{pub} \KeywordTok{fn} \NormalTok{hello() -> }\DataTypeTok{String} \NormalTok{\{}
    \StringTok{"Hello!"}\NormalTok{.to_string()}
\NormalTok{\}}
\end{Highlighting}
\end{Shaded}

And also in \texttt{src/english/farewells.rs}:

\begin{Shaded}
\begin{Highlighting}[]
\KeywordTok{pub} \KeywordTok{fn} \NormalTok{goodbye() -> }\DataTypeTok{String} \NormalTok{\{}
    \StringTok{"Goodbye."}\NormalTok{.to_string()}
\NormalTok{\}}
\end{Highlighting}
\end{Shaded}

Now, our crate compiles, albeit with warnings about not using the
\texttt{japanese} functions:

\begin{Shaded}
\begin{Highlighting}[]
\NormalTok{$ }\KeywordTok{cargo} \NormalTok{run}
   \KeywordTok{Compiling} \NormalTok{phrases v0.0.1 (file:///home/you/projects/phrases)}
\KeywordTok{src/japanese}\NormalTok{/greetings.rs:}\KeywordTok{1}\NormalTok{:1: 3:2 warning: function is never used: }\KeywordTok{`hello`}\NormalTok{, }\CommentTok{#[warn(de}
\NormalTok{↳ }\KeywordTok{ad_code}\NormalTok{)] }\KeywordTok{on} \NormalTok{by default}
\KeywordTok{src/japanese}\NormalTok{/greetings.rs:}\KeywordTok{1} \NormalTok{fn hello() }\KeywordTok{->} \NormalTok{String \{}
\KeywordTok{src/japanese}\NormalTok{/greetings.rs:}\KeywordTok{2}     \StringTok{"こんにちは"}\NormalTok{.to_string()}
\KeywordTok{src/japanese}\NormalTok{/greetings.rs:}\KeywordTok{3} \NormalTok{\}}
\KeywordTok{src/japanese}\NormalTok{/farewells.rs:}\KeywordTok{1}\NormalTok{:1: 3:2 warning: function is never used: }\KeywordTok{`goodbye`}\NormalTok{, }\CommentTok{#[warn(}
\NormalTok{↳ }\KeywordTok{dead_code}\NormalTok{)] }\KeywordTok{on} \NormalTok{by default}
\KeywordTok{src/japanese}\NormalTok{/farewells.rs:}\KeywordTok{1} \NormalTok{fn goodbye() }\KeywordTok{->} \NormalTok{String \{}
\KeywordTok{src/japanese}\NormalTok{/farewells.rs:}\KeywordTok{2}     \StringTok{"さようなら"}\NormalTok{.to_string()}
\KeywordTok{src/japanese}\NormalTok{/farewells.rs:}\KeywordTok{3} \NormalTok{\}}
     \KeywordTok{Running} \KeywordTok{`target/debug/phrases`}
\KeywordTok{Hello} \NormalTok{in English: Hello!}
\KeywordTok{Goodbye} \NormalTok{in English: Goodbye.}
\end{Highlighting}
\end{Shaded}

\texttt{pub} also applies to \texttt{struct}s and their member fields.
In keeping with Rust's tendency toward safety, simply making a
\texttt{struct} public won't automatically make its members public: you
must mark the fields individually with \texttt{pub}.

Now that our functions are public, we can use them. Great! However,
typing out \texttt{phrases::english::greetings::hello()} is very long
and repetitive. Rust has another keyword for importing names into the
current scope, so that you can refer to them with shorter names. Let's
talk about \texttt{use}.

\subsection{\texorpdfstring{Importing Modules with
\texttt{use}}{Importing Modules with use}}\label{importing-modules-with-use}

Rust has a \texttt{use} keyword, which allows us to import names into
our local scope. Let's change our \texttt{src/main.rs} to look like
this:

\begin{Shaded}
\begin{Highlighting}[]
\KeywordTok{extern} \KeywordTok{crate} \NormalTok{phrases;}

\KeywordTok{use} \NormalTok{phrases::english::greetings;}
\KeywordTok{use} \NormalTok{phrases::english::farewells;}

\KeywordTok{fn} \NormalTok{main() \{}
    \PreprocessorTok{println!}\NormalTok{(}\StringTok{"Hello in English: \{\}"}\NormalTok{, greetings::hello());}
    \PreprocessorTok{println!}\NormalTok{(}\StringTok{"Goodbye in English: \{\}"}\NormalTok{, farewells::goodbye());}
\NormalTok{\}}
\end{Highlighting}
\end{Shaded}

The two \texttt{use} lines import each module into the local scope, so
we can refer to the functions by a much shorter name. By convention,
when importing functions, it's considered best practice to import the
module, rather than the function directly. In other words, you
\emph{can} do this:

\begin{Shaded}
\begin{Highlighting}[]
\KeywordTok{extern} \KeywordTok{crate} \NormalTok{phrases;}

\KeywordTok{use} \NormalTok{phrases::english::greetings::hello;}
\KeywordTok{use} \NormalTok{phrases::english::farewells::goodbye;}

\KeywordTok{fn} \NormalTok{main() \{}
    \PreprocessorTok{println!}\NormalTok{(}\StringTok{"Hello in English: \{\}"}\NormalTok{, hello());}
    \PreprocessorTok{println!}\NormalTok{(}\StringTok{"Goodbye in English: \{\}"}\NormalTok{, goodbye());}
\NormalTok{\}}
\end{Highlighting}
\end{Shaded}

But it is not idiomatic. This is significantly more likely to introduce
a naming conflict. In our short program, it's not a big deal, but as it
grows, it becomes a problem. If we have conflicting names, Rust will
give a compilation error. For example, if we made the \texttt{japanese}
functions public, and tried to do this:

\begin{Shaded}
\begin{Highlighting}[]
\KeywordTok{extern} \KeywordTok{crate} \NormalTok{phrases;}

\KeywordTok{use} \NormalTok{phrases::english::greetings::hello;}
\KeywordTok{use} \NormalTok{phrases::japanese::greetings::hello;}

\KeywordTok{fn} \NormalTok{main() \{}
    \PreprocessorTok{println!}\NormalTok{(}\StringTok{"Hello in English: \{\}"}\NormalTok{, hello());}
    \PreprocessorTok{println!}\NormalTok{(}\StringTok{"Hello in Japanese: \{\}"}\NormalTok{, hello());}
\NormalTok{\}}
\end{Highlighting}
\end{Shaded}

Rust will give us a compile-time error:

\begin{verbatim}
   Compiling phrases v0.0.1 (file:///home/you/projects/phrases)
src/main.rs:4:5: 4:40 error: a value named `hello` has already been imported in this m
↳ odule [E0252]
src/main.rs:4 use phrases::japanese::greetings::hello;
                  ^~~~~~~~~~~~~~~~~~~~~~~~~~~~~~~~~~~
error: aborting due to previous error
Could not compile `phrases`.
\end{verbatim}

If we're importing multiple names from the same module, we don't have to
type it out twice. Instead of this:

\begin{Shaded}
\begin{Highlighting}[]
\KeywordTok{use} \NormalTok{phrases::english::greetings;}
\KeywordTok{use} \NormalTok{phrases::english::farewells;}
\end{Highlighting}
\end{Shaded}

We can use this shortcut:

\begin{Shaded}
\begin{Highlighting}[]
\KeywordTok{use} \NormalTok{phrases::english::\{greetings, farewells\};}
\end{Highlighting}
\end{Shaded}

\subsubsection{\texorpdfstring{Re-exporting with
\texttt{pub\ use}}{Re-exporting with pub use}}\label{re-exporting-with-pub-use}

You don't only use \texttt{use} to shorten identifiers. You can also use
it inside of your crate to re-export a function inside another module.
This allows you to present an external interface that may not directly
map to your internal code organization.

Let's look at an example. Modify your \texttt{src/main.rs} to read like
this:

\begin{Shaded}
\begin{Highlighting}[]
\KeywordTok{extern} \KeywordTok{crate} \NormalTok{phrases;}

\KeywordTok{use} \NormalTok{phrases::english::\{greetings,farewells\};}
\KeywordTok{use} \NormalTok{phrases::japanese;}

\KeywordTok{fn} \NormalTok{main() \{}
    \PreprocessorTok{println!}\NormalTok{(}\StringTok{"Hello in English: \{\}"}\NormalTok{, greetings::hello());}
    \PreprocessorTok{println!}\NormalTok{(}\StringTok{"Goodbye in English: \{\}"}\NormalTok{, farewells::goodbye());}

    \PreprocessorTok{println!}\NormalTok{(}\StringTok{"Hello in Japanese: \{\}"}\NormalTok{, japanese::hello());}
    \PreprocessorTok{println!}\NormalTok{(}\StringTok{"Goodbye in Japanese: \{\}"}\NormalTok{, japanese::goodbye());}
\NormalTok{\}}
\end{Highlighting}
\end{Shaded}

Then, modify your \texttt{src/lib.rs} to make the \texttt{japanese} mod
public:

\begin{Shaded}
\begin{Highlighting}[]
\KeywordTok{pub} \KeywordTok{mod} \NormalTok{english;}
\KeywordTok{pub} \KeywordTok{mod} \NormalTok{japanese;}
\end{Highlighting}
\end{Shaded}

Next, make the two functions public, first in
\texttt{src/japanese/greetings.rs}:

\begin{Shaded}
\begin{Highlighting}[]
\KeywordTok{pub} \KeywordTok{fn} \NormalTok{hello() -> }\DataTypeTok{String} \NormalTok{\{}
    \StringTok{"こんにちは"}\NormalTok{.to_string()}
\NormalTok{\}}
\end{Highlighting}
\end{Shaded}

And then in \texttt{src/japanese/farewells.rs}:

\begin{Shaded}
\begin{Highlighting}[]
\KeywordTok{pub} \KeywordTok{fn} \NormalTok{goodbye() -> }\DataTypeTok{String} \NormalTok{\{}
    \StringTok{"さようなら"}\NormalTok{.to_string()}
\NormalTok{\}}
\end{Highlighting}
\end{Shaded}

Finally, modify your \texttt{src/japanese/mod.rs} to read like this:

\begin{Shaded}
\begin{Highlighting}[]
\KeywordTok{pub} \KeywordTok{use} \KeywordTok{self}\NormalTok{::greetings::hello;}
\KeywordTok{pub} \KeywordTok{use} \KeywordTok{self}\NormalTok{::farewells::goodbye;}

\KeywordTok{mod} \NormalTok{greetings;}
\KeywordTok{mod} \NormalTok{farewells;}
\end{Highlighting}
\end{Shaded}

The \texttt{pub\ use} declaration brings the function into scope at this
part of our module hierarchy. Because we've \texttt{pub\ use}d this
inside of our \texttt{japanese} module, we now have a
\texttt{phrases::japanese::hello()} function and a
\texttt{phrases::japanese::goodbye()} function, even though the code for
them lives in \texttt{phrases::japanese::greetings::hello()} and
\texttt{phrases::japanese::farewells::goodbye()}. Our internal
organization doesn't define our external interface.

Here we have a \texttt{pub\ use} for each function we want to bring into
the \texttt{japanese} scope. We could alternatively use the wildcard
syntax to include everything from \texttt{greetings} into the current
scope: \texttt{pub\ use\ self::greetings::*}.

What about the \texttt{self}? Well, by default, \texttt{use}
declarations are absolute paths, starting from your crate root.
\texttt{self} makes that path relative to your current place in the
hierarchy instead. There's one more special form of \texttt{use}: you
can \texttt{use\ super::} to reach one level up the tree from your
current location. Some people like to think of \texttt{self} as
\texttt{.} and \texttt{super} as \texttt{..}, from many shells' display
for the current directory and the parent directory.

Outside of \texttt{use}, paths are relative: \texttt{foo::bar()} refers
to a function inside of \texttt{foo} relative to where we are. If that's
prefixed with \texttt{::}, as in \texttt{::foo::bar()}, it refers to a
different \texttt{foo}, an absolute path from your crate root.

This will build and run:

\begin{Shaded}
\begin{Highlighting}[]
\NormalTok{$ }\KeywordTok{cargo} \NormalTok{run}
   \KeywordTok{Compiling} \NormalTok{phrases v0.0.1 (file:///home/you/projects/phrases)}
     \KeywordTok{Running} \KeywordTok{`target/debug/phrases`}
\KeywordTok{Hello} \NormalTok{in English: Hello!}
\KeywordTok{Goodbye} \NormalTok{in English: Goodbye.}
\KeywordTok{Hello} \NormalTok{in Japanese: こんにちは}
\KeywordTok{Goodbye} \NormalTok{in Japanese: さようなら}
\end{Highlighting}
\end{Shaded}

\subsubsection{Complex imports}\label{complex-imports}

Rust offers several advanced options that can add compactness and
convenience to your \texttt{extern\ crate} and \texttt{use} statements.
Here is an example:

\begin{Shaded}
\begin{Highlighting}[]
\KeywordTok{extern} \KeywordTok{crate} \NormalTok{phrases }\KeywordTok{as} \NormalTok{sayings;}

\KeywordTok{use} \NormalTok{sayings::japanese::greetings }\KeywordTok{as} \NormalTok{ja_greetings;}
\KeywordTok{use} \NormalTok{sayings::japanese::farewells::*;}
\KeywordTok{use} \NormalTok{sayings::english::\{}\KeywordTok{self}\NormalTok{, greetings }\KeywordTok{as} \NormalTok{en_greetings, farewells }\KeywordTok{as} \NormalTok{en_farewells\};}

\KeywordTok{fn} \NormalTok{main() \{}
    \PreprocessorTok{println!}\NormalTok{(}\StringTok{"Hello in English; \{\}"}\NormalTok{, en_greetings::hello());}
    \PreprocessorTok{println!}\NormalTok{(}\StringTok{"And in Japanese: \{\}"}\NormalTok{, ja_greetings::hello());}
    \PreprocessorTok{println!}\NormalTok{(}\StringTok{"Goodbye in English: \{\}"}\NormalTok{, english::farewells::goodbye());}
    \PreprocessorTok{println!}\NormalTok{(}\StringTok{"Again: \{\}"}\NormalTok{, en_farewells::goodbye());}
    \PreprocessorTok{println!}\NormalTok{(}\StringTok{"And in Japanese: \{\}"}\NormalTok{, goodbye());}
\NormalTok{\}}
\end{Highlighting}
\end{Shaded}

What's going on here?

First, both \texttt{extern\ crate} and \texttt{use} allow renaming the
thing that is being imported. So the crate is still called ``phrases'',
but here we will refer to it as ``sayings''. Similarly, the first
\texttt{use} statement pulls in the \texttt{japanese::greetings} module
from the crate, but makes it available as \texttt{ja\_greetings} as
opposed to simply \texttt{greetings}. This can help to avoid ambiguity
when importing similarly-named items from different places.

The second \texttt{use} statement uses a star glob to bring in all
public symbols from the \texttt{sayings::japanese::farewells} module. As
you can see we can later refer to the Japanese \texttt{goodbye} function
with no module qualifiers. This kind of glob should be used sparingly.
It's worth noting that it only imports the public symbols, even if the
code doing the globbing is in the same module.

The third \texttt{use} statement bears more explanation. It's using
``brace expansion'' globbing to compress three \texttt{use} statements
into one (this sort of syntax may be familiar if you've written Linux
shell scripts before). The uncompressed form of this statement would be:

\begin{Shaded}
\begin{Highlighting}[]
\KeywordTok{use} \NormalTok{sayings::english;}
\KeywordTok{use} \NormalTok{sayings::english::greetings }\KeywordTok{as} \NormalTok{en_greetings;}
\KeywordTok{use} \NormalTok{sayings::english::farewells }\KeywordTok{as} \NormalTok{en_farewells;}
\end{Highlighting}
\end{Shaded}

As you can see, the curly brackets compress \texttt{use} statements for
several items under the same path, and in this context \texttt{self}
refers back to that path. Note: The curly brackets cannot be nested or
mixed with star globbing.

\hypertarget{sec--const-and-static}{\section{\texorpdfstring{\texttt{const}
and \texttt{static}}{const and static}}\label{sec--const-and-static}}

Rust has a way of defining constants with the \texttt{const} keyword:

\begin{Shaded}
\begin{Highlighting}[]
\KeywordTok{const} \NormalTok{N: }\DataTypeTok{i32} \NormalTok{= }\DecValTok{5}\NormalTok{;}
\end{Highlighting}
\end{Shaded}

Unlike \protect\hyperlink{sec--variable-bindings}{\texttt{let}}
bindings, you must annotate the type of a \texttt{const}.

Constants live for the entire lifetime of a program. More specifically,
constants in Rust have no fixed address in memory. This is because
they're effectively inlined to each place that they're used. References
to the same constant are not necessarily guaranteed to refer to the same
memory address for this reason.

\subsection{\texorpdfstring{\texttt{static}}{static}}\label{static-1}

Rust provides a `global variable' sort of facility in static items.
They're similar to constants, but static items aren't inlined upon use.
This means that there is only one instance for each value, and it's at a
fixed location in memory.

Here's an example:

\begin{Shaded}
\begin{Highlighting}[]
\KeywordTok{static} \NormalTok{N: }\DataTypeTok{i32} \NormalTok{= }\DecValTok{5}\NormalTok{;}
\end{Highlighting}
\end{Shaded}

Unlike \protect\hyperlink{sec--variable-bindings}{\texttt{let}}
bindings, you must annotate the type of a \texttt{static}.

Statics live for the entire lifetime of a program, and therefore any
reference stored in a constant has a
\protect\hyperlink{sec--lifetimes}{\texttt{\textquotesingle{}static}
lifetime}:

\begin{Shaded}
\begin{Highlighting}[]
\KeywordTok{static} \NormalTok{NAME: &}\OtherTok{'static} \DataTypeTok{str} \NormalTok{= }\StringTok{"Steve"}\NormalTok{;}
\end{Highlighting}
\end{Shaded}

\hypertarget{mutability-1}{\subsubsection{Mutability}\label{mutability-1}}

You can introduce mutability with the \texttt{mut} keyword:

\begin{Shaded}
\begin{Highlighting}[]
\KeywordTok{static} \KeywordTok{mut} \NormalTok{N: }\DataTypeTok{i32} \NormalTok{= }\DecValTok{5}\NormalTok{;}
\end{Highlighting}
\end{Shaded}

Because this is mutable, one thread could be updating \texttt{N} while
another is reading it, causing memory unsafety. As such both accessing
and mutating a \texttt{static\ mut} is
\protect\hyperlink{sec--unsafe}{\texttt{unsafe}}, and so must be done in
an \texttt{unsafe} block:

\begin{Shaded}
\begin{Highlighting}[]

\KeywordTok{unsafe} \NormalTok{\{}
    \NormalTok{N += }\DecValTok{1}\NormalTok{;}

    \PreprocessorTok{println!}\NormalTok{(}\StringTok{"N: \{\}"}\NormalTok{, N);}
\NormalTok{\}}
\end{Highlighting}
\end{Shaded}

Furthermore, any type stored in a \texttt{static} must be \texttt{Sync},
and must not have a \protect\hyperlink{sec--drop}{\texttt{Drop}}
implementation.

\subsection{Initializing}\label{initializing}

Both \texttt{const} and \texttt{static} have requirements for giving
them a value. They must be given a value that's a constant expression.
In other words, you cannot use the result of a function call or anything
similarly complex or at runtime.

\subsection{Which construct should I
use?}\label{which-construct-should-i-use}

Almost always, if you can choose between the two, choose \texttt{const}.
It's pretty rare that you actually want a memory location associated
with your constant, and using a \texttt{const} allows for optimizations
like constant propagation not only in your crate but downstream crates.

\hypertarget{sec--attributes}{\section{Attributes}\label{sec--attributes}}

Declarations can be annotated with `attributes' in Rust. They look like
this:

\begin{Shaded}
\begin{Highlighting}[]
\AttributeTok{#[}\NormalTok{test}\AttributeTok{]}
\end{Highlighting}
\end{Shaded}

or like this:

\begin{Shaded}
\begin{Highlighting}[]
\AttributeTok{#![}\NormalTok{test}\AttributeTok{]}
\end{Highlighting}
\end{Shaded}

The difference between the two is the \texttt{!}, which changes what the
attribute applies to:

\begin{Shaded}
\begin{Highlighting}[]
\AttributeTok{#[}\NormalTok{foo}\AttributeTok{]}
\KeywordTok{struct} \NormalTok{Foo;}

\KeywordTok{mod} \NormalTok{bar \{}
    \AttributeTok{#![}\NormalTok{bar}\AttributeTok{]}
\NormalTok{\}}
\end{Highlighting}
\end{Shaded}

The \texttt{\#{[}foo{]}} attribute applies to the next item, which is
the \texttt{struct} declaration. The \texttt{\#!{[}bar{]}} attribute
applies to the item enclosing it, which is the \texttt{mod} declaration.
Otherwise, they're the same. Both change the meaning of the item they're
attached to somehow.

For example, consider a function like this:

\begin{Shaded}
\begin{Highlighting}[]
\AttributeTok{#[}\NormalTok{test}\AttributeTok{]}
\KeywordTok{fn} \NormalTok{check() \{}
    \PreprocessorTok{assert_eq!}\NormalTok{(}\DecValTok{2}\NormalTok{, }\DecValTok{1} \NormalTok{+ }\DecValTok{1}\NormalTok{);}
\NormalTok{\}}
\end{Highlighting}
\end{Shaded}

It is marked with \texttt{\#{[}test{]}}. This means it's special: when
you run \protect\hyperlink{sec--testing}{tests}, this function will
execute. When you compile as usual, it won't even be included. This
function is now a test function.

Attributes may also have additional data:

\begin{Shaded}
\begin{Highlighting}[]
\AttributeTok{#[}\NormalTok{inline}\AttributeTok{(}\NormalTok{always}\AttributeTok{)]}
\KeywordTok{fn} \NormalTok{super_fast_fn() \{}
\end{Highlighting}
\end{Shaded}

Or even keys and values:

\begin{Shaded}
\begin{Highlighting}[]
\AttributeTok{#[}\NormalTok{cfg}\AttributeTok{(}\NormalTok{target_os }\AttributeTok{=} \StringTok{"macos"}\AttributeTok{)]}
\KeywordTok{mod} \NormalTok{macos_only \{}
\end{Highlighting}
\end{Shaded}

Rust attributes are used for a number of different things. There is a
full list of attributes
\href{http://doc.rust-lang.org/reference.html\#attributes}{in the
reference}. Currently, you are not allowed to create your own
attributes, the Rust compiler defines them.

\hypertarget{sec--type-aliases}{\section{\texorpdfstring{\texttt{type}
aliases}{type aliases}}\label{sec--type-aliases}}

The \texttt{type} keyword lets you declare an alias of another type:

\begin{Shaded}
\begin{Highlighting}[]
\KeywordTok{type} \NormalTok{Name = }\DataTypeTok{String}\NormalTok{;}
\end{Highlighting}
\end{Shaded}

You can then use this type as if it were a real type:

\begin{Shaded}
\begin{Highlighting}[]
\KeywordTok{type} \NormalTok{Name = }\DataTypeTok{String}\NormalTok{;}

\KeywordTok{let} \NormalTok{x: Name = }\StringTok{"Hello"}\NormalTok{.to_string();}
\end{Highlighting}
\end{Shaded}

Note, however, that this is an \emph{alias}, not a new type entirely. In
other words, because Rust is strongly typed, you'd expect a comparison
between two different types to fail:

\begin{Shaded}
\begin{Highlighting}[]
\KeywordTok{let} \NormalTok{x: }\DataTypeTok{i32} \NormalTok{= }\DecValTok{5}\NormalTok{;}
\KeywordTok{let} \NormalTok{y: }\DataTypeTok{i64} \NormalTok{= }\DecValTok{5}\NormalTok{;}

\KeywordTok{if} \NormalTok{x == y \{}
   \CommentTok{// ...}
\NormalTok{\}}
\end{Highlighting}
\end{Shaded}

this gives

\begin{verbatim}
error: mismatched types:
 expected `i32`,
    found `i64`
(expected i32,
    found i64) [E0308]
     if x == y {
             ^
\end{verbatim}

But, if we had an alias:

\begin{Shaded}
\begin{Highlighting}[]
\KeywordTok{type} \NormalTok{Num = }\DataTypeTok{i32}\NormalTok{;}

\KeywordTok{let} \NormalTok{x: }\DataTypeTok{i32} \NormalTok{= }\DecValTok{5}\NormalTok{;}
\KeywordTok{let} \NormalTok{y: Num = }\DecValTok{5}\NormalTok{;}

\KeywordTok{if} \NormalTok{x == y \{}
   \CommentTok{// ...}
\NormalTok{\}}
\end{Highlighting}
\end{Shaded}

This compiles without error. Values of a \texttt{Num} type are the same
as a value of type \texttt{i32}, in every way. You can use {[}tuple
struct{]} to really get a new type.

You can also use type aliases with generics:

\begin{Shaded}
\begin{Highlighting}[]
\KeywordTok{use} \NormalTok{std::result;}

\KeywordTok{enum} \NormalTok{ConcreteError \{}
    \NormalTok{Foo,}
    \NormalTok{Bar,}
\NormalTok{\}}

\KeywordTok{type} \NormalTok{Result<T> = result::}\DataTypeTok{Result}\NormalTok{<T, ConcreteError>;}
\end{Highlighting}
\end{Shaded}

This creates a specialized version of the \texttt{Result} type, which
always has a \texttt{ConcreteError} for the \texttt{E} part of
\texttt{Result\textless{}T,\ E\textgreater{}}. This is commonly used in
the standard library to create custom errors for each subsection. For
example,
\href{http://doc.rust-lang.org/std/io/type.Result.html}{io::Result}.

\section{Casting between types}\label{sec--casting-between-types}

Rust, with its focus on safety, provides two different ways of casting
different types between each other. The first, \texttt{as}, is for safe
casts. In contrast, \texttt{transmute} allows for arbitrary casting, and
is one of the most dangerous features of Rust!

\subsection{Coercion}\label{coercion}

Coercion between types is implicit and has no syntax of its own, but can
be spelled out with \protect\hyperlink{explicit-coercions}{\texttt{as}}.

Coercion occurs in \texttt{let}, \texttt{const}, and \texttt{static}
statements; in function call arguments; in field values in struct
initialization; and in a function result.

The most common case of coercion is removing mutability from a
reference:

\begin{itemize}
\tightlist
\item
  \texttt{\&mut\ T} to \texttt{\&T}
\end{itemize}

An analogous conversion is to remove mutability from a
\href{raw-pointers.md}{raw pointer}:

\begin{itemize}
\tightlist
\item
  \texttt{*mut\ T} to \texttt{*const\ T}
\end{itemize}

References can also be coerced to raw pointers:

\begin{itemize}
\item
  \texttt{\&T} to \texttt{*const\ T}
\item
  \texttt{\&mut\ T} to \texttt{*mut\ T}
\end{itemize}

Custom coercions may be defined using
\href{deref-coercions.md}{\texttt{Deref}}.

Coercion is transitive.

\subsection{\texorpdfstring{\texttt{as}}{as}}\label{as}

The \texttt{as} keyword does safe casting:

\begin{Shaded}
\begin{Highlighting}[]
\KeywordTok{let} \NormalTok{x: }\DataTypeTok{i32} \NormalTok{= }\DecValTok{5}\NormalTok{;}

\KeywordTok{let} \NormalTok{y = x }\KeywordTok{as} \DataTypeTok{i64}\NormalTok{;}
\end{Highlighting}
\end{Shaded}

There are three major categories of safe cast: explicit coercions, casts
between numeric types, and pointer casts.

Casting is not transitive: even if \texttt{e\ as\ U1\ as\ U2} is a valid
expression, \texttt{e\ as\ U2} is not necessarily so (in fact it will
only be valid if \texttt{U1} coerces to \texttt{U2}).

\hypertarget{explicit-coercions}{\subsubsection{Explicit
coercions}\label{explicit-coercions}}

A cast \texttt{e\ as\ U} is valid if \texttt{e} has type \texttt{T} and
\texttt{T} \emph{coerces} to \texttt{U}.

\subsubsection{Numeric casts}\label{numeric-casts}

A cast \texttt{e\ as\ U} is also valid in any of the following cases:

\begin{itemize}
\tightlist
\item
  \texttt{e} has type \texttt{T} and \texttt{T} and \texttt{U} are any
  numeric types; \emph{numeric-cast}
\item
  \texttt{e} is a C-like enum (with no data attached to the variants),
  and \texttt{U} is an integer type; \emph{enum-cast}
\item
  \texttt{e} has type \texttt{bool} or \texttt{char} and \texttt{U} is
  an integer type; \emph{prim-int-cast}
\item
  \texttt{e} has type \texttt{u8} and \texttt{U} is \texttt{char};
  \emph{u8-char-cast}
\end{itemize}

For example

\begin{Shaded}
\begin{Highlighting}[]
\KeywordTok{let} \NormalTok{one = }\ConstantTok{true} \KeywordTok{as} \DataTypeTok{u8}\NormalTok{;}
\KeywordTok{let} \NormalTok{at_sign = }\DecValTok{64} \KeywordTok{as} \DataTypeTok{char}\NormalTok{;}
\KeywordTok{let} \NormalTok{two_hundred = -}\DecValTok{56i8} \KeywordTok{as} \DataTypeTok{u8}\NormalTok{;}
\end{Highlighting}
\end{Shaded}

The semantics of numeric casts are:

\begin{itemize}
\tightlist
\item
  Casting between two integers of the same size (e.g.~i32
  -\textgreater{} u32) is a no-op
\item
  Casting from a larger integer to a smaller integer (e.g.~u32
  -\textgreater{} u8) will truncate
\item
  Casting from a smaller integer to a larger integer (e.g.~u8
  -\textgreater{} u32) will

  \begin{itemize}
  \tightlist
  \item
    zero-extend if the source is unsigned
  \item
    sign-extend if the source is signed
  \end{itemize}
\item
  Casting from a float to an integer will round the float towards zero

  \begin{itemize}
  \tightlist
  \item
    \textbf{\href{https://github.com/rust-lang/rust/issues/10184}{NOTE:
    currently this will cause Undefined Behavior if the rounded value
    cannot be represented by the target integer type}}. This includes
    Inf and NaN. This is a bug and will be fixed.
  \end{itemize}
\item
  Casting from an integer to float will produce the floating point
  representation of the integer, rounded if necessary (rounding strategy
  unspecified)
\item
  Casting from an f32 to an f64 is perfect and lossless
\item
  Casting from an f64 to an f32 will produce the closest possible value
  (rounding strategy unspecified)

  \begin{itemize}
  \tightlist
  \item
    \textbf{\href{https://github.com/rust-lang/rust/issues/15536}{NOTE:
    currently this will cause Undefined Behavior if the value is finite
    but larger or smaller than the largest or smallest finite value
    representable by f32}}. This is a bug and will be fixed.
  \end{itemize}
\end{itemize}

\subsubsection{Pointer casts}\label{pointer-casts}

Perhaps surprisingly, it is safe to cast \href{raw-pointers.md}{raw
pointers} to and from integers, and to cast between pointers to
different types subject to some constraints. It is only unsafe to
dereference the pointer:

\begin{Shaded}
\begin{Highlighting}[]
\KeywordTok{let} \NormalTok{a = }\DecValTok{300} \KeywordTok{as} \NormalTok{*}\KeywordTok{const} \DataTypeTok{char}\NormalTok{; }\CommentTok{// a pointer to location 300}
\KeywordTok{let} \NormalTok{b = a }\KeywordTok{as} \DataTypeTok{u32}\NormalTok{;}
\end{Highlighting}
\end{Shaded}

\texttt{e\ as\ U} is a valid pointer cast in any of the following cases:

\begin{itemize}
\item
  \texttt{e} has type \texttt{*T}, \texttt{U} has type \texttt{*U\_0},
  and either \texttt{U\_0:\ Sized} or
  \texttt{unsize\_kind(T)\ ==\ unsize\_kind(U\_0)}; a
  \emph{ptr-ptr-cast}
\item
  \texttt{e} has type \texttt{*T} and \texttt{U} is a numeric type,
  while \texttt{T:\ Sized}; \emph{ptr-addr-cast}
\item
  \texttt{e} is an integer and \texttt{U} is \texttt{*U\_0}, while
  \texttt{U\_0:\ Sized}; \emph{addr-ptr-cast}
\item
  \texttt{e} has type \texttt{\&{[}T;\ n{]}} and \texttt{U} is
  \texttt{*const\ T}; \emph{array-ptr-cast}
\item
  \texttt{e} is a function pointer type and \texttt{U} has type
  \texttt{*T}, while \texttt{T:\ Sized}; \emph{fptr-ptr-cast}
\item
  \texttt{e} is a function pointer type and \texttt{U} is an integer;
  \emph{fptr-addr-cast}
\end{itemize}

\subsection{\texorpdfstring{\texttt{transmute}}{transmute}}\label{transmute}

\texttt{as} only allows safe casting, and will for example reject an
attempt to cast four bytes into a \texttt{u32}:

\begin{Shaded}
\begin{Highlighting}[]
\KeywordTok{let} \NormalTok{a = [}\DecValTok{0u8}\NormalTok{, }\DecValTok{0u8}\NormalTok{, }\DecValTok{0u8}\NormalTok{, }\DecValTok{0u8}\NormalTok{];}

\KeywordTok{let} \NormalTok{b = a }\KeywordTok{as} \DataTypeTok{u32}\NormalTok{; }\CommentTok{// four eights makes 32}
\end{Highlighting}
\end{Shaded}

This errors with:

\begin{verbatim}
error: non-scalar cast: `[u8; 4]` as `u32`
let b = a as u32; // four eights makes 32
        ^~~~~~~~
\end{verbatim}

This is a `non-scalar cast' because we have multiple values here: the
four elements of the array. These kinds of casts are very dangerous,
because they make assumptions about the way that multiple underlying
structures are implemented. For this, we need something more dangerous.

The \texttt{transmute} function is provided by a
\protect\hyperlink{sec--intrinsics}{compiler intrinsic}, and what it
does is very simple, but very scary. It tells Rust to treat a value of
one type as though it were another type. It does this regardless of the
typechecking system, and completely trusts you.

In our previous example, we know that an array of four \texttt{u8}s
represents a \texttt{u32} properly, and so we want to do the cast. Using
\texttt{transmute} instead of \texttt{as}, Rust lets us:

\begin{Shaded}
\begin{Highlighting}[]
\KeywordTok{use} \NormalTok{std::mem;}

\KeywordTok{fn} \NormalTok{main() \{}
    \KeywordTok{unsafe} \NormalTok{\{}
        \KeywordTok{let} \NormalTok{a = [}\DecValTok{0u8}\NormalTok{, }\DecValTok{1u8}\NormalTok{, }\DecValTok{0u8}\NormalTok{, }\DecValTok{0u8}\NormalTok{];}
        \KeywordTok{let} \NormalTok{b = mem::transmute::<[}\DataTypeTok{u8}\NormalTok{; }\DecValTok{4}\NormalTok{], }\DataTypeTok{u32}\NormalTok{>(a);}
        \PreprocessorTok{println!}\NormalTok{(}\StringTok{"\{\}"}\NormalTok{, b); }\CommentTok{// 256}
        \CommentTok{// or, more concisely:}
        \KeywordTok{let} \NormalTok{c: }\DataTypeTok{u32} \NormalTok{= mem::transmute(a);}
        \PreprocessorTok{println!}\NormalTok{(}\StringTok{"\{\}"}\NormalTok{, c); }\CommentTok{// 256}
    \NormalTok{\}}
\NormalTok{\}}
\end{Highlighting}
\end{Shaded}

We have to wrap the operation in an \texttt{unsafe} block for this to
compile successfully. Technically, only the \texttt{mem::transmute} call
itself needs to be in the block, but it's nice in this case to enclose
everything related, so you know where to look. In this case, the details
about \texttt{a} are also important, and so they're in the block. You'll
see code in either style, sometimes the context is too far away, and
wrapping all of the code in \texttt{unsafe} isn't a great idea.

While \texttt{transmute} does very little checking, it will at least
make sure that the types are the same size. This errors:

\begin{Shaded}
\begin{Highlighting}[]
\KeywordTok{use} \NormalTok{std::mem;}

\KeywordTok{unsafe} \NormalTok{\{}
    \KeywordTok{let} \NormalTok{a = [}\DecValTok{0u8}\NormalTok{, }\DecValTok{0u8}\NormalTok{, }\DecValTok{0u8}\NormalTok{, }\DecValTok{0u8}\NormalTok{];}

    \KeywordTok{let} \NormalTok{b = mem::transmute::<[}\DataTypeTok{u8}\NormalTok{; }\DecValTok{4}\NormalTok{], }\DataTypeTok{u64}\NormalTok{>(a);}
\NormalTok{\}}
\end{Highlighting}
\end{Shaded}

with:

\begin{verbatim}
error: transmute called with differently sized types: [u8; 4] (32 bits) to u64
(64 bits)
\end{verbatim}

Other than that, you're on your own!

\hypertarget{sec--associated-types}{\section{Associated
Types}\label{sec--associated-types}}

Associated types are a powerful part of Rust's type system. They're
related to the idea of a `type family', in other words, grouping
multiple types together. That description is a bit abstract, so let's
dive right into an example. If you want to write a \texttt{Graph} trait,
you have two types to be generic over: the node type and the edge type.
So you might write a trait,
\texttt{Graph\textless{}N,\ E\textgreater{}}, that looks like this:

\begin{Shaded}
\begin{Highlighting}[]
\KeywordTok{trait} \NormalTok{Graph<N, E> \{}
    \KeywordTok{fn} \NormalTok{has_edge(&}\KeywordTok{self}\NormalTok{, &N, &N) -> }\DataTypeTok{bool}\NormalTok{;}
    \KeywordTok{fn} \NormalTok{edges(&}\KeywordTok{self}\NormalTok{, &N) -> }\DataTypeTok{Vec}\NormalTok{<E>;}
    \CommentTok{// etc}
\NormalTok{\}}
\end{Highlighting}
\end{Shaded}

While this sort of works, it ends up being awkward. For example, any
function that wants to take a \texttt{Graph} as a parameter now
\emph{also} needs to be generic over the \texttt{N}ode and \texttt{E}dge
types too:

\begin{Shaded}
\begin{Highlighting}[]
\KeywordTok{fn} \NormalTok{distance<N, E, G: Graph<N, E>>(graph: &G, start: &N, end: &N) -> }\DataTypeTok{u32} \NormalTok{\{ ... \}}
\end{Highlighting}
\end{Shaded}

Our distance calculation works regardless of our \texttt{Edge} type, so
the \texttt{E} stuff in this signature is a distraction.

What we really want to say is that a certain \texttt{E}dge and
\texttt{N}ode type come together to form each kind of \texttt{Graph}. We
can do that with associated types:

\begin{Shaded}
\begin{Highlighting}[]
\KeywordTok{trait} \NormalTok{Graph \{}
    \KeywordTok{type} \NormalTok{N;}
    \KeywordTok{type} \NormalTok{E;}

    \KeywordTok{fn} \NormalTok{has_edge(&}\KeywordTok{self}\NormalTok{, &}\KeywordTok{Self}\NormalTok{::N, &}\KeywordTok{Self}\NormalTok{::N) -> }\DataTypeTok{bool}\NormalTok{;}
    \KeywordTok{fn} \NormalTok{edges(&}\KeywordTok{self}\NormalTok{, &}\KeywordTok{Self}\NormalTok{::N) -> }\DataTypeTok{Vec}\NormalTok{<}\KeywordTok{Self}\NormalTok{::E>;}
    \CommentTok{// etc}
\NormalTok{\}}
\end{Highlighting}
\end{Shaded}

Now, our clients can be abstract over a given \texttt{Graph}:

\begin{Shaded}
\begin{Highlighting}[]
\KeywordTok{fn} \NormalTok{distance<G: Graph>(graph: &G, start: &G::N, end: &G::N) -> }\DataTypeTok{u32} \NormalTok{\{ ... \}}
\end{Highlighting}
\end{Shaded}

No need to deal with the \texttt{E}dge type here!

Let's go over all this in more detail.

\subsubsection{Defining associated
types}\label{defining-associated-types}

Let's build that \texttt{Graph} trait. Here's the definition:

\begin{Shaded}
\begin{Highlighting}[]
\KeywordTok{trait} \NormalTok{Graph \{}
    \KeywordTok{type} \NormalTok{N;}
    \KeywordTok{type} \NormalTok{E;}

    \KeywordTok{fn} \NormalTok{has_edge(&}\KeywordTok{self}\NormalTok{, &}\KeywordTok{Self}\NormalTok{::N, &}\KeywordTok{Self}\NormalTok{::N) -> }\DataTypeTok{bool}\NormalTok{;}
    \KeywordTok{fn} \NormalTok{edges(&}\KeywordTok{self}\NormalTok{, &}\KeywordTok{Self}\NormalTok{::N) -> }\DataTypeTok{Vec}\NormalTok{<}\KeywordTok{Self}\NormalTok{::E>;}
\NormalTok{\}}
\end{Highlighting}
\end{Shaded}

Simple enough. Associated types use the \texttt{type} keyword, and go
inside the body of the trait, with the functions.

These \texttt{type} declarations can have all the same thing as
functions do. For example, if we wanted our \texttt{N} type to implement
\texttt{Display}, so we can print the nodes out, we could do this:

\begin{Shaded}
\begin{Highlighting}[]
\KeywordTok{use} \NormalTok{std::fmt;}

\KeywordTok{trait} \NormalTok{Graph \{}
    \KeywordTok{type} \NormalTok{N: fmt::Display;}
    \KeywordTok{type} \NormalTok{E;}

    \KeywordTok{fn} \NormalTok{has_edge(&}\KeywordTok{self}\NormalTok{, &}\KeywordTok{Self}\NormalTok{::N, &}\KeywordTok{Self}\NormalTok{::N) -> }\DataTypeTok{bool}\NormalTok{;}
    \KeywordTok{fn} \NormalTok{edges(&}\KeywordTok{self}\NormalTok{, &}\KeywordTok{Self}\NormalTok{::N) -> }\DataTypeTok{Vec}\NormalTok{<}\KeywordTok{Self}\NormalTok{::E>;}
\NormalTok{\}}
\end{Highlighting}
\end{Shaded}

\subsubsection{Implementing associated
types}\label{implementing-associated-types}

Just like any trait, traits that use associated types use the
\texttt{impl} keyword to provide implementations. Here's a simple
implementation of Graph:

\begin{Shaded}
\begin{Highlighting}[]
\KeywordTok{struct} \NormalTok{Node;}

\KeywordTok{struct} \NormalTok{Edge;}

\KeywordTok{struct} \NormalTok{MyGraph;}

\KeywordTok{impl} \NormalTok{Graph }\KeywordTok{for} \NormalTok{MyGraph \{}
    \KeywordTok{type} \NormalTok{N = Node;}
    \KeywordTok{type} \NormalTok{E = Edge;}

    \KeywordTok{fn} \NormalTok{has_edge(&}\KeywordTok{self}\NormalTok{, n1: &Node, n2: &Node) -> }\DataTypeTok{bool} \NormalTok{\{}
        \ConstantTok{true}
    \NormalTok{\}}

    \KeywordTok{fn} \NormalTok{edges(&}\KeywordTok{self}\NormalTok{, n: &Node) -> }\DataTypeTok{Vec}\NormalTok{<Edge> \{}
        \DataTypeTok{Vec}\NormalTok{::new()}
    \NormalTok{\}}
\NormalTok{\}}
\end{Highlighting}
\end{Shaded}

This silly implementation always returns \texttt{true} and an empty
\texttt{Vec\textless{}Edge\textgreater{}}, but it gives you an idea of
how to implement this kind of thing. We first need three
\texttt{struct}s, one for the graph, one for the node, and one for the
edge. If it made more sense to use a different type, that would work as
well, we're going to use \texttt{struct}s for all three here.

Next is the \texttt{impl} line, which is an implementation like any
other trait.

From here, we use \texttt{=} to define our associated types. The name
the trait uses goes on the left of the \texttt{=}, and the concrete type
we're \texttt{impl}ementing this for goes on the right. Finally, we use
the concrete types in our function declarations.

\subsubsection{Trait objects with associated
types}\label{trait-objects-with-associated-types}

There's one more bit of syntax we should talk about: trait objects. If
you try to create a trait object from a trait with an associated type,
like this:

\begin{Shaded}
\begin{Highlighting}[]
\KeywordTok{let} \NormalTok{graph = MyGraph;}
\KeywordTok{let} \NormalTok{obj = }\DataTypeTok{Box}\NormalTok{::new(graph) }\KeywordTok{as} \DataTypeTok{Box}\NormalTok{<Graph>;}
\end{Highlighting}
\end{Shaded}

You'll get two errors:

\begin{verbatim}
error: the value of the associated type `E` (from the trait `main::Graph`) must
be specified [E0191]
let obj = Box::new(graph) as Box<Graph>;
          ^~~~~~~~~~~~~~~~~~~~~~~~~~~~~
24:44 error: the value of the associated type `N` (from the trait
`main::Graph`) must be specified [E0191]
let obj = Box::new(graph) as Box<Graph>;
          ^~~~~~~~~~~~~~~~~~~~~~~~~~~~~
\end{verbatim}

We can't create a trait object like this, because we don't know the
associated types. Instead, we can write this:

\begin{Shaded}
\begin{Highlighting}[]
\KeywordTok{let} \NormalTok{graph = MyGraph;}
\KeywordTok{let} \NormalTok{obj = }\DataTypeTok{Box}\NormalTok{::new(graph) }\KeywordTok{as} \DataTypeTok{Box}\NormalTok{<Graph<N=Node, E=Edge>>;}
\end{Highlighting}
\end{Shaded}

The \texttt{N=Node} syntax allows us to provide a concrete type,
\texttt{Node}, for the \texttt{N} type parameter. Same with
\texttt{E=Edge}. If we didn't provide this constraint, we couldn't be
sure which \texttt{impl} to match this trait object to.

\hypertarget{sec--unsized-types}{\section{Unsized
Types}\label{sec--unsized-types}}

Most types have a particular size, in bytes, that is knowable at compile
time. For example, an \texttt{i32} is thirty-two bits big, or four
bytes. However, there are some types which are useful to express, but do
not have a defined size. These are called `unsized' or `dynamically
sized' types. One example is \texttt{{[}T{]}}. This type represents a
certain number of \texttt{T} in sequence. But we don't know how many
there are, so the size is not known.

Rust understands a few of these types, but they have some restrictions.
There are three:

\begin{enumerate}
\def\labelenumi{\arabic{enumi}.}
\tightlist
\item
  We can only manipulate an instance of an unsized type via a pointer.
  An \texttt{\&{[}T{]}} works fine, but a \texttt{{[}T{]}} does not.
\item
  Variables and arguments cannot have dynamically sized types.
\item
  Only the last field in a \texttt{struct} may have a dynamically sized
  type; the other fields must not. Enum variants must not have
  dynamically sized types as data.
\end{enumerate}

So why bother? Well, because \texttt{{[}T{]}} can only be used behind a
pointer, if we didn't have language support for unsized types, it would
be impossible to write this:

\begin{Shaded}
\begin{Highlighting}[]
\KeywordTok{impl} \NormalTok{Foo }\KeywordTok{for} \DataTypeTok{str} \NormalTok{\{}
\end{Highlighting}
\end{Shaded}

or

\begin{Shaded}
\begin{Highlighting}[]
\KeywordTok{impl}\NormalTok{<T> Foo }\KeywordTok{for} \NormalTok{[T] \{}
\end{Highlighting}
\end{Shaded}

Instead, you would have to write:

\begin{Shaded}
\begin{Highlighting}[]
\KeywordTok{impl} \NormalTok{Foo }\KeywordTok{for} \NormalTok{&}\DataTypeTok{str} \NormalTok{\{}
\end{Highlighting}
\end{Shaded}

Meaning, this implementation would only work for
\protect\hyperlink{sec--references-and-borrowing}{references}, and not
other types of pointers. With the \texttt{impl\ for\ str}, all pointers,
including (at some point, there are some bugs to fix first) user-defined
custom smart pointers, can use this \texttt{impl}.

\subsection{?Sized}\label{sized}

If you want to write a function that accepts a dynamically sized type,
you can use the special bound, \texttt{?Sized}:

\begin{Shaded}
\begin{Highlighting}[]
\KeywordTok{struct} \NormalTok{Foo<T: ?}\BuiltInTok{Sized}\NormalTok{> \{}
    \NormalTok{f: T,}
\NormalTok{\}}
\end{Highlighting}
\end{Shaded}

This \texttt{?}, read as ``T may be \texttt{Sized}'', means that this
bound is special: it lets us match more kinds, not less. It's almost
like every \texttt{T} implicitly has \texttt{T:\ Sized}, and the
\texttt{?} undoes this default.

\hypertarget{sec--operators-and-overloading}{\section{Operators and
Overloading}\label{sec--operators-and-overloading}}

Rust allows for a limited form of operator overloading. There are
certain operators that are able to be overloaded. To support a
particular operator between types, there's a specific trait that you can
implement, which then overloads the operator.

For example, the \texttt{+} operator can be overloaded with the
\texttt{Add} trait:

\begin{Shaded}
\begin{Highlighting}[]
\KeywordTok{use} \NormalTok{std::ops::Add;}

\AttributeTok{#[}\NormalTok{derive}\AttributeTok{(}\BuiltInTok{Debug}\AttributeTok{)]}
\KeywordTok{struct} \NormalTok{Point \{}
    \NormalTok{x: }\DataTypeTok{i32}\NormalTok{,}
    \NormalTok{y: }\DataTypeTok{i32}\NormalTok{,}
\NormalTok{\}}

\KeywordTok{impl} \NormalTok{Add }\KeywordTok{for} \NormalTok{Point \{}
    \KeywordTok{type} \NormalTok{Output = Point;}

    \KeywordTok{fn} \NormalTok{add(}\KeywordTok{self}\NormalTok{, other: Point) -> Point \{}
        \NormalTok{Point \{ x: }\KeywordTok{self}\NormalTok{.x + other.x, y: }\KeywordTok{self}\NormalTok{.y + other.y \}}
    \NormalTok{\}}
\NormalTok{\}}

\KeywordTok{fn} \NormalTok{main() \{}
    \KeywordTok{let} \NormalTok{p1 = Point \{ x: }\DecValTok{1}\NormalTok{, y: }\DecValTok{0} \NormalTok{\};}
    \KeywordTok{let} \NormalTok{p2 = Point \{ x: }\DecValTok{2}\NormalTok{, y: }\DecValTok{3} \NormalTok{\};}

    \KeywordTok{let} \NormalTok{p3 = p1 + p2;}

    \PreprocessorTok{println!}\NormalTok{(}\StringTok{"\{:?\}"}\NormalTok{, p3);}
\NormalTok{\}}
\end{Highlighting}
\end{Shaded}

In \texttt{main}, we can use \texttt{+} on our two \texttt{Point}s,
since we've implemented
\texttt{Add\textless{}Output=Point\textgreater{}} for \texttt{Point}.

There are a number of operators that can be overloaded this way, and all
of their associated traits live in the
\href{http://doc.rust-lang.org/std/ops/index.html}{\texttt{std::ops}}
module. Check out its documentation for the full list.

Implementing these traits follows a pattern. Let's look at
\href{http://doc.rust-lang.org/std/ops/trait.Add.html}{\texttt{Add}} in
more detail:

\begin{Shaded}
\begin{Highlighting}[]
\KeywordTok{pub} \KeywordTok{trait} \NormalTok{Add<RHS = }\KeywordTok{Self}\NormalTok{> \{}
    \KeywordTok{type} \NormalTok{Output;}

    \KeywordTok{fn} \NormalTok{add(}\KeywordTok{self}\NormalTok{, rhs: RHS) -> }\KeywordTok{Self}\NormalTok{::Output;}
\NormalTok{\}}
\end{Highlighting}
\end{Shaded}

There's three types in total involved here: the type you
\texttt{impl\ Add} for, \texttt{RHS}, which defaults to \texttt{Self},
and \texttt{Output}. For an expression \texttt{let\ z\ =\ x\ +\ y},
\texttt{x} is the \texttt{Self} type, \texttt{y} is the RHS, and
\texttt{z} is the \texttt{Self::Output} type.

\begin{Shaded}
\begin{Highlighting}[]
\KeywordTok{impl} \NormalTok{Add<}\DataTypeTok{i32}\NormalTok{> }\KeywordTok{for} \NormalTok{Point \{}
    \KeywordTok{type} \NormalTok{Output = }\DataTypeTok{f64}\NormalTok{;}

    \KeywordTok{fn} \NormalTok{add(}\KeywordTok{self}\NormalTok{, rhs: }\DataTypeTok{i32}\NormalTok{) -> }\DataTypeTok{f64} \NormalTok{\{}
        \CommentTok{// add an i32 to a Point and get an f64}
    \NormalTok{\}}
\NormalTok{\}}
\end{Highlighting}
\end{Shaded}

will let you do this:

\begin{Shaded}
\begin{Highlighting}[]
\KeywordTok{let} \NormalTok{p: Point = }\CommentTok{// ...}
\KeywordTok{let} \NormalTok{x: }\DataTypeTok{f64} \NormalTok{= p + }\DecValTok{2i32}\NormalTok{;}
\end{Highlighting}
\end{Shaded}

\subsection{Using operator traits in generic
structs}\label{using-operator-traits-in-generic-structs}

Now that we know how operator traits are defined, we can define our
\texttt{HasArea} trait and \texttt{Square} struct from the
\protect\hyperlink{sec--traits}{traits chapter} more generically:

\begin{Shaded}
\begin{Highlighting}[]
\KeywordTok{use} \NormalTok{std::ops::Mul;}

\KeywordTok{trait} \NormalTok{HasArea<T> \{}
    \KeywordTok{fn} \NormalTok{area(&}\KeywordTok{self}\NormalTok{) -> T;}
\NormalTok{\}}

\KeywordTok{struct} \NormalTok{Square<T> \{}
    \NormalTok{x: T,}
    \NormalTok{y: T,}
    \NormalTok{side: T,}
\NormalTok{\}}

\KeywordTok{impl}\NormalTok{<T> HasArea<T> }\KeywordTok{for} \NormalTok{Square<T>}
        \KeywordTok{where} \NormalTok{T: Mul<Output=T> + }\BuiltInTok{Copy} \NormalTok{\{}
    \KeywordTok{fn} \NormalTok{area(&}\KeywordTok{self}\NormalTok{) -> T \{}
        \KeywordTok{self}\NormalTok{.side * }\KeywordTok{self}\NormalTok{.side}
    \NormalTok{\}}
\NormalTok{\}}

\KeywordTok{fn} \NormalTok{main() \{}
    \KeywordTok{let} \NormalTok{s = Square \{}
        \NormalTok{x: }\DecValTok{0.0f64}\NormalTok{,}
        \NormalTok{y: }\DecValTok{0.0f64}\NormalTok{,}
        \NormalTok{side: }\DecValTok{12.0f64}\NormalTok{,}
    \NormalTok{\};}

    \PreprocessorTok{println!}\NormalTok{(}\StringTok{"Area of s: \{\}"}\NormalTok{, s.area());}
\NormalTok{\}}
\end{Highlighting}
\end{Shaded}

For \texttt{HasArea} and \texttt{Square}, we declare a type parameter
\texttt{T} and replace \texttt{f64} with it. The \texttt{impl} needs
more involved modifications:

\begin{Shaded}
\begin{Highlighting}[]
\KeywordTok{impl}\NormalTok{<T> HasArea<T> }\KeywordTok{for} \NormalTok{Square<T>}
        \KeywordTok{where} \NormalTok{T: Mul<Output=T> + }\BuiltInTok{Copy} \NormalTok{\{ ... \}}
\end{Highlighting}
\end{Shaded}

The \texttt{area} method requires that we can multiply the sides, so we
declare that type \texttt{T} must implement \texttt{std::ops::Mul}. Like
\texttt{Add}, mentioned above, \texttt{Mul} itself takes an
\texttt{Output} parameter: since we know that numbers don't change type
when multiplied, we also set it to \texttt{T}. \texttt{T} must also
support copying, so Rust doesn't try to move \texttt{self.side} into the
return value.

\hypertarget{sec--deref-coercions}{\section{Deref
coercions}\label{sec--deref-coercions}}

The standard library provides a special trait,
\href{http://doc.rust-lang.org/std/ops/trait.Deref.html}{\texttt{Deref}}.
It's normally used to overload \texttt{*}, the dereference operator:

\begin{Shaded}
\begin{Highlighting}[]
\KeywordTok{use} \NormalTok{std::ops::Deref;}

\KeywordTok{struct} \NormalTok{DerefExample<T> \{}
    \NormalTok{value: T,}
\NormalTok{\}}

\KeywordTok{impl}\NormalTok{<T> Deref }\KeywordTok{for} \NormalTok{DerefExample<T> \{}
    \KeywordTok{type} \NormalTok{Target = T;}

    \KeywordTok{fn} \NormalTok{deref(&}\KeywordTok{self}\NormalTok{) -> &T \{}
        \NormalTok{&}\KeywordTok{self}\NormalTok{.value}
    \NormalTok{\}}
\NormalTok{\}}

\KeywordTok{fn} \NormalTok{main() \{}
    \KeywordTok{let} \NormalTok{x = DerefExample \{ value: }\CharTok{'a'} \NormalTok{\};}
    \PreprocessorTok{assert_eq!}\NormalTok{(}\CharTok{'a'}\NormalTok{, *x);}
\NormalTok{\}}
\end{Highlighting}
\end{Shaded}

This is useful for writing custom pointer types. However, there's a
language feature related to \texttt{Deref}: `deref coercions'. Here's
the rule: If you have a type \texttt{U}, and it implements
\texttt{Deref\textless{}Target=T\textgreater{}}, values of \texttt{\&U}
will automatically coerce to a \texttt{\&T}. Here's an example:

\begin{Shaded}
\begin{Highlighting}[]
\KeywordTok{fn} \NormalTok{foo(s: &}\DataTypeTok{str}\NormalTok{) \{}
    \CommentTok{// borrow a string for a second}
\NormalTok{\}}

\CommentTok{// String implements Deref<Target=str>}
\KeywordTok{let} \NormalTok{owned = }\StringTok{"Hello"}\NormalTok{.to_string();}

\CommentTok{// therefore, this works:}
\NormalTok{foo(&owned);}
\end{Highlighting}
\end{Shaded}

Using an ampersand in front of a value takes a reference to it. So
\texttt{owned} is a \texttt{String}, \texttt{\&owned} is an
\texttt{\&String}, and since
\texttt{impl\ Deref\textless{}Target=str\textgreater{}\ for\ String},
\texttt{\&String} will deref to \texttt{\&str}, which \texttt{foo()}
takes.

That's it. This rule is one of the only places in which Rust does an
automatic conversion for you, but it adds a lot of flexibility. For
example, the \texttt{Rc\textless{}T\textgreater{}} type implements
\texttt{Deref\textless{}Target=T\textgreater{}}, so this works:

\begin{Shaded}
\begin{Highlighting}[]
\KeywordTok{use} \NormalTok{std::rc::Rc;}

\KeywordTok{fn} \NormalTok{foo(s: &}\DataTypeTok{str}\NormalTok{) \{}
    \CommentTok{// borrow a string for a second}
\NormalTok{\}}

\CommentTok{// String implements Deref<Target=str>}
\KeywordTok{let} \NormalTok{owned = }\StringTok{"Hello"}\NormalTok{.to_string();}
\KeywordTok{let} \NormalTok{counted = Rc::new(owned);}

\CommentTok{// therefore, this works:}
\NormalTok{foo(&counted);}
\end{Highlighting}
\end{Shaded}

All we've done is wrap our \texttt{String} in an
\texttt{Rc\textless{}T\textgreater{}}. But we can now pass the
\texttt{Rc\textless{}String\textgreater{}} around anywhere we'd have a
\texttt{String}. The signature of \texttt{foo} didn't change, but works
just as well with either type. This example has two conversions:
\texttt{Rc\textless{}String\textgreater{}} to \texttt{String} and then
\texttt{String} to \texttt{\&str}. Rust will do this as many times as
possible until the types match.

Another very common implementation provided by the standard library is:

\begin{Shaded}
\begin{Highlighting}[]
\KeywordTok{fn} \NormalTok{foo(s: &[}\DataTypeTok{i32}\NormalTok{]) \{}
    \CommentTok{// borrow a slice for a second}
\NormalTok{\}}

\CommentTok{// Vec<T> implements Deref<Target=[T]>}
\KeywordTok{let} \NormalTok{owned = }\PreprocessorTok{vec!}\NormalTok{[}\DecValTok{1}\NormalTok{, }\DecValTok{2}\NormalTok{, }\DecValTok{3}\NormalTok{];}

\NormalTok{foo(&owned);}
\end{Highlighting}
\end{Shaded}

Vectors can \texttt{Deref} to a slice.

\subsubsection{Deref and method calls}\label{deref-and-method-calls}

\texttt{Deref} will also kick in when calling a method. Consider the
following example.

\begin{Shaded}
\begin{Highlighting}[]
\KeywordTok{struct} \NormalTok{Foo;}

\KeywordTok{impl} \NormalTok{Foo \{}
    \KeywordTok{fn} \NormalTok{foo(&}\KeywordTok{self}\NormalTok{) \{ }\PreprocessorTok{println!}\NormalTok{(}\StringTok{"Foo"}\NormalTok{); \}}
\NormalTok{\}}

\KeywordTok{let} \NormalTok{f = &&Foo;}

\NormalTok{f.foo();}
\end{Highlighting}
\end{Shaded}

Even though \texttt{f} is a \texttt{\&\&Foo} and \texttt{foo} takes
\texttt{\&self}, this works. That's because these things are the same:

\begin{Shaded}
\begin{Highlighting}[]
\NormalTok{f.foo();}
\NormalTok{(&f).foo();}
\NormalTok{(&&f).foo();}
\NormalTok{(&&&&&&&&f).foo();}
\end{Highlighting}
\end{Shaded}

A value of type \texttt{\&\&\&\&\&\&\&\&\&\&\&\&\&\&\&\&Foo} can still
have methods defined on \texttt{Foo} called, because the compiler will
insert as many * operations as necessary to get it right. And since it's
inserting \texttt{*}s, that uses \texttt{Deref}.

\hypertarget{sec--macros}{\section{Macros}\label{sec--macros}}

By now you've learned about many of the tools Rust provides for
abstracting and reusing code. These units of code reuse have a rich
semantic structure. For example, functions have a type signature, type
parameters have trait bounds, and overloaded functions must belong to a
particular trait.

This structure means that Rust's core abstractions have powerful
compile-time correctness checking. But this comes at the price of
reduced flexibility. If you visually identify a pattern of repeated
code, you may find it's difficult or cumbersome to express that pattern
as a generic function, a trait, or anything else within Rust's
semantics.

Macros allow us to abstract at a syntactic level. A macro invocation is
shorthand for an ``expanded'' syntactic form. This expansion happens
early in compilation, before any static checking. As a result, macros
can capture many patterns of code reuse that Rust's core abstractions
cannot.

The drawback is that macro-based code can be harder to understand,
because fewer of the built-in rules apply. Like an ordinary function, a
well-behaved macro can be used without understanding its implementation.
However, it can be difficult to design a well-behaved macro!
Additionally, compiler errors in macro code are harder to interpret,
because they describe problems in the expanded code, not the
source-level form that developers use.

These drawbacks make macros something of a ``feature of last resort''.
That's not to say that macros are bad; they are part of Rust because
sometimes they're needed for truly concise, well-abstracted code. Just
keep this tradeoff in mind.

\subsection{Defining a macro}\label{defining-a-macro}

You may have seen the \texttt{vec!} macro, used to initialize a
\protect\hyperlink{sec--vectors}{vector} with any number of elements.

\begin{Shaded}
\begin{Highlighting}[]
\KeywordTok{let} \NormalTok{x: }\DataTypeTok{Vec}\NormalTok{<}\DataTypeTok{u32}\NormalTok{> = }\PreprocessorTok{vec!}\NormalTok{[}\DecValTok{1}\NormalTok{, }\DecValTok{2}\NormalTok{, }\DecValTok{3}\NormalTok{];}
\end{Highlighting}
\end{Shaded}

This can't be an ordinary function, because it takes any number of
arguments. But we can imagine it as syntactic shorthand for

\begin{Shaded}
\begin{Highlighting}[]
\KeywordTok{let} \NormalTok{x: }\DataTypeTok{Vec}\NormalTok{<}\DataTypeTok{u32}\NormalTok{> = \{}
    \KeywordTok{let} \KeywordTok{mut} \NormalTok{temp_vec = }\DataTypeTok{Vec}\NormalTok{::new();}
    \NormalTok{temp_vec.push(}\DecValTok{1}\NormalTok{);}
    \NormalTok{temp_vec.push(}\DecValTok{2}\NormalTok{);}
    \NormalTok{temp_vec.push(}\DecValTok{3}\NormalTok{);}
    \NormalTok{temp_vec}
\NormalTok{\};}
\end{Highlighting}
\end{Shaded}

We can implement this shorthand, using a macro: \footnote{The actual
  definition of \texttt{vec!} in libcollections differs from the one
  presented here, for reasons of efficiency and reusability.}

\begin{Shaded}
\begin{Highlighting}[]
\PreprocessorTok{macro_rules!} \NormalTok{vec \{}
    \NormalTok{( $( $x:expr ),* ) => \{}
        \NormalTok{\{}
            \KeywordTok{let} \KeywordTok{mut} \NormalTok{temp_vec = }\DataTypeTok{Vec}\NormalTok{::new();}
            \NormalTok{$(}
                \NormalTok{temp_vec.push($x);}
            \NormalTok{)*}
            \NormalTok{temp_vec}
        \NormalTok{\}}
    \NormalTok{\};}
\NormalTok{\}}
\end{Highlighting}
\end{Shaded}

Whoa, that's a lot of new syntax! Let's break it down.

\begin{Shaded}
\begin{Highlighting}[]
\PreprocessorTok{macro_rules!} \NormalTok{vec \{ ... \}}
\end{Highlighting}
\end{Shaded}

This says we're defining a macro named \texttt{vec}, much as
\texttt{fn\ vec} would define a function named \texttt{vec}. In prose,
we informally write a macro's name with an exclamation point, e.g.
\texttt{vec!}. The exclamation point is part of the invocation syntax
and serves to distinguish a macro from an ordinary function.

\subsubsection{Matching}\label{matching}

The macro is defined through a series of rules, which are
pattern-matching cases. Above, we had

\begin{Shaded}
\begin{Highlighting}[]
\NormalTok{( $( $x:expr ),* ) => \{ ... \};}
\end{Highlighting}
\end{Shaded}

This is like a \texttt{match} expression arm, but the matching happens
on Rust syntax trees, at compile time. The semicolon is optional on the
last (here, only) case. The ``pattern'' on the left-hand side of
\texttt{=\textgreater{}} is known as a `matcher'. These have {[}their
own little grammar{]} within the language.

The matcher \texttt{\$x:expr} will match any Rust expression, binding
that syntax tree to the `metavariable' \texttt{\$x}. The identifier
\texttt{expr} is a `fragment specifier'; the full possibilities are
enumerated later in this chapter. Surrounding the matcher with
\texttt{\$(...),*} will match zero or more expressions, separated by
commas.

Aside from the special matcher syntax, any Rust tokens that appear in a
matcher must match exactly. For example,

\begin{Shaded}
\begin{Highlighting}[]
\PreprocessorTok{macro_rules!} \NormalTok{foo \{}
    \NormalTok{(x => $e:expr) => (}\PreprocessorTok{println!}\NormalTok{(}\StringTok{"mode X: \{\}"}\NormalTok{, $e));}
    \NormalTok{(y => $e:expr) => (}\PreprocessorTok{println!}\NormalTok{(}\StringTok{"mode Y: \{\}"}\NormalTok{, $e));}
\NormalTok{\}}

\KeywordTok{fn} \NormalTok{main() \{}
    \PreprocessorTok{foo!}\NormalTok{(y => }\DecValTok{3}\NormalTok{);}
\NormalTok{\}}
\end{Highlighting}
\end{Shaded}

will print

\begin{verbatim}
mode Y: 3
\end{verbatim}

With

\begin{Shaded}
\begin{Highlighting}[]
\PreprocessorTok{foo!}\NormalTok{(z => }\DecValTok{3}\NormalTok{);}
\end{Highlighting}
\end{Shaded}

we get the compiler error

\begin{verbatim}
error: no rules expected the token `z`
\end{verbatim}

\subsubsection{Expansion}\label{expansion}

The right-hand side of a macro rule is ordinary Rust syntax, for the
most part. But we can splice in bits of syntax captured by the matcher.
From the original example:

\begin{Shaded}
\begin{Highlighting}[]
\NormalTok{$(}
    \NormalTok{temp_vec.push($x);}
\NormalTok{)*}
\end{Highlighting}
\end{Shaded}

Each matched expression \texttt{\$x} will produce a single \texttt{push}
statement in the macro expansion. The repetition in the expansion
proceeds in ``lockstep'' with repetition in the matcher (more on this in
a moment).

Because \texttt{\$x} was already declared as matching an expression, we
don't repeat \texttt{:expr} on the right-hand side. Also, we don't
include a separating comma as part of the repetition operator. Instead,
we have a terminating semicolon within the repeated block.

Another detail: the \texttt{vec!} macro has \emph{two} pairs of braces
on the right-hand side. They are often combined like so:

\begin{Shaded}
\begin{Highlighting}[]
\PreprocessorTok{macro_rules!} \NormalTok{foo \{}
    \NormalTok{() => \{\{}
        \NormalTok{...}
    \NormalTok{\}\}}
\NormalTok{\}}
\end{Highlighting}
\end{Shaded}

The outer braces are part of the syntax of \texttt{macro\_rules!}. In
fact, you can use \texttt{()} or \texttt{{[}{]}} instead. They simply
delimit the right-hand side as a whole.

The inner braces are part of the expanded syntax. Remember, the
\texttt{vec!} macro is used in an expression context. To write an
expression with multiple statements, including \texttt{let}-bindings, we
use a block. If your macro expands to a single expression, you don't
need this extra layer of braces.

Note that we never \emph{declared} that the macro produces an
expression. In fact, this is not determined until we use the macro as an
expression. With care, you can write a macro whose expansion works in
several contexts. For example, shorthand for a data type could be valid
as either an expression or a pattern.

\subsubsection{Repetition}\label{repetition}

The repetition operator follows two principal rules:

\begin{enumerate}
\def\labelenumi{\arabic{enumi}.}
\tightlist
\item
  \texttt{\$(...)*} walks through one ``layer'' of repetitions, for all
  of the \texttt{\$name}s it contains, in lockstep, and
\item
  each \texttt{\$name} must be under at least as many \texttt{\$(...)*}s
  as it was matched against. If it is under more, it'll be duplicated,
  as appropriate.
\end{enumerate}

This baroque macro illustrates the duplication of variables from outer
repetition levels.

\begin{Shaded}
\begin{Highlighting}[]
\PreprocessorTok{macro_rules!} \NormalTok{o_O \{}
    \NormalTok{(}
        \NormalTok{$(}
            \NormalTok{$x:expr; [ $( $y:expr ),* ]}
        \NormalTok{);*}
    \NormalTok{) => \{}
        \NormalTok{&[ $($( $x + $y ),*),* ]}
    \NormalTok{\}}
\NormalTok{\}}

\KeywordTok{fn} \NormalTok{main() \{}
    \KeywordTok{let} \NormalTok{a: &[}\DataTypeTok{i32}\NormalTok{]}
        \NormalTok{= }\PreprocessorTok{o_O!}\NormalTok{(}\DecValTok{10}\NormalTok{; [}\DecValTok{1}\NormalTok{, }\DecValTok{2}\NormalTok{, }\DecValTok{3}\NormalTok{];}
               \DecValTok{20}\NormalTok{; [}\DecValTok{4}\NormalTok{, }\DecValTok{5}\NormalTok{, }\DecValTok{6}\NormalTok{]);}

    \PreprocessorTok{assert_eq!}\NormalTok{(a, [}\DecValTok{11}\NormalTok{, }\DecValTok{12}\NormalTok{, }\DecValTok{13}\NormalTok{, }\DecValTok{24}\NormalTok{, }\DecValTok{25}\NormalTok{, }\DecValTok{26}\NormalTok{]);}
\NormalTok{\}}
\end{Highlighting}
\end{Shaded}

That's most of the matcher syntax. These examples use \texttt{\$(...)*},
which is a ``zero or more'' match. Alternatively you can write
\texttt{\$(...)+} for a ``one or more'' match. Both forms optionally
include a separator, which can be any token except \texttt{+} or
\texttt{*}.

This system is based on
``\href{https://www.cs.indiana.edu/ftp/techreports/TR206.pdf}{Macro-by-Example}''
(PDF link).

\subsection{Hygiene}\label{hygiene}

Some languages implement macros using simple text substitution, which
leads to various problems. For example, this C program prints
\texttt{13} instead of the expected \texttt{25}.

\begin{verbatim}
#define FIVE_TIMES(x) 5 * x

int main() {
    printf("%d\n", FIVE_TIMES(2 + 3));
    return 0;
}
\end{verbatim}

After expansion we have \texttt{5\ *\ 2\ +\ 3}, and multiplication has
greater precedence than addition. If you've used C macros a lot, you
probably know the standard idioms for avoiding this problem, as well as
five or six others. In Rust, we don't have to worry about it.

\begin{Shaded}
\begin{Highlighting}[]
\PreprocessorTok{macro_rules!} \NormalTok{five_times \{}
    \NormalTok{($x:expr) => (}\DecValTok{5} \NormalTok{* $x);}
\NormalTok{\}}

\KeywordTok{fn} \NormalTok{main() \{}
    \PreprocessorTok{assert_eq!}\NormalTok{(}\DecValTok{25}\NormalTok{, }\PreprocessorTok{five_times!}\NormalTok{(}\DecValTok{2} \NormalTok{+ }\DecValTok{3}\NormalTok{));}
\NormalTok{\}}
\end{Highlighting}
\end{Shaded}

The metavariable \texttt{\$x} is parsed as a single expression node, and
keeps its place in the syntax tree even after substitution.

Another common problem in macro systems is `variable capture'. Here's a
C macro, using {[}a GNU C extension{]} to emulate Rust's expression
blocks.

\begin{verbatim}
#define LOG(msg) ({ \
    int state = get_log_state(); \
    if (state > 0) { \
        printf("log(%d): %s\n", state, msg); \
    } \
})
\end{verbatim}

Here's a simple use case that goes terribly wrong:

\begin{verbatim}
const char *state = "reticulating splines";
LOG(state)
\end{verbatim}

This expands to

\begin{verbatim}
const char *state = "reticulating splines";
{
    int state = get_log_state();
    if (state > 0) {
        printf("log(%d): %s\n", state, state);
    }
}
\end{verbatim}

The second variable named \texttt{state} shadows the first one. This is
a problem because the print statement should refer to both of them.

The equivalent Rust macro has the desired behavior.

\begin{Shaded}
\begin{Highlighting}[]
\PreprocessorTok{macro_rules!} \NormalTok{log \{}
    \NormalTok{($msg:expr) => \{\{}
        \KeywordTok{let} \NormalTok{state: }\DataTypeTok{i32} \NormalTok{= get_log_state();}
        \KeywordTok{if} \NormalTok{state > }\DecValTok{0} \NormalTok{\{}
            \PreprocessorTok{println!}\NormalTok{(}\StringTok{"log(\{\}): \{\}"}\NormalTok{, state, $msg);}
        \NormalTok{\}}
    \NormalTok{\}\};}
\NormalTok{\}}

\KeywordTok{fn} \NormalTok{main() \{}
    \KeywordTok{let} \NormalTok{state: &}\DataTypeTok{str} \NormalTok{= }\StringTok{"reticulating splines"}\NormalTok{;}
    \PreprocessorTok{log!}\NormalTok{(state);}
\NormalTok{\}}
\end{Highlighting}
\end{Shaded}

This works because Rust has a {[}hygienic macro system{]}. Each macro
expansion happens in a distinct `syntax context', and each variable is
tagged with the syntax context where it was introduced. It's as though
the variable \texttt{state} inside \texttt{main} is painted a different
``color'' from the variable \texttt{state} inside the macro, and
therefore they don't conflict.

This also restricts the ability of macros to introduce new bindings at
the invocation site. Code such as the following will not work:

\begin{Shaded}
\begin{Highlighting}[]
\PreprocessorTok{macro_rules!} \NormalTok{foo \{}
    \NormalTok{() => (}\KeywordTok{let} \NormalTok{x = }\DecValTok{3}\NormalTok{);}
\NormalTok{\}}

\KeywordTok{fn} \NormalTok{main() \{}
    \PreprocessorTok{foo!}\NormalTok{();}
    \PreprocessorTok{println!}\NormalTok{(}\StringTok{"\{\}"}\NormalTok{, x);}
\NormalTok{\}}
\end{Highlighting}
\end{Shaded}

Instead you need to pass the variable name into the invocation, so that
it's tagged with the right syntax context.

\begin{Shaded}
\begin{Highlighting}[]
\PreprocessorTok{macro_rules!} \NormalTok{foo \{}
    \NormalTok{($v:ident) => (}\KeywordTok{let} \NormalTok{$v = }\DecValTok{3}\NormalTok{);}
\NormalTok{\}}

\KeywordTok{fn} \NormalTok{main() \{}
    \PreprocessorTok{foo!}\NormalTok{(x);}
    \PreprocessorTok{println!}\NormalTok{(}\StringTok{"\{\}"}\NormalTok{, x);}
\NormalTok{\}}
\end{Highlighting}
\end{Shaded}

This holds for \texttt{let} bindings and loop labels, but not for
\href{http://doc.rust-lang.org/reference.html\#items}{items}. So the
following code does compile:

\begin{Shaded}
\begin{Highlighting}[]
\PreprocessorTok{macro_rules!} \NormalTok{foo \{}
    \NormalTok{() => (}\KeywordTok{fn} \NormalTok{x() \{ \});}
\NormalTok{\}}

\KeywordTok{fn} \NormalTok{main() \{}
    \PreprocessorTok{foo!}\NormalTok{();}
    \NormalTok{x();}
\NormalTok{\}}
\end{Highlighting}
\end{Shaded}

\subsection{Recursive macros}\label{recursive-macros}

A macro's expansion can include more macro invocations, including
invocations of the very same macro being expanded. These recursive
macros are useful for processing tree-structured input, as illustrated
by this (simplistic) HTML shorthand:

\begin{Shaded}
\begin{Highlighting}[]
\PreprocessorTok{macro_rules!} \NormalTok{write_html \{}
    \NormalTok{($w:expr, ) => (());}

    \NormalTok{($w:expr, $e:tt) => (}\PreprocessorTok{write!}\NormalTok{($w, }\StringTok{"\{\}"}\NormalTok{, $e));}

    \NormalTok{($w:expr, $tag:ident [ $($inner:tt)* ] $($rest:tt)*) => \{\{}
        \PreprocessorTok{write!}\NormalTok{($w, }\StringTok{"<\{\}>"}\NormalTok{, }\PreprocessorTok{stringify!}\NormalTok{($tag));}
        \PreprocessorTok{write_html!}\NormalTok{($w, $($inner)*);}
        \PreprocessorTok{write!}\NormalTok{($w, }\StringTok{"</\{\}>"}\NormalTok{, }\PreprocessorTok{stringify!}\NormalTok{($tag));}
        \PreprocessorTok{write_html!}\NormalTok{($w, $($rest)*);}
    \NormalTok{\}\};}
\NormalTok{\}}

\KeywordTok{fn} \NormalTok{main() \{}
    \KeywordTok{use} \NormalTok{std::fmt::Write;}
    \KeywordTok{let} \KeywordTok{mut} \NormalTok{out = }\DataTypeTok{String}\NormalTok{::new();}

    \PreprocessorTok{write_html!}\NormalTok{(&}\KeywordTok{mut} \NormalTok{out,}
        \NormalTok{html[}
            \NormalTok{head[title[}\StringTok{"Macros guide"}\NormalTok{]]}
            \NormalTok{body[h1[}\StringTok{"Macros are the best!"}\NormalTok{]]}
        \NormalTok{]);}

    \PreprocessorTok{assert_eq!}\NormalTok{(out,}
        \StringTok{"<html><head><title>Macros guide</title></head>}\SpecialCharTok{\textbackslash{}}
\StringTok{         <body><h1>Macros are the best!</h1></body></html>"}\NormalTok{);}
\NormalTok{\}}
\end{Highlighting}
\end{Shaded}

\hypertarget{debugging-macro-code}{\subsection{Debugging macro
code}\label{debugging-macro-code}}

To see the results of expanding macros, run
\texttt{rustc\ -\/-pretty\ expanded}. The output represents a whole
crate, so you can also feed it back in to \texttt{rustc}, which will
sometimes produce better error messages than the original compilation.
Note that the \texttt{-\/-pretty\ expanded} output may have a different
meaning if multiple variables of the same name (but different syntax
contexts) are in play in the same scope. In this case
\texttt{-\/-pretty\ expanded,hygiene} will tell you about the syntax
contexts.

\texttt{rustc} provides two syntax extensions that help with macro
debugging. For now, they are unstable and require feature gates.

\begin{itemize}
\item
  \texttt{log\_syntax!(...)} will print its arguments to standard
  output, at compile time, and ``expand'' to nothing.
\item
  \texttt{trace\_macros!(true)} will enable a compiler message every
  time a macro is expanded. Use \texttt{trace\_macros!(false)} later in
  expansion to turn it off.
\end{itemize}

\subsection{Syntactic requirements}\label{syntactic-requirements}

Even when Rust code contains un-expanded macros, it can be parsed as a
full \protect\hyperlink{abstract-syntax-tree}{syntax tree}. This
property can be very useful for editors and other tools that process
code. It also has a few consequences for the design of Rust's macro
system.

One consequence is that Rust must determine, when it parses a macro
invocation, whether the macro stands in for

\begin{itemize}
\tightlist
\item
  zero or more items,
\item
  zero or more methods,
\item
  an expression,
\item
  a statement, or
\item
  a pattern.
\end{itemize}

A macro invocation within a block could stand for some items, or for an
expression / statement. Rust uses a simple rule to resolve this
ambiguity. A macro invocation that stands for items must be either

\begin{itemize}
\tightlist
\item
  delimited by curly braces, e.g. \texttt{foo!\ \{\ ...\ \}}, or
\item
  terminated by a semicolon, e.g. \texttt{foo!(...);}
\end{itemize}

Another consequence of pre-expansion parsing is that the macro
invocation must consist of valid Rust tokens. Furthermore, parentheses,
brackets, and braces must be balanced within a macro invocation. For
example, \texttt{foo!({[})} is forbidden. This allows Rust to know where
the macro invocation ends.

More formally, the macro invocation body must be a sequence of `token
trees'. A token tree is defined recursively as either

\begin{itemize}
\tightlist
\item
  a sequence of token trees surrounded by matching \texttt{()},
  \texttt{{[}{]}}, or \texttt{\{\}}, or
\item
  any other single token.
\end{itemize}

Within a matcher, each metavariable has a `fragment specifier',
identifying which syntactic form it matches.

\begin{itemize}
\tightlist
\item
  \texttt{ident}: an identifier. Examples: \texttt{x}; \texttt{foo}.
\item
  \texttt{path}: a qualified name. Example: \texttt{T::SpecialA}.
\item
  \texttt{expr}: an expression. Examples: \texttt{2\ +\ 2};
  \texttt{if\ true\ \{\ 1\ \}\ else\ \{\ 2\ \}}; \texttt{f(42)}.
\item
  \texttt{ty}: a type. Examples: \texttt{i32};
  \texttt{Vec\textless{}(char,\ String)\textgreater{}}; \texttt{\&T}.
\item
  \texttt{pat}: a pattern. Examples: \texttt{Some(t)};
  \texttt{(17,\ \textquotesingle{}a\textquotesingle{})}; \texttt{\_}.
\item
  \texttt{stmt}: a single statement. Example: \texttt{let\ x\ =\ 3}.
\item
  \texttt{block}: a brace-delimited sequence of statements and
  optionally an expression. Example:
  \texttt{\{\ log(error,\ "hi");\ return\ 12;\ \}}.
\item
  \texttt{item}: an
  \href{http://doc.rust-lang.org/reference.html\#items}{item}. Examples:
  \texttt{fn\ foo()\ \{\ \}}; \texttt{struct\ Bar;}.
\item
  \texttt{meta}: a ``meta item'', as found in attributes. Example:
  \texttt{cfg(target\_os\ =\ "windows")}.
\item
  \texttt{tt}: a single token tree.
\end{itemize}

There are additional rules regarding the next token after a
metavariable:

\begin{itemize}
\tightlist
\item
  \texttt{expr} and \texttt{stmt} variables may only be followed by one
  of: \texttt{=\textgreater{}\ ,\ ;}
\item
  \texttt{ty} and \texttt{path} variables may only be followed by one
  of:
  \texttt{=\textgreater{}\ ,\ =\ \textbar{}\ ;\ :\ \textgreater{}\ {[}\ \{\ as\ where}
\item
  \texttt{pat} variables may only be followed by one of:
  \texttt{=\textgreater{}\ ,\ =\ \textbar{}\ if\ in}
\item
  Other variables may be followed by any token.
\end{itemize}

These rules provide some flexibility for Rust's syntax to evolve without
breaking existing macros.

The macro system does not deal with parse ambiguity at all. For example,
the grammar \texttt{\$(\$i:ident)*\ \$e:expr} will always fail to parse,
because the parser would be forced to choose between parsing
\texttt{\$i} and parsing \texttt{\$e}. Changing the invocation syntax to
put a distinctive token in front can solve the problem. In this case,
you can write \texttt{\$(I\ \$i:ident)*\ E\ \$e:expr}.

\subsection{Scoping and macro
import/export}\label{scoping-and-macro-importexport}

Macros are expanded at an early stage in compilation, before name
resolution. One downside is that scoping works differently for macros,
compared to other constructs in the language.

Definition and expansion of macros both happen in a single depth-first,
lexical-order traversal of a crate's source. So a macro defined at
module scope is visible to any subsequent code in the same module, which
includes the body of any subsequent child \texttt{mod} items.

A macro defined within the body of a single \texttt{fn}, or anywhere
else not at module scope, is visible only within that item.

If a module has the \texttt{macro\_use} attribute, its macros are also
visible in its parent module after the child's \texttt{mod} item. If the
parent also has \texttt{macro\_use} then the macros will be visible in
the grandparent after the parent's \texttt{mod} item, and so forth.

The \texttt{macro\_use} attribute can also appear on
\texttt{extern\ crate}. In this context it controls which macros are
loaded from the external crate, e.g.

\begin{Shaded}
\begin{Highlighting}[]
\AttributeTok{#[}\NormalTok{macro_use}\AttributeTok{(}\NormalTok{foo}\AttributeTok{,} \NormalTok{bar}\AttributeTok{)]}
\KeywordTok{extern} \KeywordTok{crate} \NormalTok{baz;}
\end{Highlighting}
\end{Shaded}

If the attribute is given simply as \texttt{\#{[}macro\_use{]}}, all
macros are loaded. If there is no \texttt{\#{[}macro\_use{]}} attribute
then no macros are loaded. Only macros defined with the
\texttt{\#{[}macro\_export{]}} attribute may be loaded.

To load a crate's macros without linking it into the output, use
\texttt{\#{[}no\_link{]}} as well.

An example:

\begin{Shaded}
\begin{Highlighting}[]
\PreprocessorTok{macro_rules!} \NormalTok{m1 \{ () => (()) \}}

\CommentTok{// visible here: m1}

\KeywordTok{mod} \NormalTok{foo \{}
    \CommentTok{// visible here: m1}

    \AttributeTok{#[}\NormalTok{macro_export}\AttributeTok{]}
    \PreprocessorTok{macro_rules!} \NormalTok{m2 \{ () => (()) \}}

    \CommentTok{// visible here: m1, m2}
\NormalTok{\}}

\CommentTok{// visible here: m1}

\PreprocessorTok{macro_rules!} \NormalTok{m3 \{ () => (()) \}}

\CommentTok{// visible here: m1, m3}

\AttributeTok{#[}\NormalTok{macro_use}\AttributeTok{]}
\KeywordTok{mod} \NormalTok{bar \{}
    \CommentTok{// visible here: m1, m3}

    \PreprocessorTok{macro_rules!} \NormalTok{m4 \{ () => (()) \}}

    \CommentTok{// visible here: m1, m3, m4}
\NormalTok{\}}

\CommentTok{// visible here: m1, m3, m4}
\end{Highlighting}
\end{Shaded}

When this library is loaded with
\texttt{\#{[}macro\_use{]}\ extern\ crate}, only \texttt{m2} will be
imported.

The Rust Reference has a
\href{http://doc.rust-lang.org/reference.html\#macro-related-attributes}{listing
of macro-related attributes}.

\subsection{\texorpdfstring{The variable
\texttt{\$crate}}{The variable \$crate}}\label{the-variable-crate}

A further difficulty occurs when a macro is used in multiple crates. Say
that \texttt{mylib} defines

\begin{Shaded}
\begin{Highlighting}[]
\KeywordTok{pub} \KeywordTok{fn} \NormalTok{increment(x: }\DataTypeTok{u32}\NormalTok{) -> }\DataTypeTok{u32} \NormalTok{\{}
    \NormalTok{x + }\DecValTok{1}
\NormalTok{\}}

\AttributeTok{#[}\NormalTok{macro_export}\AttributeTok{]}
\PreprocessorTok{macro_rules!} \NormalTok{inc_a \{}
    \NormalTok{($x:expr) => ( ::increment($x) )}
\NormalTok{\}}

\AttributeTok{#[}\NormalTok{macro_export}\AttributeTok{]}
\PreprocessorTok{macro_rules!} \NormalTok{inc_b \{}
    \NormalTok{($x:expr) => ( ::mylib::increment($x) )}
\NormalTok{\}}
\end{Highlighting}
\end{Shaded}

\texttt{inc\_a} only works within \texttt{mylib}, while \texttt{inc\_b}
only works outside the library. Furthermore, \texttt{inc\_b} will break
if the user imports \texttt{mylib} under another name.

Rust does not (yet) have a hygiene system for crate references, but it
does provide a simple workaround for this problem. Within a macro
imported from a crate named \texttt{foo}, the special macro variable
\texttt{\$crate} will expand to \texttt{::foo}. By contrast, when a
macro is defined and then used in the same crate, \texttt{\$crate} will
expand to nothing. This means we can write

\begin{Shaded}
\begin{Highlighting}[]
\AttributeTok{#[}\NormalTok{macro_export}\AttributeTok{]}
\PreprocessorTok{macro_rules!} \NormalTok{inc \{}
    \NormalTok{($x:expr) => ( $crate::increment($x) )}
\NormalTok{\}}
\end{Highlighting}
\end{Shaded}

to define a single macro that works both inside and outside our library.
The function name will expand to either \texttt{::increment} or
\texttt{::mylib::increment}.

To keep this system simple and correct,
\texttt{\#{[}macro\_use{]}\ extern\ crate\ ...} may only appear at the
root of your crate, not inside \texttt{mod}.

\subsection{The deep end}\label{the-deep-end}

The introductory chapter mentioned recursive macros, but it did not give
the full story. Recursive macros are useful for another reason: Each
recursive invocation gives you another opportunity to pattern-match the
macro's arguments.

As an extreme example, it is possible, though hardly advisable, to
implement the
\href{https://esolangs.org/wiki/Bitwise_Cyclic_Tag}{Bitwise Cyclic Tag}
automaton within Rust's macro system.

\begin{Shaded}
\begin{Highlighting}[]
\PreprocessorTok{macro_rules!} \NormalTok{bct \{}
    \CommentTok{// cmd 0:  d ... => ...}
    \NormalTok{(}\DecValTok{0}\NormalTok{, $($ps:tt),* ; $_d:tt)}
        \NormalTok{=> (}\PreprocessorTok{bct!}\NormalTok{($($ps),*, }\DecValTok{0} \NormalTok{; ));}
    \NormalTok{(}\DecValTok{0}\NormalTok{, $($ps:tt),* ; $_d:tt, $($ds:tt),*)}
        \NormalTok{=> (}\PreprocessorTok{bct!}\NormalTok{($($ps),*, }\DecValTok{0} \NormalTok{; $($ds),*));}

    \CommentTok{// cmd 1p:  1 ... => 1 ... p}
    \NormalTok{(}\DecValTok{1}\NormalTok{, $p:tt, $($ps:tt),* ; }\DecValTok{1}\NormalTok{)}
        \NormalTok{=> (}\PreprocessorTok{bct!}\NormalTok{($($ps),*, }\DecValTok{1}\NormalTok{, $p ; }\DecValTok{1}\NormalTok{, $p));}
    \NormalTok{(}\DecValTok{1}\NormalTok{, $p:tt, $($ps:tt),* ; }\DecValTok{1}\NormalTok{, $($ds:tt),*)}
        \NormalTok{=> (}\PreprocessorTok{bct!}\NormalTok{($($ps),*, }\DecValTok{1}\NormalTok{, $p ; }\DecValTok{1}\NormalTok{, $($ds),*, $p));}

    \CommentTok{// cmd 1p:  0 ... => 0 ...}
    \NormalTok{(}\DecValTok{1}\NormalTok{, $p:tt, $($ps:tt),* ; $($ds:tt),*)}
        \NormalTok{=> (}\PreprocessorTok{bct!}\NormalTok{($($ps),*, }\DecValTok{1}\NormalTok{, $p ; $($ds),*));}

    \CommentTok{// halt on empty data string}
    \NormalTok{( $($ps:tt),* ; )}
        \NormalTok{=> (());}
\NormalTok{\}}
\end{Highlighting}
\end{Shaded}

Exercise: use macros to reduce duplication in the above definition of
the \texttt{bct!} macro.

\subsection{Common macros}\label{common-macros}

Here are some common macros you'll see in Rust code.

\subsubsection{panic!}\label{panic}

This macro causes the current thread to panic. You can give it a message
to panic with:

\begin{Shaded}
\begin{Highlighting}[]
\PreprocessorTok{panic!}\NormalTok{(}\StringTok{"oh no!"}\NormalTok{);}
\end{Highlighting}
\end{Shaded}

\subsubsection{vec!}\label{vec}

The \texttt{vec!} macro is used throughout the book, so you've probably
seen it already. It creates \texttt{Vec\textless{}T\textgreater{}}s with
ease:

\begin{Shaded}
\begin{Highlighting}[]
\KeywordTok{let} \NormalTok{v = }\PreprocessorTok{vec!}\NormalTok{[}\DecValTok{1}\NormalTok{, }\DecValTok{2}\NormalTok{, }\DecValTok{3}\NormalTok{, }\DecValTok{4}\NormalTok{, }\DecValTok{5}\NormalTok{];}
\end{Highlighting}
\end{Shaded}

It also lets you make vectors with repeating values. For example, a
hundred zeroes:

\begin{Shaded}
\begin{Highlighting}[]
\KeywordTok{let} \NormalTok{v = }\PreprocessorTok{vec!}\NormalTok{[}\DecValTok{0}\NormalTok{; }\DecValTok{100}\NormalTok{];}
\end{Highlighting}
\end{Shaded}

\subsubsection{assert! and assert\_eq!}\label{assert-and-assertux5feq}

These two macros are used in tests. \texttt{assert!} takes a boolean.
\texttt{assert\_eq!} takes two values and checks them for equality.
\texttt{true} passes, \texttt{false} \texttt{panic!}s. Like this:

\begin{Shaded}
\begin{Highlighting}[]
\CommentTok{// A-ok!}

\PreprocessorTok{assert!}\NormalTok{(}\ConstantTok{true}\NormalTok{);}
\PreprocessorTok{assert_eq!}\NormalTok{(}\DecValTok{5}\NormalTok{, }\DecValTok{3} \NormalTok{+ }\DecValTok{2}\NormalTok{);}

\CommentTok{// nope :(}

\PreprocessorTok{assert!}\NormalTok{(}\DecValTok{5} \NormalTok{< }\DecValTok{3}\NormalTok{);}
\PreprocessorTok{assert_eq!}\NormalTok{(}\DecValTok{5}\NormalTok{, }\DecValTok{3}\NormalTok{);}
\end{Highlighting}
\end{Shaded}

\subsubsection{try!}\label{try}

\texttt{try!} is used for error handling. It takes something that can
return a \texttt{Result\textless{}T,\ E\textgreater{}}, and gives
\texttt{T} if it's a \texttt{Ok\textless{}T\textgreater{}}, and
\texttt{return}s with the \texttt{Err(E)} if it's that. Like this:

\begin{Shaded}
\begin{Highlighting}[]
\KeywordTok{use} \NormalTok{std::fs::File;}

\KeywordTok{fn} \NormalTok{foo() -> std::io::}\DataTypeTok{Result}\NormalTok{<()> \{}
    \KeywordTok{let} \NormalTok{f = }\PreprocessorTok{try!}\NormalTok{(File::create(}\StringTok{"foo.txt"}\NormalTok{));}

    \ConstantTok{Ok}\NormalTok{(())}
\NormalTok{\}}
\end{Highlighting}
\end{Shaded}

This is cleaner than doing this:

\begin{Shaded}
\begin{Highlighting}[]
\KeywordTok{use} \NormalTok{std::fs::File;}

\KeywordTok{fn} \NormalTok{foo() -> std::io::}\DataTypeTok{Result}\NormalTok{<()> \{}
    \KeywordTok{let} \NormalTok{f = File::create(}\StringTok{"foo.txt"}\NormalTok{);}

    \KeywordTok{let} \NormalTok{f = }\KeywordTok{match} \NormalTok{f \{}
        \ConstantTok{Ok}\NormalTok{(t) => t,}
        \ConstantTok{Err}\NormalTok{(e) => }\KeywordTok{return} \ConstantTok{Err}\NormalTok{(e),}
    \NormalTok{\};}

    \ConstantTok{Ok}\NormalTok{(())}
\NormalTok{\}}
\end{Highlighting}
\end{Shaded}

\subsubsection{unreachable!}\label{unreachable}

This macro is used when you think some code should never execute:

\begin{Shaded}
\begin{Highlighting}[]
\KeywordTok{if} \ConstantTok{false} \NormalTok{\{}
    \PreprocessorTok{unreachable!}\NormalTok{();}
\NormalTok{\}}
\end{Highlighting}
\end{Shaded}

Sometimes, the compiler may make you have a different branch that you
know will never, ever run. In these cases, use this macro, so that if
you end up wrong, you'll get a \texttt{panic!} about it.

\begin{Shaded}
\begin{Highlighting}[]
\KeywordTok{let} \NormalTok{x: }\DataTypeTok{Option}\NormalTok{<}\DataTypeTok{i32}\NormalTok{> = }\ConstantTok{None}\NormalTok{;}

\KeywordTok{match} \NormalTok{x \{}
    \ConstantTok{Some}\NormalTok{(_) => }\PreprocessorTok{unreachable!}\NormalTok{(),}
    \ConstantTok{None} \NormalTok{=> }\PreprocessorTok{println!}\NormalTok{(}\StringTok{"I know x is None!"}\NormalTok{),}
\NormalTok{\}}
\end{Highlighting}
\end{Shaded}

\subsubsection{unimplemented!}\label{unimplemented}

The \texttt{unimplemented!} macro can be used when you're trying to get
your functions to typecheck, and don't want to worry about writing out
the body of the function. One example of this situation is implementing
a trait with multiple required methods, where you want to tackle one at
a time. Define the others as \texttt{unimplemented!} until you're ready
to write them.

\subsection{Procedural macros}\label{procedural-macros}

If Rust's macro system can't do what you need, you may want to write a
\protect\hyperlink{sec--compiler-plugins}{compiler plugin} instead.
Compared to \texttt{macro\_rules!} macros, this is significantly more
work, the interfaces are much less stable, and bugs can be much harder
to track down. In exchange you get the flexibility of running arbitrary
Rust code within the compiler. Syntax extension plugins are sometimes
called `procedural macros' for this reason.

\hypertarget{sec--raw-pointers}{\section{Raw
Pointers}\label{sec--raw-pointers}}

Rust has a number of different smart pointer types in its standard
library, but there are two types that are extra-special. Much of Rust's
safety comes from compile-time checks, but raw pointers don't have such
guarantees, and are \protect\hyperlink{sec--unsafe}{unsafe} to use.

\texttt{*const\ T} and \texttt{*mut\ T} are called `raw pointers' in
Rust. Sometimes, when writing certain kinds of libraries, you'll need to
get around Rust's safety guarantees for some reason. In this case, you
can use raw pointers to implement your library, while exposing a safe
interface for your users. For example, \texttt{*} pointers are allowed
to alias, allowing them to be used to write shared-ownership types, and
even thread-safe shared memory types (the
\texttt{Rc\textless{}T\textgreater{}} and
\texttt{Arc\textless{}T\textgreater{}} types are both implemented
entirely in Rust).

Here are some things to remember about raw pointers that are different
than other pointer types. They:

\begin{itemize}
\tightlist
\item
  are not guaranteed to point to valid memory and are not even
  guaranteed to be non-null (unlike both \texttt{Box} and \texttt{\&});
\item
  do not have any automatic clean-up, unlike \texttt{Box}, and so
  require manual resource management;
\item
  are plain-old-data, that is, they don't move ownership, again unlike
  \texttt{Box}, hence the Rust compiler cannot protect against bugs like
  use-after-free;
\item
  lack any form of lifetimes, unlike \texttt{\&}, and so the compiler
  cannot reason about dangling pointers; and
\item
  have no guarantees about aliasing or mutability other than mutation
  not being allowed directly through a \texttt{*const\ T}.
\end{itemize}

\subsection{Basics}\label{basics}

Creating a raw pointer is perfectly safe:

\begin{Shaded}
\begin{Highlighting}[]
\KeywordTok{let} \NormalTok{x = }\DecValTok{5}\NormalTok{;}
\KeywordTok{let} \NormalTok{raw = &x }\KeywordTok{as} \NormalTok{*}\KeywordTok{const} \DataTypeTok{i32}\NormalTok{;}

\KeywordTok{let} \KeywordTok{mut} \NormalTok{y = }\DecValTok{10}\NormalTok{;}
\KeywordTok{let} \NormalTok{raw_mut = &}\KeywordTok{mut} \NormalTok{y }\KeywordTok{as} \NormalTok{*}\KeywordTok{mut} \DataTypeTok{i32}\NormalTok{;}
\end{Highlighting}
\end{Shaded}

However, dereferencing one is not. This won't work:

\begin{Shaded}
\begin{Highlighting}[]
\KeywordTok{let} \NormalTok{x = }\DecValTok{5}\NormalTok{;}
\KeywordTok{let} \NormalTok{raw = &x }\KeywordTok{as} \NormalTok{*}\KeywordTok{const} \DataTypeTok{i32}\NormalTok{;}

\PreprocessorTok{println!}\NormalTok{(}\StringTok{"raw points at \{\}"}\NormalTok{, *raw);}
\end{Highlighting}
\end{Shaded}

It gives this error:

\begin{verbatim}
error: dereference of raw pointer requires unsafe function or block [E0133]
     println!("raw points at {}", *raw);
                                  ^~~~
\end{verbatim}

When you dereference a raw pointer, you're taking responsibility that
it's not pointing somewhere that would be incorrect. As such, you need
\texttt{unsafe}:

\begin{Shaded}
\begin{Highlighting}[]
\KeywordTok{let} \NormalTok{x = }\DecValTok{5}\NormalTok{;}
\KeywordTok{let} \NormalTok{raw = &x }\KeywordTok{as} \NormalTok{*}\KeywordTok{const} \DataTypeTok{i32}\NormalTok{;}

\KeywordTok{let} \NormalTok{points_at = }\KeywordTok{unsafe} \NormalTok{\{ *raw \};}

\PreprocessorTok{println!}\NormalTok{(}\StringTok{"raw points at \{\}"}\NormalTok{, points_at);}
\end{Highlighting}
\end{Shaded}

For more operations on raw pointers, see
\href{http://doc.rust-lang.org/std/primitive.pointer.html}{their API
documentation}.

\subsection{FFI}\label{ffi}

Raw pointers are useful for FFI: Rust's \texttt{*const\ T} and
\texttt{*mut\ T} are similar to C's \texttt{const\ T*} and \texttt{T*},
respectively. For more about this use, consult the
\protect\hyperlink{sec--ffi}{FFI chapter}.

\subsection{References and raw
pointers}\label{references-and-raw-pointers}

At runtime, a raw pointer \texttt{*} and a reference pointing to the
same piece of data have an identical representation. In fact, an
\texttt{\&T} reference will implicitly coerce to an \texttt{*const\ T}
raw pointer in safe code and similarly for the \texttt{mut} variants
(both coercions can be performed explicitly with, respectively,
\texttt{value\ as\ *const\ T} and \texttt{value\ as\ *mut\ T}).

Going the opposite direction, from \texttt{*const} to a reference
\texttt{\&}, is not safe. A \texttt{\&T} is always valid, and so, at a
minimum, the raw pointer \texttt{*const\ T} has to point to a valid
instance of type \texttt{T}. Furthermore, the resulting pointer must
satisfy the aliasing and mutability laws of references. The compiler
assumes these properties are true for any references, no matter how they
are created, and so any conversion from raw pointers is asserting that
they hold. The programmer \emph{must} guarantee this.

The recommended method for the conversion is:

\begin{Shaded}
\begin{Highlighting}[]
\CommentTok{// explicit cast}
\KeywordTok{let} \NormalTok{i: }\DataTypeTok{u32} \NormalTok{= }\DecValTok{1}\NormalTok{;}
\KeywordTok{let} \NormalTok{p_imm: *}\KeywordTok{const} \DataTypeTok{u32} \NormalTok{= &i }\KeywordTok{as} \NormalTok{*}\KeywordTok{const} \DataTypeTok{u32}\NormalTok{;}

\CommentTok{// implicit coercion}
\KeywordTok{let} \KeywordTok{mut} \NormalTok{m: }\DataTypeTok{u32} \NormalTok{= }\DecValTok{2}\NormalTok{;}
\KeywordTok{let} \NormalTok{p_mut: *}\KeywordTok{mut} \DataTypeTok{u32} \NormalTok{= &}\KeywordTok{mut} \NormalTok{m;}

\KeywordTok{unsafe} \NormalTok{\{}
    \KeywordTok{let} \NormalTok{ref_imm: &}\DataTypeTok{u32} \NormalTok{= &*p_imm;}
    \KeywordTok{let} \NormalTok{ref_mut: &}\KeywordTok{mut} \DataTypeTok{u32} \NormalTok{= &}\KeywordTok{mut} \NormalTok{*p_mut;}
\NormalTok{\}}
\end{Highlighting}
\end{Shaded}

The \texttt{\&*x} dereferencing style is preferred to using a
\texttt{transmute}. The latter is far more powerful than necessary, and
the more restricted operation is harder to use incorrectly; for example,
it requires that \texttt{x} is a pointer (unlike \texttt{transmute}).

\hypertarget{sec--unsafe}{\section{\texorpdfstring{\texttt{unsafe}}{unsafe}}\label{sec--unsafe}}

Rust's main draw is its powerful static guarantees about behavior. But
safety checks are conservative by nature: there are some programs that
are actually safe, but the compiler is not able to verify this is true.
To write these kinds of programs, we need to tell the compiler to relax
its restrictions a bit. For this, Rust has a keyword, \texttt{unsafe}.
Code using \texttt{unsafe} has fewer restrictions than normal code does.

Let's go over the syntax, and then we'll talk semantics. \texttt{unsafe}
is used in four contexts. The first one is to mark a function as unsafe:

\begin{Shaded}
\begin{Highlighting}[]
\KeywordTok{unsafe} \KeywordTok{fn} \NormalTok{danger_will_robinson() \{}
    \CommentTok{// scary stuff}
\NormalTok{\}}
\end{Highlighting}
\end{Shaded}

All functions called from \protect\hyperlink{sec--ffi}{FFI} must be
marked as \texttt{unsafe}, for example. The second use of
\texttt{unsafe} is an unsafe block:

\begin{Shaded}
\begin{Highlighting}[]
\KeywordTok{unsafe} \NormalTok{\{}
    \CommentTok{// scary stuff}
\NormalTok{\}}
\end{Highlighting}
\end{Shaded}

The third is for unsafe traits:

\begin{Shaded}
\begin{Highlighting}[]
\KeywordTok{unsafe} \KeywordTok{trait} \NormalTok{Scary \{ \}}
\end{Highlighting}
\end{Shaded}

And the fourth is for \texttt{impl}ementing one of those traits:

\begin{Shaded}
\begin{Highlighting}[]
\KeywordTok{unsafe} \KeywordTok{impl} \NormalTok{Scary }\KeywordTok{for} \DataTypeTok{i32} \NormalTok{\{\}}
\end{Highlighting}
\end{Shaded}

It's important to be able to explicitly delineate code that may have
bugs that cause big problems. If a Rust program segfaults, you can be
sure the cause is related to something marked \texttt{unsafe}.

\subsection{\texorpdfstring{What does `safe'
mean?}{What does safe mean?}}\label{what-does-safe-mean}

Safe, in the context of Rust, means `doesn't do anything unsafe'. It's
also important to know that there are certain behaviors that are
probably not desirable in your code, but are expressly \emph{not}
unsafe:

\begin{itemize}
\tightlist
\item
  Deadlocks
\item
  Leaks of memory or other resources
\item
  Exiting without calling destructors
\item
  Integer overflow
\end{itemize}

Rust cannot prevent all kinds of software problems. Buggy code can and
will be written in Rust. These things aren't great, but they don't
qualify as \texttt{unsafe} specifically.

In addition, the following are all undefined behaviors in Rust, and must
be avoided, even when writing \texttt{unsafe} code:

\begin{itemize}
\tightlist
\item
  Data races
\item
  Dereferencing a null/dangling raw pointer
\item
  Reads of
  \href{http://llvm.org/docs/LangRef.html\#undefined-values}{undef}
  (uninitialized) memory
\item
  Breaking the
  \href{http://llvm.org/docs/LangRef.html\#pointer-aliasing-rules}{pointer
  aliasing rules} with raw pointers.
\item
  \texttt{\&mut\ T} and \texttt{\&T} follow LLVM's scoped
  \href{http://llvm.org/docs/LangRef.html\#noalias}{noalias} model,
  except if the \texttt{\&T} contains an
  \texttt{UnsafeCell\textless{}U\textgreater{}}. Unsafe code must not
  violate these aliasing guarantees.
\item
  Mutating an immutable value/reference without
  \texttt{UnsafeCell\textless{}U\textgreater{}}
\item
  Invoking undefined behavior via compiler intrinsics:
\item
  Indexing outside of the bounds of an object with
  \texttt{std::ptr::offset} (\texttt{offset} intrinsic), with the
  exception of one byte past the end which is permitted.
\item
  Using \texttt{std::ptr::copy\_nonoverlapping\_memory}
  (\texttt{memcpy32}/\texttt{memcpy64} intrinsics) on overlapping
  buffers
\item
  Invalid values in primitive types, even in private fields/locals:
\item
  Null/dangling references or boxes
\item
  A value other than \texttt{false} (0) or \texttt{true} (1) in a
  \texttt{bool}
\item
  A discriminant in an \texttt{enum} not included in its type definition
\item
  A value in a \texttt{char} which is a surrogate or above
  \texttt{char::MAX}
\item
  Non-UTF-8 byte sequences in a \texttt{str}
\item
  Unwinding into Rust from foreign code or unwinding from Rust into
  foreign code.
\end{itemize}

\subsection{Unsafe Superpowers}\label{unsafe-superpowers}

In both unsafe functions and unsafe blocks, Rust will let you do three
things that you normally can not do. Just three. Here they are:

\begin{enumerate}
\def\labelenumi{\arabic{enumi}.}
\tightlist
\item
  Access or update a \protect\hyperlink{static}{static mutable
  variable}.
\item
  Dereference a raw pointer.
\item
  Call unsafe functions. This is the most powerful ability.
\end{enumerate}

That's it. It's important that \texttt{unsafe} does not, for example,
`turn off the borrow checker'. Adding \texttt{unsafe} to some random
Rust code doesn't change its semantics, it won't start accepting
anything. But it will let you write things that \emph{do} break some of
the rules.

You will also encounter the \texttt{unsafe} keyword when writing
bindings to foreign (non-Rust) interfaces. You're encouraged to write a
safe, native Rust interface around the methods provided by the library.

Let's go over the basic three abilities listed, in order.

\subsubsection{\texorpdfstring{Access or update a
\texttt{static\ mut}}{Access or update a static mut}}\label{access-or-update-a-static-mut}

Rust has a feature called `\texttt{static\ mut}' which allows for
mutable global state. Doing so can cause a data race, and as such is
inherently not safe. For more details, see the
\protect\hyperlink{static}{static} section of the book.

\subsubsection{Dereference a raw
pointer}\label{dereference-a-raw-pointer}

Raw pointers let you do arbitrary pointer arithmetic, and can cause a
number of different memory safety and security issues. In some senses,
the ability to dereference an arbitrary pointer is one of the most
dangerous things you can do. For more on raw pointers, see
\protect\hyperlink{sec--raw-pointers}{their section of the book}.

\subsubsection{Call unsafe functions}\label{call-unsafe-functions}

This last ability works with both aspects of \texttt{unsafe}: you can
only call functions marked \texttt{unsafe} from inside an unsafe block.

This ability is powerful and varied. Rust exposes some
\protect\hyperlink{sec--intrinsics}{compiler intrinsics} as unsafe
functions, and some unsafe functions bypass safety checks, trading
safety for speed.

I'll repeat again: even though you \emph{can} do arbitrary things in
unsafe blocks and functions doesn't mean you should. The compiler will
act as though you're upholding its invariants, so be careful!

\hypertarget{sec--effective-rust}{\chapter{Effective
Rust}\label{sec--effective-rust}}

So you've learned how to write some Rust code. But there's a difference
between writing \emph{any} Rust code and writing \emph{good} Rust code.

This chapter consists of relatively independent tutorials which show you
how to take your Rust to the next level. Common patterns and standard
library features will be introduced. Read these sections in any order of
your choosing.

\hypertarget{sec--the-stack-and-the-heap}{\section{The Stack and the
Heap}\label{sec--the-stack-and-the-heap}}

As a systems language, Rust operates at a low level. If you're coming
from a high-level language, there are some aspects of systems
programming that you may not be familiar with. The most important one is
how memory works, with a stack and a heap. If you're familiar with how
C-like languages use stack allocation, this chapter will be a refresher.
If you're not, you'll learn about this more general concept, but with a
Rust-y focus.

As with most things, when learning about them, we'll use a simplified
model to start. This lets you get a handle on the basics, without
getting bogged down with details which are, for now, irrelevant. The
examples we'll use aren't 100\% accurate, but are representative for the
level we're trying to learn at right now. Once you have the basics down,
learning more about how allocators are implemented, virtual memory, and
other advanced topics will reveal the leaks in this particular
abstraction.

\subsection{Memory management}\label{memory-management}

These two terms are about memory management. The stack and the heap are
abstractions that help you determine when to allocate and deallocate
memory.

Here's a high-level comparison:

The stack is very fast, and is where memory is allocated in Rust by
default. But the allocation is local to a function call, and is limited
in size. The heap, on the other hand, is slower, and is explicitly
allocated by your program. But it's effectively unlimited in size, and
is globally accessible.

\hypertarget{the-stack}{\subsection{The Stack}\label{the-stack}}

Let's talk about this Rust program:

\begin{Shaded}
\begin{Highlighting}[]
\KeywordTok{fn} \NormalTok{main() \{}
    \KeywordTok{let} \NormalTok{x = }\DecValTok{42}\NormalTok{;}
\NormalTok{\}}
\end{Highlighting}
\end{Shaded}

This program has one variable binding, \texttt{x}. This memory needs to
be allocated from somewhere. Rust `stack allocates' by default, which
means that basic values `go on the stack'. What does that mean?

Well, when a function gets called, some memory gets allocated for all of
its local variables and some other information. This is called a `stack
frame', and for the purpose of this tutorial, we're going to ignore the
extra information and only consider the local variables we're
allocating. So in this case, when \texttt{main()} is run, we'll allocate
a single 32-bit integer for our stack frame. This is automatically
handled for you, as you can see; we didn't have to write any special
Rust code or anything.

When the function exits, its stack frame gets deallocated. This happens
automatically as well.

That's all there is for this simple program. The key thing to understand
here is that stack allocation is very, very fast. Since we know all the
local variables we have ahead of time, we can grab the memory all at
once. And since we'll throw them all away at the same time as well, we
can get rid of it very fast too.

The downside is that we can't keep values around if we need them for
longer than a single function. We also haven't talked about what the
word, `stack', means. To do that, we need a slightly more complicated
example:

\begin{Shaded}
\begin{Highlighting}[]
\KeywordTok{fn} \NormalTok{foo() \{}
    \KeywordTok{let} \NormalTok{y = }\DecValTok{5}\NormalTok{;}
    \KeywordTok{let} \NormalTok{z = }\DecValTok{100}\NormalTok{;}
\NormalTok{\}}

\KeywordTok{fn} \NormalTok{main() \{}
    \KeywordTok{let} \NormalTok{x = }\DecValTok{42}\NormalTok{;}

    \NormalTok{foo();}
\NormalTok{\}}
\end{Highlighting}
\end{Shaded}

This program has three variables total: two in \texttt{foo()}, one in
\texttt{main()}. Just as before, when \texttt{main()} is called, a
single integer is allocated for its stack frame. But before we can show
what happens when \texttt{foo()} is called, we need to visualize what's
going on with memory. Your operating system presents a view of memory to
your program that's pretty simple: a huge list of addresses, from 0 to a
large number, representing how much RAM your computer has. For example,
if you have a gigabyte of RAM, your addresses go from \texttt{0} to
\texttt{1,073,741,823}. That number comes from 2\textsuperscript{30},
the number of bytes in a gigabyte. \footnote{`Gigabyte' can mean two
  things: 10\^{}9, or 2\^{}30. The SI standard resolved this by stating
  that `gigabyte' is 10\^{}9, and `gibibyte' is 2\^{}30. However, very
  few people use this terminology, and rely on context to differentiate.
  We follow in that tradition here.}

This memory is kind of like a giant array: addresses start at zero and
go up to the final number. So here's a diagram of our first stack frame:

\begin{longtable}[c]{@{}lll@{}}
\toprule
Address & Name & Value\tabularnewline
\midrule
\endhead
0 & x & 42\tabularnewline
\bottomrule
\end{longtable}

We've got \texttt{x} located at address \texttt{0}, with the value
\texttt{42}.

When \texttt{foo()} is called, a new stack frame is allocated:

\begin{longtable}[c]{@{}lll@{}}
\toprule
Address & Name & Value\tabularnewline
\midrule
\endhead
2 & z & 100\tabularnewline
1 & y & 5\tabularnewline
0 & x & 42\tabularnewline
\bottomrule
\end{longtable}

Because \texttt{0} was taken by the first frame, \texttt{1} and
\texttt{2} are used for \texttt{foo()}'s stack frame. It grows upward,
the more functions we call.

There are some important things we have to take note of here. The
numbers 0, 1, and 2 are all solely for illustrative purposes, and bear
no relationship to the address values the computer will use in reality.
In particular, the series of addresses are in reality going to be
separated by some number of bytes that separate each address, and that
separation may even exceed the size of the value being stored.

After \texttt{foo()} is over, its frame is deallocated:

\begin{longtable}[c]{@{}lll@{}}
\toprule
Address & Name & Value\tabularnewline
\midrule
\endhead
0 & x & 42\tabularnewline
\bottomrule
\end{longtable}

And then, after \texttt{main()}, even this last value goes away. Easy!

It's called a `stack' because it works like a stack of dinner plates:
the first plate you put down is the last plate to pick back up. Stacks
are sometimes called `last in, first out queues' for this reason, as the
last value you put on the stack is the first one you retrieve from it.

Let's try a three-deep example:

\begin{Shaded}
\begin{Highlighting}[]
\KeywordTok{fn} \NormalTok{italic() \{}
    \KeywordTok{let} \NormalTok{i = }\DecValTok{6}\NormalTok{;}
\NormalTok{\}}

\KeywordTok{fn} \NormalTok{bold() \{}
    \KeywordTok{let} \NormalTok{a = }\DecValTok{5}\NormalTok{;}
    \KeywordTok{let} \NormalTok{b = }\DecValTok{100}\NormalTok{;}
    \KeywordTok{let} \NormalTok{c = }\DecValTok{1}\NormalTok{;}

    \NormalTok{italic();}
\NormalTok{\}}

\KeywordTok{fn} \NormalTok{main() \{}
    \KeywordTok{let} \NormalTok{x = }\DecValTok{42}\NormalTok{;}

    \NormalTok{bold();}
\NormalTok{\}}
\end{Highlighting}
\end{Shaded}

We have some kooky function names to make the diagrams clearer.

Okay, first, we call \texttt{main()}:

\begin{longtable}[c]{@{}lll@{}}
\toprule
Address & Name & Value\tabularnewline
\midrule
\endhead
0 & x & 42\tabularnewline
\bottomrule
\end{longtable}

Next up, \texttt{main()} calls \texttt{bold()}:

\begin{longtable}[c]{@{}lll@{}}
\toprule
Address & Name & Value\tabularnewline
\midrule
\endhead
\textbf{3} & \textbf{c} & \textbf{1}\tabularnewline
\textbf{2} & \textbf{b} & \textbf{100}\tabularnewline
\textbf{1} & \textbf{a} & \textbf{5}\tabularnewline
0 & x & 42\tabularnewline
\bottomrule
\end{longtable}

And then \texttt{bold()} calls \texttt{italic()}:

\begin{longtable}[c]{@{}lll@{}}
\toprule
Address & Name & Value\tabularnewline
\midrule
\endhead
\emph{4} & \emph{i} & \emph{6}\tabularnewline
\textbf{3} & \textbf{c} & \textbf{1}\tabularnewline
\textbf{2} & \textbf{b} & \textbf{100}\tabularnewline
\textbf{1} & \textbf{a} & \textbf{5}\tabularnewline
0 & x & 42\tabularnewline
\bottomrule
\end{longtable}

Whew! Our stack is growing tall.

After \texttt{italic()} is over, its frame is deallocated, leaving only
\texttt{bold()} and \texttt{main()}:

\begin{longtable}[c]{@{}lll@{}}
\toprule
Address & Name & Value\tabularnewline
\midrule
\endhead
\textbf{3} & \textbf{c} & \textbf{1}\tabularnewline
\textbf{2} & \textbf{b} & \textbf{100}\tabularnewline
\textbf{1} & \textbf{a} & \textbf{5}\tabularnewline
0 & x & 42\tabularnewline
\bottomrule
\end{longtable}

And then \texttt{bold()} ends, leaving only \texttt{main()}:

\begin{longtable}[c]{@{}lll@{}}
\toprule
Address & Name & Value\tabularnewline
\midrule
\endhead
0 & x & 42\tabularnewline
\bottomrule
\end{longtable}

And then we're done. Getting the hang of it? It's like piling up dishes:
you add to the top, you take away from the top.

\subsection{The Heap}\label{the-heap}

Now, this works pretty well, but not everything can work like this.
Sometimes, you need to pass some memory between different functions, or
keep it alive for longer than a single function's execution. For this,
we can use the heap.

In Rust, you can allocate memory on the heap with the
\href{http://doc.rust-lang.org/std/boxed/index.html}{\texttt{Box\textless{}T\textgreater{}}
type}. Here's an example:

\begin{Shaded}
\begin{Highlighting}[]
\KeywordTok{fn} \NormalTok{main() \{}
    \KeywordTok{let} \NormalTok{x = }\DataTypeTok{Box}\NormalTok{::new(}\DecValTok{5}\NormalTok{);}
    \KeywordTok{let} \NormalTok{y = }\DecValTok{42}\NormalTok{;}
\NormalTok{\}}
\end{Highlighting}
\end{Shaded}

Here's what happens in memory when \texttt{main()} is called:

\begin{longtable}[c]{@{}lll@{}}
\toprule
Address & Name & Value\tabularnewline
\midrule
\endhead
1 & y & 42\tabularnewline
0 & x & ??????\tabularnewline
\bottomrule
\end{longtable}

We allocate space for two variables on the stack. \texttt{y} is
\texttt{42}, as it always has been, but what about \texttt{x}? Well,
\texttt{x} is a \texttt{Box\textless{}i32\textgreater{}}, and boxes
allocate memory on the heap. The actual value of the box is a structure
which has a pointer to `the heap'. When we start executing the function,
and \texttt{Box::new()} is called, it allocates some memory for the
heap, and puts \texttt{5} there. The memory now looks like this:

\begin{longtable}[c]{@{}lll@{}}
\toprule
Address & Name & Value\tabularnewline
\midrule
\endhead
(2\textsuperscript{30}) - 1 & & 5\tabularnewline
\ldots{} & \ldots{} & \ldots{}\tabularnewline
1 & y & 42\tabularnewline
0 & x & → (2\textsuperscript{30}) - 1\tabularnewline
\bottomrule
\end{longtable}

We have (2\textsuperscript{30}) - 1 addresses in our hypothetical
computer with 1GB of RAM. And since our stack grows from zero, the
easiest place to allocate memory is from the other end. So our first
value is at the highest place in memory. And the value of the struct at
\texttt{x} has a \protect\hyperlink{sec--raw-pointers}{raw pointer} to
the place we've allocated on the heap, so the value of \texttt{x} is
(2\textsuperscript{30}) - 1, the memory location we've asked for.

We haven't really talked too much about what it actually means to
allocate and deallocate memory in these contexts. Getting into very deep
detail is out of the scope of this tutorial, but what's important to
point out here is that the heap isn't a stack that grows from the
opposite end. We'll have an example of this later in the book, but
because the heap can be allocated and freed in any order, it can end up
with `holes'. Here's a diagram of the memory layout of a program which
has been running for a while now:

\begin{longtable}[c]{@{}lll@{}}
\toprule
Address & Name & Value\tabularnewline
\midrule
\endhead
(2\textsuperscript{30}) - 1 & & 5\tabularnewline
(2\textsuperscript{30}) - 2 & &\tabularnewline
(2\textsuperscript{30}) - 3 & &\tabularnewline
(2\textsuperscript{30}) - 4 & & 42\tabularnewline
\ldots{} & \ldots{} & \ldots{}\tabularnewline
2 & z & → (2\textsuperscript{30}) - 4\tabularnewline
1 & y & 42\tabularnewline
0 & x & → (2\textsuperscript{30}) - 1\tabularnewline
\bottomrule
\end{longtable}

In this case, we've allocated four things on the heap, but deallocated
two of them. There's a gap between (2\textsuperscript{30}) - 1 and
(2\textsuperscript{30}) - 4 which isn't currently being used. The
specific details of how and why this happens depends on what kind of
strategy you use to manage the heap. Different programs can use
different `memory allocators', which are libraries that manage this for
you. Rust programs use
\href{http://www.canonware.com/jemalloc/}{jemalloc} for this purpose.

Anyway, back to our example. Since this memory is on the heap, it can
stay alive longer than the function which allocates the box. In this
case, however, it doesn't.\footnote{We can make the memory live longer
  by transferring ownership, sometimes called `moving out of the box'.
  More complex examples will be covered later.} When the function is
over, we need to free the stack frame for \texttt{main()}.
\texttt{Box\textless{}T\textgreater{}}, though, has a trick up its
sleeve: \protect\hyperlink{sec--drop}{Drop}. The implementation of
\texttt{Drop} for \texttt{Box} deallocates the memory that was allocated
when it was created. Great! So when \texttt{x} goes away, it first frees
the memory allocated on the heap:

\begin{longtable}[c]{@{}lll@{}}
\toprule
Address & Name & Value\tabularnewline
\midrule
\endhead
1 & y & 42\tabularnewline
0 & x & ??????\tabularnewline
\bottomrule
\end{longtable}

And then the stack frame goes away, freeing all of our memory.

\subsection{Arguments and borrowing}\label{arguments-and-borrowing}

We've got some basic examples with the stack and the heap going, but
what about function arguments and borrowing? Here's a small Rust
program:

\begin{Shaded}
\begin{Highlighting}[]
\KeywordTok{fn} \NormalTok{foo(i: &}\DataTypeTok{i32}\NormalTok{) \{}
    \KeywordTok{let} \NormalTok{z = }\DecValTok{42}\NormalTok{;}
\NormalTok{\}}

\KeywordTok{fn} \NormalTok{main() \{}
    \KeywordTok{let} \NormalTok{x = }\DecValTok{5}\NormalTok{;}
    \KeywordTok{let} \NormalTok{y = &x;}

    \NormalTok{foo(y);}
\NormalTok{\}}
\end{Highlighting}
\end{Shaded}

When we enter \texttt{main()}, memory looks like this:

\begin{longtable}[c]{@{}lll@{}}
\toprule
Address & Name & Value\tabularnewline
\midrule
\endhead
1 & y & → 0\tabularnewline
0 & x & 5\tabularnewline
\bottomrule
\end{longtable}

\texttt{x} is a plain old \texttt{5}, and \texttt{y} is a reference to
\texttt{x}. So its value is the memory location that \texttt{x} lives
at, which in this case is \texttt{0}.

What about when we call \texttt{foo()}, passing \texttt{y} as an
argument?

\begin{longtable}[c]{@{}lll@{}}
\toprule
Address & Name & Value\tabularnewline
\midrule
\endhead
3 & z & 42\tabularnewline
2 & i & → 0\tabularnewline
1 & y & → 0\tabularnewline
0 & x & 5\tabularnewline
\bottomrule
\end{longtable}

Stack frames aren't only for local bindings, they're for arguments too.
So in this case, we need to have both \texttt{i}, our argument, and
\texttt{z}, our local variable binding. \texttt{i} is a copy of the
argument, \texttt{y}. Since \texttt{y}'s value is \texttt{0}, so is
\texttt{i}'s.

This is one reason why borrowing a variable doesn't deallocate any
memory: the value of a reference is a pointer to a memory location. If
we got rid of the underlying memory, things wouldn't work very well.

\subsection{A complex example}\label{a-complex-example}

Okay, let's go through this complex program step-by-step:

\begin{Shaded}
\begin{Highlighting}[]
\KeywordTok{fn} \NormalTok{foo(x: &}\DataTypeTok{i32}\NormalTok{) \{}
    \KeywordTok{let} \NormalTok{y = }\DecValTok{10}\NormalTok{;}
    \KeywordTok{let} \NormalTok{z = &y;}

    \NormalTok{baz(z);}
    \NormalTok{bar(x, z);}
\NormalTok{\}}

\KeywordTok{fn} \NormalTok{bar(a: &}\DataTypeTok{i32}\NormalTok{, b: &}\DataTypeTok{i32}\NormalTok{) \{}
    \KeywordTok{let} \NormalTok{c = }\DecValTok{5}\NormalTok{;}
    \KeywordTok{let} \NormalTok{d = }\DataTypeTok{Box}\NormalTok{::new(}\DecValTok{5}\NormalTok{);}
    \KeywordTok{let} \NormalTok{e = &d;}

    \NormalTok{baz(e);}
\NormalTok{\}}

\KeywordTok{fn} \NormalTok{baz(f: &}\DataTypeTok{i32}\NormalTok{) \{}
    \KeywordTok{let} \NormalTok{g = }\DecValTok{100}\NormalTok{;}
\NormalTok{\}}

\KeywordTok{fn} \NormalTok{main() \{}
    \KeywordTok{let} \NormalTok{h = }\DecValTok{3}\NormalTok{;}
    \KeywordTok{let} \NormalTok{i = }\DataTypeTok{Box}\NormalTok{::new(}\DecValTok{20}\NormalTok{);}
    \KeywordTok{let} \NormalTok{j = &h;}

    \NormalTok{foo(j);}
\NormalTok{\}}
\end{Highlighting}
\end{Shaded}

First, we call \texttt{main()}:

\begin{longtable}[c]{@{}lll@{}}
\toprule
Address & Name & Value\tabularnewline
\midrule
\endhead
(2\textsuperscript{30}) - 1 & & 20\tabularnewline
\ldots{} & \ldots{} & \ldots{}\tabularnewline
2 & j & → 0\tabularnewline
1 & i & → (2\textsuperscript{30}) - 1\tabularnewline
0 & h & 3\tabularnewline
\bottomrule
\end{longtable}

We allocate memory for \texttt{j}, \texttt{i}, and \texttt{h}.
\texttt{i} is on the heap, and so has a value pointing there.

Next, at the end of \texttt{main()}, \texttt{foo()} gets called:

\begin{longtable}[c]{@{}lll@{}}
\toprule
Address & Name & Value\tabularnewline
\midrule
\endhead
(2\textsuperscript{30}) - 1 & & 20\tabularnewline
\ldots{} & \ldots{} & \ldots{}\tabularnewline
5 & z & → 4\tabularnewline
4 & y & 10\tabularnewline
3 & x & → 0\tabularnewline
2 & j & → 0\tabularnewline
1 & i & → (2\textsuperscript{30}) - 1\tabularnewline
0 & h & 3\tabularnewline
\bottomrule
\end{longtable}

Space gets allocated for \texttt{x}, \texttt{y}, and \texttt{z}. The
argument \texttt{x} has the same value as \texttt{j}, since that's what
we passed it in. It's a pointer to the \texttt{0} address, since
\texttt{j} points at \texttt{h}.

Next, \texttt{foo()} calls \texttt{baz()}, passing \texttt{z}:

\begin{longtable}[c]{@{}lll@{}}
\toprule
Address & Name & Value\tabularnewline
\midrule
\endhead
(2\textsuperscript{30}) - 1 & & 20\tabularnewline
\ldots{} & \ldots{} & \ldots{}\tabularnewline
7 & g & 100\tabularnewline
6 & f & → 4\tabularnewline
5 & z & → 4\tabularnewline
4 & y & 10\tabularnewline
3 & x & → 0\tabularnewline
2 & j & → 0\tabularnewline
1 & i & → (2\textsuperscript{30}) - 1\tabularnewline
0 & h & 3\tabularnewline
\bottomrule
\end{longtable}

We've allocated memory for \texttt{f} and \texttt{g}. \texttt{baz()} is
very short, so when it's over, we get rid of its stack frame:

\begin{longtable}[c]{@{}lll@{}}
\toprule
Address & Name & Value\tabularnewline
\midrule
\endhead
(2\textsuperscript{30}) - 1 & & 20\tabularnewline
\ldots{} & \ldots{} & \ldots{}\tabularnewline
5 & z & → 4\tabularnewline
4 & y & 10\tabularnewline
3 & x & → 0\tabularnewline
2 & j & → 0\tabularnewline
1 & i & → (2\textsuperscript{30}) - 1\tabularnewline
0 & h & 3\tabularnewline
\bottomrule
\end{longtable}

Next, \texttt{foo()} calls \texttt{bar()} with \texttt{x} and
\texttt{z}:

\begin{longtable}[c]{@{}lll@{}}
\toprule
Address & Name & Value\tabularnewline
\midrule
\endhead
(2\textsuperscript{30}) - 1 & & 20\tabularnewline
(2\textsuperscript{30}) - 2 & & 5\tabularnewline
\ldots{} & \ldots{} & \ldots{}\tabularnewline
10 & e & → 9\tabularnewline
9 & d & → (2\textsuperscript{30}) - 2\tabularnewline
8 & c & 5\tabularnewline
7 & b & → 4\tabularnewline
6 & a & → 0\tabularnewline
5 & z & → 4\tabularnewline
4 & y & 10\tabularnewline
3 & x & → 0\tabularnewline
2 & j & → 0\tabularnewline
1 & i & → (2\textsuperscript{30}) - 1\tabularnewline
0 & h & 3\tabularnewline
\bottomrule
\end{longtable}

We end up allocating another value on the heap, and so we have to
subtract one from (2\textsuperscript{30}) - 1. It's easier to write that
than \texttt{1,073,741,822}. In any case, we set up the variables as
usual.

At the end of \texttt{bar()}, it calls \texttt{baz()}:

\begin{longtable}[c]{@{}lll@{}}
\toprule
Address & Name & Value\tabularnewline
\midrule
\endhead
(2\textsuperscript{30}) - 1 & & 20\tabularnewline
(2\textsuperscript{30}) - 2 & & 5\tabularnewline
\ldots{} & \ldots{} & \ldots{}\tabularnewline
12 & g & 100\tabularnewline
11 & f & → (2\textsuperscript{30}) - 2\tabularnewline
10 & e & → 9\tabularnewline
9 & d & → (2\textsuperscript{30}) - 2\tabularnewline
8 & c & 5\tabularnewline
7 & b & → 4\tabularnewline
6 & a & → 0\tabularnewline
5 & z & → 4\tabularnewline
4 & y & 10\tabularnewline
3 & x & → 0\tabularnewline
2 & j & → 0\tabularnewline
1 & i & → (2\textsuperscript{30}) - 1\tabularnewline
0 & h & 3\tabularnewline
\bottomrule
\end{longtable}

With this, we're at our deepest point! Whew! Congrats for following
along this far.

After \texttt{baz()} is over, we get rid of \texttt{f} and \texttt{g}:

\begin{longtable}[c]{@{}lll@{}}
\toprule
Address & Name & Value\tabularnewline
\midrule
\endhead
(2\textsuperscript{30}) - 1 & & 20\tabularnewline
(2\textsuperscript{30}) - 2 & & 5\tabularnewline
\ldots{} & \ldots{} & \ldots{}\tabularnewline
10 & e & → 9\tabularnewline
9 & d & → (2\textsuperscript{30}) - 2\tabularnewline
8 & c & 5\tabularnewline
7 & b & → 4\tabularnewline
6 & a & → 0\tabularnewline
5 & z & → 4\tabularnewline
4 & y & 10\tabularnewline
3 & x & → 0\tabularnewline
2 & j & → 0\tabularnewline
1 & i & → (2\textsuperscript{30}) - 1\tabularnewline
0 & h & 3\tabularnewline
\bottomrule
\end{longtable}

Next, we return from \texttt{bar()}. \texttt{d} in this case is a
\texttt{Box\textless{}T\textgreater{}}, so it also frees what it points
to: (2\textsuperscript{30}) - 2.

\begin{longtable}[c]{@{}lll@{}}
\toprule
Address & Name & Value\tabularnewline
\midrule
\endhead
(2\textsuperscript{30}) - 1 & & 20\tabularnewline
\ldots{} & \ldots{} & \ldots{}\tabularnewline
5 & z & → 4\tabularnewline
4 & y & 10\tabularnewline
3 & x & → 0\tabularnewline
2 & j & → 0\tabularnewline
1 & i & → (2\textsuperscript{30}) - 1\tabularnewline
0 & h & 3\tabularnewline
\bottomrule
\end{longtable}

And after that, \texttt{foo()} returns:

\begin{longtable}[c]{@{}lll@{}}
\toprule
Address & Name & Value\tabularnewline
\midrule
\endhead
(2\textsuperscript{30}) - 1 & & 20\tabularnewline
\ldots{} & \ldots{} & \ldots{}\tabularnewline
2 & j & → 0\tabularnewline
1 & i & → (2\textsuperscript{30}) - 1\tabularnewline
0 & h & 3\tabularnewline
\bottomrule
\end{longtable}

And then, finally, \texttt{main()}, which cleans the rest up. When
\texttt{i} is \texttt{Drop}ped, it will clean up the last of the heap
too.

\subsection{What do other languages
do?}\label{what-do-other-languages-do}

Most languages with a garbage collector heap-allocate by default. This
means that every value is boxed. There are a number of reasons why this
is done, but they're out of scope for this tutorial. There are some
possible optimizations that don't make it true 100\% of the time, too.
Rather than relying on the stack and \texttt{Drop} to clean up memory,
the garbage collector deals with the heap instead.

\subsection{Which to use?}\label{which-to-use}

So if the stack is faster and easier to manage, why do we need the heap?
A big reason is that Stack-allocation alone means you only have `Last In
First Out (LIFO)' semantics for reclaiming storage. Heap-allocation is
strictly more general, allowing storage to be taken from and returned to
the pool in arbitrary order, but at a complexity cost.

Generally, you should prefer stack allocation, and so, Rust
stack-allocates by default. The LIFO model of the stack is simpler, at a
fundamental level. This has two big impacts: runtime efficiency and
semantic impact.

\subsubsection{Runtime Efficiency}\label{runtime-efficiency}

Managing the memory for the stack is trivial: The machine increments or
decrements a single value, the so-called ``stack pointer''. Managing
memory for the heap is non-trivial: heap-allocated memory is freed at
arbitrary points, and each block of heap-allocated memory can be of
arbitrary size, so the memory manager must generally work much harder to
identify memory for reuse.

If you'd like to dive into this topic in greater detail,
\href{http://citeseerx.ist.psu.edu/viewdoc/summary?doi=10.1.1.143.4688}{this
paper} is a great introduction.

\subsubsection{Semantic impact}\label{semantic-impact}

Stack-allocation impacts the Rust language itself, and thus the
developer's mental model. The LIFO semantics is what drives how the Rust
language handles automatic memory management. Even the deallocation of a
uniquely-owned heap-allocated box can be driven by the stack-based LIFO
semantics, as discussed throughout this chapter. The flexibility
(i.e.~expressiveness) of non LIFO-semantics means that in general the
compiler cannot automatically infer at compile-time where memory should
be freed; it has to rely on dynamic protocols, potentially from outside
the language itself, to drive deallocation (reference counting, as used
by \texttt{Rc\textless{}T\textgreater{}} and
\texttt{Arc\textless{}T\textgreater{}}, is one example of this).

When taken to the extreme, the increased expressive power of heap
allocation comes at the cost of either significant runtime support
(e.g.~in the form of a garbage collector) or significant programmer
effort (in the form of explicit memory management calls that require
verification not provided by the Rust compiler).

\hypertarget{sec--testing}{\section{Testing}\label{sec--testing}}

\begin{quote}
Program testing can be a very effective way to show the presence of
bugs, but it is hopelessly inadequate for showing their absence.

Edsger W. Dijkstra, ``The Humble Programmer'' (1972)
\end{quote}

Let's talk about how to test Rust code. What we will not be talking
about is the right way to test Rust code. There are many schools of
thought regarding the right and wrong way to write tests. All of these
approaches use the same basic tools, and so we'll show you the syntax
for using them.

\subsection{\texorpdfstring{The \texttt{test}
attribute}{The test attribute}}\label{the-test-attribute}

At its simplest, a test in Rust is a function that's annotated with the
\texttt{test} attribute. Let's make a new project with Cargo called
\texttt{adder}:

\begin{Shaded}
\begin{Highlighting}[]
\NormalTok{$ }\KeywordTok{cargo} \NormalTok{new adder}
\NormalTok{$ }\KeywordTok{cd} \NormalTok{adder}
\end{Highlighting}
\end{Shaded}

Cargo will automatically generate a simple test when you make a new
project. Here's the contents of \texttt{src/lib.rs}:

\begin{Shaded}
\begin{Highlighting}[]
\AttributeTok{#[}\NormalTok{test}\AttributeTok{]}
\KeywordTok{fn} \NormalTok{it_works() \{}
\NormalTok{\}}
\end{Highlighting}
\end{Shaded}

Note the \texttt{\#{[}test{]}}. This attribute indicates that this is a
test function. It currently has no body. That's good enough to pass! We
can run the tests with \texttt{cargo\ test}:

\begin{Shaded}
\begin{Highlighting}[]
\NormalTok{$ }\KeywordTok{cargo} \NormalTok{test}
   \KeywordTok{Compiling} \NormalTok{adder v0.0.1 (file:///home/you/projects/adder)}
     \KeywordTok{Running} \NormalTok{target/adder-91b3e234d4ed382a}

\KeywordTok{running} \NormalTok{1 test}
\KeywordTok{test} \NormalTok{it_works ... ok}

\KeywordTok{test} \NormalTok{result: ok. 1 passed}\KeywordTok{;} \KeywordTok{0} \NormalTok{failed}\KeywordTok{;} \KeywordTok{0} \NormalTok{ignored}\KeywordTok{;} \KeywordTok{0} \NormalTok{measured}

   \KeywordTok{Doc-tests} \NormalTok{adder}

\KeywordTok{running} \NormalTok{0 tests}

\KeywordTok{test} \NormalTok{result: ok. 0 passed}\KeywordTok{;} \KeywordTok{0} \NormalTok{failed}\KeywordTok{;} \KeywordTok{0} \NormalTok{ignored}\KeywordTok{;} \KeywordTok{0} \NormalTok{measured}
\end{Highlighting}
\end{Shaded}

Cargo compiled and ran our tests. There are two sets of output here: one
for the test we wrote, and another for documentation tests. We'll talk
about those later. For now, see this line:

\begin{verbatim}
test it_works ... ok
\end{verbatim}

Note the \texttt{it\_works}. This comes from the name of our function:

\begin{Shaded}
\begin{Highlighting}[]
\KeywordTok{fn} \NormalTok{it_works() \{}
\end{Highlighting}
\end{Shaded}

We also get a summary line:

\begin{verbatim}
test result: ok. 1 passed; 0 failed; 0 ignored; 0 measured
\end{verbatim}

So why does our do-nothing test pass? Any test which doesn't
\texttt{panic!} passes, and any test that does \texttt{panic!} fails.
Let's make our test fail:

\begin{Shaded}
\begin{Highlighting}[]
\AttributeTok{#[}\NormalTok{test}\AttributeTok{]}
\KeywordTok{fn} \NormalTok{it_works() \{}
    \PreprocessorTok{assert!}\NormalTok{(}\ConstantTok{false}\NormalTok{);}
\NormalTok{\}}
\end{Highlighting}
\end{Shaded}

\texttt{assert!} is a macro provided by Rust which takes one argument:
if the argument is \texttt{true}, nothing happens. If the argument is
\texttt{false}, it will \texttt{panic!}. Let's run our tests again:

\begin{Shaded}
\begin{Highlighting}[]
\NormalTok{$ }\KeywordTok{cargo} \NormalTok{test}
   \KeywordTok{Compiling} \NormalTok{adder v0.0.1 (file:///home/you/projects/adder)}
     \KeywordTok{Running} \NormalTok{target/adder-91b3e234d4ed382a}

\KeywordTok{running} \NormalTok{1 test}
\KeywordTok{test} \NormalTok{it_works ... FAILED}

\KeywordTok{failures}\NormalTok{:}

\KeywordTok{----} \NormalTok{it_works stdout ----}
        \KeywordTok{thread} \StringTok{'it_works'} \NormalTok{panicked at }\StringTok{'assertion failed: false'}\NormalTok{, /home/steve/tmp/adder}
\NormalTok{↳ }\KeywordTok{/src}\NormalTok{/lib.rs:}\KeywordTok{3}



\KeywordTok{failures}\NormalTok{:}
    \KeywordTok{it_works}

\KeywordTok{test} \NormalTok{result: FAILED. 0 passed}\KeywordTok{;} \KeywordTok{1} \NormalTok{failed}\KeywordTok{;} \KeywordTok{0} \NormalTok{ignored}\KeywordTok{;} \KeywordTok{0} \NormalTok{measured}

\KeywordTok{thread} \StringTok{'<main>'} \NormalTok{panicked at }\StringTok{'Some tests failed'}\NormalTok{, /home/steve/src/rust/src/libtest/lib.}
\NormalTok{↳ }\KeywordTok{rs}\NormalTok{:247}
\end{Highlighting}
\end{Shaded}

Rust indicates that our test failed:

\begin{verbatim}
test it_works ... FAILED
\end{verbatim}

And that's reflected in the summary line:

\begin{verbatim}
test result: FAILED. 0 passed; 1 failed; 0 ignored; 0 measured
\end{verbatim}

We also get a non-zero status code. We can use \texttt{\$?} on OS X and
Linux:

\begin{Shaded}
\begin{Highlighting}[]
\NormalTok{$ }\KeywordTok{echo} \OtherTok{$?}
\KeywordTok{101}
\end{Highlighting}
\end{Shaded}

On Windows, if you're using \texttt{cmd}:

\begin{Shaded}
\begin{Highlighting}[]
\KeywordTok{>} \KeywordTok{echo} \NormalTok
\end{Highlighting}
\end{Shaded}

And if you're using PowerShell:

\begin{Shaded}
\begin{Highlighting}[]
\KeywordTok{>} \KeywordTok{echo} \OtherTok{$LASTEXITCODE} \CommentTok{# the code itself}
\KeywordTok{>} \KeywordTok{echo} \OtherTok{$?} \CommentTok{# a boolean, fail or succeed}
\end{Highlighting}
\end{Shaded}

This is useful if you want to integrate \texttt{cargo\ test} into other
tooling.

We can invert our test's failure with another attribute:
\texttt{should\_panic}:

\begin{Shaded}
\begin{Highlighting}[]
\AttributeTok{#[}\NormalTok{test}\AttributeTok{]}
\AttributeTok{#[}\NormalTok{should_panic}\AttributeTok{]}
\KeywordTok{fn} \NormalTok{it_works() \{}
    \PreprocessorTok{assert!}\NormalTok{(}\ConstantTok{false}\NormalTok{);}
\NormalTok{\}}
\end{Highlighting}
\end{Shaded}

This test will now succeed if we \texttt{panic!} and fail if we
complete. Let's try it:

\begin{Shaded}
\begin{Highlighting}[]
\NormalTok{$ }\KeywordTok{cargo} \NormalTok{test}
   \KeywordTok{Compiling} \NormalTok{adder v0.0.1 (file:///home/you/projects/adder)}
     \KeywordTok{Running} \NormalTok{target/adder-91b3e234d4ed382a}

\KeywordTok{running} \NormalTok{1 test}
\KeywordTok{test} \NormalTok{it_works ... ok}

\KeywordTok{test} \NormalTok{result: ok. 1 passed}\KeywordTok{;} \KeywordTok{0} \NormalTok{failed}\KeywordTok{;} \KeywordTok{0} \NormalTok{ignored}\KeywordTok{;} \KeywordTok{0} \NormalTok{measured}

   \KeywordTok{Doc-tests} \NormalTok{adder}

\KeywordTok{running} \NormalTok{0 tests}

\KeywordTok{test} \NormalTok{result: ok. 0 passed}\KeywordTok{;} \KeywordTok{0} \NormalTok{failed}\KeywordTok{;} \KeywordTok{0} \NormalTok{ignored}\KeywordTok{;} \KeywordTok{0} \NormalTok{measured}
\end{Highlighting}
\end{Shaded}

Rust provides another macro, \texttt{assert\_eq!}, that compares two
arguments for equality:

\begin{Shaded}
\begin{Highlighting}[]
\AttributeTok{#[}\NormalTok{test}\AttributeTok{]}
\AttributeTok{#[}\NormalTok{should_panic}\AttributeTok{]}
\KeywordTok{fn} \NormalTok{it_works() \{}
    \PreprocessorTok{assert_eq!}\NormalTok{(}\StringTok{"Hello"}\NormalTok{, }\StringTok{"world"}\NormalTok{);}
\NormalTok{\}}
\end{Highlighting}
\end{Shaded}

Does this test pass or fail? Because of the \texttt{should\_panic}
attribute, it passes:

\begin{Shaded}
\begin{Highlighting}[]
\NormalTok{$ }\KeywordTok{cargo} \NormalTok{test}
   \KeywordTok{Compiling} \NormalTok{adder v0.0.1 (file:///home/you/projects/adder)}
     \KeywordTok{Running} \NormalTok{target/adder-91b3e234d4ed382a}

\KeywordTok{running} \NormalTok{1 test}
\KeywordTok{test} \NormalTok{it_works ... ok}

\KeywordTok{test} \NormalTok{result: ok. 1 passed}\KeywordTok{;} \KeywordTok{0} \NormalTok{failed}\KeywordTok{;} \KeywordTok{0} \NormalTok{ignored}\KeywordTok{;} \KeywordTok{0} \NormalTok{measured}

   \KeywordTok{Doc-tests} \NormalTok{adder}

\KeywordTok{running} \NormalTok{0 tests}

\KeywordTok{test} \NormalTok{result: ok. 0 passed}\KeywordTok{;} \KeywordTok{0} \NormalTok{failed}\KeywordTok{;} \KeywordTok{0} \NormalTok{ignored}\KeywordTok{;} \KeywordTok{0} \NormalTok{measured}
\end{Highlighting}
\end{Shaded}

\texttt{should\_panic} tests can be fragile, as it's hard to guarantee
that the test didn't fail for an unexpected reason. To help with this,
an optional \texttt{expected} parameter can be added to the
\texttt{should\_panic} attribute. The test harness will make sure that
the failure message contains the provided text. A safer version of the
example above would be:

\begin{Shaded}
\begin{Highlighting}[]
\AttributeTok{#[}\NormalTok{test}\AttributeTok{]}
\AttributeTok{#[}\NormalTok{should_panic}\AttributeTok{(}\NormalTok{expected }\AttributeTok{=} \StringTok{"assertion failed"}\AttributeTok{)]}
\KeywordTok{fn} \NormalTok{it_works() \{}
    \PreprocessorTok{assert_eq!}\NormalTok{(}\StringTok{"Hello"}\NormalTok{, }\StringTok{"world"}\NormalTok{);}
\NormalTok{\}}
\end{Highlighting}
\end{Shaded}

That's all there is to the basics! Let's write one `real' test:

\begin{Shaded}
\begin{Highlighting}[]
\KeywordTok{pub} \KeywordTok{fn} \NormalTok{add_two(a: }\DataTypeTok{i32}\NormalTok{) -> }\DataTypeTok{i32} \NormalTok{\{}
    \NormalTok{a + }\DecValTok{2}
\NormalTok{\}}

\AttributeTok{#[}\NormalTok{test}\AttributeTok{]}
\KeywordTok{fn} \NormalTok{it_works() \{}
    \PreprocessorTok{assert_eq!}\NormalTok{(}\DecValTok{4}\NormalTok{, add_two(}\DecValTok{2}\NormalTok{));}
\NormalTok{\}}
\end{Highlighting}
\end{Shaded}

This is a very common use of \texttt{assert\_eq!}: call some function
with some known arguments and compare it to the expected output.

\subsection{\texorpdfstring{The \texttt{ignore}
attribute}{The ignore attribute}}\label{the-ignore-attribute}

Sometimes a few specific tests can be very time-consuming to execute.
These can be disabled by default by using the \texttt{ignore} attribute:

\begin{Shaded}
\begin{Highlighting}[]
\AttributeTok{#[}\NormalTok{test}\AttributeTok{]}
\KeywordTok{fn} \NormalTok{it_works() \{}
    \PreprocessorTok{assert_eq!}\NormalTok{(}\DecValTok{4}\NormalTok{, add_two(}\DecValTok{2}\NormalTok{));}
\NormalTok{\}}

\AttributeTok{#[}\NormalTok{test}\AttributeTok{]}
\AttributeTok{#[}\NormalTok{ignore}\AttributeTok{]}
\KeywordTok{fn} \NormalTok{expensive_test() \{}
    \CommentTok{// code that takes an hour to run}
\NormalTok{\}}
\end{Highlighting}
\end{Shaded}

Now we run our tests and see that \texttt{it\_works} is run, but
\texttt{expensive\_test} is not:

\begin{Shaded}
\begin{Highlighting}[]
\NormalTok{$ }\KeywordTok{cargo} \NormalTok{test}
   \KeywordTok{Compiling} \NormalTok{adder v0.0.1 (file:///home/you/projects/adder)}
     \KeywordTok{Running} \NormalTok{target/adder-91b3e234d4ed382a}

\KeywordTok{running} \NormalTok{2 tests}
\KeywordTok{test} \NormalTok{expensive_test ... ignored}
\KeywordTok{test} \NormalTok{it_works ... ok}

\KeywordTok{test} \NormalTok{result: ok. 1 passed}\KeywordTok{;} \KeywordTok{0} \NormalTok{failed}\KeywordTok{;} \KeywordTok{1} \NormalTok{ignored}\KeywordTok{;} \KeywordTok{0} \NormalTok{measured}

   \KeywordTok{Doc-tests} \NormalTok{adder}

\KeywordTok{running} \NormalTok{0 tests}

\KeywordTok{test} \NormalTok{result: ok. 0 passed}\KeywordTok{;} \KeywordTok{0} \NormalTok{failed}\KeywordTok{;} \KeywordTok{0} \NormalTok{ignored}\KeywordTok{;} \KeywordTok{0} \NormalTok{measured}
\end{Highlighting}
\end{Shaded}

The expensive tests can be run explicitly using
\texttt{cargo\ test\ -\/-\ -\/-ignored}:

\begin{Shaded}
\begin{Highlighting}[]
\NormalTok{$ }\KeywordTok{cargo} \NormalTok{test -- --ignored}
     \KeywordTok{Running} \NormalTok{target/adder-91b3e234d4ed382a}

\KeywordTok{running} \NormalTok{1 test}
\KeywordTok{test} \NormalTok{expensive_test ... ok}

\KeywordTok{test} \NormalTok{result: ok. 1 passed}\KeywordTok{;} \KeywordTok{0} \NormalTok{failed}\KeywordTok{;} \KeywordTok{0} \NormalTok{ignored}\KeywordTok{;} \KeywordTok{0} \NormalTok{measured}

   \KeywordTok{Doc-tests} \NormalTok{adder}

\KeywordTok{running} \NormalTok{0 tests}

\KeywordTok{test} \NormalTok{result: ok. 0 passed}\KeywordTok{;} \KeywordTok{0} \NormalTok{failed}\KeywordTok{;} \KeywordTok{0} \NormalTok{ignored}\KeywordTok{;} \KeywordTok{0} \NormalTok{measured}
\end{Highlighting}
\end{Shaded}

The \texttt{-\/-ignored} argument is an argument to the test binary, and
not to Cargo, which is why the command is
\texttt{cargo\ test\ -\/-\ -\/-ignored}.

\subsection{\texorpdfstring{The \texttt{tests}
module}{The tests module}}\label{the-tests-module}

There is one way in which our existing example is not idiomatic: it's
missing the \texttt{tests} module. The idiomatic way of writing our
example looks like this:

\begin{Shaded}
\begin{Highlighting}[]
\KeywordTok{pub} \KeywordTok{fn} \NormalTok{add_two(a: }\DataTypeTok{i32}\NormalTok{) -> }\DataTypeTok{i32} \NormalTok{\{}
    \NormalTok{a + }\DecValTok{2}
\NormalTok{\}}

\AttributeTok{#[}\NormalTok{cfg}\AttributeTok{(}\NormalTok{test}\AttributeTok{)]}
\KeywordTok{mod} \NormalTok{tests \{}
    \KeywordTok{use} \KeywordTok{super}\NormalTok{::add_two;}

    \AttributeTok{#[}\NormalTok{test}\AttributeTok{]}
    \KeywordTok{fn} \NormalTok{it_works() \{}
        \PreprocessorTok{assert_eq!}\NormalTok{(}\DecValTok{4}\NormalTok{, add_two(}\DecValTok{2}\NormalTok{));}
    \NormalTok{\}}
\NormalTok{\}}
\end{Highlighting}
\end{Shaded}

There's a few changes here. The first is the introduction of a
\texttt{mod\ tests} with a \texttt{cfg} attribute. The module allows us
to group all of our tests together, and to also define helper functions
if needed, that don't become a part of the rest of our crate. The
\texttt{cfg} attribute only compiles our test code if we're currently
trying to run the tests. This can save compile time, and also ensures
that our tests are entirely left out of a normal build.

The second change is the \texttt{use} declaration. Because we're in an
inner module, we need to bring our test function into scope. This can be
annoying if you have a large module, and so this is a common use of
globs. Let's change our \texttt{src/lib.rs} to make use of it:

\begin{Shaded}
\begin{Highlighting}[]
\KeywordTok{pub} \KeywordTok{fn} \NormalTok{add_two(a: }\DataTypeTok{i32}\NormalTok{) -> }\DataTypeTok{i32} \NormalTok{\{}
    \NormalTok{a + }\DecValTok{2}
\NormalTok{\}}

\AttributeTok{#[}\NormalTok{cfg}\AttributeTok{(}\NormalTok{test}\AttributeTok{)]}
\KeywordTok{mod} \NormalTok{tests \{}
    \KeywordTok{use} \KeywordTok{super}\NormalTok{::*;}

    \AttributeTok{#[}\NormalTok{test}\AttributeTok{]}
    \KeywordTok{fn} \NormalTok{it_works() \{}
        \PreprocessorTok{assert_eq!}\NormalTok{(}\DecValTok{4}\NormalTok{, add_two(}\DecValTok{2}\NormalTok{));}
    \NormalTok{\}}
\NormalTok{\}}
\end{Highlighting}
\end{Shaded}

Note the different \texttt{use} line. Now we run our tests:

\begin{Shaded}
\begin{Highlighting}[]
\NormalTok{$ }\KeywordTok{cargo} \NormalTok{test}
    \KeywordTok{Updating} \NormalTok{registry }\KeywordTok{`https}\NormalTok{://github.com/rust-lang/crates.io-index}\KeywordTok{`}
   \KeywordTok{Compiling} \NormalTok{adder v0.0.1 (file:///home/you/projects/adder)}
     \KeywordTok{Running} \NormalTok{target/adder-91b3e234d4ed382a}

\KeywordTok{running} \NormalTok{1 test}
\KeywordTok{test} \NormalTok{tests::it_works ... ok}

\KeywordTok{test} \NormalTok{result: ok. 1 passed}\KeywordTok{;} \KeywordTok{0} \NormalTok{failed}\KeywordTok{;} \KeywordTok{0} \NormalTok{ignored}\KeywordTok{;} \KeywordTok{0} \NormalTok{measured}

   \KeywordTok{Doc-tests} \NormalTok{adder}

\KeywordTok{running} \NormalTok{0 tests}

\KeywordTok{test} \NormalTok{result: ok. 0 passed}\KeywordTok{;} \KeywordTok{0} \NormalTok{failed}\KeywordTok{;} \KeywordTok{0} \NormalTok{ignored}\KeywordTok{;} \KeywordTok{0} \NormalTok{measured}
\end{Highlighting}
\end{Shaded}

It works!

The current convention is to use the \texttt{tests} module to hold your
``unit-style'' tests. Anything that tests one small bit of functionality
makes sense to go here. But what about ``integration-style'' tests
instead? For that, we have the \texttt{tests} directory.

\subsection{\texorpdfstring{The \texttt{tests}
directory}{The tests directory}}\label{the-tests-directory}

To write an integration test, let's make a \texttt{tests} directory, and
put a \texttt{tests/lib.rs} file inside, with this as its contents:

\begin{Shaded}
\begin{Highlighting}[]
\KeywordTok{extern} \KeywordTok{crate} \NormalTok{adder;}

\AttributeTok{#[}\NormalTok{test}\AttributeTok{]}
\KeywordTok{fn} \NormalTok{it_works() \{}
    \PreprocessorTok{assert_eq!}\NormalTok{(}\DecValTok{4}\NormalTok{, adder::add_two(}\DecValTok{2}\NormalTok{));}
\NormalTok{\}}
\end{Highlighting}
\end{Shaded}

This looks similar to our previous tests, but slightly different. We now
have an \texttt{extern\ crate\ adder} at the top. This is because the
tests in the \texttt{tests} directory are an entirely separate crate,
and so we need to import our library. This is also why \texttt{tests} is
a suitable place to write integration-style tests: they use the library
like any other consumer of it would.

Let's run them:

\begin{Shaded}
\begin{Highlighting}[]
\NormalTok{$ }\KeywordTok{cargo} \NormalTok{test}
   \KeywordTok{Compiling} \NormalTok{adder v0.0.1 (file:///home/you/projects/adder)}
     \KeywordTok{Running} \NormalTok{target/adder-91b3e234d4ed382a}

\KeywordTok{running} \NormalTok{1 test}
\KeywordTok{test} \NormalTok{tests::it_works ... ok}

\KeywordTok{test} \NormalTok{result: ok. 1 passed}\KeywordTok{;} \KeywordTok{0} \NormalTok{failed}\KeywordTok{;} \KeywordTok{0} \NormalTok{ignored}\KeywordTok{;} \KeywordTok{0} \NormalTok{measured}

     \KeywordTok{Running} \NormalTok{target/lib-c18e7d3494509e74}

\KeywordTok{running} \NormalTok{1 test}
\KeywordTok{test} \NormalTok{it_works ... ok}

\KeywordTok{test} \NormalTok{result: ok. 1 passed}\KeywordTok{;} \KeywordTok{0} \NormalTok{failed}\KeywordTok{;} \KeywordTok{0} \NormalTok{ignored}\KeywordTok{;} \KeywordTok{0} \NormalTok{measured}

   \KeywordTok{Doc-tests} \NormalTok{adder}

\KeywordTok{running} \NormalTok{0 tests}

\KeywordTok{test} \NormalTok{result: ok. 0 passed}\KeywordTok{;} \KeywordTok{0} \NormalTok{failed}\KeywordTok{;} \KeywordTok{0} \NormalTok{ignored}\KeywordTok{;} \KeywordTok{0} \NormalTok{measured}
\end{Highlighting}
\end{Shaded}

Now we have three sections: our previous test is also run, as well as
our new one.

That's all there is to the \texttt{tests} directory. The \texttt{tests}
module isn't needed here, since the whole thing is focused on tests.

Let's finally check out that third section: documentation tests.

\subsection{Documentation tests}\label{documentation-tests}

Nothing is better than documentation with examples. Nothing is worse
than examples that don't actually work, because the code has changed
since the documentation has been written. To this end, Rust supports
automatically running examples in your documentation (\textbf{note:}
this only works in library crates, not binary crates). Here's a
fleshed-out \texttt{src/lib.rs} with examples:

\begin{Shaded}
\begin{Highlighting}[]
\CommentTok{//! The `adder` crate provides functions that add numbers to other numbers.}
\CommentTok{//!}
\CommentTok{//! # Examples}
\CommentTok{//!}
\CommentTok{//! ```}
\CommentTok{//! assert_eq!(4, adder::add_two(2));}
\CommentTok{//! ```}

\CommentTok{/// This function adds two to its argument.}
\CommentTok{///}
\CommentTok{/// # Examples}
\CommentTok{///}
\CommentTok{/// ```}
\CommentTok{/// use adder::add_two;}
\CommentTok{///}
\CommentTok{/// assert_eq!(4, add_two(2));}
\CommentTok{/// ```}
\KeywordTok{pub} \KeywordTok{fn} \NormalTok{add_two(a: }\DataTypeTok{i32}\NormalTok{) -> }\DataTypeTok{i32} \NormalTok{\{}
    \NormalTok{a + }\DecValTok{2}
\NormalTok{\}}

\AttributeTok{#[}\NormalTok{cfg}\AttributeTok{(}\NormalTok{test}\AttributeTok{)]}
\KeywordTok{mod} \NormalTok{tests \{}
    \KeywordTok{use} \KeywordTok{super}\NormalTok{::*;}

    \AttributeTok{#[}\NormalTok{test}\AttributeTok{]}
    \KeywordTok{fn} \NormalTok{it_works() \{}
        \PreprocessorTok{assert_eq!}\NormalTok{(}\DecValTok{4}\NormalTok{, add_two(}\DecValTok{2}\NormalTok{));}
    \NormalTok{\}}
\NormalTok{\}}
\end{Highlighting}
\end{Shaded}

Note the module-level documentation with \texttt{//!} and the
function-level documentation with \texttt{///}. Rust's documentation
supports Markdown in comments, and so triple graves mark code blocks. It
is conventional to include the \texttt{\#\ Examples} section, exactly
like that, with examples following.

Let's run the tests again:

\begin{Shaded}
\begin{Highlighting}[]
\NormalTok{$ }\KeywordTok{cargo} \NormalTok{test}
   \KeywordTok{Compiling} \NormalTok{adder v0.0.1 (file:///home/steve/tmp/adder)}
     \KeywordTok{Running} \NormalTok{target/adder-91b3e234d4ed382a}

\KeywordTok{running} \NormalTok{1 test}
\KeywordTok{test} \NormalTok{tests::it_works ... ok}

\KeywordTok{test} \NormalTok{result: ok. 1 passed}\KeywordTok{;} \KeywordTok{0} \NormalTok{failed}\KeywordTok{;} \KeywordTok{0} \NormalTok{ignored}\KeywordTok{;} \KeywordTok{0} \NormalTok{measured}

     \KeywordTok{Running} \NormalTok{target/lib-c18e7d3494509e74}

\KeywordTok{running} \NormalTok{1 test}
\KeywordTok{test} \NormalTok{it_works ... ok}

\KeywordTok{test} \NormalTok{result: ok. 1 passed}\KeywordTok{;} \KeywordTok{0} \NormalTok{failed}\KeywordTok{;} \KeywordTok{0} \NormalTok{ignored}\KeywordTok{;} \KeywordTok{0} \NormalTok{measured}

   \KeywordTok{Doc-tests} \NormalTok{adder}

\KeywordTok{running} \NormalTok{2 tests}
\KeywordTok{test} \NormalTok{add_two_0 ... ok}
\KeywordTok{test} \NormalTok{_0 ... ok}

\KeywordTok{test} \NormalTok{result: ok. 2 passed}\KeywordTok{;} \KeywordTok{0} \NormalTok{failed}\KeywordTok{;} \KeywordTok{0} \NormalTok{ignored}\KeywordTok{;} \KeywordTok{0} \NormalTok{measured}
\end{Highlighting}
\end{Shaded}

Now we have all three kinds of tests running! Note the names of the
documentation tests: the \texttt{\_0} is generated for the module test,
and \texttt{add\_two\_0} for the function test. These will auto
increment with names like \texttt{add\_two\_1} as you add more examples.

We haven't covered all of the details with writing documentation tests.
For more, please see the
\protect\hyperlink{sec--documentation}{Documentation chapter}.

\section{Conditional Compilation}\label{sec--conditional-compilation}

Rust has a special attribute, \texttt{\#{[}cfg{]}}, which allows you to
compile code based on a flag passed to the compiler. It has two forms:

\begin{Shaded}
\begin{Highlighting}[]
\AttributeTok{#[}\NormalTok{cfg}\AttributeTok{(}\NormalTok{foo}\AttributeTok{)]}

\AttributeTok{#[}\NormalTok{cfg}\AttributeTok{(}\NormalTok{bar }\AttributeTok{=} \StringTok{"baz"}\AttributeTok{)]}
\end{Highlighting}
\end{Shaded}

They also have some helpers:

\begin{Shaded}
\begin{Highlighting}[]
\AttributeTok{#[}\NormalTok{cfg}\AttributeTok{(}\NormalTok{any}\AttributeTok{(}\NormalTok{unix}\AttributeTok{,} \NormalTok{windows}\AttributeTok{))]}

\AttributeTok{#[}\NormalTok{cfg}\AttributeTok{(}\NormalTok{all}\AttributeTok{(}\NormalTok{unix}\AttributeTok{,} \NormalTok{target_pointer_width }\AttributeTok{=} \StringTok{"32"}\AttributeTok{))]}

\AttributeTok{#[}\NormalTok{cfg}\AttributeTok{(}\NormalTok{not}\AttributeTok{(}\NormalTok{foo}\AttributeTok{))]}
\end{Highlighting}
\end{Shaded}

These can nest arbitrarily:

\begin{Shaded}
\begin{Highlighting}[]
\AttributeTok{#[}\NormalTok{cfg}\AttributeTok{(}\NormalTok{any}\AttributeTok{(}\NormalTok{not}\AttributeTok{(}\NormalTok{unix}\AttributeTok{),} \NormalTok{all}\AttributeTok{(}\NormalTok{target_os}\AttributeTok{=}\StringTok{"macos"}\AttributeTok{,} \NormalTok{target_arch }\AttributeTok{=} \StringTok{"powerpc"}\AttributeTok{)))]}
\end{Highlighting}
\end{Shaded}

As for how to enable or disable these switches, if you're using Cargo,
they get set in the
\href{http://doc.crates.io/manifest.html\#the-features-section}{\texttt{{[}features{]}}
section} of your \texttt{Cargo.toml}:

\begin{verbatim}
[features]
# no features by default
default = []

# The “secure-password” feature depends on the bcrypt package.
secure-password = ["bcrypt"]
\end{verbatim}

When you do this, Cargo passes along a flag to \texttt{rustc}:

\begin{verbatim}
--cfg feature="${feature_name}"
\end{verbatim}

The sum of these \texttt{cfg} flags will determine which ones get
activated, and therefore, which code gets compiled. Let's take this
code:

\begin{Shaded}
\begin{Highlighting}[]
\AttributeTok{#[}\NormalTok{cfg}\AttributeTok{(}\NormalTok{feature }\AttributeTok{=} \StringTok{"foo"}\AttributeTok{)]}
\KeywordTok{mod} \NormalTok{foo \{}
\NormalTok{\}}
\end{Highlighting}
\end{Shaded}

If we compile it with \texttt{cargo\ build\ -\/-features\ "foo"}, it
will send the \texttt{-\/-cfg\ feature="foo"} flag to \texttt{rustc},
and the output will have the \texttt{mod\ foo} in it. If we compile it
with a regular \texttt{cargo\ build}, no extra flags get passed on, and
so, no \texttt{foo} module will exist.

\subsection{cfg\_attr}\label{cfgux5fattr}

You can also set another attribute based on a \texttt{cfg} variable with
\texttt{cfg\_attr}:

\begin{Shaded}
\begin{Highlighting}[]
\AttributeTok{#[}\NormalTok{cfg_attr}\AttributeTok{(}\NormalTok{a}\AttributeTok{,} \NormalTok{b}\AttributeTok{)]}
\end{Highlighting}
\end{Shaded}

Will be the same as \texttt{\#{[}b{]}} if \texttt{a} is set by
\texttt{cfg} attribute, and nothing otherwise.

\subsection{cfg!}\label{cfg}

The \texttt{cfg!} \protect\hyperlink{sec--compiler-plugins}{syntax
extension} lets you use these kinds of flags elsewhere in your code,
too:

\begin{Shaded}
\begin{Highlighting}[]
\KeywordTok{if} \PreprocessorTok{cfg!}\NormalTok{(target_os = }\StringTok{"macos"}\NormalTok{) || }\PreprocessorTok{cfg!}\NormalTok{(target_os = }\StringTok{"ios"}\NormalTok{) \{}
    \PreprocessorTok{println!}\NormalTok{(}\StringTok{"Think Different!"}\NormalTok{);}
\NormalTok{\}}
\end{Highlighting}
\end{Shaded}

These will be replaced by a \texttt{true} or \texttt{false} at
compile-time, depending on the configuration settings.

\hypertarget{sec--documentation}{\section{Documentation}\label{sec--documentation}}

Documentation is an important part of any software project, and it's
first-class in Rust. Let's talk about the tooling Rust gives you to
document your project.

\subsubsection{\texorpdfstring{About
\texttt{rustdoc}}{About rustdoc}}\label{about-rustdoc}

The Rust distribution includes a tool, \texttt{rustdoc}, that generates
documentation. \texttt{rustdoc} is also used by Cargo through
\texttt{cargo\ doc}.

Documentation can be generated in two ways: from source code, and from
standalone Markdown files.

\subsubsection{Documenting source code}\label{documenting-source-code}

The primary way of documenting a Rust project is through annotating the
source code. You can use documentation comments for this purpose:

\begin{Shaded}
\begin{Highlighting}[]
\CommentTok{/// Constructs a new `Rc<T>`.}
\CommentTok{///}
\CommentTok{/// # Examples}
\CommentTok{///}
\CommentTok{/// ```}
\CommentTok{/// use std::rc::Rc;}
\CommentTok{///}
\CommentTok{/// let five = Rc::new(5);}
\CommentTok{/// ```}
\KeywordTok{pub} \KeywordTok{fn} \NormalTok{new(value: T) -> Rc<T> \{}
    \CommentTok{// implementation goes here}
\NormalTok{\}}
\end{Highlighting}
\end{Shaded}

This code generates documentation that looks
\href{https://doc.rust-lang.org/nightly/std/rc/struct.Rc.html\#method.new}{like
this}. I've left the implementation out, with a regular comment in its
place.

The first thing to notice about this annotation is that it uses
\texttt{///} instead of \texttt{//}. The triple slash indicates a
documentation comment.

Documentation comments are written in Markdown.

Rust keeps track of these comments, and uses them when generating
documentation. This is important when documenting things like enums:

\begin{Shaded}
\begin{Highlighting}[]
\CommentTok{/// The `Option` type. See [the module level documentation](#sec--index) for more.}
\KeywordTok{enum} \DataTypeTok{Option}\NormalTok{<T> \{}
    \CommentTok{/// No value}
    \ConstantTok{None}\NormalTok{,}
    \CommentTok{/// Some value `T`}
    \ConstantTok{Some}\NormalTok{(T),}
\NormalTok{\}}
\end{Highlighting}
\end{Shaded}

The above works, but this does not:

\begin{Shaded}
\begin{Highlighting}[]
\CommentTok{/// The `Option` type. See [the module level documentation](#sec--index) for more.}
\KeywordTok{enum} \DataTypeTok{Option}\NormalTok{<T> \{}
    \ConstantTok{None}\NormalTok{, }\CommentTok{/// No value}
    \ConstantTok{Some}\NormalTok{(T), }\CommentTok{/// Some value `T`}
\NormalTok{\}}
\end{Highlighting}
\end{Shaded}

You'll get an error:

\begin{verbatim}
hello.rs:4:1: 4:2 error: expected ident, found `}`
hello.rs:4 }
           ^
\end{verbatim}

This \href{https://github.com/rust-lang/rust/issues/22547}{unfortunate
error} is correct; documentation comments apply to the thing after them,
and there's nothing after that last comment.

\paragraph{Writing documentation
comments}\label{writing-documentation-comments}

Anyway, let's cover each part of this comment in detail:

\begin{Shaded}
\begin{Highlighting}[]
\CommentTok{/// Constructs a new `Rc<T>`.}
\end{Highlighting}
\end{Shaded}

The first line of a documentation comment should be a short summary of
its functionality. One sentence. Just the basics. High level.

\begin{Shaded}
\begin{Highlighting}[]
\CommentTok{///}
\CommentTok{/// Other details about constructing `Rc<T>`s, maybe describing complicated}
\CommentTok{/// semantics, maybe additional options, all kinds of stuff.}
\CommentTok{///}
\end{Highlighting}
\end{Shaded}

Our original example had just a summary line, but if we had more things
to say, we could have added more explanation in a new paragraph.

\subparagraph{Special sections}\label{special-sections}

Next, are special sections. These are indicated with a header,
\texttt{\#}. There are four kinds of headers that are commonly used.
They aren't special syntax, just convention, for now.

\begin{Shaded}
\begin{Highlighting}[]
\CommentTok{/// # Panics}
\end{Highlighting}
\end{Shaded}

Unrecoverable misuses of a function (i.e.~programming errors) in Rust
are usually indicated by panics, which kill the whole current thread at
the very least. If your function has a non-trivial contract like this,
that is detected/enforced by panics, documenting it is very important.

\begin{Shaded}
\begin{Highlighting}[]
\CommentTok{/// # Errors}
\end{Highlighting}
\end{Shaded}

If your function or method returns a
\texttt{Result\textless{}T,\ E\textgreater{}}, then describing the
conditions under which it returns \texttt{Err(E)} is a nice thing to do.
This is slightly less important than \texttt{Panics}, because failure is
encoded into the type system, but it's still a good thing to do.

\begin{Shaded}
\begin{Highlighting}[]
\CommentTok{/// # Safety}
\end{Highlighting}
\end{Shaded}

If your function is \texttt{unsafe}, you should explain which invariants
the caller is responsible for upholding.

\begin{Shaded}
\begin{Highlighting}[]
\CommentTok{/// # Examples}
\CommentTok{///}
\CommentTok{/// ```}
\CommentTok{/// use std::rc::Rc;}
\CommentTok{///}
\CommentTok{/// let five = Rc::new(5);}
\CommentTok{/// ```}
\end{Highlighting}
\end{Shaded}

Fourth, \texttt{Examples}. Include one or more examples of using your
function or method, and your users will love you for it. These examples
go inside of code block annotations, which we'll talk about in a moment,
and can have more than one section:

\begin{Shaded}
\begin{Highlighting}[]
\CommentTok{/// # Examples}
\CommentTok{///}
\CommentTok{/// Simple `&str` patterns:}
\CommentTok{///}
\CommentTok{/// ```}
\CommentTok{/// let v: Vec<&str> = "Mary had a little lamb".split(' ').collect();}
\CommentTok{/// assert_eq!(v, vec!["Mary", "had", "a", "little", "lamb"]);}
\CommentTok{/// ```}
\CommentTok{///}
\CommentTok{/// More complex patterns with a lambda:}
\CommentTok{///}
\CommentTok{/// ```}
\CommentTok{/// let v: Vec<&str> = "abc1def2ghi".split(|c: char| c.is_numeric()).collect();}
\CommentTok{/// assert_eq!(v, vec!["abc", "def", "ghi"]);}
\CommentTok{/// ```}
\end{Highlighting}
\end{Shaded}

Let's discuss the details of these code blocks.

\subparagraph{Code block annotations}\label{code-block-annotations}

To write some Rust code in a comment, use the triple graves:

\begin{Shaded}
\begin{Highlighting}[]
\CommentTok{/// ```}
\CommentTok{/// println!("Hello, world");}
\CommentTok{/// ```}
\end{Highlighting}
\end{Shaded}

If you want something that's not Rust code, you can add an annotation:

\begin{Shaded}
\begin{Highlighting}[]
\CommentTok{/// ```c}
\CommentTok{/// printf("Hello, world\textbackslash{}n");}
\CommentTok{/// ```}
\end{Highlighting}
\end{Shaded}

This will highlight according to whatever language you're showing off.
If you're only showing plain text, choose \texttt{text}.

It's important to choose the correct annotation here, because
\texttt{rustdoc} uses it in an interesting way: It can be used to
actually test your examples in a library crate, so that they don't get
out of date. If you have some C code but \texttt{rustdoc} thinks it's
Rust because you left off the annotation, \texttt{rustdoc} will complain
when trying to generate the documentation.

\subsubsection{Documentation as tests}\label{documentation-as-tests}

Let's discuss our sample example documentation:

\begin{Shaded}
\begin{Highlighting}[]
\CommentTok{/// ```}
\CommentTok{/// println!("Hello, world");}
\CommentTok{/// ```}
\end{Highlighting}
\end{Shaded}

You'll notice that you don't need a \texttt{fn\ main()} or anything
here. \texttt{rustdoc} will automatically add a \texttt{main()} wrapper
around your code, using heuristics to attempt to put it in the right
place. For example:

\begin{Shaded}
\begin{Highlighting}[]
\CommentTok{/// ```}
\CommentTok{/// use std::rc::Rc;}
\CommentTok{///}
\CommentTok{/// let five = Rc::new(5);}
\CommentTok{/// ```}
\end{Highlighting}
\end{Shaded}

This will end up testing:

\begin{Shaded}
\begin{Highlighting}[]
\KeywordTok{fn} \NormalTok{main() \{}
    \KeywordTok{use} \NormalTok{std::rc::Rc;}
    \KeywordTok{let} \NormalTok{five = Rc::new(}\DecValTok{5}\NormalTok{);}
\NormalTok{\}}
\end{Highlighting}
\end{Shaded}

Here's the full algorithm rustdoc uses to preprocess examples:

\begin{enumerate}
\def\labelenumi{\arabic{enumi}.}
\tightlist
\item
  Any leading \texttt{\#!{[}foo{]}} attributes are left intact as crate
  attributes.
\item
  Some common \texttt{allow} attributes are inserted, including
  \texttt{unused\_variables}, \texttt{unused\_assignments},
  \texttt{unused\_mut}, \texttt{unused\_attributes}, and
  \texttt{dead\_code}. Small examples often trigger these lints.
\item
  If the example does not contain \texttt{extern\ crate}, then
  \texttt{extern\ crate\ \ \ \ \textless{}mycrate\textgreater{};} is
  inserted (note the lack of \texttt{\#{[}macro\_use{]}}).
\item
  Finally, if the example does not contain \texttt{fn\ main}, the
  remainder of the text is wrapped in
  \texttt{fn\ main()\ \{\ your\_code\ \}}.
\end{enumerate}

This generated \texttt{fn\ main} can be a problem! If you have
\texttt{extern\ crate} or a \texttt{mod} statements in the example code
that are referred to by \texttt{use} statements, they will fail to
resolve unless you include at least \texttt{fn\ main()\ \{\}} to inhibit
step 4. \texttt{\#{[}macro\_use{]}\ extern\ crate} also does not work
except at the crate root, so when testing macros an explicit
\texttt{main} is always required. It doesn't have to clutter up your
docs, though -- keep reading!

Sometimes this algorithm isn't enough, though. For example, all of these
code samples with \texttt{///} we've been talking about? The raw text:

\begin{verbatim}
/// Some documentation.
# fn foo() {}
\end{verbatim}

looks different than the output:

\begin{Shaded}
\begin{Highlighting}[]
\CommentTok{/// Some documentation.}
\end{Highlighting}
\end{Shaded}

Yes, that's right: you can add lines that start with \texttt{\#}, and
they will be hidden from the output, but will be used when compiling
your code. You can use this to your advantage. In this case,
documentation comments need to apply to some kind of function, so if I
want to show you just a documentation comment, I need to add a little
function definition below it. At the same time, it's only there to
satisfy the compiler, so hiding it makes the example more clear. You can
use this technique to explain longer examples in detail, while still
preserving the testability of your documentation.

For example, imagine that we wanted to document this code:

\begin{Shaded}
\begin{Highlighting}[]
\KeywordTok{let} \NormalTok{x = }\DecValTok{5}\NormalTok{;}
\KeywordTok{let} \NormalTok{y = }\DecValTok{6}\NormalTok{;}
\PreprocessorTok{println!}\NormalTok{(}\StringTok{"\{\}"}\NormalTok{, x + y);}
\end{Highlighting}
\end{Shaded}

We might want the documentation to end up looking like this:

\begin{quote}
First, we set \texttt{x} to five:

\begin{Shaded}
\begin{Highlighting}[]
\KeywordTok{let} \NormalTok{x = }\DecValTok{5}\NormalTok{;}
\NormalTok{# }\KeywordTok{let} \NormalTok{y = }\DecValTok{6}\NormalTok{;}
\NormalTok{# }\PreprocessorTok{println!}\NormalTok{(}\StringTok{"\{\}"}\NormalTok{, x + y);}
\end{Highlighting}
\end{Shaded}

Next, we set \texttt{y} to six:

\begin{Shaded}
\begin{Highlighting}[]
\NormalTok{# }\KeywordTok{let} \NormalTok{x = }\DecValTok{5}\NormalTok{;}
\KeywordTok{let} \NormalTok{y = }\DecValTok{6}\NormalTok{;}
\NormalTok{# }\PreprocessorTok{println!}\NormalTok{(}\StringTok{"\{\}"}\NormalTok{, x + y);}
\end{Highlighting}
\end{Shaded}

Finally, we print the sum of \texttt{x} and \texttt{y}:

\begin{Shaded}
\begin{Highlighting}[]
\NormalTok{# }\KeywordTok{let} \NormalTok{x = }\DecValTok{5}\NormalTok{;}
\NormalTok{# }\KeywordTok{let} \NormalTok{y = }\DecValTok{6}\NormalTok{;}
\PreprocessorTok{println!}\NormalTok{(}\StringTok{"\{\}"}\NormalTok{, x + y);}
\end{Highlighting}
\end{Shaded}
\end{quote}

To keep each code block testable, we want the whole program in each
block, but we don't want the reader to see every line every time. Here's
what we put in our source code:

\begin{verbatim}
    First, we set `x` to five:

    ```rust
    let x = 5;
    # let y = 6;
    # println!("{}", x + y);
    ```

    Next, we set `y` to six:

    ```rust
    # let x = 5;
    let y = 6;
    # println!("{}", x + y);
    ```

    Finally, we print the sum of `x` and `y`:

    ```rust
    # let x = 5;
    # let y = 6;
    println!("{}", x + y);
    ```
\end{verbatim}

By repeating all parts of the example, you can ensure that your example
still compiles, while only showing the parts that are relevant to that
part of your explanation.

\paragraph{Documenting macros}\label{documenting-macros}

Here's an example of documenting a macro:

\begin{Shaded}
\begin{Highlighting}[]
\CommentTok{/// Panic with a given message unless an expression evaluates to true.}
\CommentTok{///}
\CommentTok{/// # Examples}
\CommentTok{///}
\CommentTok{/// ```}
\CommentTok{/// # #[macro_use] extern crate foo;}
\CommentTok{/// # fn main() \{}
\CommentTok{/// panic_unless!(1 + 1 == 2, “Math is broken.”);}
\CommentTok{/// # \}}
\CommentTok{/// ```}
\CommentTok{///}
\CommentTok{/// ```rust,should_panic}
\CommentTok{/// # #[macro_use] extern crate foo;}
\CommentTok{/// # fn main() \{}
\CommentTok{/// panic_unless!(true == false, “I’m broken.”);}
\CommentTok{/// # \}}
\CommentTok{/// ```}
\AttributeTok{#[}\NormalTok{macro_export}\AttributeTok{]}
\PreprocessorTok{macro_rules!} \NormalTok{panic_unless \{}
    \NormalTok{($condition:expr, $($rest:expr),+) => (\{ }\KeywordTok{if} \NormalTok{! $condition \{ }\PreprocessorTok{panic!}\NormalTok{($($rest),+); \} \}}
\NormalTok{↳ );}
\NormalTok{\}}
\end{Highlighting}
\end{Shaded}

You'll note three things: we need to add our own \texttt{extern\ crate}
line, so that we can add the \texttt{\#{[}macro\_use{]}} attribute.
Second, we'll need to add our own \texttt{main()} as well (for reasons
discussed above). Finally, a judicious use of \texttt{\#} to comment out
those two things, so they don't show up in the output.

Another case where the use of \texttt{\#} is handy is when you want to
ignore error handling. Lets say you want the following,

\begin{Shaded}
\begin{Highlighting}[]
\CommentTok{/// use std::io;}
\CommentTok{/// let mut input = String::new();}
\CommentTok{/// try!(io::stdin().read_line(&mut input));}
\end{Highlighting}
\end{Shaded}

The problem is that \texttt{try!} returns a
\texttt{Result\textless{}T,\ E\textgreater{}} and test functions don't
return anything so this will give a mismatched types error.

\begin{Shaded}
\begin{Highlighting}[]
\CommentTok{/// A doc test using try!}
\CommentTok{///}
\CommentTok{/// ```}
\CommentTok{/// use std::io;}
\CommentTok{/// # fn foo() -> io::Result<()> \{}
\CommentTok{/// let mut input = String::new();}
\CommentTok{/// try!(io::stdin().read_line(&mut input));}
\CommentTok{/// # Ok(())}
\CommentTok{/// # \}}
\CommentTok{/// ```}
\end{Highlighting}
\end{Shaded}

You can get around this by wrapping the code in a function. This catches
and swallows the \texttt{Result\textless{}T,\ E\textgreater{}} when
running tests on the docs. This pattern appears regularly in the
standard library.

\paragraph{Running documentation
tests}\label{running-documentation-tests}

To run the tests, either:

\begin{Shaded}
\begin{Highlighting}[]
\NormalTok{$ }\KeywordTok{rustdoc} \NormalTok{--test path/to/my/crate/root.rs}
\CommentTok{# or}
\NormalTok{$ }\KeywordTok{cargo} \NormalTok{test}
\end{Highlighting}
\end{Shaded}

That's right, \texttt{cargo\ test} tests embedded documentation too.
\textbf{However, \texttt{cargo\ test} will not test binary crates, only
library ones.} This is due to the way \texttt{rustdoc} works: it links
against the library to be tested, but with a binary, there's nothing to
link to.

There are a few more annotations that are useful to help
\texttt{rustdoc} do the right thing when testing your code:

\begin{Shaded}
\begin{Highlighting}[]
\CommentTok{/// ```rust,ignore}
\CommentTok{/// fn foo() \{}
\CommentTok{/// ```}
\end{Highlighting}
\end{Shaded}

The \texttt{ignore} directive tells Rust to ignore your code. This is
almost never what you want, as it's the most generic. Instead, consider
annotating it with \texttt{text} if it's not code, or using \texttt{\#}s
to get a working example that only shows the part you care about.

\begin{Shaded}
\begin{Highlighting}[]
\CommentTok{/// ```rust,should_panic}
\CommentTok{/// assert!(false);}
\CommentTok{/// ```}
\end{Highlighting}
\end{Shaded}

\texttt{should\_panic} tells \texttt{rustdoc} that the code should
compile correctly, but not actually pass as a test.

\begin{Shaded}
\begin{Highlighting}[]
\CommentTok{/// ```rust,no_run}
\CommentTok{/// loop \{}
\CommentTok{///     println!("Hello, world");}
\CommentTok{/// \}}
\CommentTok{/// ```}
\end{Highlighting}
\end{Shaded}

The \texttt{no\_run} attribute will compile your code, but not run it.
This is important for examples such as ``Here's how to start up a
network service,'' which you would want to make sure compile, but might
run in an infinite loop!

\paragraph{Documenting modules}\label{documenting-modules}

Rust has another kind of doc comment, \texttt{//!}. This comment doesn't
document the next item, but the enclosing item. In other words:

\begin{Shaded}
\begin{Highlighting}[]
\KeywordTok{mod} \NormalTok{foo \{}
    \CommentTok{//! This is documentation for the `foo` module.}
    \CommentTok{//!}
    \CommentTok{//! # Examples}

    \CommentTok{// ...}
\NormalTok{\}}
\end{Highlighting}
\end{Shaded}

This is where you'll see \texttt{//!} used most often: for module
documentation. If you have a module in \texttt{foo.rs}, you'll often
open its code and see this:

\begin{Shaded}
\begin{Highlighting}[]
\CommentTok{//! A module for using `foo`s.}
\CommentTok{//!}
\CommentTok{//! The `foo` module contains a lot of useful functionality blah blah blah}
\end{Highlighting}
\end{Shaded}

\paragraph{Documentation comment
style}\label{documentation-comment-style}

Check out
\href{https://github.com/rust-lang/rfcs/blob/master/text/0505-api-comment-conventions.md}{RFC
505} for full conventions around the style and format of documentation.

\subsubsection{Other documentation}\label{other-documentation}

All of this behavior works in non-Rust source files too. Because
comments are written in Markdown, they're often \texttt{.md} files.

When you write documentation in Markdown files, you don't need to prefix
the documentation with comments. For example:

\begin{Shaded}
\begin{Highlighting}[]
\CommentTok{/// # Examples}
\CommentTok{///}
\CommentTok{/// ```}
\CommentTok{/// use std::rc::Rc;}
\CommentTok{///}
\CommentTok{/// let five = Rc::new(5);}
\CommentTok{/// ```}
\end{Highlighting}
\end{Shaded}

is:

\begin{Shaded}
\begin{Highlighting}[]
\FunctionTok{### Examples}

\NormalTok{```}
\NormalTok{use std::rc::Rc;}

\NormalTok{let five = Rc::new(5);}
\NormalTok{```}
\end{Highlighting}
\end{Shaded}

when it's in a Markdown file. There is one wrinkle though: Markdown
files need to have a title like this:

\begin{Shaded}
\begin{Highlighting}[]
\NormalTok line needs to be the very first line of the file.

\subsubsection{\texorpdfstring{\texttt{doc}
attributes}{doc attributes}}\label{doc-attributes}

At a deeper level, documentation comments are syntactic sugar for
documentation attributes:

\begin{Shaded}
\begin{Highlighting}[]
\CommentTok{/// this}

\AttributeTok{#[}\NormalTok{doc}\AttributeTok{=}\StringTok{"this"}\AttributeTok{]}
\end{Highlighting}
\end{Shaded}

are the same, as are these:

\begin{Shaded}
\begin{Highlighting}[]
\CommentTok{//! this}

\AttributeTok{#![}\NormalTok{doc}\AttributeTok{=}\StringTok{"this"}\AttributeTok{]}
\end{Highlighting}
\end{Shaded}

You won't often see this attribute used for writing documentation, but
it can be useful when changing some options, or when writing a macro.

\paragraph{Re-exports}\label{re-exports}

\texttt{rustdoc} will show the documentation for a public re-export in
both places:

\begin{Shaded}
\begin{Highlighting}[]
\KeywordTok{extern} \KeywordTok{crate} \NormalTok{foo;}

\KeywordTok{pub} \KeywordTok{use} \NormalTok{foo::bar;}
\end{Highlighting}
\end{Shaded}

This will create documentation for \texttt{bar} both inside the
documentation for the crate \texttt{foo}, as well as the documentation
for your crate. It will use the same documentation in both places.

This behavior can be suppressed with \texttt{no\_inline}:

\begin{Shaded}
\begin{Highlighting}[]
\KeywordTok{extern} \KeywordTok{crate} \NormalTok{foo;}

\AttributeTok{#[}\NormalTok{doc}\AttributeTok{(}\NormalTok{no_inline}\AttributeTok{)]}
\KeywordTok{pub} \KeywordTok{use} \NormalTok{foo::bar;}
\end{Highlighting}
\end{Shaded}

\subsubsection{Missing documentation}\label{missing-documentation}

Sometimes you want to make sure that every single public thing in your
project is documented, especially when you are working on a library.
Rust allows you to to generate warnings or errors, when an item is
missing documentation. To generate warnings you use \texttt{warn}:

\begin{Shaded}
\begin{Highlighting}[]
\AttributeTok{#![}\NormalTok{warn}\AttributeTok{(}\NormalTok{missing_docs}\AttributeTok{)]}
\end{Highlighting}
\end{Shaded}

And to generate errors you use \texttt{deny}:

\begin{Shaded}
\begin{Highlighting}[]
\AttributeTok{#![}\NormalTok{deny}\AttributeTok{(}\NormalTok{missing_docs}\AttributeTok{)]}
\end{Highlighting}
\end{Shaded}

There are cases where you want to disable these warnings/errors to
explicitly leave something undocumented. This is done by using
\texttt{allow}:

\begin{Shaded}
\begin{Highlighting}[]
\AttributeTok{#[}\NormalTok{allow}\AttributeTok{(}\NormalTok{missing_docs}\AttributeTok{)]}
\KeywordTok{struct} \NormalTok{Undocumented;}
\end{Highlighting}
\end{Shaded}

You might even want to hide items from the documentation completely:

\begin{Shaded}
\begin{Highlighting}[]
\AttributeTok{#[}\NormalTok{doc}\AttributeTok{(}\NormalTok{hidden}\AttributeTok{)]}
\KeywordTok{struct} \NormalTok{Hidden;}
\end{Highlighting}
\end{Shaded}

\paragraph{Controlling HTML}\label{controlling-html}

You can control a few aspects of the HTML that \texttt{rustdoc}
generates through the \texttt{\#!{[}doc{]}} version of the attribute:

\begin{Shaded}
\begin{Highlighting}[]
\AttributeTok{#![}\NormalTok{doc}\AttributeTok{(}\NormalTok{html_logo_url }\AttributeTok{=} \StringTok{"https://www.rust-lang.org/logos/rust-logo-128x128-blk-v2.png"}\AttributeTok{,}
       \NormalTok{html_favicon_url }\AttributeTok{=} \StringTok{"https://www.rust-lang.org/favicon.ico"}\AttributeTok{,}
       \NormalTok{html_root_url }\AttributeTok{=} \StringTok{"https://doc.rust-lang.org/"}\AttributeTok{)]}
\end{Highlighting}
\end{Shaded}

This sets a few different options, with a logo, favicon, and a root URL.

\paragraph{Configuring documentation
tests}\label{configuring-documentation-tests}

You can also configure the way that \texttt{rustdoc} tests your
documentation examples through the \texttt{\#!{[}doc(test(..)){]}}
attribute.

\begin{Shaded}
\begin{Highlighting}[]
\AttributeTok{#![}\NormalTok{doc}\AttributeTok{(}\NormalTok{test}\AttributeTok{(}\NormalTok{attr}\AttributeTok{(}\NormalTok{allow}\AttributeTok{(}\NormalTok{unused_variables}\AttributeTok{),} \NormalTok{deny}\AttributeTok{(}\NormalTok{warnings}\AttributeTok{))))]}
\end{Highlighting}
\end{Shaded}

This allows unused variables within the examples, but will fail the test
for any other lint warning thrown.

\subsubsection{Generation options}\label{generation-options}

\texttt{rustdoc} also contains a few other options on the command line,
for further customization:

\begin{itemize}
\tightlist
\item
  \texttt{-\/-html-in-header\ FILE}: includes the contents of FILE at
  the end of the
  \texttt{\textless{}head\textgreater{}...\textless{}/head\textgreater{}}
  section.
\item
  \texttt{-\/-html-before-content\ FILE}: includes the contents of FILE
  directly after \texttt{\textless{}body\textgreater{}}, before the
  rendered content (including the search bar).
\item
  \texttt{-\/-html-after-content\ FILE}: includes the contents of FILE
  after all the rendered content.
\end{itemize}

\subsubsection{Security note}\label{security-note}

The Markdown in documentation comments is placed without processing into
the final webpage. Be careful with literal HTML:

\begin{Shaded}
\begin{Highlighting}[]
\CommentTok{/// <script>alert(document.cookie)</script>}
\end{Highlighting}
\end{Shaded}

\hypertarget{sec--iterators}{\section{Iterators}\label{sec--iterators}}

Let's talk about loops.

Remember Rust's \texttt{for} loop? Here's an example:

\begin{Shaded}
\begin{Highlighting}[]
\KeywordTok{for} \NormalTok{x }\KeywordTok{in} \DecValTok{0.}\NormalTok{.}\DecValTok{10} \NormalTok{\{}
    \PreprocessorTok{println!}\NormalTok{(}\StringTok{"\{\}"}\NormalTok{, x);}
\NormalTok{\}}
\end{Highlighting}
\end{Shaded}

Now that you know more Rust, we can talk in detail about how this works.
Ranges (the \texttt{0..10}) are `iterators'. An iterator is something
that we can call the \texttt{.next()} method on repeatedly, and it gives
us a sequence of things.

(By the way, a range with two dots like \texttt{0..10} is inclusive on
the left (so it starts at 0) and exclusive on the right (so it ends at
9). A mathematician would write ``{[}0, 10)''. To get a range that goes
all the way up to 10 you can write \texttt{0...10}.)

Like this:

\begin{Shaded}
\begin{Highlighting}[]
\KeywordTok{let} \KeywordTok{mut} \NormalTok{range = }\DecValTok{0.}\NormalTok{.}\DecValTok{10}\NormalTok{;}

\KeywordTok{loop} \NormalTok{\{}
    \KeywordTok{match} \NormalTok{range.next() \{}
        \ConstantTok{Some}\NormalTok{(x) => \{}
            \PreprocessorTok{println!}\NormalTok{(}\StringTok{"\{\}"}\NormalTok{, x);}
        \NormalTok{\},}
        \ConstantTok{None} \NormalTok{=> \{ }\KeywordTok{break} \NormalTok{\}}
    \NormalTok{\}}
\NormalTok{\}}
\end{Highlighting}
\end{Shaded}

We make a mutable binding to the range, which is our iterator. We then
\texttt{loop}, with an inner \texttt{match}. This \texttt{match} is used
on the result of \texttt{range.next()}, which gives us a reference to
the next value of the iterator. \texttt{next} returns an
\texttt{Option\textless{}i32\textgreater{}}, in this case, which will be
\texttt{Some(i32)} when we have a value and \texttt{None} once we run
out. If we get \texttt{Some(i32)}, we print it out, and if we get
\texttt{None}, we \texttt{break} out of the loop.

This code sample is basically the same as our \texttt{for} loop version.
The \texttt{for} loop is a handy way to write this
\texttt{loop}/\texttt{match}/\texttt{break} construct.

\texttt{for} loops aren't the only thing that uses iterators, however.
Writing your own iterator involves implementing the \texttt{Iterator}
trait. While doing that is outside of the scope of this guide, Rust
provides a number of useful iterators to accomplish various tasks. But
first, a few notes about limitations of ranges.

Ranges are very primitive, and we often can use better alternatives.
Consider the following Rust anti-pattern: using ranges to emulate a
C-style \texttt{for} loop. Let's suppose you needed to iterate over the
contents of a vector. You may be tempted to write this:

\begin{Shaded}
\begin{Highlighting}[]
\KeywordTok{let} \NormalTok{nums = }\PreprocessorTok{vec!}\NormalTok{[}\DecValTok{1}\NormalTok{, }\DecValTok{2}\NormalTok{, }\DecValTok{3}\NormalTok{];}

\KeywordTok{for} \NormalTok{i }\KeywordTok{in} \DecValTok{0.}\NormalTok{.nums.len() \{}
    \PreprocessorTok{println!}\NormalTok{(}\StringTok{"\{\}"}\NormalTok{, nums[i]);}
\NormalTok{\}}
\end{Highlighting}
\end{Shaded}

This is strictly worse than using an actual iterator. You can iterate
over vectors directly, so write this:

\begin{Shaded}
\begin{Highlighting}[]
\KeywordTok{let} \NormalTok{nums = }\PreprocessorTok{vec!}\NormalTok{[}\DecValTok{1}\NormalTok{, }\DecValTok{2}\NormalTok{, }\DecValTok{3}\NormalTok{];}

\KeywordTok{for} \NormalTok{num }\KeywordTok{in} \NormalTok{&nums \{}
    \PreprocessorTok{println!}\NormalTok{(}\StringTok{"\{\}"}\NormalTok{, num);}
\NormalTok{\}}
\end{Highlighting}
\end{Shaded}

There are two reasons for this. First, this more directly expresses what
we mean. We iterate through the entire vector, rather than iterating
through indexes, and then indexing the vector. Second, this version is
more efficient: the first version will have extra bounds checking
because it used indexing, \texttt{nums{[}i{]}}. But since we yield a
reference to each element of the vector in turn with the iterator,
there's no bounds checking in the second example. This is very common
with iterators: we can ignore unnecessary bounds checks, but still know
that we're safe.

There's another detail here that's not 100\% clear because of how
\texttt{println!} works. \texttt{num} is actually of type
\texttt{\&i32}. That is, it's a reference to an \texttt{i32}, not an
\texttt{i32} itself. \texttt{println!} handles the dereferencing for us,
so we don't see it. This code works fine too:

\begin{Shaded}
\begin{Highlighting}[]
\KeywordTok{let} \NormalTok{nums = }\PreprocessorTok{vec!}\NormalTok{[}\DecValTok{1}\NormalTok{, }\DecValTok{2}\NormalTok{, }\DecValTok{3}\NormalTok{];}

\KeywordTok{for} \NormalTok{num }\KeywordTok{in} \NormalTok{&nums \{}
    \PreprocessorTok{println!}\NormalTok{(}\StringTok{"\{\}"}\NormalTok{, *num);}
\NormalTok{\}}
\end{Highlighting}
\end{Shaded}

Now we're explicitly dereferencing \texttt{num}. Why does
\texttt{\&nums} give us references? Firstly, because we explicitly asked
it to with \texttt{\&}. Secondly, if it gave us the data itself, we
would have to be its owner, which would involve making a copy of the
data and giving us the copy. With references, we're only borrowing a
reference to the data, and so it's only passing a reference, without
needing to do the move.

So, now that we've established that ranges are often not what you want,
let's talk about what you do want instead.

There are three broad classes of things that are relevant here:
iterators, \emph{iterator adaptors}, and \emph{consumers}. Here's some
definitions:

\begin{itemize}
\tightlist
\item
  \emph{iterators} give you a sequence of values.
\item
  \emph{iterator adaptors} operate on an iterator, producing a new
  iterator with a different output sequence.
\item
  \emph{consumers} operate on an iterator, producing some final set of
  values.
\end{itemize}

Let's talk about consumers first, since you've already seen an iterator,
ranges.

\subsubsection{Consumers}\label{consumers}

A \emph{consumer} operates on an iterator, returning some kind of value
or values. The most common consumer is \texttt{collect()}. This code
doesn't quite compile, but it shows the intention:

\begin{Shaded}
\begin{Highlighting}[]
\KeywordTok{let} \NormalTok{one_to_one_hundred = (}\DecValTok{1.}\NormalTok{.}\DecValTok{101}\NormalTok{).collect();}
\end{Highlighting}
\end{Shaded}

As you can see, we call \texttt{collect()} on our iterator.
\texttt{collect()} takes as many values as the iterator will give it,
and returns a collection of the results. So why won't this compile? Rust
can't determine what type of things you want to collect, and so you need
to let it know. Here's the version that does compile:

\begin{Shaded}
\begin{Highlighting}[]
\KeywordTok{let} \NormalTok{one_to_one_hundred = (}\DecValTok{1.}\NormalTok{.}\DecValTok{101}\NormalTok{).collect::<}\DataTypeTok{Vec}\NormalTok{<}\DataTypeTok{i32}\NormalTok{>>();}
\end{Highlighting}
\end{Shaded}

If you remember, the \texttt{::\textless{}\textgreater{}} syntax allows
us to give a type hint, and so we tell it that we want a vector of
integers. You don't always need to use the whole type, though. Using a
\texttt{\_} will let you provide a partial hint:

\begin{Shaded}
\begin{Highlighting}[]
\KeywordTok{let} \NormalTok{one_to_one_hundred = (}\DecValTok{1.}\NormalTok{.}\DecValTok{101}\NormalTok{).collect::<}\DataTypeTok{Vec}\NormalTok{<_>>();}
\end{Highlighting}
\end{Shaded}

This says ``Collect into a \texttt{Vec\textless{}T\textgreater{}},
please, but infer what the \texttt{T} is for me.'' \texttt{\_} is
sometimes called a ``type placeholder'' for this reason.

\texttt{collect()} is the most common consumer, but there are others
too. \texttt{find()} is one:

\begin{Shaded}
\begin{Highlighting}[]
\KeywordTok{let} \NormalTok{greater_than_forty_two = (}\DecValTok{0.}\NormalTok{.}\DecValTok{100}\NormalTok{)}
                             \NormalTok{.find(|x| *x > }\DecValTok{42}\NormalTok{);}

\KeywordTok{match} \NormalTok{greater_than_forty_two \{}
    \ConstantTok{Some}\NormalTok{(_) => }\PreprocessorTok{println!}\NormalTok{(}\StringTok{"Found a match!"}\NormalTok{),}
    \ConstantTok{None} \NormalTok{=> }\PreprocessorTok{println!}\NormalTok{(}\StringTok{"No match found :("}\NormalTok{),}
\NormalTok{\}}
\end{Highlighting}
\end{Shaded}

\texttt{find} takes a closure, and works on a reference to each element
of an iterator. This closure returns \texttt{true} if the element is the
element we're looking for, and \texttt{false} otherwise. \texttt{find}
returns the first element satisfying the specified predicate. Because we
might not find a matching element, \texttt{find} returns an
\texttt{Option} rather than the element itself.

Another important consumer is \texttt{fold}. Here's what it looks like:

\begin{Shaded}
\begin{Highlighting}[]
\KeywordTok{let} \NormalTok{sum = (}\DecValTok{1.}\NormalTok{.}\DecValTok{4}\NormalTok{).fold(}\DecValTok{0}\NormalTok{, |sum, x| sum + x);}
\end{Highlighting}
\end{Shaded}

\texttt{fold()} is a consumer that looks like this:
\texttt{fold(base,\ \textbar{}accumulator,\ element\textbar{}\ ...)}. It
takes two arguments: the first is an element called the \emph{base}. The
second is a closure that itself takes two arguments: the first is called
the \emph{accumulator}, and the second is an \emph{element}. Upon each
iteration, the closure is called, and the result is the value of the
accumulator on the next iteration. On the first iteration, the base is
the value of the accumulator.

Okay, that's a bit confusing. Let's examine the values of all of these
things in this iterator:

\begin{longtable}[c]{@{}llll@{}}
\toprule
base & accumulator & element & closure result\tabularnewline
\midrule
\endhead
0 & 0 & 1 & 1\tabularnewline
0 & 1 & 2 & 3\tabularnewline
0 & 3 & 3 & 6\tabularnewline
\bottomrule
\end{longtable}

We called \texttt{fold()} with these arguments:

\begin{Shaded}
\begin{Highlighting}[]
\NormalTok{.fold(}\DecValTok{0}\NormalTok{, |sum, x| sum + x);}
\end{Highlighting}
\end{Shaded}

So, \texttt{0} is our base, \texttt{sum} is our accumulator, and
\texttt{x} is our element. On the first iteration, we set \texttt{sum}
to \texttt{0}, and \texttt{x} is the first element of \texttt{nums},
\texttt{1}. We then add \texttt{sum} and \texttt{x}, which gives us
\texttt{0\ +\ 1\ =\ 1}. On the second iteration, that value becomes our
accumulator, \texttt{sum}, and the element is the second element of the
array, \texttt{2}. \texttt{1\ +\ 2\ =\ 3}, and so that becomes the value
of the accumulator for the last iteration. On that iteration, \texttt{x}
is the last element, \texttt{3}, and \texttt{3\ +\ 3\ =\ 6}, which is
our final result for our sum. \texttt{1\ +\ 2\ +\ 3\ =\ 6}, and that's
the result we got.

Whew. \texttt{fold} can be a bit strange the first few times you see it,
but once it clicks, you can use it all over the place. Any time you have
a list of things, and you want a single result, \texttt{fold} is
appropriate.

Consumers are important due to one additional property of iterators we
haven't talked about yet: laziness. Let's talk some more about
iterators, and you'll see why consumers matter.

\hypertarget{iterators}{\subsubsection{Iterators}\label{iterators}}

As we've said before, an iterator is something that we can call the
\texttt{.next()} method on repeatedly, and it gives us a sequence of
things. Because you need to call the method, this means that iterators
can be \emph{lazy} and not generate all of the values upfront. This
code, for example, does not actually generate the numbers \texttt{1-99},
instead creating a value that merely represents the sequence:

\begin{Shaded}
\begin{Highlighting}[]
\KeywordTok{let} \NormalTok{nums = }\DecValTok{1.}\NormalTok{.}\DecValTok{100}\NormalTok{;}
\end{Highlighting}
\end{Shaded}

Since we didn't do anything with the range, it didn't generate the
sequence. Let's add the consumer:

\begin{Shaded}
\begin{Highlighting}[]
\KeywordTok{let} \NormalTok{nums = (}\DecValTok{1.}\NormalTok{.}\DecValTok{100}\NormalTok{).collect::<}\DataTypeTok{Vec}\NormalTok{<}\DataTypeTok{i32}\NormalTok{>>();}
\end{Highlighting}
\end{Shaded}

Now, \texttt{collect()} will require that the range gives it some
numbers, and so it will do the work of generating the sequence.

Ranges are one of two basic iterators that you'll see. The other is
\texttt{iter()}. \texttt{iter()} can turn a vector into a simple
iterator that gives you each element in turn:

\begin{Shaded}
\begin{Highlighting}[]
\KeywordTok{let} \NormalTok{nums = }\PreprocessorTok{vec!}\NormalTok{[}\DecValTok{1}\NormalTok{, }\DecValTok{2}\NormalTok{, }\DecValTok{3}\NormalTok{];}

\KeywordTok{for} \NormalTok{num }\KeywordTok{in} \NormalTok{nums.iter() \{}
   \PreprocessorTok{println!}\NormalTok{(}\StringTok{"\{\}"}\NormalTok{, num);}
\NormalTok{\}}
\end{Highlighting}
\end{Shaded}

These two basic iterators should serve you well. There are some more
advanced iterators, including ones that are infinite.

That's enough about iterators. Iterator adaptors are the last concept we
need to talk about with regards to iterators. Let's get to it!

\subsubsection{Iterator adaptors}\label{iterator-adaptors}

\emph{Iterator adaptors} take an iterator and modify it somehow,
producing a new iterator. The simplest one is called \texttt{map}:

\begin{Shaded}
\begin{Highlighting}[]
\NormalTok{(}\DecValTok{1.}\NormalTok{.}\DecValTok{100}\NormalTok{).map(|x| x + }\DecValTok{1}\NormalTok{);}
\end{Highlighting}
\end{Shaded}

\texttt{map} is called upon another iterator, and produces a new
iterator where each element reference has the closure it's been given as
an argument called on it. So this would give us the numbers from
\texttt{2-100}. Well, almost! If you compile the example, you'll get a
warning:

\begin{verbatim}
warning: unused result which must be used: iterator adaptors are lazy and
         do nothing unless consumed, #[warn(unused_must_use)] on by default
(1..100).map(|x| x + 1);
 ^~~~~~~~~~~~~~~~~~~~~~~~~~~~~~~~
\end{verbatim}

Laziness strikes again! That closure will never execute. This example
doesn't print any numbers:

\begin{Shaded}
\begin{Highlighting}[]
\NormalTok{(}\DecValTok{1.}\NormalTok{.}\DecValTok{100}\NormalTok{).map(|x| }\PreprocessorTok{println!}\NormalTok{(}\StringTok{"\{\}"}\NormalTok{, x));}
\end{Highlighting}
\end{Shaded}

If you are trying to execute a closure on an iterator for its side
effects, use \texttt{for} instead.

There are tons of interesting iterator adaptors. \texttt{take(n)} will
return an iterator over the next \texttt{n} elements of the original
iterator. Let's try it out with an infinite iterator:

\begin{Shaded}
\begin{Highlighting}[]
\KeywordTok{for} \NormalTok{i }\KeywordTok{in} \NormalTok{(}\DecValTok{1.}\NormalTok{.).take(}\DecValTok{5}\NormalTok{) \{}
    \PreprocessorTok{println!}\NormalTok{(}\StringTok{"\{\}"}\NormalTok{, i);}
\NormalTok{\}}
\end{Highlighting}
\end{Shaded}

This will print

\begin{verbatim}
1
2
3
4
5
\end{verbatim}

\texttt{filter()} is an adapter that takes a closure as an argument.
This closure returns \texttt{true} or \texttt{false}. The new iterator
\texttt{filter()} produces only the elements that the closure returns
\texttt{true} for:

\begin{Shaded}
\begin{Highlighting}[]
\KeywordTok{for} \NormalTok{i }\KeywordTok{in} \NormalTok{(}\DecValTok{1.}\NormalTok{.}\DecValTok{100}\NormalTok{).filter(|&x| x % }\DecValTok{2} \NormalTok{== }\DecValTok{0}\NormalTok{) \{}
    \PreprocessorTok{println!}\NormalTok{(}\StringTok{"\{\}"}\NormalTok{, i);}
\NormalTok{\}}
\end{Highlighting}
\end{Shaded}

This will print all of the even numbers between one and a hundred. (Note
that, unlike \texttt{map}, the closure passed to \texttt{filter} is
passed a reference to the element instead of the element itself. The
filter predicate here uses the \texttt{\&x} pattern to extract the
integer. The filter closure is passed a reference because it returns
\texttt{true} or \texttt{false} instead of the element, so the
\texttt{filter} implementation must retain ownership to put the elements
into the newly constructed iterator.)

You can chain all three things together: start with an iterator, adapt
it a few times, and then consume the result. Check it out:

\begin{Shaded}
\begin{Highlighting}[]
\NormalTok{(}\DecValTok{1.}\NormalTok{.)}
    \NormalTok{.filter(|&x| x % }\DecValTok{2} \NormalTok{== }\DecValTok{0}\NormalTok{)}
    \NormalTok{.filter(|&x| x % }\DecValTok{3} \NormalTok{== }\DecValTok{0}\NormalTok{)}
    \NormalTok{.take(}\DecValTok{5}\NormalTok{)}
    \NormalTok{.collect::<}\DataTypeTok{Vec}\NormalTok{<}\DataTypeTok{i32}\NormalTok{>>();}
\end{Highlighting}
\end{Shaded}

This will give you a vector containing \texttt{6}, \texttt{12},
\texttt{18}, \texttt{24}, and \texttt{30}.

This is just a small taste of what iterators, iterator adaptors, and
consumers can help you with. There are a number of really useful
iterators, and you can write your own as well. Iterators provide a safe,
efficient way to manipulate all kinds of lists. They're a little unusual
at first, but if you play with them, you'll get hooked. For a full list
of the different iterators and consumers, check out the
\href{http://doc.rust-lang.org/std/iter/index.html}{iterator module
documentation}.

\hypertarget{sec--concurrency}{\section{Concurrency}\label{sec--concurrency}}

Concurrency and parallelism are incredibly important topics in computer
science, and are also a hot topic in industry today. Computers are
gaining more and more cores, yet many programmers aren't prepared to
fully utilize them.

Rust's memory safety features also apply to its concurrency story too.
Even concurrent Rust programs must be memory safe, having no data races.
Rust's type system is up to the task, and gives you powerful ways to
reason about concurrent code at compile time.

Before we talk about the concurrency features that come with Rust, it's
important to understand something: Rust is low-level enough that the
vast majority of this is provided by the standard library, not by the
language. This means that if you don't like some aspect of the way Rust
handles concurrency, you can implement an alternative way of doing
things. \href{https://github.com/carllerche/mio}{mio} is a real-world
example of this principle in action.

\subsubsection{\texorpdfstring{Background: \texttt{Send} and
\texttt{Sync}}{Background: Send and Sync}}\label{background-send-and-sync}

Concurrency is difficult to reason about. In Rust, we have a strong,
static type system to help us reason about our code. As such, Rust gives
us two traits to help us make sense of code that can possibly be
concurrent.

\paragraph{\texorpdfstring{\texttt{Send}}{Send}}\label{send}

The first trait we're going to talk about is
\href{http://doc.rust-lang.org/std/marker/trait.Send.html}{\texttt{Send}}.
When a type \texttt{T} implements \texttt{Send}, it indicates that
something of this type is able to have ownership transferred safely
between threads.

This is important to enforce certain restrictions. For example, if we
have a channel connecting two threads, we would want to be able to send
some data down the channel and to the other thread. Therefore, we'd
ensure that \texttt{Send} was implemented for that type.

In the opposite way, if we were wrapping a library with
\protect\hyperlink{sec--ffi}{FFI} that isn't threadsafe, we wouldn't
want to implement \texttt{Send}, and so the compiler will help us
enforce that it can't leave the current thread.

\paragraph{\texorpdfstring{\texttt{Sync}}{Sync}}\label{sync}

The second of these traits is called
\href{http://doc.rust-lang.org/std/marker/trait.Sync.html}{\texttt{Sync}}.
When a type \texttt{T} implements \texttt{Sync}, it indicates that
something of this type has no possibility of introducing memory unsafety
when used from multiple threads concurrently through shared references.
This implies that types which don't have
\protect\hyperlink{sec--mutability}{interior mutability} are inherently
\texttt{Sync}, which includes simple primitive types (like \texttt{u8})
and aggregate types containing them.

For sharing references across threads, Rust provides a wrapper type
called \texttt{Arc\textless{}T\textgreater{}}.
\texttt{Arc\textless{}T\textgreater{}} implements \texttt{Send} and
\texttt{Sync} if and only if \texttt{T} implements both \texttt{Send}
and \texttt{Sync}. For example, an object of type
\texttt{Arc\textless{}RefCell\textless{}U\textgreater{}\textgreater{}}
cannot be transferred across threads because
\protect\hyperlink{refcellt}{\texttt{RefCell}} does not implement
\texttt{Sync}, consequently
\texttt{Arc\textless{}RefCell\textless{}U\textgreater{}\textgreater{}}
would not implement \texttt{Send}.

These two traits allow you to use the type system to make strong
guarantees about the properties of your code under concurrency. Before
we demonstrate why, we need to learn how to create a concurrent Rust
program in the first place!

\subsubsection{Threads}\label{threads}

Rust's standard library provides a library for threads, which allow you
to run Rust code in parallel. Here's a basic example of using
\texttt{std::thread}:

\begin{Shaded}
\begin{Highlighting}[]
\KeywordTok{use} \NormalTok{std::thread;}

\KeywordTok{fn} \NormalTok{main() \{}
    \NormalTok{thread::spawn(|| \{}
        \PreprocessorTok{println!}\NormalTok{(}\StringTok{"Hello from a thread!"}\NormalTok{);}
    \NormalTok{\});}
\NormalTok{\}}
\end{Highlighting}
\end{Shaded}

The \texttt{thread::spawn()} method accepts a
\protect\hyperlink{sec--closures}{closure}, which is executed in a new
thread. It returns a handle to the thread, that can be used to wait for
the child thread to finish and extract its result:

\begin{Shaded}
\begin{Highlighting}[]
\KeywordTok{use} \NormalTok{std::thread;}

\KeywordTok{fn} \NormalTok{main() \{}
    \KeywordTok{let} \NormalTok{handle = thread::spawn(|| \{}
        \StringTok{"Hello from a thread!"}
    \NormalTok{\});}

    \PreprocessorTok{println!}\NormalTok{(}\StringTok{"\{\}"}\NormalTok{, handle.join().unwrap());}
\NormalTok{\}}
\end{Highlighting}
\end{Shaded}

As closures can capture variables from their environment, we can also
try to bring some data into the other thread:

\begin{Shaded}
\begin{Highlighting}[]
\KeywordTok{use} \NormalTok{std::thread;}

\KeywordTok{fn} \NormalTok{main() \{}
    \KeywordTok{let} \NormalTok{x = }\DecValTok{1}\NormalTok{;}
    \NormalTok{thread::spawn(|| \{}
        \PreprocessorTok{println!}\NormalTok{(}\StringTok{"x is \{\}"}\NormalTok{, x);}
    \NormalTok{\});}
\NormalTok{\}}
\end{Highlighting}
\end{Shaded}

However, this gives us an error:

\begin{verbatim}
5:19: 7:6 error: closure may outlive the current function, but it
                 borrows `x`, which is owned by the current function
...
5:19: 7:6 help: to force the closure to take ownership of `x` (and any other reference
↳ d variables),
          use the `move` keyword, as shown:
      thread::spawn(move || {
          println!("x is {}", x);
      });
\end{verbatim}

This is because by default closures capture variables by reference, and
thus the closure only captures a \emph{reference to \texttt{x}}. This is
a problem, because the thread may outlive the scope of \texttt{x},
leading to a dangling pointer.

To fix this, we use a \texttt{move} closure as mentioned in the error
message. \texttt{move} closures are explained in depth
\protect\hyperlink{move-closures}{here}; basically they move variables
from their environment into themselves.

\begin{Shaded}
\begin{Highlighting}[]
\KeywordTok{use} \NormalTok{std::thread;}

\KeywordTok{fn} \NormalTok{main() \{}
    \KeywordTok{let} \NormalTok{x = }\DecValTok{1}\NormalTok{;}
    \NormalTok{thread::spawn(}\KeywordTok{move} \NormalTok{|| \{}
        \PreprocessorTok{println!}\NormalTok{(}\StringTok{"x is \{\}"}\NormalTok{, x);}
    \NormalTok{\});}
\NormalTok{\}}
\end{Highlighting}
\end{Shaded}

Many languages have the ability to execute threads, but it's wildly
unsafe. There are entire books about how to prevent errors that occur
from shared mutable state. Rust helps out with its type system here as
well, by preventing data races at compile time. Let's talk about how you
actually share things between threads.

\subsubsection{Safe Shared Mutable
State}\label{safe-shared-mutable-state}

Due to Rust's type system, we have a concept that sounds like a lie:
``safe shared mutable state.'' Many programmers agree that shared
mutable state is very, very bad.

Someone once said this:

\begin{quote}
Shared mutable state is the root of all evil. Most languages attempt to
deal with this problem through the `mutable' part, but Rust deals with
it by solving the `shared' part.
\end{quote}

The same \protect\hyperlink{sec--ownership}{ownership system} that helps
prevent using pointers incorrectly also helps rule out data races, one
of the worst kinds of concurrency bugs.

As an example, here is a Rust program that would have a data race in
many languages. It will not compile:

\begin{Shaded}
\begin{Highlighting}[]
\KeywordTok{use} \NormalTok{std::thread;}
\KeywordTok{use} \NormalTok{std::time::Duration;}

\KeywordTok{fn} \NormalTok{main() \{}
    \KeywordTok{let} \KeywordTok{mut} \NormalTok{data = }\PreprocessorTok{vec!}\NormalTok{[}\DecValTok{1}\NormalTok{, }\DecValTok{2}\NormalTok{, }\DecValTok{3}\NormalTok{];}

    \KeywordTok{for} \NormalTok{i }\KeywordTok{in} \DecValTok{0.}\NormalTok{.}\DecValTok{3} \NormalTok{\{}
        \NormalTok{thread::spawn(}\KeywordTok{move} \NormalTok{|| \{}
            \NormalTok{data[}\DecValTok{0}\NormalTok{] += i;}
        \NormalTok{\});}
    \NormalTok{\}}

    \NormalTok{thread::sleep(Duration::from_millis(}\DecValTok{50}\NormalTok{));}
\NormalTok{\}}
\end{Highlighting}
\end{Shaded}

This gives us an error:

\begin{verbatim}
8:17 error: capture of moved value: `data`
        data[0] += i;
        ^~~~
\end{verbatim}

Rust knows this wouldn't be safe! If we had a reference to \texttt{data}
in each thread, and the thread takes ownership of the reference, we'd
have three owners! \texttt{data} gets moved out of \texttt{main} in the
first call to \texttt{spawn()}, so subsequent calls in the loop cannot
use this variable.

So, we need some type that lets us have more than one owning reference
to a value. Usually, we'd use \texttt{Rc\textless{}T\textgreater{}} for
this, which is a reference counted type that provides shared ownership.
It has some runtime bookkeeping that keeps track of the number of
references to it, hence the ``reference count'' part of its name.

Calling \texttt{clone()} on an \texttt{Rc\textless{}T\textgreater{}}
will return a new owned reference and bump the internal reference count.
We create one of these for each thread:

\begin{Shaded}
\begin{Highlighting}[]
\KeywordTok{use} \NormalTok{std::thread;}
\KeywordTok{use} \NormalTok{std::time::Duration;}
\KeywordTok{use} \NormalTok{std::rc::Rc;}

\KeywordTok{fn} \NormalTok{main() \{}
    \KeywordTok{let} \KeywordTok{mut} \NormalTok{data = Rc::new(}\PreprocessorTok{vec!}\NormalTok{[}\DecValTok{1}\NormalTok{, }\DecValTok{2}\NormalTok{, }\DecValTok{3}\NormalTok{]);}

    \KeywordTok{for} \NormalTok{i }\KeywordTok{in} \DecValTok{0.}\NormalTok{.}\DecValTok{3} \NormalTok{\{}
        \CommentTok{// create a new owned reference}
        \KeywordTok{let} \NormalTok{data_ref = data.clone();}

        \CommentTok{// use it in a thread}
        \NormalTok{thread::spawn(}\KeywordTok{move} \NormalTok{|| \{}
            \NormalTok{data_ref[}\DecValTok{0}\NormalTok{] += i;}
        \NormalTok{\});}
    \NormalTok{\}}

    \NormalTok{thread::sleep(Duration::from_millis(}\DecValTok{50}\NormalTok{));}
\NormalTok{\}}
\end{Highlighting}
\end{Shaded}

This won't work, however, and will give us the error:

\begin{verbatim}
13:9: 13:22 error: the trait bound `alloc::rc::Rc<collections::vec::Vec<i32>> : core::
↳ marker::Send`
            is not satisfied
...
13:9: 13:22 note: `alloc::rc::Rc<collections::vec::Vec<i32>>`
            cannot be sent between threads safely
\end{verbatim}

As the error message mentions, \texttt{Rc} cannot be sent between
threads safely. This is because the internal reference count is not
maintained in a thread safe matter and can have a data race.

To solve this, we'll use \texttt{Arc\textless{}T\textgreater{}}, Rust's
standard atomic reference count type.

The Atomic part means \texttt{Arc\textless{}T\textgreater{}} can safely
be accessed from multiple threads. To do this the compiler guarantees
that mutations of the internal count use indivisible operations which
can't have data races.

In essence, \texttt{Arc\textless{}T\textgreater{}} is a type that lets
us share ownership of data \emph{across threads}.

\begin{Shaded}
\begin{Highlighting}[]
\KeywordTok{use} \NormalTok{std::thread;}
\KeywordTok{use} \NormalTok{std::sync::Arc;}
\KeywordTok{use} \NormalTok{std::time::Duration;}

\KeywordTok{fn} \NormalTok{main() \{}
    \KeywordTok{let} \KeywordTok{mut} \NormalTok{data = Arc::new(}\PreprocessorTok{vec!}\NormalTok{[}\DecValTok{1}\NormalTok{, }\DecValTok{2}\NormalTok{, }\DecValTok{3}\NormalTok{]);}

    \KeywordTok{for} \NormalTok{i }\KeywordTok{in} \DecValTok{0.}\NormalTok{.}\DecValTok{3} \NormalTok{\{}
        \KeywordTok{let} \NormalTok{data = data.clone();}
        \NormalTok{thread::spawn(}\KeywordTok{move} \NormalTok{|| \{}
            \NormalTok{data[}\DecValTok{0}\NormalTok{] += i;}
        \NormalTok{\});}
    \NormalTok{\}}

    \NormalTok{thread::sleep(Duration::from_millis(}\DecValTok{50}\NormalTok{));}
\NormalTok{\}}
\end{Highlighting}
\end{Shaded}

Similarly to last time, we use \texttt{clone()} to create a new owned
handle. This handle is then moved into the new thread.

And\ldots{} still gives us an error.

\begin{verbatim}
<anon>:11:24 error: cannot borrow immutable borrowed content as mutable
<anon>:11                    data[0] += i;
                             ^~~~
\end{verbatim}

\texttt{Arc\textless{}T\textgreater{}} by default has immutable
contents. It allows the \emph{sharing} of data between threads, but
shared mutable data is unsafe and when threads are involved can cause
data races!

Usually when we wish to make something in an immutable position mutable,
we use \texttt{Cell\textless{}T\textgreater{}} or
\texttt{RefCell\textless{}T\textgreater{}} which allow safe mutation via
runtime checks or otherwise (see also:
\protect\hyperlink{sec--choosing-your-guarantees}{Choosing Your
Guarantees}). However, similar to \texttt{Rc}, these are not thread
safe. If we try using these, we will get an error about these types not
being \texttt{Sync}, and the code will fail to compile.

It looks like we need some type that allows us to safely mutate a shared
value across threads, for example a type that can ensure only one thread
at a time is able to mutate the value inside it at any one time.

For that, we can use the \texttt{Mutex\textless{}T\textgreater{}} type!

Here's the working version:

\begin{Shaded}
\begin{Highlighting}[]
\KeywordTok{use} \NormalTok{std::sync::\{Arc, Mutex\};}
\KeywordTok{use} \NormalTok{std::thread;}
\KeywordTok{use} \NormalTok{std::time::Duration;}

\KeywordTok{fn} \NormalTok{main() \{}
    \KeywordTok{let} \NormalTok{data = Arc::new(Mutex::new(}\PreprocessorTok{vec!}\NormalTok{[}\DecValTok{1}\NormalTok{, }\DecValTok{2}\NormalTok{, }\DecValTok{3}\NormalTok{]));}

    \KeywordTok{for} \NormalTok{i }\KeywordTok{in} \DecValTok{0.}\NormalTok{.}\DecValTok{3} \NormalTok{\{}
        \KeywordTok{let} \NormalTok{data = data.clone();}
        \NormalTok{thread::spawn(}\KeywordTok{move} \NormalTok{|| \{}
            \KeywordTok{let} \KeywordTok{mut} \NormalTok{data = data.lock().unwrap();}
            \NormalTok{data[}\DecValTok{0}\NormalTok{] += i;}
        \NormalTok{\});}
    \NormalTok{\}}

    \NormalTok{thread::sleep(Duration::from_millis(}\DecValTok{50}\NormalTok{));}
\NormalTok{\}}
\end{Highlighting}
\end{Shaded}

Note that the value of \texttt{i} is bound (copied) to the closure and
not shared among the threads.

We're ``locking'' the mutex here. A mutex (short for ``mutual
exclusion''), as mentioned, only allows one thread at a time to access a
value. When we wish to access the value, we use \texttt{lock()} on it.
This will ``lock'' the mutex, and no other thread will be able to lock
it (and hence, do anything with the value) until we're done with it. If
a thread attempts to lock a mutex which is already locked, it will wait
until the other thread releases the lock.

The lock ``release'' here is implicit; when the result of the lock (in
this case, \texttt{data}) goes out of scope, the lock is automatically
released.

Note that
\href{http://doc.rust-lang.org/std/sync/struct.Mutex.html\#method.lock}{\texttt{lock}}
method of
\href{http://doc.rust-lang.org/std/sync/struct.Mutex.html}{\texttt{Mutex}}
has this signature:

\begin{Shaded}
\begin{Highlighting}[]
\KeywordTok{fn} \NormalTok{lock(&}\KeywordTok{self}\NormalTok{) -> LockResult<MutexGuard<T>>}
\end{Highlighting}
\end{Shaded}

and because \texttt{Send} is not implemented for
\texttt{MutexGuard\textless{}T\textgreater{}}, the guard cannot cross
thread boundaries, ensuring thread-locality of lock acquire and release.

Let's examine the body of the thread more closely:

\begin{Shaded}
\begin{Highlighting}[]
\NormalTok{thread::spawn(}\KeywordTok{move} \NormalTok{|| \{}
    \KeywordTok{let} \KeywordTok{mut} \NormalTok{data = data.lock().unwrap();}
    \NormalTok{data[}\DecValTok{0}\NormalTok{] += i;}
\NormalTok{\});}
\end{Highlighting}
\end{Shaded}

First, we call \texttt{lock()}, which acquires the mutex's lock. Because
this may fail, it returns a
\texttt{Result\textless{}T,\ E\textgreater{}}, and because this is just
an example, we \texttt{unwrap()} it to get a reference to the data. Real
code would have more robust error handling here. We're then free to
mutate it, since we have the lock.

Lastly, while the threads are running, we wait on a short timer. But
this is not ideal: we may have picked a reasonable amount of time to
wait but it's more likely we'll either be waiting longer than necessary
or not long enough, depending on just how much time the threads actually
take to finish computing when the program runs.

A more precise alternative to the timer would be to use one of the
mechanisms provided by the Rust standard library for synchronizing
threads with each other. Let's talk about one of them: channels.

\subsubsection{Channels}\label{channels}

Here's a version of our code that uses channels for synchronization,
rather than waiting for a specific time:

\begin{Shaded}
\begin{Highlighting}[]
\KeywordTok{use} \NormalTok{std::sync::\{Arc, Mutex\};}
\KeywordTok{use} \NormalTok{std::thread;}
\KeywordTok{use} \NormalTok{std::sync::mpsc;}

\KeywordTok{fn} \NormalTok{main() \{}
    \KeywordTok{let} \NormalTok{data = Arc::new(Mutex::new(}\DecValTok{0}\NormalTok{));}

    \CommentTok{// `tx` is the "transmitter" or "sender"}
    \CommentTok{// `rx` is the "receiver"}
    \KeywordTok{let} \NormalTok{(tx, rx) = mpsc::channel();}

    \KeywordTok{for} \NormalTok{_ }\KeywordTok{in} \DecValTok{0.}\NormalTok{.}\DecValTok{10} \NormalTok{\{}
        \KeywordTok{let} \NormalTok{(data, tx) = (data.clone(), tx.clone());}

        \NormalTok{thread::spawn(}\KeywordTok{move} \NormalTok{|| \{}
            \KeywordTok{let} \KeywordTok{mut} \NormalTok{data = data.lock().unwrap();}
            \NormalTok{*data += }\DecValTok{1}\NormalTok{;}

            \NormalTok{tx.send(()).unwrap();}
        \NormalTok{\});}
    \NormalTok{\}}

    \KeywordTok{for} \NormalTok{_ }\KeywordTok{in} \DecValTok{0.}\NormalTok{.}\DecValTok{10} \NormalTok{\{}
        \NormalTok{rx.recv().unwrap();}
    \NormalTok{\}}
\NormalTok{\}}
\end{Highlighting}
\end{Shaded}

We use the \texttt{mpsc::channel()} method to construct a new channel.
We \texttt{send} a simple \texttt{()} down the channel, and then wait
for ten of them to come back.

While this channel is sending a generic signal, we can send any data
that is \texttt{Send} over the channel!

\begin{Shaded}
\begin{Highlighting}[]
\KeywordTok{use} \NormalTok{std::thread;}
\KeywordTok{use} \NormalTok{std::sync::mpsc;}

\KeywordTok{fn} \NormalTok{main() \{}
    \KeywordTok{let} \NormalTok{(tx, rx) = mpsc::channel();}

    \KeywordTok{for} \NormalTok{i }\KeywordTok{in} \DecValTok{0.}\NormalTok{.}\DecValTok{10} \NormalTok{\{}
        \KeywordTok{let} \NormalTok{tx = tx.clone();}

        \NormalTok{thread::spawn(}\KeywordTok{move} \NormalTok{|| \{}
            \KeywordTok{let} \NormalTok{answer = i * i;}

            \NormalTok{tx.send(answer).unwrap();}
        \NormalTok{\});}
    \NormalTok{\}}

    \KeywordTok{for} \NormalTok{_ }\KeywordTok{in} \DecValTok{0.}\NormalTok{.}\DecValTok{10} \NormalTok{\{}
        \PreprocessorTok{println!}\NormalTok{(}\StringTok{"\{\}"}\NormalTok{, rx.recv().unwrap());}
    \NormalTok{\}}
\NormalTok{\}}
\end{Highlighting}
\end{Shaded}

Here we create 10 threads, asking each to calculate the square of a
number (\texttt{i} at the time of \texttt{spawn()}), and then
\texttt{send()} back the answer over the channel.

\subsubsection{Panics}\label{panics}

A \texttt{panic!} will crash the currently executing thread. You can use
Rust's threads as a simple isolation mechanism:

\begin{Shaded}
\begin{Highlighting}[]
\KeywordTok{use} \NormalTok{std::thread;}

\KeywordTok{let} \NormalTok{handle = thread::spawn(}\KeywordTok{move} \NormalTok{|| \{}
    \PreprocessorTok{panic!}\NormalTok{(}\StringTok{"oops!"}\NormalTok{);}
\NormalTok{\});}

\KeywordTok{let} \NormalTok{result = handle.join();}

\PreprocessorTok{assert!}\NormalTok{(result.is_err());}
\end{Highlighting}
\end{Shaded}

\texttt{Thread.join()} gives us a \texttt{Result} back, which allows us
to check if the thread has panicked or not.

\hypertarget{sec--error-handling}{\section{Error
Handling}\label{sec--error-handling}}

Like most programming languages, Rust encourages the programmer to
handle errors in a particular way. Generally speaking, error handling is
divided into two broad categories: exceptions and return values. Rust
opts for return values.

In this section, we intend to provide a comprehensive treatment of how
to deal with errors in Rust. More than that, we will attempt to
introduce error handling one piece at a time so that you'll come away
with a solid working knowledge of how everything fits together.

When done naïvely, error handling in Rust can be verbose and annoying.
This section will explore those stumbling blocks and demonstrate how to
use the standard library to make error handling concise and ergonomic.

\subsection{Table of Contents}\label{table-of-contents}

This section is very long, mostly because we start at the very beginning
with sum types and combinators, and try to motivate the way Rust does
error handling incrementally. As such, programmers with experience in
other expressive type systems may want to jump around.

\begin{itemize}
\tightlist
\item
  \protect\hyperlink{the-basics}{The Basics}

  \begin{itemize}
  \tightlist
  \item
    \protect\hyperlink{unwrapping-explained}{Unwrapping explained}
  \item
    \protect\hyperlink{the-option-type}{The \texttt{Option} type}

    \begin{itemize}
    \tightlist
    \item
      \protect\hyperlink{composing-optiont-values}{Composing
      \texttt{Option\textless{}T\textgreater{}} values}
    \end{itemize}
  \item
    \protect\hyperlink{the-result-type}{The \texttt{Result} type}

    \begin{itemize}
    \tightlist
    \item
      \protect\hyperlink{parsing-integers}{Parsing integers}
    \item
      \protect\hyperlink{the-result-type-alias-idiom}{The
      \texttt{Result} type alias idiom}
    \end{itemize}
  \item
    \protect\hyperlink{a-brief-interlude-unwrapping-isnt-evil}{A brief
    interlude: unwrapping isn't evil}
  \end{itemize}
\item
  \protect\hyperlink{working-with-multiple-error-types}{Working with
  multiple error types}

  \begin{itemize}
  \tightlist
  \item
    \protect\hyperlink{composing-option-and-result}{Composing
    \texttt{Option} and \texttt{Result}}
  \item
    \protect\hyperlink{the-limits-of-combinators}{The limits of
    combinators}
  \item
    \protect\hyperlink{early-returns}{Early returns}
  \item
    \protect\hyperlink{the-try-macro}{The \texttt{try!} macro}
  \item
    \protect\hyperlink{defining-your-own-error-type}{Defining your own
    error type}
  \end{itemize}
\item
  \protect\hyperlink{standard-library-traits-used-for-error-handling}{Standard
  library traits used for error handling}

  \begin{itemize}
  \tightlist
  \item
    \protect\hyperlink{the-error-trait}{The \texttt{Error} trait}
  \item
    \protect\hyperlink{the-from-trait}{The \texttt{From} trait}
  \item
    \protect\hyperlink{the-real-try-macro}{The real \texttt{try!} macro}
  \item
    \protect\hyperlink{composing-custom-error-types}{Composing custom
    error types}
  \item
    \protect\hyperlink{advice-for-library-writers}{Advice for library
    writers}
  \end{itemize}
\item
  \protect\hyperlink{case-study-a-program-to-read-population-data}{Case
  study: A program to read population data}

  \begin{itemize}
  \tightlist
  \item
    \protect\hyperlink{initial-setup}{Initial setup}
  \item
    \protect\hyperlink{argument-parsing}{Argument parsing}
  \item
    \protect\hyperlink{writing-the-logic}{Writing the logic}
  \item
    \protect\hyperlink{error-handling-with-boxerror}{Error handling with
    \texttt{Box\textless{}Error\textgreater{}}}
  \item
    \protect\hyperlink{reading-from-stdin}{Reading from stdin}
  \item
    \protect\hyperlink{error-handling-with-a-custom-type}{Error handling
    with a custom type}
  \item
    \protect\hyperlink{adding-functionality}{Adding functionality}
  \end{itemize}
\item
  \protect\hyperlink{the-short-story}{The short story}
\end{itemize}

\hypertarget{the-basics}{\subsection{The Basics}\label{the-basics}}

You can think of error handling as using \emph{case analysis} to
determine whether a computation was successful or not. As you will see,
the key to ergonomic error handling is reducing the amount of explicit
case analysis the programmer has to do while keeping code composable.

Keeping code composable is important, because without that requirement,
we could
\href{http://doc.rust-lang.org/std/macro.panic!.html}{\texttt{panic}}
whenever we come across something unexpected. (\texttt{panic} causes the
current task to unwind, and in most cases, the entire program aborts.)
Here's an example:

\begin{Shaded}
\begin{Highlighting}[]
\CommentTok{// Guess a number between 1 and 10.}
\CommentTok{// If it matches the number we had in mind, return true. Else, return false.}
\KeywordTok{fn} \NormalTok{guess(n: }\DataTypeTok{i32}\NormalTok{) -> }\DataTypeTok{bool} \NormalTok{\{}
    \KeywordTok{if} \NormalTok{n < }\DecValTok{1} \NormalTok{|| n > }\DecValTok{10} \NormalTok{\{}
        \PreprocessorTok{panic!}\NormalTok{(}\StringTok{"Invalid number: \{\}"}\NormalTok{, n);}
    \NormalTok{\}}
    \NormalTok{n == }\DecValTok{5}
\NormalTok{\}}

\KeywordTok{fn} \NormalTok{main() \{}
    \NormalTok{guess(}\DecValTok{11}\NormalTok{);}
\NormalTok{\}}
\end{Highlighting}
\end{Shaded}

If you try running this code, the program will crash with a message like
this:

\begin{verbatim}
thread '<main>' panicked at 'Invalid number: 11', src/bin/panic-simple.rs:5
\end{verbatim}

Here's another example that is slightly less contrived. A program that
accepts an integer as an argument, doubles it and prints it.

\protect\hypertarget{code-unwrap-double}{}{}

\begin{Shaded}
\begin{Highlighting}[]
\KeywordTok{use} \NormalTok{std::env;}

\KeywordTok{fn} \NormalTok{main() \{}
    \KeywordTok{let} \KeywordTok{mut} \NormalTok{argv = env::args();}
    \KeywordTok{let} \NormalTok{arg: }\DataTypeTok{String} \NormalTok{= argv.nth(}\DecValTok{1}\NormalTok{).unwrap(); }\CommentTok{// error 1}
    \KeywordTok{let} \NormalTok{n: }\DataTypeTok{i32} \NormalTok{= arg.parse().unwrap(); }\CommentTok{// error 2}
    \PreprocessorTok{println!}\NormalTok{(}\StringTok{"\{\}"}\NormalTok{, }\DecValTok{2} \NormalTok{* n);}
\NormalTok{\}}
\end{Highlighting}
\end{Shaded}

If you give this program zero arguments (error 1) or if the first
argument isn't an integer (error 2), the program will panic just like in
the first example.

You can think of this style of error handling as similar to a bull
running through a china shop. The bull will get to where it wants to go,
but it will trample everything in the process.

\hypertarget{unwrapping-explained}{\subsubsection{Unwrapping
explained}\label{unwrapping-explained}}

In the previous example, we claimed that the program would simply panic
if it reached one of the two error conditions, yet, the program does not
include an explicit call to \texttt{panic} like the first example. This
is because the panic is embedded in the calls to \texttt{unwrap}.

To ``unwrap'' something in Rust is to say, ``Give me the result of the
computation, and if there was an error, panic and stop the program.'' It
would be better if we showed the code for unwrapping because it is so
simple, but to do that, we will first need to explore the
\texttt{Option} and \texttt{Result} types. Both of these types have a
method called \texttt{unwrap} defined on them.

\hypertarget{the-option-type}{\paragraph{\texorpdfstring{The
\texttt{Option} type}{The Option type}}\label{the-option-type}}

The \texttt{Option} type is
\href{http://doc.rust-lang.org/std/option/enum.Option.html}{defined in
the standard library}:

\begin{Shaded}
\begin{Highlighting}[]
\KeywordTok{enum} \DataTypeTok{Option}\NormalTok{<T> \{}
    \ConstantTok{None}\NormalTok{,}
    \ConstantTok{Some}\NormalTok{(T),}
\NormalTok{\}}
\end{Highlighting}
\end{Shaded}

The \texttt{Option} type is a way to use Rust's type system to express
the \emph{possibility of absence}. Encoding the possibility of absence
into the type system is an important concept because it will cause the
compiler to force the programmer to handle that absence. Let's take a
look at an example that tries to find a character in a string:

\protect\hypertarget{code-option-ex-string-find}{}{}

\begin{Shaded}
\begin{Highlighting}[]
\CommentTok{// Searches `haystack` for the Unicode character `needle`. If one is found, the}
\CommentTok{// byte offset of the character is returned. Otherwise, `None` is returned.}
\KeywordTok{fn} \NormalTok{find(haystack: &}\DataTypeTok{str}\NormalTok{, needle: }\DataTypeTok{char}\NormalTok{) -> }\DataTypeTok{Option}\NormalTok{<}\DataTypeTok{usize}\NormalTok{> \{}
    \KeywordTok{for} \NormalTok{(offset, c) }\KeywordTok{in} \NormalTok{haystack.char_indices() \{}
        \KeywordTok{if} \NormalTok{c == needle \{}
            \KeywordTok{return} \ConstantTok{Some}\NormalTok{(offset);}
        \NormalTok{\}}
    \NormalTok{\}}
    \ConstantTok{None}
\NormalTok{\}}
\end{Highlighting}
\end{Shaded}

Notice that when this function finds a matching character, it doesn't
only return the \texttt{offset}. Instead, it returns
\texttt{Some(offset)}. \texttt{Some} is a variant or a \emph{value
constructor} for the \texttt{Option} type. You can think of it as a
function with the type
\texttt{fn\textless{}T\textgreater{}(value:\ T)\ -\textgreater{}\ Option\textless{}T\textgreater{}}.
Correspondingly, \texttt{None} is also a value constructor, except it
has no arguments. You can think of \texttt{None} as a function with the
type
\texttt{fn\textless{}T\textgreater{}()\ -\textgreater{}\ Option\textless{}T\textgreater{}}.

This might seem like much ado about nothing, but this is only half of
the story. The other half is \emph{using} the \texttt{find} function
we've written. Let's try to use it to find the extension in a file name.

\begin{Shaded}
\begin{Highlighting}[]
\KeywordTok{fn} \NormalTok{main() \{}
    \KeywordTok{let} \NormalTok{file_name = }\StringTok{"foobar.rs"}\NormalTok{;}
    \KeywordTok{match} \NormalTok{find(file_name, }\CharTok{'.'}\NormalTok{) \{}
        \ConstantTok{None} \NormalTok{=> }\PreprocessorTok{println!}\NormalTok{(}\StringTok{"No file extension found."}\NormalTok{),}
        \ConstantTok{Some}\NormalTok{(i) => }\PreprocessorTok{println!}\NormalTok{(}\StringTok{"File extension: \{\}"}\NormalTok{, &file_name[i+}\DecValTok{1.}\NormalTok{.]),}
    \NormalTok{\}}
\NormalTok{\}}
\end{Highlighting}
\end{Shaded}

This code uses
\href{http://doc.rust-lang.org/book/patterns.html}{pattern matching} to
do \emph{case analysis} on the
\texttt{Option\textless{}usize\textgreater{}} returned by the
\texttt{find} function. In fact, case analysis is the only way to get at
the value stored inside an \texttt{Option\textless{}T\textgreater{}}.
This means that you, as the programmer, must handle the case when an
\texttt{Option\textless{}T\textgreater{}} is \texttt{None} instead of
\texttt{Some(t)}.

But wait, what about \texttt{unwrap}, which we used
\protect\hyperlink{code-unwrap-double}{previously}? There was no case
analysis there! Instead, the case analysis was put inside the
\texttt{unwrap} method for you. You could define it yourself if you
want:

\protect\hypertarget{code-option-def-unwrap}{}{}

\begin{Shaded}
\begin{Highlighting}[]
\KeywordTok{enum} \DataTypeTok{Option}\NormalTok{<T> \{}
    \ConstantTok{None}\NormalTok{,}
    \ConstantTok{Some}\NormalTok{(T),}
\NormalTok{\}}

\KeywordTok{impl}\NormalTok{<T> }\DataTypeTok{Option}\NormalTok{<T> \{}
    \KeywordTok{fn} \NormalTok{unwrap(}\KeywordTok{self}\NormalTok{) -> T \{}
        \KeywordTok{match} \KeywordTok{self} \NormalTok{\{}
            \DataTypeTok{Option}\NormalTok{::}\ConstantTok{Some}\NormalTok{(val) => val,}
            \DataTypeTok{Option}\NormalTok{::}\ConstantTok{None} \NormalTok{=>}
              \PreprocessorTok{panic!}\NormalTok{(}\StringTok{"called `Option::unwrap()` on a `None` value"}\NormalTok{),}
        \NormalTok{\}}
    \NormalTok{\}}
\NormalTok{\}}
\end{Highlighting}
\end{Shaded}

The \texttt{unwrap} method \emph{abstracts away the case analysis}. This
is precisely the thing that makes \texttt{unwrap} ergonomic to use.
Unfortunately, that \texttt{panic!} means that \texttt{unwrap} is not
composable: it is the bull in the china shop.

\hypertarget{composing-optiont-values}{\paragraph{\texorpdfstring{Composing
\texttt{Option\textless{}T\textgreater{}}
values}{Composing Option\textless{}T\textgreater{} values}}\label{composing-optiont-values}}

In an \protect\hyperlink{code-option-ex-string-find}{example from
before}, we saw how to use \texttt{find} to discover the extension in a
file name. Of course, not all file names have a \texttt{.} in them, so
it's possible that the file name has no extension. This
\emph{possibility of absence} is encoded into the types using
\texttt{Option\textless{}T\textgreater{}}. In other words, the compiler
will force us to address the possibility that an extension does not
exist. In our case, we only print out a message saying as such.

Getting the extension of a file name is a pretty common operation, so it
makes sense to put it into a function:

\begin{Shaded}
\begin{Highlighting}[]
\CommentTok{// Returns the extension of the given file name, where the extension is defined}
\CommentTok{// as all characters following the first `.`.}
\CommentTok{// If `file_name` has no `.`, then `None` is returned.}
\KeywordTok{fn} \NormalTok{extension_explicit(file_name: &}\DataTypeTok{str}\NormalTok{) -> }\DataTypeTok{Option}\NormalTok{<&}\DataTypeTok{str}\NormalTok{> \{}
    \KeywordTok{match} \NormalTok{find(file_name, }\CharTok{'.'}\NormalTok{) \{}
        \ConstantTok{None} \NormalTok{=> }\ConstantTok{None}\NormalTok{,}
        \ConstantTok{Some}\NormalTok{(i) => }\ConstantTok{Some}\NormalTok{(&file_name[i+}\DecValTok{1.}\NormalTok{.]),}
    \NormalTok{\}}
\NormalTok{\}}
\end{Highlighting}
\end{Shaded}

(Pro-tip: don't use this code. Use the
\href{http://doc.rust-lang.org/std/path/struct.Path.html\#method.extension}{\texttt{extension}}
method in the standard library instead.)

The code stays simple, but the important thing to notice is that the
type of \texttt{find} forces us to consider the possibility of absence.
This is a good thing because it means the compiler won't let us
accidentally forget about the case where a file name doesn't have an
extension. On the other hand, doing explicit case analysis like we've
done in \texttt{extension\_explicit} every time can get a bit tiresome.

In fact, the case analysis in \texttt{extension\_explicit} follows a
very common pattern: \emph{map} a function on to the value inside of an
\texttt{Option\textless{}T\textgreater{}}, unless the option is
\texttt{None}, in which case, return \texttt{None}.

Rust has parametric polymorphism, so it is very easy to define a
combinator that abstracts this pattern:

\protect\hypertarget{code-option-map}{}{}

\begin{Shaded}
\begin{Highlighting}[]
\KeywordTok{fn} \NormalTok{map<F, T, A>(option: }\DataTypeTok{Option}\NormalTok{<T>, f: F) -> }\DataTypeTok{Option}\NormalTok{<A> }\KeywordTok{where} \NormalTok{F: }\BuiltInTok{FnOnce}\NormalTok{(T) -> A \{}
    \KeywordTok{match} \NormalTok{option \{}
        \ConstantTok{None} \NormalTok{=> }\ConstantTok{None}\NormalTok{,}
        \ConstantTok{Some}\NormalTok{(value) => }\ConstantTok{Some}\NormalTok{(f(value)),}
    \NormalTok{\}}
\NormalTok{\}}
\end{Highlighting}
\end{Shaded}

Indeed, \texttt{map} is
\href{http://doc.rust-lang.org/std/option/enum.Option.html\#method.map}{defined
as a method} on \texttt{Option\textless{}T\textgreater{}} in the
standard library. As a method, it has a slightly different signature:
methods take \texttt{self}, \texttt{\&self}, or \texttt{\&mut\ self} as
their first argument.

Armed with our new combinator, we can rewrite our
\texttt{extension\_explicit} method to get rid of the case analysis:

\begin{Shaded}
\begin{Highlighting}[]
\CommentTok{// Returns the extension of the given file name, where the extension is defined}
\CommentTok{// as all characters following the first `.`.}
\CommentTok{// If `file_name` has no `.`, then `None` is returned.}
\KeywordTok{fn} \NormalTok{extension(file_name: &}\DataTypeTok{str}\NormalTok{) -> }\DataTypeTok{Option}\NormalTok{<&}\DataTypeTok{str}\NormalTok{> \{}
    \NormalTok{find(file_name, }\CharTok{'.'}\NormalTok{).map(|i| &file_name[i+}\DecValTok{1.}\NormalTok{.])}
\NormalTok{\}}
\end{Highlighting}
\end{Shaded}

One other pattern we commonly find is assigning a default value to the
case when an \texttt{Option} value is \texttt{None}. For example, maybe
your program assumes that the extension of a file is \texttt{rs} even if
none is present. As you might imagine, the case analysis for this is not
specific to file extensions - it can work with any
\texttt{Option\textless{}T\textgreater{}}:

\begin{Shaded}
\begin{Highlighting}[]
\KeywordTok{fn} \NormalTok{unwrap_or<T>(option: }\DataTypeTok{Option}\NormalTok{<T>, default: T) -> T \{}
    \KeywordTok{match} \NormalTok{option \{}
        \ConstantTok{None} \NormalTok{=> default,}
        \ConstantTok{Some}\NormalTok{(value) => value,}
    \NormalTok{\}}
\NormalTok{\}}
\end{Highlighting}
\end{Shaded}

Like with \texttt{map} above, the standard library implementation is a
method instead of a free function.

The trick here is that the default value must have the same type as the
value that might be inside the
\texttt{Option\textless{}T\textgreater{}}. Using it is dead simple in
our case:

\begin{Shaded}
\begin{Highlighting}[]
\KeywordTok{fn} \NormalTok{main() \{}
    \PreprocessorTok{assert_eq!}\NormalTok{(extension(}\StringTok{"foobar.csv"}\NormalTok{).unwrap_or(}\StringTok{"rs"}\NormalTok{), }\StringTok{"csv"}\NormalTok{);}
    \PreprocessorTok{assert_eq!}\NormalTok{(extension(}\StringTok{"foobar"}\NormalTok{).unwrap_or(}\StringTok{"rs"}\NormalTok{), }\StringTok{"rs"}\NormalTok{);}
\NormalTok{\}}
\end{Highlighting}
\end{Shaded}

(Note that \texttt{unwrap\_or} is
\href{http://doc.rust-lang.org/std/option/enum.Option.html\#method.unwrap_or}{defined
as a method} on \texttt{Option\textless{}T\textgreater{}} in the
standard library, so we use that here instead of the free-standing
function we defined above. Don't forget to check out the more general
\href{http://doc.rust-lang.org/std/option/enum.Option.html\#method.unwrap_or_else}{\texttt{unwrap\_or\_else}}
method.)

There is one more combinator that we think is worth paying special
attention to: \texttt{and\_then}. It makes it easy to compose distinct
computations that admit the \emph{possibility of absence}. For example,
much of the code in this section is about finding an extension given a
file name. In order to do this, you first need the file name which is
typically extracted from a file \emph{path}. While most file paths have
a file name, not \emph{all} of them do. For example, \texttt{.},
\texttt{..} or \texttt{/}.

So, we are tasked with the challenge of finding an extension given a
file \emph{path}. Let's start with explicit case analysis:

\begin{Shaded}
\begin{Highlighting}[]
\KeywordTok{fn} \NormalTok{file_path_ext_explicit(file_path: &}\DataTypeTok{str}\NormalTok{) -> }\DataTypeTok{Option}\NormalTok{<&}\DataTypeTok{str}\NormalTok{> \{}
    \KeywordTok{match} \NormalTok{file_name(file_path) \{}
        \ConstantTok{None} \NormalTok{=> }\ConstantTok{None}\NormalTok{,}
        \ConstantTok{Some}\NormalTok{(name) => }\KeywordTok{match} \NormalTok{extension(name) \{}
            \ConstantTok{None} \NormalTok{=> }\ConstantTok{None}\NormalTok{,}
            \ConstantTok{Some}\NormalTok{(ext) => }\ConstantTok{Some}\NormalTok{(ext),}
        \NormalTok{\}}
    \NormalTok{\}}
\NormalTok{\}}

\KeywordTok{fn} \NormalTok{file_name(file_path: &}\DataTypeTok{str}\NormalTok{) -> }\DataTypeTok{Option}\NormalTok{<&}\DataTypeTok{str}\NormalTok{> \{}
  \CommentTok{// implementation elided}
  \PreprocessorTok{unimplemented!}\NormalTok{()}
\NormalTok{\}}
\end{Highlighting}
\end{Shaded}

You might think that we could use the \texttt{map} combinator to reduce
the case analysis, but its type doesn't quite fit\ldots{}

\begin{Shaded}
\begin{Highlighting}[]
\KeywordTok{fn} \NormalTok{file_path_ext(file_path: &}\DataTypeTok{str}\NormalTok{) -> }\DataTypeTok{Option}\NormalTok{<&}\DataTypeTok{str}\NormalTok{> \{}
    \NormalTok{file_name(file_path).map(|x| extension(x)) }\CommentTok{//Compilation error}
\NormalTok{\}}
\end{Highlighting}
\end{Shaded}

The \texttt{map} function here wraps the value returned by the
\texttt{extension} function inside an
\texttt{Option\textless{}\_\textgreater{}} and since the
\texttt{extension} function itself returns an
\texttt{Option\textless{}\&str\textgreater{}} the expression
\texttt{file\_name(file\_path).map(\textbar{}x\textbar{}\ extension(x))}
actually returns an
\texttt{Option\textless{}Option\textless{}\&str\textgreater{}\textgreater{}}.

But since \texttt{file\_path\_ext} just returns
\texttt{Option\textless{}\&str\textgreater{}} (and not
\texttt{Option\textless{}Option\textless{}\&str\textgreater{}\textgreater{}})
we get a compilation error.

The result of the function taken by map as input is \emph{always}
\protect\hyperlink{code-option-map}{rewrapped with \texttt{Some}}.
Instead, we need something like \texttt{map}, but which allows the
caller to return a \texttt{Option\textless{}\_\textgreater{}} directly
without wrapping it in another
\texttt{Option\textless{}\_\textgreater{}}.

Its generic implementation is even simpler than \texttt{map}:

\begin{Shaded}
\begin{Highlighting}[]
\KeywordTok{fn} \NormalTok{and_then<F, T, A>(option: }\DataTypeTok{Option}\NormalTok{<T>, f: F) -> }\DataTypeTok{Option}\NormalTok{<A>}
        \KeywordTok{where} \NormalTok{F: }\BuiltInTok{FnOnce}\NormalTok{(T) -> }\DataTypeTok{Option}\NormalTok{<A> \{}
    \KeywordTok{match} \NormalTok{option \{}
        \ConstantTok{None} \NormalTok{=> }\ConstantTok{None}\NormalTok{,}
        \ConstantTok{Some}\NormalTok{(value) => f(value),}
    \NormalTok{\}}
\NormalTok{\}}
\end{Highlighting}
\end{Shaded}

Now we can rewrite our \texttt{file\_path\_ext} function without
explicit case analysis:

\begin{Shaded}
\begin{Highlighting}[]
\KeywordTok{fn} \NormalTok{file_path_ext(file_path: &}\DataTypeTok{str}\NormalTok{) -> }\DataTypeTok{Option}\NormalTok{<&}\DataTypeTok{str}\NormalTok{> \{}
    \NormalTok{file_name(file_path).and_then(extension)}
\NormalTok{\}}
\end{Highlighting}
\end{Shaded}

Side note: Since \texttt{and\_then} essentially works like \texttt{map}
but returns an \texttt{Option\textless{}\_\textgreater{}} instead of an
\texttt{Option\textless{}Option\textless{}\_\textgreater{}\textgreater{}}
it is known as \texttt{flatmap} in some other languages.

The \texttt{Option} type has many other combinators
\href{http://doc.rust-lang.org/std/option/enum.Option.html}{defined in
the standard library}. It is a good idea to skim this list and
familiarize yourself with what's available---they can often reduce case
analysis for you. Familiarizing yourself with these combinators will pay
dividends because many of them are also defined (with similar semantics)
for \texttt{Result}, which we will talk about next.

Combinators make using types like \texttt{Option} ergonomic because they
reduce explicit case analysis. They are also composable because they
permit the caller to handle the possibility of absence in their own way.
Methods like \texttt{unwrap} remove choices because they will panic if
\texttt{Option\textless{}T\textgreater{}} is \texttt{None}.

\hypertarget{the-result-type}{\subsubsection{\texorpdfstring{The
\texttt{Result} type}{The Result type}}\label{the-result-type}}

The \texttt{Result} type is also
\href{http://doc.rust-lang.org/std/result/}{defined in the standard
library}:

\protect\hypertarget{code-result-def}{}{}

\begin{Shaded}
\begin{Highlighting}[]
\KeywordTok{enum} \DataTypeTok{Result}\NormalTok{<T, E> \{}
    \ConstantTok{Ok}\NormalTok{(T),}
    \ConstantTok{Err}\NormalTok{(E),}
\NormalTok{\}}
\end{Highlighting}
\end{Shaded}

The \texttt{Result} type is a richer version of \texttt{Option}. Instead
of expressing the possibility of \emph{absence} like \texttt{Option}
does, \texttt{Result} expresses the possibility of \emph{error}.
Usually, the \emph{error} is used to explain why the execution of some
computation failed. This is a strictly more general form of
\texttt{Option}. Consider the following type alias, which is
semantically equivalent to the real
\texttt{Option\textless{}T\textgreater{}} in every way:

\begin{Shaded}
\begin{Highlighting}[]
\KeywordTok{type} \NormalTok{Option<T> = }\DataTypeTok{Result}\NormalTok{<T, ()>;}
\end{Highlighting}
\end{Shaded}

This fixes the second type parameter of \texttt{Result} to always be
\texttt{()} (pronounced ``unit'' or ``empty tuple''). Exactly one value
inhabits the \texttt{()} type: \texttt{()}. (Yup, the type and value
level terms have the same notation!)

The \texttt{Result} type is a way of representing one of two possible
outcomes in a computation. By convention, one outcome is meant to be
expected or ``\texttt{Ok}'' while the other outcome is meant to be
unexpected or ``\texttt{Err}''.

Just like \texttt{Option}, the \texttt{Result} type also has an
\href{http://doc.rust-lang.org/std/result/enum.Result.html\#method.unwrap}{\texttt{unwrap}
method defined} in the standard library. Let's define it:

\begin{Shaded}
\begin{Highlighting}[]
\KeywordTok{impl}\NormalTok{<T, E: ::std::fmt::}\BuiltInTok{Debug}\NormalTok{> }\DataTypeTok{Result}\NormalTok{<T, E> \{}
    \KeywordTok{fn} \NormalTok{unwrap(}\KeywordTok{self}\NormalTok{) -> T \{}
        \KeywordTok{match} \KeywordTok{self} \NormalTok{\{}
            \DataTypeTok{Result}\NormalTok{::}\ConstantTok{Ok}\NormalTok{(val) => val,}
            \DataTypeTok{Result}\NormalTok{::}\ConstantTok{Err}\NormalTok{(err) =>}
              \PreprocessorTok{panic!}\NormalTok{(}\StringTok{"called `Result::unwrap()` on an `Err` value: \{:?\}"}\NormalTok{, err),}
        \NormalTok{\}}
    \NormalTok{\}}
\NormalTok{\}}
\end{Highlighting}
\end{Shaded}

This is effectively the same as our
\protect\hyperlink{code-option-def-unwrap}{definition for
\texttt{Option::unwrap}}, except it includes the error value in the
\texttt{panic!} message. This makes debugging easier, but it also
requires us to add a
\href{http://doc.rust-lang.org/std/fmt/trait.Debug.html}{\texttt{Debug}}
constraint on the \texttt{E} type parameter (which represents our error
type). Since the vast majority of types should satisfy the
\texttt{Debug} constraint, this tends to work out in practice.
(\texttt{Debug} on a type simply means that there's a reasonable way to
print a human readable description of values with that type.)

OK, let's move on to an example.

\hypertarget{parsing-integers}{\paragraph{Parsing
integers}\label{parsing-integers}}

The Rust standard library makes converting strings to integers dead
simple. It's so easy in fact, that it is very tempting to write
something like the following:

\begin{Shaded}
\begin{Highlighting}[]
\KeywordTok{fn} \NormalTok{double_number(number_str: &}\DataTypeTok{str}\NormalTok{) -> }\DataTypeTok{i32} \NormalTok{\{}
    \DecValTok{2} \NormalTok{* number_str.parse::<}\DataTypeTok{i32}\NormalTok{>().unwrap()}
\NormalTok{\}}

\KeywordTok{fn} \NormalTok{main() \{}
    \KeywordTok{let} \NormalTok{n: }\DataTypeTok{i32} \NormalTok{= double_number(}\StringTok{"10"}\NormalTok{);}
    \PreprocessorTok{assert_eq!}\NormalTok{(n, }\DecValTok{20}\NormalTok{);}
\NormalTok{\}}
\end{Highlighting}
\end{Shaded}

At this point, you should be skeptical of calling \texttt{unwrap}. For
example, if the string doesn't parse as a number, you'll get a panic:

\begin{verbatim}
thread '<main>' panicked at 'called `Result::unwrap()` on an `Err` value: ParseIntErro
↳ r { kind: InvalidDigit }', /home/rustbuild/src/rust-buildbot/slave/beta-dist-rustc-l
↳ inux/build/src/libcore/result.rs:729
\end{verbatim}

This is rather unsightly, and if this happened inside a library you're
using, you might be understandably annoyed. Instead, we should try to
handle the error in our function and let the caller decide what to do.
This means changing the return type of \texttt{double\_number}. But to
what? Well, that requires looking at the signature of the
\href{http://doc.rust-lang.org/std/primitive.str.html\#method.parse}{\texttt{parse}
method} in the standard library:

\begin{Shaded}
\begin{Highlighting}[]
\KeywordTok{impl} \DataTypeTok{str} \NormalTok{\{}
    \KeywordTok{fn} \NormalTok{parse<F: FromStr>(&}\KeywordTok{self}\NormalTok{) -> }\DataTypeTok{Result}\NormalTok{<F, F::}\ConstantTok{Err}\NormalTok{>;}
\NormalTok{\}}
\end{Highlighting}
\end{Shaded}

Hmm. So we at least know that we need to use a \texttt{Result}.
Certainly, it's possible that this could have returned an
\texttt{Option}. After all, a string either parses as a number or it
doesn't, right? That's certainly a reasonable way to go, but the
implementation internally distinguishes \emph{why} the string didn't
parse as an integer. (Whether it's an empty string, an invalid digit,
too big or too small.) Therefore, using a \texttt{Result} makes sense
because we want to provide more information than simply ``absence.'' We
want to say \emph{why} the parsing failed. You should try to emulate
this line of reasoning when faced with a choice between \texttt{Option}
and \texttt{Result}. If you can provide detailed error information, then
you probably should. (We'll see more on this later.)

OK, but how do we write our return type? The \texttt{parse} method as
defined above is generic over all the different number types defined in
the standard library. We could (and probably should) also make our
function generic, but let's favor explicitness for the moment. We only
care about \texttt{i32}, so we need to
\href{http://doc.rust-lang.org/std/primitive.i32.html}{find its
implementation of \texttt{FromStr}} (do a \texttt{CTRL-F} in your
browser for ``FromStr'') and look at its
\href{http://doc.rust-lang.org/book/associated-types.html}{associated
type} \texttt{Err}. We did this so we can find the concrete error type.
In this case, it's
\href{http://doc.rust-lang.org/std/num/struct.ParseIntError.html}{\texttt{std::num::ParseIntError}}.
Finally, we can rewrite our function:

\begin{Shaded}
\begin{Highlighting}[]
\KeywordTok{use} \NormalTok{std::num::ParseIntError;}

\KeywordTok{fn} \NormalTok{double_number(number_str: &}\DataTypeTok{str}\NormalTok{) -> }\DataTypeTok{Result}\NormalTok{<}\DataTypeTok{i32}\NormalTok{, ParseIntError> \{}
    \KeywordTok{match} \NormalTok{number_str.parse::<}\DataTypeTok{i32}\NormalTok{>() \{}
        \ConstantTok{Ok}\NormalTok{(n) => }\ConstantTok{Ok}\NormalTok{(}\DecValTok{2} \NormalTok{* n),}
        \ConstantTok{Err}\NormalTok{(err) => }\ConstantTok{Err}\NormalTok{(err),}
    \NormalTok{\}}
\NormalTok{\}}

\KeywordTok{fn} \NormalTok{main() \{}
    \KeywordTok{match} \NormalTok{double_number(}\StringTok{"10"}\NormalTok{) \{}
        \ConstantTok{Ok}\NormalTok{(n) => }\PreprocessorTok{assert_eq!}\NormalTok{(n, }\DecValTok{20}\NormalTok{),}
        \ConstantTok{Err}\NormalTok{(err) => }\PreprocessorTok{println!}\NormalTok{(}\StringTok{"Error: \{:?\}"}\NormalTok{, err),}
    \NormalTok{\}}
\NormalTok{\}}
\end{Highlighting}
\end{Shaded}

This is a little better, but now we've written a lot more code! The case
analysis has once again bitten us.

Combinators to the rescue! Just like \texttt{Option}, \texttt{Result}
has lots of combinators defined as methods. There is a large
intersection of common combinators between \texttt{Result} and
\texttt{Option}. In particular, \texttt{map} is part of that
intersection:

\begin{Shaded}
\begin{Highlighting}[]
\KeywordTok{use} \NormalTok{std::num::ParseIntError;}

\KeywordTok{fn} \NormalTok{double_number(number_str: &}\DataTypeTok{str}\NormalTok{) -> }\DataTypeTok{Result}\NormalTok{<}\DataTypeTok{i32}\NormalTok{, ParseIntError> \{}
    \NormalTok{number_str.parse::<}\DataTypeTok{i32}\NormalTok{>().map(|n| }\DecValTok{2} \NormalTok{* n)}
\NormalTok{\}}

\KeywordTok{fn} \NormalTok{main() \{}
    \KeywordTok{match} \NormalTok{double_number(}\StringTok{"10"}\NormalTok{) \{}
        \ConstantTok{Ok}\NormalTok{(n) => }\PreprocessorTok{assert_eq!}\NormalTok{(n, }\DecValTok{20}\NormalTok{),}
        \ConstantTok{Err}\NormalTok{(err) => }\PreprocessorTok{println!}\NormalTok{(}\StringTok{"Error: \{:?\}"}\NormalTok{, err),}
    \NormalTok{\}}
\NormalTok{\}}
\end{Highlighting}
\end{Shaded}

The usual suspects are all there for \texttt{Result}, including
\href{http://doc.rust-lang.org/std/result/enum.Result.html\#method.unwrap_or}{\texttt{unwrap\_or}}
and
\href{http://doc.rust-lang.org/std/result/enum.Result.html\#method.and_then}{\texttt{and\_then}}.
Additionally, since \texttt{Result} has a second type parameter, there
are combinators that affect only the error type, such as
\href{http://doc.rust-lang.org/std/result/enum.Result.html\#method.map_err}{\texttt{map\_err}}
(instead of \texttt{map}) and
\href{http://doc.rust-lang.org/std/result/enum.Result.html\#method.or_else}{\texttt{or\_else}}
(instead of \texttt{and\_then}).

\hypertarget{the-result-type-alias-idiom}{\paragraph{\texorpdfstring{The
\texttt{Result} type alias
idiom}{The Result type alias idiom}}\label{the-result-type-alias-idiom}}

In the standard library, you may frequently see types like
\texttt{Result\textless{}i32\textgreater{}}. But wait,
\protect\hyperlink{code-result-def}{we defined \texttt{Result}} to have
two type parameters. How can we get away with only specifying one? The
key is to define a \texttt{Result} type alias that \emph{fixes} one of
the type parameters to a particular type. Usually the fixed type is the
error type. For example, our previous example parsing integers could be
rewritten like this:

\begin{Shaded}
\begin{Highlighting}[]
\KeywordTok{use} \NormalTok{std::num::ParseIntError;}
\KeywordTok{use} \NormalTok{std::result;}

\KeywordTok{type} \NormalTok{Result<T> = result::}\DataTypeTok{Result}\NormalTok{<T, ParseIntError>;}

\KeywordTok{fn} \NormalTok{double_number(number_str: &}\DataTypeTok{str}\NormalTok{) -> }\DataTypeTok{Result}\NormalTok{<}\DataTypeTok{i32}\NormalTok{> \{}
    \PreprocessorTok{unimplemented!}\NormalTok{();}
\NormalTok{\}}
\end{Highlighting}
\end{Shaded}

Why would we do this? Well, if we have a lot of functions that could
return \texttt{ParseIntError}, then it's much more convenient to define
an alias that always uses \texttt{ParseIntError} so that we don't have
to write it out all the time.

The most prominent place this idiom is used in the standard library is
with
\href{http://doc.rust-lang.org/std/io/type.Result.html}{\texttt{io::Result}}.
Typically, one writes \texttt{io::Result\textless{}T\textgreater{}},
which makes it clear that you're using the \texttt{io} module's type
alias instead of the plain definition from \texttt{std::result}. (This
idiom is also used for
\href{http://doc.rust-lang.org/std/fmt/type.Result.html}{\texttt{fmt::Result}}.)

\hypertarget{a-brief-interlude-unwrapping-isnt-evil}{\subsubsection{A
brief interlude: unwrapping isn't
evil}\label{a-brief-interlude-unwrapping-isnt-evil}}

If you've been following along, you might have noticed that I've taken a
pretty hard line against calling methods like \texttt{unwrap} that could
\texttt{panic} and abort your program. \emph{Generally speaking}, this
is good advice.

However, \texttt{unwrap} can still be used judiciously. What exactly
justifies use of \texttt{unwrap} is somewhat of a grey area and
reasonable people can disagree. I'll summarize some of my
\emph{opinions} on the matter.

\begin{itemize}
\tightlist
\item
  \textbf{In examples and quick `n' dirty code.} Sometimes you're
  writing examples or a quick program, and error handling simply isn't
  important. Beating the convenience of \texttt{unwrap} can be hard in
  such scenarios, so it is very appealing.
\item
  \textbf{When panicking indicates a bug in the program.} When the
  invariants of your code should prevent a certain case from happening
  (like, say, popping from an empty stack), then panicking can be
  permissible. This is because it exposes a bug in your program. This
  can be explicit, like from an \texttt{assert!} failing, or it could be
  because your index into an array was out of bounds.
\end{itemize}

This is probably not an exhaustive list. Moreover, when using an
\texttt{Option}, it is often better to use its
\href{http://doc.rust-lang.org/std/option/enum.Option.html\#method.expect}{\texttt{expect}}
method. \texttt{expect} does exactly the same thing as \texttt{unwrap},
except it prints a message you give to \texttt{expect}. This makes the
resulting panic a bit nicer to deal with, since it will show your
message instead of ``called unwrap on a \texttt{None} value.''

My advice boils down to this: use good judgment. There's a reason why
the words ``never do X'' or ``Y is considered harmful'' don't appear in
my writing. There are trade offs to all things, and it is up to you as
the programmer to determine what is acceptable for your use cases. My
goal is only to help you evaluate trade offs as accurately as possible.

Now that we've covered the basics of error handling in Rust, and
explained unwrapping, let's start exploring more of the standard
library.

\hypertarget{working-with-multiple-error-types}{\subsection{Working with
multiple error types}\label{working-with-multiple-error-types}}

Thus far, we've looked at error handling where everything was either an
\texttt{Option\textless{}T\textgreater{}} or a
\texttt{Result\textless{}T,\ SomeError\textgreater{}}. But what happens
when you have both an \texttt{Option} and a \texttt{Result}? Or what if
you have a \texttt{Result\textless{}T,\ Error1\textgreater{}} and a
\texttt{Result\textless{}T,\ Error2\textgreater{}}? Handling
\emph{composition of distinct error types} is the next challenge in
front of us, and it will be the major theme throughout the rest of this
section.

\hypertarget{composing-option-and-result}{\subsubsection{\texorpdfstring{Composing
\texttt{Option} and
\texttt{Result}}{Composing Option and Result}}\label{composing-option-and-result}}

So far, I've talked about combinators defined for \texttt{Option} and
combinators defined for \texttt{Result}. We can use these combinators to
compose results of different computations without doing explicit case
analysis.

Of course, in real code, things aren't always as clean. Sometimes you
have a mix of \texttt{Option} and \texttt{Result} types. Must we resort
to explicit case analysis, or can we continue using combinators?

For now, let's revisit one of the first examples in this section:

\begin{Shaded}
\begin{Highlighting}[]
\KeywordTok{use} \NormalTok{std::env;}

\KeywordTok{fn} \NormalTok{main() \{}
    \KeywordTok{let} \KeywordTok{mut} \NormalTok{argv = env::args();}
    \KeywordTok{let} \NormalTok{arg: }\DataTypeTok{String} \NormalTok{= argv.nth(}\DecValTok{1}\NormalTok{).unwrap(); }\CommentTok{// error 1}
    \KeywordTok{let} \NormalTok{n: }\DataTypeTok{i32} \NormalTok{= arg.parse().unwrap(); }\CommentTok{// error 2}
    \PreprocessorTok{println!}\NormalTok{(}\StringTok{"\{\}"}\NormalTok{, }\DecValTok{2} \NormalTok{* n);}
\NormalTok{\}}
\end{Highlighting}
\end{Shaded}

Given our new found knowledge of \texttt{Option}, \texttt{Result} and
their various combinators, we should try to rewrite this so that errors
are handled properly and the program doesn't panic if there's an error.

The tricky aspect here is that \texttt{argv.nth(1)} produces an
\texttt{Option} while \texttt{arg.parse()} produces a \texttt{Result}.
These aren't directly composable. When faced with both an
\texttt{Option} and a \texttt{Result}, the solution is \emph{usually} to
convert the \texttt{Option} to a \texttt{Result}. In our case, the
absence of a command line parameter (from \texttt{env::args()}) means
the user didn't invoke the program correctly. We could use a
\texttt{String} to describe the error. Let's try:

\protect\hypertarget{code-error-double-string}{}{}

\begin{Shaded}
\begin{Highlighting}[]
\KeywordTok{use} \NormalTok{std::env;}

\KeywordTok{fn} \NormalTok{double_arg(}\KeywordTok{mut} \NormalTok{argv: env::Args) -> }\DataTypeTok{Result}\NormalTok{<}\DataTypeTok{i32}\NormalTok{, }\DataTypeTok{String}\NormalTok{> \{}
    \NormalTok{argv.nth(}\DecValTok{1}\NormalTok{)}
        \NormalTok{.ok_or(}\StringTok{"Please give at least one argument"}\NormalTok{.to_owned())}
        \NormalTok{.and_then(|arg| arg.parse::<}\DataTypeTok{i32}\NormalTok{>().map_err(|err| err.to_string()))}
        \NormalTok{.map(|n| }\DecValTok{2} \NormalTok{* n)}
\NormalTok{\}}

\KeywordTok{fn} \NormalTok{main() \{}
    \KeywordTok{match} \NormalTok{double_arg(env::args()) \{}
        \ConstantTok{Ok}\NormalTok{(n) => }\PreprocessorTok{println!}\NormalTok{(}\StringTok{"\{\}"}\NormalTok{, n),}
        \ConstantTok{Err}\NormalTok{(err) => }\PreprocessorTok{println!}\NormalTok{(}\StringTok{"Error: \{\}"}\NormalTok{, err),}
    \NormalTok{\}}
\NormalTok{\}}
\end{Highlighting}
\end{Shaded}

There are a couple new things in this example. The first is the use of
the
\href{http://doc.rust-lang.org/std/option/enum.Option.html\#method.ok_or}{\texttt{Option::ok\_or}}
combinator. This is one way to convert an \texttt{Option} into a
\texttt{Result}. The conversion requires you to specify what error to
use if \texttt{Option} is \texttt{None}. Like the other combinators
we've seen, its definition is very simple:

\begin{Shaded}
\begin{Highlighting}[]
\KeywordTok{fn} \NormalTok{ok_or<T, E>(option: }\DataTypeTok{Option}\NormalTok{<T>, err: E) -> }\DataTypeTok{Result}\NormalTok{<T, E> \{}
    \KeywordTok{match} \NormalTok{option \{}
        \ConstantTok{Some}\NormalTok{(val) => }\ConstantTok{Ok}\NormalTok{(val),}
        \ConstantTok{None} \NormalTok{=> }\ConstantTok{Err}\NormalTok{(err),}
    \NormalTok{\}}
\NormalTok{\}}
\end{Highlighting}
\end{Shaded}

The other new combinator used here is
\href{http://doc.rust-lang.org/std/result/enum.Result.html\#method.map_err}{\texttt{Result::map\_err}}.
This is like \texttt{Result::map}, except it maps a function on to the
\emph{error} portion of a \texttt{Result} value. If the \texttt{Result}
is an \texttt{Ok(...)} value, then it is returned unmodified.

We use \texttt{map\_err} here because it is necessary for the error
types to remain the same (because of our use of \texttt{and\_then}).
Since we chose to convert the
\texttt{Option\textless{}String\textgreater{}} (from
\texttt{argv.nth(1)}) to a
\texttt{Result\textless{}String,\ String\textgreater{}}, we must also
convert the \texttt{ParseIntError} from \texttt{arg.parse()} to a
\texttt{String}.

\hypertarget{the-limits-of-combinators}{\subsubsection{The limits of
combinators}\label{the-limits-of-combinators}}

Doing IO and parsing input is a very common task, and it's one that I
personally have done a lot of in Rust. Therefore, we will use (and
continue to use) IO and various parsing routines to exemplify error
handling.

Let's start simple. We are tasked with opening a file, reading all of
its contents and converting its contents to a number. Then we multiply
it by \texttt{2} and print the output.

Although I've tried to convince you not to use \texttt{unwrap}, it can
be useful to first write your code using \texttt{unwrap}. It allows you
to focus on your problem instead of the error handling, and it exposes
the points where proper error handling need to occur. Let's start there
so we can get a handle on the code, and then refactor it to use better
error handling.

\begin{Shaded}
\begin{Highlighting}[]
\KeywordTok{use} \NormalTok{std::fs::File;}
\KeywordTok{use} \NormalTok{std::io::Read;}
\KeywordTok{use} \NormalTok{std::path::Path;}

\KeywordTok{fn} \NormalTok{file_double<P: AsRef<Path>>(file_path: P) -> }\DataTypeTok{i32} \NormalTok{\{}
    \KeywordTok{let} \KeywordTok{mut} \NormalTok{file = File::open(file_path).unwrap(); }\CommentTok{// error 1}
    \KeywordTok{let} \KeywordTok{mut} \NormalTok{contents = }\DataTypeTok{String}\NormalTok{::new();}
    \NormalTok{file.read_to_string(&}\KeywordTok{mut} \NormalTok{contents).unwrap(); }\CommentTok{// error 2}
    \KeywordTok{let} \NormalTok{n: }\DataTypeTok{i32} \NormalTok{= contents.trim().parse().unwrap(); }\CommentTok{// error 3}
    \DecValTok{2} \NormalTok{* n}
\NormalTok{\}}

\KeywordTok{fn} \NormalTok{main() \{}
    \KeywordTok{let} \NormalTok{doubled = file_double(}\StringTok{"foobar"}\NormalTok{);}
    \PreprocessorTok{println!}\NormalTok{(}\StringTok{"\{\}"}\NormalTok{, doubled);}
\NormalTok{\}}
\end{Highlighting}
\end{Shaded}

(N.B. The \texttt{AsRef\textless{}Path\textgreater{}} is used because
those are the
\href{http://doc.rust-lang.org/std/fs/struct.File.html\#method.open}{same
bounds used on \texttt{std::fs::File::open}}. This makes it ergonomic to
use any kind of string as a file path.)

There are three different errors that can occur here:

\begin{enumerate}
\def\labelenumi{\arabic{enumi}.}
\tightlist
\item
  A problem opening the file.
\item
  A problem reading data from the file.
\item
  A problem parsing the data as a number.
\end{enumerate}

The first two problems are described via the
\href{http://doc.rust-lang.org/std/io/struct.Error.html}{\texttt{std::io::Error}}
type. We know this because of the return types of
\href{http://doc.rust-lang.org/std/fs/struct.File.html\#method.open}{\texttt{std::fs::File::open}}
and
\href{http://doc.rust-lang.org/std/io/trait.Read.html\#method.read_to_string}{\texttt{std::io::Read::read\_to\_string}}.
(Note that they both use the
\protect\hyperlink{the-result-type-alias-idiom}{\texttt{Result} type
alias idiom} described previously. If you click on the \texttt{Result}
type, you'll \href{http://doc.rust-lang.org/std/io/type.Result.html}{see
the type alias}, and consequently, the underlying \texttt{io::Error}
type.) The third problem is described by the
\href{http://doc.rust-lang.org/std/num/struct.ParseIntError.html}{\texttt{std::num::ParseIntError}}
type. The \texttt{io::Error} type in particular is \emph{pervasive}
throughout the standard library. You will see it again and again.

Let's start the process of refactoring the \texttt{file\_double}
function. To make this function composable with other components of the
program, it should \emph{not} panic if any of the above error conditions
are met. Effectively, this means that the function should \emph{return
an error} if any of its operations fail. Our problem is that the return
type of \texttt{file\_double} is \texttt{i32}, which does not give us
any useful way of reporting an error. Thus, we must start by changing
the return type from \texttt{i32} to something else.

The first thing we need to decide: should we use \texttt{Option} or
\texttt{Result}? We certainly could use \texttt{Option} very easily. If
any of the three errors occur, we could simply return \texttt{None}.
This will work \emph{and it is better than panicking}, but we can do a
lot better. Instead, we should pass some detail about the error that
occurred. Since we want to express the \emph{possibility of error}, we
should use \texttt{Result\textless{}i32,\ E\textgreater{}}. But what
should \texttt{E} be? Since two \emph{different} types of errors can
occur, we need to convert them to a common type. One such type is
\texttt{String}. Let's see how that impacts our code:

\begin{Shaded}
\begin{Highlighting}[]
\KeywordTok{use} \NormalTok{std::fs::File;}
\KeywordTok{use} \NormalTok{std::io::Read;}
\KeywordTok{use} \NormalTok{std::path::Path;}

\KeywordTok{fn} \NormalTok{file_double<P: AsRef<Path>>(file_path: P) -> }\DataTypeTok{Result}\NormalTok{<}\DataTypeTok{i32}\NormalTok{, }\DataTypeTok{String}\NormalTok{> \{}
    \NormalTok{File::open(file_path)}
         \NormalTok{.map_err(|err| err.to_string())}
         \NormalTok{.and_then(|}\KeywordTok{mut} \NormalTok{file| \{}
              \KeywordTok{let} \KeywordTok{mut} \NormalTok{contents = }\DataTypeTok{String}\NormalTok{::new();}
              \NormalTok{file.read_to_string(&}\KeywordTok{mut} \NormalTok{contents)}
                  \NormalTok{.map_err(|err| err.to_string())}
                  \NormalTok{.map(|_| contents)}
         \NormalTok{\})}
         \NormalTok{.and_then(|contents| \{}
              \NormalTok{contents.trim().parse::<}\DataTypeTok{i32}\NormalTok{>()}
                      \NormalTok{.map_err(|err| err.to_string())}
         \NormalTok{\})}
         \NormalTok{.map(|n| }\DecValTok{2} \NormalTok{* n)}
\NormalTok{\}}

\KeywordTok{fn} \NormalTok{main() \{}
    \KeywordTok{match} \NormalTok{file_double(}\StringTok{"foobar"}\NormalTok{) \{}
        \ConstantTok{Ok}\NormalTok{(n) => }\PreprocessorTok{println!}\NormalTok{(}\StringTok{"\{\}"}\NormalTok{, n),}
        \ConstantTok{Err}\NormalTok{(err) => }\PreprocessorTok{println!}\NormalTok{(}\StringTok{"Error: \{\}"}\NormalTok{, err),}
    \NormalTok{\}}
\NormalTok{\}}
\end{Highlighting}
\end{Shaded}

This code looks a bit hairy. It can take quite a bit of practice before
code like this becomes easy to write. The way we write it is by
\emph{following the types}. As soon as we changed the return type of
\texttt{file\_double} to
\texttt{Result\textless{}i32,\ String\textgreater{}}, we had to start
looking for the right combinators. In this case, we only used three
different combinators: \texttt{and\_then}, \texttt{map} and
\texttt{map\_err}.

\texttt{and\_then} is used to chain multiple computations where each
computation could return an error. After opening the file, there are two
more computations that could fail: reading from the file and parsing the
contents as a number. Correspondingly, there are two calls to
\texttt{and\_then}.

\texttt{map} is used to apply a function to the \texttt{Ok(...)} value
of a \texttt{Result}. For example, the very last call to \texttt{map}
multiplies the \texttt{Ok(...)} value (which is an \texttt{i32}) by
\texttt{2}. If an error had occurred before that point, this operation
would have been skipped because of how \texttt{map} is defined.

\texttt{map\_err} is the trick that makes all of this work.
\texttt{map\_err} is like \texttt{map}, except it applies a function to
the \texttt{Err(...)} value of a \texttt{Result}. In this case, we want
to convert all of our errors to one type: \texttt{String}. Since both
\texttt{io::Error} and \texttt{num::ParseIntError} implement
\texttt{ToString}, we can call the \texttt{to\_string()} method to
convert them.

With all of that said, the code is still hairy. Mastering use of
combinators is important, but they have their limits. Let's try a
different approach: early returns.

\subsubsection{Early returns}\label{early-returns-1}

I'd like to take the code from the previous section and rewrite it using
\emph{early returns}. Early returns let you exit the function early. We
can't return early in \texttt{file\_double} from inside another closure,
so we'll need to revert back to explicit case analysis.

\begin{Shaded}
\begin{Highlighting}[]
\KeywordTok{use} \NormalTok{std::fs::File;}
\KeywordTok{use} \NormalTok{std::io::Read;}
\KeywordTok{use} \NormalTok{std::path::Path;}

\KeywordTok{fn} \NormalTok{file_double<P: AsRef<Path>>(file_path: P) -> }\DataTypeTok{Result}\NormalTok{<}\DataTypeTok{i32}\NormalTok{, }\DataTypeTok{String}\NormalTok{> \{}
    \KeywordTok{let} \KeywordTok{mut} \NormalTok{file = }\KeywordTok{match} \NormalTok{File::open(file_path) \{}
        \ConstantTok{Ok}\NormalTok{(file) => file,}
        \ConstantTok{Err}\NormalTok{(err) => }\KeywordTok{return} \ConstantTok{Err}\NormalTok{(err.to_string()),}
    \NormalTok{\};}
    \KeywordTok{let} \KeywordTok{mut} \NormalTok{contents = }\DataTypeTok{String}\NormalTok{::new();}
    \KeywordTok{if} \KeywordTok{let} \ConstantTok{Err}\NormalTok{(err) = file.read_to_string(&}\KeywordTok{mut} \NormalTok{contents) \{}
        \KeywordTok{return} \ConstantTok{Err}\NormalTok{(err.to_string());}
    \NormalTok{\}}
    \KeywordTok{let} \NormalTok{n: }\DataTypeTok{i32} \NormalTok{= }\KeywordTok{match} \NormalTok{contents.trim().parse() \{}
        \ConstantTok{Ok}\NormalTok{(n) => n,}
        \ConstantTok{Err}\NormalTok{(err) => }\KeywordTok{return} \ConstantTok{Err}\NormalTok{(err.to_string()),}
    \NormalTok{\};}
    \ConstantTok{Ok}\NormalTok{(}\DecValTok{2} \NormalTok{* n)}
\NormalTok{\}}

\KeywordTok{fn} \NormalTok{main() \{}
    \KeywordTok{match} \NormalTok{file_double(}\StringTok{"foobar"}\NormalTok{) \{}
        \ConstantTok{Ok}\NormalTok{(n) => }\PreprocessorTok{println!}\NormalTok{(}\StringTok{"\{\}"}\NormalTok{, n),}
        \ConstantTok{Err}\NormalTok{(err) => }\PreprocessorTok{println!}\NormalTok{(}\StringTok{"Error: \{\}"}\NormalTok{, err),}
    \NormalTok{\}}
\NormalTok{\}}
\end{Highlighting}
\end{Shaded}

Reasonable people can disagree over whether this code is better than the
code that uses combinators, but if you aren't familiar with the
combinator approach, this code looks simpler to read to me. It uses
explicit case analysis with \texttt{match} and \texttt{if\ let}. If an
error occurs, it simply stops executing the function and returns the
error (by converting it to a string).

Isn't this a step backwards though? Previously, we said that the key to
ergonomic error handling is reducing explicit case analysis, yet we've
reverted back to explicit case analysis here. It turns out, there are
\emph{multiple} ways to reduce explicit case analysis. Combinators
aren't the only way.

\hypertarget{the-try-macro}{\subsubsection{\texorpdfstring{The
\texttt{try!} macro}{The try! macro}}\label{the-try-macro}}

A cornerstone of error handling in Rust is the \texttt{try!} macro. The
\texttt{try!} macro abstracts case analysis like combinators, but unlike
combinators, it also abstracts \emph{control flow}. Namely, it can
abstract the \emph{early return} pattern seen above.

Here is a simplified definition of a \texttt{try!} macro:

\protect\hypertarget{code-try-def-simple}{}{}

\begin{Shaded}
\begin{Highlighting}[]
\PreprocessorTok{macro_rules!} \NormalTok{try \{}
    \NormalTok{($e:expr) => (}\KeywordTok{match} \NormalTok{$e \{}
        \ConstantTok{Ok}\NormalTok{(val) => val,}
        \ConstantTok{Err}\NormalTok{(err) => }\KeywordTok{return} \ConstantTok{Err}\NormalTok{(err),}
    \NormalTok{\});}
\NormalTok{\}}
\end{Highlighting}
\end{Shaded}

(The \href{http://doc.rust-lang.org/std/macro.try!.html}{real
definition} is a bit more sophisticated. We will address that later.)

Using the \texttt{try!} macro makes it very easy to simplify our last
example. Since it does the case analysis and the early return for us, we
get tighter code that is easier to read:

\begin{Shaded}
\begin{Highlighting}[]
\KeywordTok{use} \NormalTok{std::fs::File;}
\KeywordTok{use} \NormalTok{std::io::Read;}
\KeywordTok{use} \NormalTok{std::path::Path;}

\KeywordTok{fn} \NormalTok{file_double<P: AsRef<Path>>(file_path: P) -> }\DataTypeTok{Result}\NormalTok{<}\DataTypeTok{i32}\NormalTok{, }\DataTypeTok{String}\NormalTok{> \{}
    \KeywordTok{let} \KeywordTok{mut} \NormalTok{file = }\PreprocessorTok{try!}\NormalTok{(File::open(file_path).map_err(|e| e.to_string()));}
    \KeywordTok{let} \KeywordTok{mut} \NormalTok{contents = }\DataTypeTok{String}\NormalTok{::new();}
    \PreprocessorTok{try!}\NormalTok{(file.read_to_string(&}\KeywordTok{mut} \NormalTok{contents).map_err(|e| e.to_string()));}
    \KeywordTok{let} \NormalTok{n = }\PreprocessorTok{try!}\NormalTok{(contents.trim().parse::<}\DataTypeTok{i32}\NormalTok{>().map_err(|e| e.to_string()));}
    \ConstantTok{Ok}\NormalTok{(}\DecValTok{2} \NormalTok{* n)}
\NormalTok{\}}

\KeywordTok{fn} \NormalTok{main() \{}
    \KeywordTok{match} \NormalTok{file_double(}\StringTok{"foobar"}\NormalTok{) \{}
        \ConstantTok{Ok}\NormalTok{(n) => }\PreprocessorTok{println!}\NormalTok{(}\StringTok{"\{\}"}\NormalTok{, n),}
        \ConstantTok{Err}\NormalTok{(err) => }\PreprocessorTok{println!}\NormalTok{(}\StringTok{"Error: \{\}"}\NormalTok{, err),}
    \NormalTok{\}}
\NormalTok{\}}
\end{Highlighting}
\end{Shaded}

The \texttt{map\_err} calls are still necessary given
\protect\hyperlink{code-try-def-simple}{our definition of
\texttt{try!}}. This is because the error types still need to be
converted to \texttt{String}. The good news is that we will soon learn
how to remove those \texttt{map\_err} calls! The bad news is that we
will need to learn a bit more about a couple important traits in the
standard library before we can remove the \texttt{map\_err} calls.

\hypertarget{defining-your-own-error-type}{\subsubsection{Defining your
own error type}\label{defining-your-own-error-type}}

Before we dive into some of the standard library error traits, I'd like
to wrap up this section by removing the use of \texttt{String} as our
error type in the previous examples.

Using \texttt{String} as we did in our previous examples is convenient
because it's easy to convert errors to strings, or even make up your own
errors as strings on the spot. However, using \texttt{String} for your
errors has some downsides.

The first downside is that the error messages tend to clutter your code.
It's possible to define the error messages elsewhere, but unless you're
unusually disciplined, it is very tempting to embed the error message
into your code. Indeed, we did exactly this in a
\protect\hyperlink{code-error-double-string}{previous example}.

The second and more important downside is that \texttt{String}s are
\emph{lossy}. That is, if all errors are converted to strings, then the
errors we pass to the caller become completely opaque. The only
reasonable thing the caller can do with a \texttt{String} error is show
it to the user. Certainly, inspecting the string to determine the type
of error is not robust. (Admittedly, this downside is far more important
inside of a library as opposed to, say, an application.)

For example, the \texttt{io::Error} type embeds an
\href{http://doc.rust-lang.org/std/io/enum.ErrorKind.html}{\texttt{io::ErrorKind}},
which is \emph{structured data} that represents what went wrong during
an IO operation. This is important because you might want to react
differently depending on the error. (e.g., A \texttt{BrokenPipe} error
might mean quitting your program gracefully while a \texttt{NotFound}
error might mean exiting with an error code and showing an error to the
user.) With \texttt{io::ErrorKind}, the caller can examine the type of
an error with case analysis, which is strictly superior to trying to
tease out the details of an error inside of a \texttt{String}.

Instead of using a \texttt{String} as an error type in our previous
example of reading an integer from a file, we can define our own error
type that represents errors with \emph{structured data}. We endeavor to
not drop information from underlying errors in case the caller wants to
inspect the details.

The ideal way to represent \emph{one of many possibilities} is to define
our own sum type using \texttt{enum}. In our case, an error is either an
\texttt{io::Error} or a \texttt{num::ParseIntError}, so a natural
definition arises:

\begin{Shaded}
\begin{Highlighting}[]
\KeywordTok{use} \NormalTok{std::io;}
\KeywordTok{use} \NormalTok{std::num;}

\CommentTok{// We derive `Debug` because all types should probably derive `Debug`.}
\CommentTok{// This gives us a reasonable human readable description of `CliError` values.}
\AttributeTok{#[}\NormalTok{derive}\AttributeTok{(}\BuiltInTok{Debug}\AttributeTok{)]}
\KeywordTok{enum} \NormalTok{CliError \{}
    \NormalTok{Io(io::Error),}
    \NormalTok{Parse(num::ParseIntError),}
\NormalTok{\}}
\end{Highlighting}
\end{Shaded}

Tweaking our code is very easy. Instead of converting errors to strings,
we simply convert them to our \texttt{CliError} type using the
corresponding value constructor:

\begin{Shaded}
\begin{Highlighting}[]
\KeywordTok{use} \NormalTok{std::fs::File;}
\KeywordTok{use} \NormalTok{std::io::Read;}
\KeywordTok{use} \NormalTok{std::path::Path;}

\KeywordTok{fn} \NormalTok{file_double<P: AsRef<Path>>(file_path: P) -> }\DataTypeTok{Result}\NormalTok{<}\DataTypeTok{i32}\NormalTok{, CliError> \{}
    \KeywordTok{let} \KeywordTok{mut} \NormalTok{file = }\PreprocessorTok{try!}\NormalTok{(File::open(file_path).map_err(CliError::Io));}
    \KeywordTok{let} \KeywordTok{mut} \NormalTok{contents = }\DataTypeTok{String}\NormalTok{::new();}
    \PreprocessorTok{try!}\NormalTok{(file.read_to_string(&}\KeywordTok{mut} \NormalTok{contents).map_err(CliError::Io));}
    \KeywordTok{let} \NormalTok{n: }\DataTypeTok{i32} \NormalTok{= }\PreprocessorTok{try!}\NormalTok{(contents.trim().parse().map_err(CliError::Parse));}
    \ConstantTok{Ok}\NormalTok{(}\DecValTok{2} \NormalTok{* n)}
\NormalTok{\}}

\KeywordTok{fn} \NormalTok{main() \{}
    \KeywordTok{match} \NormalTok{file_double(}\StringTok{"foobar"}\NormalTok{) \{}
        \ConstantTok{Ok}\NormalTok{(n) => }\PreprocessorTok{println!}\NormalTok{(}\StringTok{"\{\}"}\NormalTok{, n),}
        \ConstantTok{Err}\NormalTok{(err) => }\PreprocessorTok{println!}\NormalTok{(}\StringTok{"Error: \{:?\}"}\NormalTok{, err),}
    \NormalTok{\}}
\NormalTok{\}}
\end{Highlighting}
\end{Shaded}

The only change here is switching
\texttt{map\_err(\textbar{}e\textbar{}\ e.to\_string())} (which converts
errors to strings) to \texttt{map\_err(CliError::Io)} or
\texttt{map\_err(CliError::Parse)}. The \emph{caller} gets to decide the
level of detail to report to the user. In effect, using a
\texttt{String} as an error type removes choices from the caller while
using a custom \texttt{enum} error type like \texttt{CliError} gives the
caller all of the conveniences as before in addition to \emph{structured
data} describing the error.

A rule of thumb is to define your own error type, but a \texttt{String}
error type will do in a pinch, particularly if you're writing an
application. If you're writing a library, defining your own error type
should be strongly preferred so that you don't remove choices from the
caller unnecessarily.

\hypertarget{standard-library-traits-used-for-error-handling}{\subsection{Standard
library traits used for error
handling}\label{standard-library-traits-used-for-error-handling}}

The standard library defines two integral traits for error handling:
\href{http://doc.rust-lang.org/std/error/trait.Error.html}{\texttt{std::error::Error}}
and
\href{http://doc.rust-lang.org/std/convert/trait.From.html}{\texttt{std::convert::From}}.
While \texttt{Error} is designed specifically for generically describing
errors, the \texttt{From} trait serves a more general role for
converting values between two distinct types.

\hypertarget{the-error-trait}{\subsubsection{\texorpdfstring{The
\texttt{Error} trait}{The Error trait}}\label{the-error-trait}}

The \texttt{Error} trait is
\href{http://doc.rust-lang.org/std/error/trait.Error.html}{defined in
the standard library}:

\begin{Shaded}
\begin{Highlighting}[]
\KeywordTok{use} \NormalTok{std::fmt::\{}\BuiltInTok{Debug}\NormalTok{, }\BuiltInTok{Display}\NormalTok{\};}

\KeywordTok{trait} \NormalTok{Error: }\BuiltInTok{Debug} \NormalTok{+ }\BuiltInTok{Display} \NormalTok{\{}
  \CommentTok{/// A short description of the error.}
  \KeywordTok{fn} \NormalTok{description(&}\KeywordTok{self}\NormalTok{) -> &}\DataTypeTok{str}\NormalTok{;}

  \CommentTok{/// The lower level cause of this error, if any.}
  \KeywordTok{fn} \NormalTok{cause(&}\KeywordTok{self}\NormalTok{) -> }\DataTypeTok{Option}\NormalTok{<&Error> \{ }\ConstantTok{None} \NormalTok{\}}
\NormalTok{\}}
\end{Highlighting}
\end{Shaded}

This trait is super generic because it is meant to be implemented for
\emph{all} types that represent errors. This will prove useful for
writing composable code as we'll see later. Otherwise, the trait allows
you to do at least the following things:

\begin{itemize}
\tightlist
\item
  Obtain a \texttt{Debug} representation of the error.
\item
  Obtain a user-facing \texttt{Display} representation of the error.
\item
  Obtain a short description of the error (via the \texttt{description}
  method).
\item
  Inspect the causal chain of an error, if one exists (via the
  \texttt{cause} method).
\end{itemize}

The first two are a result of \texttt{Error} requiring impls for both
\texttt{Debug} and \texttt{Display}. The latter two are from the two
methods defined on \texttt{Error}. The power of \texttt{Error} comes
from the fact that all error types impl \texttt{Error}, which means
errors can be existentially quantified as a
\href{http://doc.rust-lang.org/book/trait-objects.html}{trait object}.
This manifests as either \texttt{Box\textless{}Error\textgreater{}} or
\texttt{\&Error}. Indeed, the \texttt{cause} method returns an
\texttt{\&Error}, which is itself a trait object. We'll revisit the
\texttt{Error} trait's utility as a trait object later.

For now, it suffices to show an example implementing the \texttt{Error}
trait. Let's use the error type we defined in the
\protect\hyperlink{defining-your-own-error-type}{previous section}:

\begin{Shaded}
\begin{Highlighting}[]
\KeywordTok{use} \NormalTok{std::io;}
\KeywordTok{use} \NormalTok{std::num;}

\CommentTok{// We derive `Debug` because all types should probably derive `Debug`.}
\CommentTok{// This gives us a reasonable human readable description of `CliError` values.}
\AttributeTok{#[}\NormalTok{derive}\AttributeTok{(}\BuiltInTok{Debug}\AttributeTok{)]}
\KeywordTok{enum} \NormalTok{CliError \{}
    \NormalTok{Io(io::Error),}
    \NormalTok{Parse(num::ParseIntError),}
\NormalTok{\}}
\end{Highlighting}
\end{Shaded}

This particular error type represents the possibility of two types of
errors occurring: an error dealing with I/O or an error converting a
string to a number. The error could represent as many error types as you
want by adding new variants to the \texttt{enum} definition.

Implementing \texttt{Error} is pretty straight-forward. It's mostly
going to be a lot explicit case analysis.

\begin{Shaded}
\begin{Highlighting}[]
\KeywordTok{use} \NormalTok{std::error;}
\KeywordTok{use} \NormalTok{std::fmt;}

\KeywordTok{impl} \NormalTok{fmt::}\BuiltInTok{Display} \KeywordTok{for} \NormalTok{CliError \{}
    \KeywordTok{fn} \NormalTok{fmt(&}\KeywordTok{self}\NormalTok{, f: &}\KeywordTok{mut} \NormalTok{fmt::Formatter) -> fmt::}\DataTypeTok{Result} \NormalTok{\{}
        \KeywordTok{match} \NormalTok{*}\KeywordTok{self} \NormalTok{\{}
            \CommentTok{// Both underlying errors already impl `Display`, so we defer to}
            \CommentTok{// their implementations.}
            \NormalTok{CliError::Io(}\KeywordTok{ref} \NormalTok{err) => }\PreprocessorTok{write!}\NormalTok{(f, }\StringTok{"IO error: \{\}"}\NormalTok{, err),}
            \NormalTok{CliError::Parse(}\KeywordTok{ref} \NormalTok{err) => }\PreprocessorTok{write!}\NormalTok{(f, }\StringTok{"Parse error: \{\}"}\NormalTok{, err),}
        \NormalTok{\}}
    \NormalTok{\}}
\NormalTok{\}}

\KeywordTok{impl} \NormalTok{error::Error }\KeywordTok{for} \NormalTok{CliError \{}
    \KeywordTok{fn} \NormalTok{description(&}\KeywordTok{self}\NormalTok{) -> &}\DataTypeTok{str} \NormalTok{\{}
        \CommentTok{// Both underlying errors already impl `Error`, so we defer to their}
        \CommentTok{// implementations.}
        \KeywordTok{match} \NormalTok{*}\KeywordTok{self} \NormalTok{\{}
            \NormalTok{CliError::Io(}\KeywordTok{ref} \NormalTok{err) => err.description(),}
            \NormalTok{CliError::Parse(}\KeywordTok{ref} \NormalTok{err) => err.description(),}
        \NormalTok{\}}
    \NormalTok{\}}

    \KeywordTok{fn} \NormalTok{cause(&}\KeywordTok{self}\NormalTok{) -> }\DataTypeTok{Option}\NormalTok{<&error::Error> \{}
        \KeywordTok{match} \NormalTok{*}\KeywordTok{self} \NormalTok{\{}
            \CommentTok{// N.B. Both of these implicitly cast `err` from their concrete}
            \CommentTok{// types (either `&io::Error` or `&num::ParseIntError`)}
            \CommentTok{// to a trait object `&Error`. This works because both error types}
            \CommentTok{// implement `Error`.}
            \NormalTok{CliError::Io(}\KeywordTok{ref} \NormalTok{err) => }\ConstantTok{Some}\NormalTok{(err),}
            \NormalTok{CliError::Parse(}\KeywordTok{ref} \NormalTok{err) => }\ConstantTok{Some}\NormalTok{(err),}
        \NormalTok{\}}
    \NormalTok{\}}
\NormalTok{\}}
\end{Highlighting}
\end{Shaded}

We note that this is a very typical implementation of \texttt{Error}:
match on your different error types and satisfy the contracts defined
for \texttt{description} and \texttt{cause}.

\hypertarget{the-from-trait}{\subsubsection{\texorpdfstring{The
\texttt{From} trait}{The From trait}}\label{the-from-trait}}

The \texttt{std::convert::From} trait is
\href{http://doc.rust-lang.org/std/convert/trait.From.html}{defined in
the standard library}:

\protect\hypertarget{code-from-def}{}{}

\begin{Shaded}
\begin{Highlighting}[]
\KeywordTok{trait} \NormalTok{From<T> \{}
    \KeywordTok{fn} \NormalTok{from(T) -> }\KeywordTok{Self}\NormalTok{;}
\NormalTok{\}}
\end{Highlighting}
\end{Shaded}

Deliciously simple, yes? \texttt{From} is very useful because it gives
us a generic way to talk about conversion \emph{from} a particular type
\texttt{T} to some other type (in this case, ``some other type'' is the
subject of the impl, or \texttt{Self}). The crux of \texttt{From} is the
\href{http://doc.rust-lang.org/std/convert/trait.From.html}{set of
implementations provided by the standard library}.

Here are a few simple examples demonstrating how \texttt{From} works:

\begin{Shaded}
\begin{Highlighting}[]
\KeywordTok{let} \NormalTok{string: }\DataTypeTok{String} \NormalTok{= From::from(}\StringTok{"foo"}\NormalTok{);}
\KeywordTok{let} \NormalTok{bytes: }\DataTypeTok{Vec}\NormalTok{<}\DataTypeTok{u8}\NormalTok{> = From::from(}\StringTok{"foo"}\NormalTok{);}
\KeywordTok{let} \NormalTok{cow: ::std::borrow::Cow<}\DataTypeTok{str}\NormalTok{> = From::from(}\StringTok{"foo"}\NormalTok{);}
\end{Highlighting}
\end{Shaded}

OK, so \texttt{From} is useful for converting between strings. But what
about errors? It turns out, there is one critical impl:

\begin{Shaded}
\begin{Highlighting}[]
\KeywordTok{impl}\NormalTok{<}\OtherTok{'a}\NormalTok{, E: Error + }\OtherTok{'a}\NormalTok{> From<E> }\KeywordTok{for} \DataTypeTok{Box}\NormalTok{<Error + }\OtherTok{'a}\NormalTok{>}
\end{Highlighting}
\end{Shaded}

This impl says that for \emph{any} type that impls \texttt{Error}, we
can convert it to a trait object
\texttt{Box\textless{}Error\textgreater{}}. This may not seem terribly
surprising, but it is useful in a generic context.

Remember the two errors we were dealing with previously? Specifically,
\texttt{io::Error} and \texttt{num::ParseIntError}. Since both impl
\texttt{Error}, they work with \texttt{From}:

\begin{Shaded}
\begin{Highlighting}[]
\KeywordTok{use} \NormalTok{std::error::Error;}
\KeywordTok{use} \NormalTok{std::fs;}
\KeywordTok{use} \NormalTok{std::io;}
\KeywordTok{use} \NormalTok{std::num;}

\CommentTok{// We have to jump through some hoops to actually get error values.}
\KeywordTok{let} \NormalTok{io_err: io::Error = io::Error::last_os_error();}
\KeywordTok{let} \NormalTok{parse_err: num::ParseIntError = }\StringTok{"not a number"}\NormalTok{.parse::<}\DataTypeTok{i32}\NormalTok{>().unwrap_err();}

\CommentTok{// OK, here are the conversions.}
\KeywordTok{let} \NormalTok{err1: }\DataTypeTok{Box}\NormalTok{<Error> = From::from(io_err);}
\KeywordTok{let} \NormalTok{err2: }\DataTypeTok{Box}\NormalTok{<Error> = From::from(parse_err);}
\end{Highlighting}
\end{Shaded}

There is a really important pattern to recognize here. Both
\texttt{err1} and \texttt{err2} have the \emph{same type}. This is
because they are existentially quantified types, or trait objects. In
particular, their underlying type is \emph{erased} from the compiler's
knowledge, so it truly sees \texttt{err1} and \texttt{err2} as exactly
the same. Additionally, we constructed \texttt{err1} and \texttt{err2}
using precisely the same function call: \texttt{From::from}. This is
because \texttt{From::from} is overloaded on both its argument and its
return type.

This pattern is important because it solves a problem we had earlier: it
gives us a way to reliably convert errors to the same type using the
same function.

Time to revisit an old friend; the \texttt{try!} macro.

\hypertarget{the-real-try-macro}{\subsubsection{\texorpdfstring{The real
\texttt{try!} macro}{The real try! macro}}\label{the-real-try-macro}}

Previously, we presented this definition of \texttt{try!}:

\begin{Shaded}
\begin{Highlighting}[]
\PreprocessorTok{macro_rules!} \NormalTok{try \{}
    \NormalTok{($e:expr) => (}\KeywordTok{match} \NormalTok{$e \{}
        \ConstantTok{Ok}\NormalTok{(val) => val,}
        \ConstantTok{Err}\NormalTok{(err) => }\KeywordTok{return} \ConstantTok{Err}\NormalTok{(err),}
    \NormalTok{\});}
\NormalTok{\}}
\end{Highlighting}
\end{Shaded}

This is not its real definition. Its real definition is
\href{http://doc.rust-lang.org/std/macro.try!.html}{in the standard
library}:

\protect\hypertarget{code-try-def}{}{}

\begin{Shaded}
\begin{Highlighting}[]
\PreprocessorTok{macro_rules!} \NormalTok{try \{}
    \NormalTok{($e:expr) => (}\KeywordTok{match} \NormalTok{$e \{}
        \ConstantTok{Ok}\NormalTok{(val) => val,}
        \ConstantTok{Err}\NormalTok{(err) => }\KeywordTok{return} \ConstantTok{Err}\NormalTok{(::std::convert::From::from(err)),}
    \NormalTok{\});}
\NormalTok{\}}
\end{Highlighting}
\end{Shaded}

There's one tiny but powerful change: the error value is passed through
\texttt{From::from}. This makes the \texttt{try!} macro a lot more
powerful because it gives you automatic type conversion for free.

Armed with our more powerful \texttt{try!} macro, let's take a look at
code we wrote previously to read a file and convert its contents to an
integer:

\begin{Shaded}
\begin{Highlighting}[]
\KeywordTok{use} \NormalTok{std::fs::File;}
\KeywordTok{use} \NormalTok{std::io::Read;}
\KeywordTok{use} \NormalTok{std::path::Path;}

\KeywordTok{fn} \NormalTok{file_double<P: AsRef<Path>>(file_path: P) -> }\DataTypeTok{Result}\NormalTok{<}\DataTypeTok{i32}\NormalTok{, }\DataTypeTok{String}\NormalTok{> \{}
    \KeywordTok{let} \KeywordTok{mut} \NormalTok{file = }\PreprocessorTok{try!}\NormalTok{(File::open(file_path).map_err(|e| e.to_string()));}
    \KeywordTok{let} \KeywordTok{mut} \NormalTok{contents = }\DataTypeTok{String}\NormalTok{::new();}
    \PreprocessorTok{try!}\NormalTok{(file.read_to_string(&}\KeywordTok{mut} \NormalTok{contents).map_err(|e| e.to_string()));}
    \KeywordTok{let} \NormalTok{n = }\PreprocessorTok{try!}\NormalTok{(contents.trim().parse::<}\DataTypeTok{i32}\NormalTok{>().map_err(|e| e.to_string()));}
    \ConstantTok{Ok}\NormalTok{(}\DecValTok{2} \NormalTok{* n)}
\NormalTok{\}}
\end{Highlighting}
\end{Shaded}

Earlier, we promised that we could get rid of the \texttt{map\_err}
calls. Indeed, all we have to do is pick a type that \texttt{From} works
with. As we saw in the previous section, \texttt{From} has an impl that
lets it convert any error type into a
\texttt{Box\textless{}Error\textgreater{}}:

\begin{Shaded}
\begin{Highlighting}[]
\KeywordTok{use} \NormalTok{std::error::Error;}
\KeywordTok{use} \NormalTok{std::fs::File;}
\KeywordTok{use} \NormalTok{std::io::Read;}
\KeywordTok{use} \NormalTok{std::path::Path;}

\KeywordTok{fn} \NormalTok{file_double<P: AsRef<Path>>(file_path: P) -> }\DataTypeTok{Result}\NormalTok{<}\DataTypeTok{i32}\NormalTok{, }\DataTypeTok{Box}\NormalTok{<Error>> \{}
    \KeywordTok{let} \KeywordTok{mut} \NormalTok{file = }\PreprocessorTok{try!}\NormalTok{(File::open(file_path));}
    \KeywordTok{let} \KeywordTok{mut} \NormalTok{contents = }\DataTypeTok{String}\NormalTok{::new();}
    \PreprocessorTok{try!}\NormalTok{(file.read_to_string(&}\KeywordTok{mut} \NormalTok{contents));}
    \KeywordTok{let} \NormalTok{n = }\PreprocessorTok{try!}\NormalTok{(contents.trim().parse::<}\DataTypeTok{i32}\NormalTok{>());}
    \ConstantTok{Ok}\NormalTok{(}\DecValTok{2} \NormalTok{* n)}
\NormalTok{\}}
\end{Highlighting}
\end{Shaded}

We are getting very close to ideal error handling. Our code has very
little overhead as a result from error handling because the
\texttt{try!} macro encapsulates three things simultaneously:

\begin{enumerate}
\def\labelenumi{\arabic{enumi}.}
\tightlist
\item
  Case analysis.
\item
  Control flow.
\item
  Error type conversion.
\end{enumerate}

When all three things are combined, we get code that is unencumbered by
combinators, calls to \texttt{unwrap} or case analysis.

There's one little nit left: the
\texttt{Box\textless{}Error\textgreater{}} type is \emph{opaque}. If we
return a \texttt{Box\textless{}Error\textgreater{}} to the caller, the
caller can't (easily) inspect underlying error type. The situation is
certainly better than \texttt{String} because the caller can call
methods like
\href{http://doc.rust-lang.org/std/error/trait.Error.html\#tymethod.description}{\texttt{description}}
and
\href{http://doc.rust-lang.org/std/error/trait.Error.html\#method.cause}{\texttt{cause}},
but the limitation remains: \texttt{Box\textless{}Error\textgreater{}}
is opaque. (N.B. This isn't entirely true because Rust does have runtime
reflection, which is useful in some scenarios that are
\href{https://crates.io/crates/error}{beyond the scope of this
section}.)

It's time to revisit our custom \texttt{CliError} type and tie
everything together.

\hypertarget{composing-custom-error-types}{\subsubsection{Composing
custom error types}\label{composing-custom-error-types}}

In the last section, we looked at the real \texttt{try!} macro and how
it does automatic type conversion for us by calling \texttt{From::from}
on the error value. In particular, we converted errors to
\texttt{Box\textless{}Error\textgreater{}}, which works, but the type is
opaque to callers.

To fix this, we use the same remedy that we're already familiar with: a
custom error type. Once again, here is the code that reads the contents
of a file and converts it to an integer:

\begin{Shaded}
\begin{Highlighting}[]
\KeywordTok{use} \NormalTok{std::fs::File;}
\KeywordTok{use} \NormalTok{std::io::\{}\KeywordTok{self}\NormalTok{, Read\};}
\KeywordTok{use} \NormalTok{std::num;}
\KeywordTok{use} \NormalTok{std::path::Path;}

\CommentTok{// We derive `Debug` because all types should probably derive `Debug`.}
\CommentTok{// This gives us a reasonable human readable description of `CliError` values.}
\AttributeTok{#[}\NormalTok{derive}\AttributeTok{(}\BuiltInTok{Debug}\AttributeTok{)]}
\KeywordTok{enum} \NormalTok{CliError \{}
    \NormalTok{Io(io::Error),}
    \NormalTok{Parse(num::ParseIntError),}
\NormalTok{\}}

\KeywordTok{fn} \NormalTok{file_double_verbose<P: AsRef<Path>>(file_path: P) -> }\DataTypeTok{Result}\NormalTok{<}\DataTypeTok{i32}\NormalTok{, CliError> \{}
    \KeywordTok{let} \KeywordTok{mut} \NormalTok{file = }\PreprocessorTok{try!}\NormalTok{(File::open(file_path).map_err(CliError::Io));}
    \KeywordTok{let} \KeywordTok{mut} \NormalTok{contents = }\DataTypeTok{String}\NormalTok{::new();}
    \PreprocessorTok{try!}\NormalTok{(file.read_to_string(&}\KeywordTok{mut} \NormalTok{contents).map_err(CliError::Io));}
    \KeywordTok{let} \NormalTok{n: }\DataTypeTok{i32} \NormalTok{= }\PreprocessorTok{try!}\NormalTok{(contents.trim().parse().map_err(CliError::Parse));}
    \ConstantTok{Ok}\NormalTok{(}\DecValTok{2} \NormalTok{* n)}
\NormalTok{\}}
\end{Highlighting}
\end{Shaded}

Notice that we still have the calls to \texttt{map\_err}. Why? Well,
recall the definitions of
\protect\hyperlink{code-try-def}{\texttt{try!}} and
\protect\hyperlink{code-from-def}{\texttt{From}}. The problem is that
there is no \texttt{From} impl that allows us to convert from error
types like \texttt{io::Error} and \texttt{num::ParseIntError} to our own
custom \texttt{CliError}. Of course, it is easy to fix this! Since we
defined \texttt{CliError}, we can impl \texttt{From} with it:

\begin{Shaded}
\begin{Highlighting}[]
\KeywordTok{use} \NormalTok{std::io;}
\KeywordTok{use} \NormalTok{std::num;}

\KeywordTok{impl} \NormalTok{From<io::Error> }\KeywordTok{for} \NormalTok{CliError \{}
    \KeywordTok{fn} \NormalTok{from(err: io::Error) -> CliError \{}
        \NormalTok{CliError::Io(err)}
    \NormalTok{\}}
\NormalTok{\}}

\KeywordTok{impl} \NormalTok{From<num::ParseIntError> }\KeywordTok{for} \NormalTok{CliError \{}
    \KeywordTok{fn} \NormalTok{from(err: num::ParseIntError) -> CliError \{}
        \NormalTok{CliError::Parse(err)}
    \NormalTok{\}}
\NormalTok{\}}
\end{Highlighting}
\end{Shaded}

All these impls are doing is teaching \texttt{From} how to create a
\texttt{CliError} from other error types. In our case, construction is
as simple as invoking the corresponding value constructor. Indeed, it is
\emph{typically} this easy.

We can finally rewrite \texttt{file\_double}:

\begin{Shaded}
\begin{Highlighting}[]

\KeywordTok{use} \NormalTok{std::fs::File;}
\KeywordTok{use} \NormalTok{std::io::Read;}
\KeywordTok{use} \NormalTok{std::path::Path;}

\KeywordTok{fn} \NormalTok{file_double<P: AsRef<Path>>(file_path: P) -> }\DataTypeTok{Result}\NormalTok{<}\DataTypeTok{i32}\NormalTok{, CliError> \{}
    \KeywordTok{let} \KeywordTok{mut} \NormalTok{file = }\PreprocessorTok{try!}\NormalTok{(File::open(file_path));}
    \KeywordTok{let} \KeywordTok{mut} \NormalTok{contents = }\DataTypeTok{String}\NormalTok{::new();}
    \PreprocessorTok{try!}\NormalTok{(file.read_to_string(&}\KeywordTok{mut} \NormalTok{contents));}
    \KeywordTok{let} \NormalTok{n: }\DataTypeTok{i32} \NormalTok{= }\PreprocessorTok{try!}\NormalTok{(contents.trim().parse());}
    \ConstantTok{Ok}\NormalTok{(}\DecValTok{2} \NormalTok{* n)}
\NormalTok{\}}
\end{Highlighting}
\end{Shaded}

The only thing we did here was remove the calls to \texttt{map\_err}.
They are no longer needed because the \texttt{try!} macro invokes
\texttt{From::from} on the error value. This works because we've
provided \texttt{From} impls for all the error types that could appear.

If we modified our \texttt{file\_double} function to perform some other
operation, say, convert a string to a float, then we'd need to add a new
variant to our error type:

\begin{Shaded}
\begin{Highlighting}[]
\KeywordTok{use} \NormalTok{std::io;}
\KeywordTok{use} \NormalTok{std::num;}

\KeywordTok{enum} \NormalTok{CliError \{}
    \NormalTok{Io(io::Error),}
    \NormalTok{ParseInt(num::ParseIntError),}
    \NormalTok{ParseFloat(num::ParseFloatError),}
\NormalTok{\}}
\end{Highlighting}
\end{Shaded}

And add a new \texttt{From} impl:

\begin{Shaded}
\begin{Highlighting}[]

\KeywordTok{use} \NormalTok{std::num;}

\KeywordTok{impl} \NormalTok{From<num::ParseFloatError> }\KeywordTok{for} \NormalTok{CliError \{}
    \KeywordTok{fn} \NormalTok{from(err: num::ParseFloatError) -> CliError \{}
        \NormalTok{CliError::ParseFloat(err)}
    \NormalTok{\}}
\NormalTok{\}}
\end{Highlighting}
\end{Shaded}

And that's it!

\hypertarget{advice-for-library-writers}{\subsubsection{Advice for
library writers}\label{advice-for-library-writers}}

If your library needs to report custom errors, then you should probably
define your own error type. It's up to you whether or not to expose its
representation (like
\href{http://doc.rust-lang.org/std/io/enum.ErrorKind.html}{\texttt{ErrorKind}})
or keep it hidden (like
\href{http://doc.rust-lang.org/std/num/struct.ParseIntError.html}{\texttt{ParseIntError}}).
Regardless of how you do it, it's usually good practice to at least
provide some information about the error beyond its \texttt{String}
representation. But certainly, this will vary depending on use cases.

At a minimum, you should probably implement the
\href{http://doc.rust-lang.org/std/error/trait.Error.html}{\texttt{Error}}
trait. This will give users of your library some minimum flexibility for
\protect\hyperlink{the-real-try-macro}{composing errors}. Implementing
the \texttt{Error} trait also means that users are guaranteed the
ability to obtain a string representation of an error (because it
requires impls for both \texttt{fmt::Debug} and \texttt{fmt::Display}).

Beyond that, it can also be useful to provide implementations of
\texttt{From} on your error types. This allows you (the library author)
and your users to
\protect\hyperlink{composing-custom-error-types}{compose more detailed
errors}. For example,
\href{http://burntsushi.net/rustdoc/csv/enum.Error.html}{\texttt{csv::Error}}
provides \texttt{From} impls for both \texttt{io::Error} and
\texttt{byteorder::Error}.

Finally, depending on your tastes, you may also want to define a
\protect\hyperlink{the-result-type-alias-idiom}{\texttt{Result} type
alias}, particularly if your library defines a single error type. This
is used in the standard library for
\href{http://doc.rust-lang.org/std/io/type.Result.html}{\texttt{io::Result}}
and
\href{http://doc.rust-lang.org/std/fmt/type.Result.html}{\texttt{fmt::Result}}.

\hypertarget{case-study-a-program-to-read-population-data}{\subsection{Case
study: A program to read population
data}\label{case-study-a-program-to-read-population-data}}

This section was long, and depending on your background, it might be
rather dense. While there is plenty of example code to go along with the
prose, most of it was specifically designed to be pedagogical. So, we're
going to do something new: a case study.

For this, we're going to build up a command line program that lets you
query world population data. The objective is simple: you give it a
location and it will tell you the population. Despite the simplicity,
there is a lot that can go wrong!

The data we'll be using comes from the
\href{https://github.com/petewarden/dstkdata}{Data Science Toolkit}.
I've prepared some data from it for this exercise. You can either grab
the \href{http://burntsushi.net/stuff/worldcitiespop.csv.gz}{world
population data} (41MB gzip compressed, 145MB uncompressed) or only the
\href{http://burntsushi.net/stuff/uscitiespop.csv.gz}{US population
data} (2.2MB gzip compressed, 7.2MB uncompressed).

Up until now, we've kept the code limited to Rust's standard library.
For a real task like this though, we'll want to at least use something
to parse CSV data, parse the program arguments and decode that stuff
into Rust types automatically. For that, we'll use the
\href{https://crates.io/crates/csv}{\texttt{csv}}, and
\href{https://crates.io/crates/rustc-serialize}{\texttt{rustc-serialize}}
crates.

\hypertarget{initial-setup}{\subsubsection{Initial
setup}\label{initial-setup}}

We're not going to spend a lot of time on setting up a project with
Cargo because it is already covered well in
\protect\hyperlink{hello-cargo}{the Cargo section} and
\href{http://doc.crates.io/guide.html}{Cargo's documentation}.

To get started from scratch, run \texttt{cargo\ new\ -\/-bin\ city-pop}
and make sure your \texttt{Cargo.toml} looks something like this:

\begin{verbatim}
[package]
name = "city-pop"
version = "0.1.0"
authors = ["Andrew Gallant <jamslam@gmail.com>"]

[[bin]]
name = "city-pop"

[dependencies]
csv = "0.*"
rustc-serialize = "0.*"
getopts = "0.*"
\end{verbatim}

You should already be able to run:

\begin{verbatim}
cargo build --release
./target/release/city-pop
# Outputs: Hello, world!
\end{verbatim}

\hypertarget{argument-parsing}{\subsubsection{Argument
parsing}\label{argument-parsing}}

Let's get argument parsing out of the way. We won't go into too much
detail on Getopts, but there is
\href{http://doc.rust-lang.org/getopts/getopts/index.html}{some good
documentation} describing it. The short story is that Getopts generates
an argument parser and a help message from a vector of options (The fact
that it is a vector is hidden behind a struct and a set of methods).
Once the parsing is done, the parser returns a struct that records
matches for defined options, and remaining ``free'' arguments. From
there, we can get information about the flags, for instance, whether
they were passed in, and what arguments they had. Here's our program
with the appropriate \texttt{extern\ crate} statements, and the basic
argument setup for Getopts:

\begin{Shaded}
\begin{Highlighting}[]
\KeywordTok{extern} \KeywordTok{crate} \NormalTok{getopts;}
\KeywordTok{extern} \KeywordTok{crate} \NormalTok{rustc_serialize;}

\KeywordTok{use} \NormalTok{getopts::Options;}
\KeywordTok{use} \NormalTok{std::env;}

\KeywordTok{fn} \NormalTok{print_usage(program: &}\DataTypeTok{str}\NormalTok{, opts: Options) \{}
    \PreprocessorTok{println!}\NormalTok{(}\StringTok{"\{\}"}\NormalTok{, opts.usage(&}\PreprocessorTok{format!}\NormalTok{(}\StringTok{"Usage: \{\} [options] <data-path> <city>"}\NormalTok{, progr}
\NormalTok{↳ am)));}
\NormalTok{\}}

\KeywordTok{fn} \NormalTok{main() \{}
    \KeywordTok{let} \NormalTok{args: }\DataTypeTok{Vec}\NormalTok{<}\DataTypeTok{String}\NormalTok{> = env::args().collect();}
    \KeywordTok{let} \NormalTok{program = &args[}\DecValTok{0}\NormalTok{];}

    \KeywordTok{let} \KeywordTok{mut} \NormalTok{opts = Options::new();}
    \NormalTok{opts.optflag(}\StringTok{"h"}\NormalTok{, }\StringTok{"help"}\NormalTok{, }\StringTok{"Show this usage message."}\NormalTok{);}

    \KeywordTok{let} \NormalTok{matches = }\KeywordTok{match} \NormalTok{opts.parse(&args[}\DecValTok{1.}\NormalTok{.]) \{}
        \ConstantTok{Ok}\NormalTok{(m)  => \{ m \}}
        \ConstantTok{Err}\NormalTok{(e) => \{ }\PreprocessorTok{panic!}\NormalTok{(e.to_string()) \}}
    \NormalTok{\};}
    \KeywordTok{if} \NormalTok{matches.opt_present(}\StringTok{"h"}\NormalTok{) \{}
        \NormalTok{print_usage(&program, opts);}
        \KeywordTok{return}\NormalTok{;}
    \NormalTok{\}}
    \KeywordTok{let} \NormalTok{data_path = &matches.free[}\DecValTok{0}\NormalTok{];}
    \KeywordTok{let} \NormalTok{city: &}\DataTypeTok{str} \NormalTok{= &matches.free[}\DecValTok{1}\NormalTok{];}

    \CommentTok{// Do stuff with information}
\NormalTok{\}}
\end{Highlighting}
\end{Shaded}

First, we get a vector of the arguments passed into our program. We then
store the first one, knowing that it is our program's name. Once that's
done, we set up our argument flags, in this case a simplistic help
message flag. Once we have the argument flags set up, we use
\texttt{Options.parse} to parse the argument vector (starting from index
one, because index 0 is the program name). If this was successful, we
assign matches to the parsed object, if not, we panic. Once past that,
we test if the user passed in the help flag, and if so print the usage
message. The option help messages are constructed by Getopts, so all we
have to do to print the usage message is tell it what we want it to
print for the program name and template. If the user has not passed in
the help flag, we assign the proper variables to their corresponding
arguments.

\hypertarget{writing-the-logic}{\subsubsection{Writing the
logic}\label{writing-the-logic}}

We all write code differently, but error handling is usually the last
thing we want to think about. This isn't great for the overall design of
a program, but it can be useful for rapid prototyping. Because Rust
forces us to be explicit about error handling (by making us call
\texttt{unwrap}), it is easy to see which parts of our program can cause
errors.

In this case study, the logic is really simple. All we need to do is
parse the CSV data given to us and print out a field in matching rows.
Let's do it. (Make sure to add \texttt{extern\ crate\ csv;} to the top
of your file.)

\begin{Shaded}
\begin{Highlighting}[]
\KeywordTok{use} \NormalTok{std::fs::File;}

\CommentTok{// This struct represents the data in each row of the CSV file.}
\CommentTok{// Type based decoding absolves us of a lot of the nitty gritty error}
\CommentTok{// handling, like parsing strings as integers or floats.}
\AttributeTok{#[}\NormalTok{derive}\AttributeTok{(}\BuiltInTok{Debug}\AttributeTok{,} \NormalTok{RustcDecodable}\AttributeTok{)]}
\KeywordTok{struct} \NormalTok{Row \{}
    \NormalTok{country: }\DataTypeTok{String}\NormalTok{,}
    \NormalTok{city: }\DataTypeTok{String}\NormalTok{,}
    \NormalTok{accent_city: }\DataTypeTok{String}\NormalTok{,}
    \NormalTok{region: }\DataTypeTok{String}\NormalTok{,}

    \CommentTok{// Not every row has data for the population, latitude or longitude!}
    \CommentTok{// So we express them as `Option` types, which admits the possibility of}
    \CommentTok{// absence. The CSV parser will fill in the correct value for us.}
    \NormalTok{population: }\DataTypeTok{Option}\NormalTok{<}\DataTypeTok{u64}\NormalTok{>,}
    \NormalTok{latitude: }\DataTypeTok{Option}\NormalTok{<}\DataTypeTok{f64}\NormalTok{>,}
    \NormalTok{longitude: }\DataTypeTok{Option}\NormalTok{<}\DataTypeTok{f64}\NormalTok{>,}
\NormalTok{\}}

\KeywordTok{fn} \NormalTok{print_usage(program: &}\DataTypeTok{str}\NormalTok{, opts: Options) \{}
    \PreprocessorTok{println!}\NormalTok{(}\StringTok{"\{\}"}\NormalTok{, opts.usage(&}\PreprocessorTok{format!}\NormalTok{(}\StringTok{"Usage: \{\} [options] <data-path> <city>"}\NormalTok{, progr}
\NormalTok{↳ am)));}
\NormalTok{\}}

\KeywordTok{fn} \NormalTok{main() \{}
    \KeywordTok{let} \NormalTok{args: }\DataTypeTok{Vec}\NormalTok{<}\DataTypeTok{String}\NormalTok{> = env::args().collect();}
    \KeywordTok{let} \NormalTok{program = &args[}\DecValTok{0}\NormalTok{];}

    \KeywordTok{let} \KeywordTok{mut} \NormalTok{opts = Options::new();}
    \NormalTok{opts.optflag(}\StringTok{"h"}\NormalTok{, }\StringTok{"help"}\NormalTok{, }\StringTok{"Show this usage message."}\NormalTok{);}

    \KeywordTok{let} \NormalTok{matches = }\KeywordTok{match} \NormalTok{opts.parse(&args[}\DecValTok{1.}\NormalTok{.]) \{}
        \ConstantTok{Ok}\NormalTok{(m)  => \{ m \}}
        \ConstantTok{Err}\NormalTok{(e) => \{ }\PreprocessorTok{panic!}\NormalTok{(e.to_string()) \}}
    \NormalTok{\};}

    \KeywordTok{if} \NormalTok{matches.opt_present(}\StringTok{"h"}\NormalTok{) \{}
        \NormalTok{print_usage(&program, opts);}
        \KeywordTok{return}\NormalTok{;}
    \NormalTok{\}}

    \KeywordTok{let} \NormalTok{data_path = &matches.free[}\DecValTok{0}\NormalTok{];}
    \KeywordTok{let} \NormalTok{city: &}\DataTypeTok{str} \NormalTok{= &matches.free[}\DecValTok{1}\NormalTok{];}

    \KeywordTok{let} \NormalTok{file = File::open(data_path).unwrap();}
    \KeywordTok{let} \KeywordTok{mut} \NormalTok{rdr = csv::Reader::from_reader(file);}

    \KeywordTok{for} \NormalTok{row }\KeywordTok{in} \NormalTok{rdr.decode::<Row>() \{}
        \KeywordTok{let} \NormalTok{row = row.unwrap();}

        \KeywordTok{if} \NormalTok{row.city == city \{}
            \PreprocessorTok{println!}\NormalTok{(}\StringTok{"\{\}, \{\}: \{:?\}"}\NormalTok{,}
                \NormalTok{row.city, row.country,}
                \NormalTok{row.population.expect(}\StringTok{"population count"}\NormalTok{));}
        \NormalTok{\}}
    \NormalTok{\}}
\NormalTok{\}}
\end{Highlighting}
\end{Shaded}

Let's outline the errors. We can start with the obvious: the three
places that \texttt{unwrap} is called:

\begin{enumerate}
\def\labelenumi{\arabic{enumi}.}
\tightlist
\item
  \href{http://doc.rust-lang.org/std/fs/struct.File.html\#method.open}{\texttt{File::open}}
  can return an
  \href{http://doc.rust-lang.org/std/io/struct.Error.html}{\texttt{io::Error}}.
\item
  \href{http://burntsushi.net/rustdoc/csv/struct.Reader.html\#method.decode}{\texttt{csv::Reader::decode}}
  decodes one record at a time, and
  \href{http://burntsushi.net/rustdoc/csv/struct.DecodedRecords.html}{decoding
  a record} (look at the \texttt{Item} associated type on the
  \texttt{Iterator} impl) can produce a
  \href{http://burntsushi.net/rustdoc/csv/enum.Error.html}{\texttt{csv::Error}}.
\item
  If \texttt{row.population} is \texttt{None}, then calling
  \texttt{expect} will panic.
\end{enumerate}

Are there any others? What if we can't find a matching city? Tools like
\texttt{grep} will return an error code, so we probably should too. So
we have logic errors specific to our problem, IO errors and CSV parsing
errors. We're going to explore two different ways to approach handling
these errors.

I'd like to start with \texttt{Box\textless{}Error\textgreater{}}.
Later, we'll see how defining our own error type can be useful.

\hypertarget{error-handling-with-boxerror}{\subsubsection{\texorpdfstring{Error
handling with
\texttt{Box\textless{}Error\textgreater{}}}{Error handling with Box\textless{}Error\textgreater{}}}\label{error-handling-with-boxerror}}

\texttt{Box\textless{}Error\textgreater{}} is nice because it \emph{just
works}. You don't need to define your own error types and you don't need
any \texttt{From} implementations. The downside is that since
\texttt{Box\textless{}Error\textgreater{}} is a trait object, it
\emph{erases the type}, which means the compiler can no longer reason
about its underlying type.

\protect\hyperlink{the-limits-of-combinators}{Previously} we started
refactoring our code by changing the type of our function from
\texttt{T} to \texttt{Result\textless{}T,\ OurErrorType\textgreater{}}.
In this case, \texttt{OurErrorType} is only
\texttt{Box\textless{}Error\textgreater{}}. But what's \texttt{T}? And
can we add a return type to \texttt{main}?

The answer to the second question is no, we can't. That means we'll need
to write a new function. But what is \texttt{T}? The simplest thing we
can do is to return a list of matching \texttt{Row} values as a
\texttt{Vec\textless{}Row\textgreater{}}. (Better code would return an
iterator, but that is left as an exercise to the reader.)

Let's refactor our code into its own function, but keep the calls to
\texttt{unwrap}. Note that we opt to handle the possibility of a missing
population count by simply ignoring that row.

\begin{Shaded}
\begin{Highlighting}[]
\KeywordTok{use} \NormalTok{std::path::Path;}

\KeywordTok{struct} \NormalTok{Row \{}
    \CommentTok{// unchanged}
\NormalTok{\}}

\KeywordTok{struct} \NormalTok{PopulationCount \{}
    \NormalTok{city: }\DataTypeTok{String}\NormalTok{,}
    \NormalTok{country: }\DataTypeTok{String}\NormalTok{,}
    \CommentTok{// This is no longer an `Option` because values of this type are only}
    \CommentTok{// constructed if they have a population count.}
    \NormalTok{count: }\DataTypeTok{u64}\NormalTok{,}
\NormalTok{\}}

\KeywordTok{fn} \NormalTok{print_usage(program: &}\DataTypeTok{str}\NormalTok{, opts: Options) \{}
    \PreprocessorTok{println!}\NormalTok{(}\StringTok{"\{\}"}\NormalTok{, opts.usage(&}\PreprocessorTok{format!}\NormalTok{(}\StringTok{"Usage: \{\} [options] <data-path> <city>"}\NormalTok{, progr}
\NormalTok{↳ am)));}
\NormalTok{\}}

\KeywordTok{fn} \NormalTok{search<P: AsRef<Path>>(file_path: P, city: &}\DataTypeTok{str}\NormalTok{) -> }\DataTypeTok{Vec}\NormalTok{<PopulationCount> \{}
    \KeywordTok{let} \KeywordTok{mut} \NormalTok{found = }\PreprocessorTok{vec!}\NormalTok{[];}
    \KeywordTok{let} \NormalTok{file = File::open(file_path).unwrap();}
    \KeywordTok{let} \KeywordTok{mut} \NormalTok{rdr = csv::Reader::from_reader(file);}
    \KeywordTok{for} \NormalTok{row }\KeywordTok{in} \NormalTok{rdr.decode::<Row>() \{}
        \KeywordTok{let} \NormalTok{row = row.unwrap();}
        \KeywordTok{match} \NormalTok{row.population \{}
            \ConstantTok{None} \NormalTok{=> \{ \} }\CommentTok{// skip it}
            \ConstantTok{Some}\NormalTok{(count) => }\KeywordTok{if} \NormalTok{row.city == city \{}
                \NormalTok{found.push(PopulationCount \{}
                    \NormalTok{city: row.city,}
                    \NormalTok{country: row.country,}
                    \NormalTok{count: count,}
                \NormalTok{\});}
            \NormalTok{\},}
        \NormalTok{\}}
    \NormalTok{\}}
    \NormalTok{found}
\NormalTok{\}}

\KeywordTok{fn} \NormalTok{main() \{}
    \KeywordTok{let} \NormalTok{args: }\DataTypeTok{Vec}\NormalTok{<}\DataTypeTok{String}\NormalTok{> = env::args().collect();}
    \KeywordTok{let} \NormalTok{program = &args[}\DecValTok{0}\NormalTok{];}

    \KeywordTok{let} \KeywordTok{mut} \NormalTok{opts = Options::new();}
    \NormalTok{opts.optflag(}\StringTok{"h"}\NormalTok{, }\StringTok{"help"}\NormalTok{, }\StringTok{"Show this usage message."}\NormalTok{);}

    \KeywordTok{let} \NormalTok{matches = }\KeywordTok{match} \NormalTok{opts.parse(&args[}\DecValTok{1.}\NormalTok{.]) \{}
        \ConstantTok{Ok}\NormalTok{(m)  => \{ m \}}
        \ConstantTok{Err}\NormalTok{(e) => \{ }\PreprocessorTok{panic!}\NormalTok{(e.to_string()) \}}
    \NormalTok{\};}

    \KeywordTok{if} \NormalTok{matches.opt_present(}\StringTok{"h"}\NormalTok{) \{}
        \NormalTok{print_usage(&program, opts);}
        \KeywordTok{return}\NormalTok{;}
    \NormalTok{\}}

    \KeywordTok{let} \NormalTok{data_path = &matches.free[}\DecValTok{0}\NormalTok{];}
    \KeywordTok{let} \NormalTok{city: &}\DataTypeTok{str} \NormalTok{= &matches.free[}\DecValTok{1}\NormalTok{];}

    \KeywordTok{for} \NormalTok{pop }\KeywordTok{in} \NormalTok{search(data_path, city) \{}
        \PreprocessorTok{println!}\NormalTok{(}\StringTok{"\{\}, \{\}: \{:?\}"}\NormalTok{, pop.city, pop.country, pop.count);}
    \NormalTok{\}}
\NormalTok{\}}
\end{Highlighting}
\end{Shaded}

While we got rid of one use of \texttt{expect} (which is a nicer variant
of \texttt{unwrap}), we still should handle the absence of any search
results.

To convert this to proper error handling, we need to do the following:

\begin{enumerate}
\def\labelenumi{\arabic{enumi}.}
\tightlist
\item
  Change the return type of \texttt{search} to be
  \texttt{Result\textless{}Vec\textless{}PopulationCount\textgreater{},\ \ \ \ Box\textless{}Error\textgreater{}\textgreater{}}.
\item
  Use the \protect\hyperlink{code-try-def}{\texttt{try!} macro} so that
  errors are returned to the caller instead of panicking the program.
\item
  Handle the error in \texttt{main}.
\end{enumerate}

Let's try it:

\begin{Shaded}
\begin{Highlighting}[]
\KeywordTok{use} \NormalTok{std::error::Error;}

\CommentTok{// The rest of the code before this is unchanged}

\KeywordTok{fn} \NormalTok{search<P: AsRef<Path>>}
         \NormalTok{(file_path: P, city: &}\DataTypeTok{str}\NormalTok{)}
         \NormalTok{-> }\DataTypeTok{Result}\NormalTok{<}\DataTypeTok{Vec}\NormalTok{<PopulationCount>, }\DataTypeTok{Box}\NormalTok{<Error+}\BuiltInTok{Send}\NormalTok{+}\BuiltInTok{Sync}\NormalTok{>> \{}
    \KeywordTok{let} \KeywordTok{mut} \NormalTok{found = }\PreprocessorTok{vec!}\NormalTok{[];}
    \KeywordTok{let} \NormalTok{file = }\PreprocessorTok{try!}\NormalTok{(File::open(file_path));}
    \KeywordTok{let} \KeywordTok{mut} \NormalTok{rdr = csv::Reader::from_reader(file);}
    \KeywordTok{for} \NormalTok{row }\KeywordTok{in} \NormalTok{rdr.decode::<Row>() \{}
        \KeywordTok{let} \NormalTok{row = }\PreprocessorTok{try!}\NormalTok{(row);}
        \KeywordTok{match} \NormalTok{row.population \{}
            \ConstantTok{None} \NormalTok{=> \{ \} }\CommentTok{// skip it}
            \ConstantTok{Some}\NormalTok{(count) => }\KeywordTok{if} \NormalTok{row.city == city \{}
                \NormalTok{found.push(PopulationCount \{}
                    \NormalTok{city: row.city,}
                    \NormalTok{country: row.country,}
                    \NormalTok{count: count,}
                \NormalTok{\});}
            \NormalTok{\},}
        \NormalTok{\}}
    \NormalTok{\}}
    \KeywordTok{if} \NormalTok{found.is_empty() \{}
        \ConstantTok{Err}\NormalTok{(From::from(}\StringTok{"No matching cities with a population were found."}\NormalTok{))}
    \NormalTok{\} }\KeywordTok{else} \NormalTok{\{}
        \ConstantTok{Ok}\NormalTok{(found)}
    \NormalTok{\}}
\NormalTok{\}}
\end{Highlighting}
\end{Shaded}

Instead of \texttt{x.unwrap()}, we now have \texttt{try!(x)}. Since our
function returns a \texttt{Result\textless{}T,\ E\textgreater{}}, the
\texttt{try!} macro will return early from the function if an error
occurs.

There is one big gotcha in this code: we used
\texttt{Box\textless{}Error\ +\ Send\ +\ Sync\textgreater{}} instead of
\texttt{Box\textless{}Error\textgreater{}}. We did this so we could
convert a plain string to an error type. We need these extra bounds so
that we can use the
\href{http://doc.rust-lang.org/std/convert/trait.From.html}{corresponding
\texttt{From} impls}:

\begin{Shaded}
\begin{Highlighting}[]
\CommentTok{// We are making use of this impl in the code above, since we call `From::from`}
\CommentTok{// on a `&'static str`.}
\KeywordTok{impl}\NormalTok{<}\OtherTok{'a}\NormalTok{, }\OtherTok{'b}\NormalTok{> From<&}\OtherTok{'b} \DataTypeTok{str}\NormalTok{> }\KeywordTok{for} \DataTypeTok{Box}\NormalTok{<Error + }\BuiltInTok{Send} \NormalTok{+ }\BuiltInTok{Sync} \NormalTok{+ }\OtherTok{'a}\NormalTok{>}

\CommentTok{// But this is also useful when you need to allocate a new string for an}
\CommentTok{// error message, usually with `format!`.}
\KeywordTok{impl} \NormalTok{From<}\DataTypeTok{String}\NormalTok{> }\KeywordTok{for} \DataTypeTok{Box}\NormalTok{<Error + }\BuiltInTok{Send} \NormalTok{+ }\BuiltInTok{Sync}\NormalTok{>}
\end{Highlighting}
\end{Shaded}

Since \texttt{search} now returns a
\texttt{Result\textless{}T,\ E\textgreater{}}, \texttt{main} should use
case analysis when calling \texttt{search}:

\begin{Shaded}
\begin{Highlighting}[]
\NormalTok{...}
    \KeywordTok{match} \NormalTok{search(data_path, city) \{}
        \ConstantTok{Ok}\NormalTok{(pops) => \{}
            \KeywordTok{for} \NormalTok{pop }\KeywordTok{in} \NormalTok{pops \{}
                \PreprocessorTok{println!}\NormalTok{(}\StringTok{"\{\}, \{\}: \{:?\}"}\NormalTok{, pop.city, pop.country, pop.count);}
            \NormalTok{\}}
        \NormalTok{\}}
        \ConstantTok{Err}\NormalTok{(err) => }\PreprocessorTok{println!}\NormalTok{(}\StringTok{"\{\}"}\NormalTok{, err)}
    \NormalTok{\}}
\NormalTok{...}
\end{Highlighting}
\end{Shaded}

Now that we've seen how to do proper error handling with
\texttt{Box\textless{}Error\textgreater{}}, let's try a different
approach with our own custom error type. But first, let's take a quick
break from error handling and add support for reading from
\texttt{stdin}.

\hypertarget{reading-from-stdin}{\subsubsection{Reading from
stdin}\label{reading-from-stdin}}

In our program, we accept a single file for input and do one pass over
the data. This means we probably should be able to accept input on
stdin. But maybe we like the current format too---so let's have both!

Adding support for stdin is actually quite easy. There are only three
things we have to do:

\begin{enumerate}
\def\labelenumi{\arabic{enumi}.}
\tightlist
\item
  Tweak the program arguments so that a single parameter---the
  city---can be accepted while the population data is read from stdin.
\item
  Modify the program so that an option \texttt{-f} can take the file, if
  it is not passed into stdin.
\item
  Modify the \texttt{search} function to take an \emph{optional} file
  path. When \texttt{None}, it should know to read from stdin.
\end{enumerate}

First, here's the new usage:

\begin{Shaded}
\begin{Highlighting}[]
\KeywordTok{fn} \NormalTok{print_usage(program: &}\DataTypeTok{str}\NormalTok{, opts: Options) \{}
    \PreprocessorTok{println!}\NormalTok{(}\StringTok{"\{\}"}\NormalTok{, opts.usage(&}\PreprocessorTok{format!}\NormalTok{(}\StringTok{"Usage: \{\} [options] <city>"}\NormalTok{, program)));}
\NormalTok{\}}
\end{Highlighting}
\end{Shaded}

Of course we need to adapt the argument handling code:

\begin{Shaded}
\begin{Highlighting}[]
\NormalTok{...}
    \KeywordTok{let} \KeywordTok{mut} \NormalTok{opts = Options::new();}
    \NormalTok{opts.optopt(}\StringTok{"f"}\NormalTok{, }\StringTok{"file"}\NormalTok{, }\StringTok{"Choose an input file, instead of using STDIN."}\NormalTok{, }\StringTok{"NAME"}\NormalTok{);}
    \NormalTok{opts.optflag(}\StringTok{"h"}\NormalTok{, }\StringTok{"help"}\NormalTok{, }\StringTok{"Show this usage message."}\NormalTok{);}
    \NormalTok{...}
    \KeywordTok{let} \NormalTok{data_path = matches.opt_str(}\StringTok{"f"}\NormalTok{);}

    \KeywordTok{let} \NormalTok{city = }\KeywordTok{if} \NormalTok{!matches.free.is_empty() \{}
        \NormalTok{&matches.free[}\DecValTok{0}\NormalTok{]}
    \NormalTok{\} }\KeywordTok{else} \NormalTok{\{}
        \NormalTok{print_usage(&program, opts);}
        \KeywordTok{return}\NormalTok{;}
    \NormalTok{\};}

    \KeywordTok{match} \NormalTok{search(&data_path, city) \{}
        \ConstantTok{Ok}\NormalTok{(pops) => \{}
            \KeywordTok{for} \NormalTok{pop }\KeywordTok{in} \NormalTok{pops \{}
                \PreprocessorTok{println!}\NormalTok{(}\StringTok{"\{\}, \{\}: \{:?\}"}\NormalTok{, pop.city, pop.country, pop.count);}
            \NormalTok{\}}
        \NormalTok{\}}
        \ConstantTok{Err}\NormalTok{(err) => }\PreprocessorTok{println!}\NormalTok{(}\StringTok{"\{\}"}\NormalTok{, err)}
    \NormalTok{\}}
\NormalTok{...}
\end{Highlighting}
\end{Shaded}

We've made the user experience a bit nicer by showing the usage message,
instead of a panic from an out-of-bounds index, when \texttt{city}, the
remaining free argument, is not present.

Modifying \texttt{search} is slightly trickier. The \texttt{csv} crate
can build a parser out of
\href{http://burntsushi.net/rustdoc/csv/struct.Reader.html\#method.from_reader}{any
type that implements \texttt{io::Read}}. But how can we use the same
code over both types? There's actually a couple ways we could go about
this. One way is to write \texttt{search} such that it is generic on
some type parameter \texttt{R} that satisfies \texttt{io::Read}. Another
way is to use trait objects:

\begin{Shaded}
\begin{Highlighting}[]
\KeywordTok{use} \NormalTok{std::io;}

\CommentTok{// The rest of the code before this is unchanged}

\KeywordTok{fn} \NormalTok{search<P: AsRef<Path>>}
         \NormalTok{(file_path: &}\DataTypeTok{Option}\NormalTok{<P>, city: &}\DataTypeTok{str}\NormalTok{)}
         \NormalTok{-> }\DataTypeTok{Result}\NormalTok{<}\DataTypeTok{Vec}\NormalTok{<PopulationCount>, }\DataTypeTok{Box}\NormalTok{<Error+}\BuiltInTok{Send}\NormalTok{+}\BuiltInTok{Sync}\NormalTok{>> \{}
    \KeywordTok{let} \KeywordTok{mut} \NormalTok{found = }\PreprocessorTok{vec!}\NormalTok{[];}
    \KeywordTok{let} \NormalTok{input: }\DataTypeTok{Box}\NormalTok{<io::Read> = }\KeywordTok{match} \NormalTok{*file_path \{}
        \ConstantTok{None} \NormalTok{=> }\DataTypeTok{Box}\NormalTok{::new(io::stdin()),}
        \ConstantTok{Some}\NormalTok{(}\KeywordTok{ref} \NormalTok{file_path) => }\DataTypeTok{Box}\NormalTok{::new(}\PreprocessorTok{try!}\NormalTok{(File::open(file_path))),}
    \NormalTok{\};}
    \KeywordTok{let} \KeywordTok{mut} \NormalTok{rdr = csv::Reader::from_reader(input);}
    \CommentTok{// The rest remains unchanged!}
\NormalTok{\}}
\end{Highlighting}
\end{Shaded}

\hypertarget{error-handling-with-a-custom-type}{\subsubsection{Error
handling with a custom type}\label{error-handling-with-a-custom-type}}

Previously, we learned how to
\protect\hyperlink{composing-custom-error-types}{compose errors using a
custom error type}. We did this by defining our error type as an
\texttt{enum} and implementing \texttt{Error} and \texttt{From}.

Since we have three distinct errors (IO, CSV parsing and not found),
let's define an \texttt{enum} with three variants:

\begin{Shaded}
\begin{Highlighting}[]
\AttributeTok{#[}\NormalTok{derive}\AttributeTok{(}\BuiltInTok{Debug}\AttributeTok{)]}
\KeywordTok{enum} \NormalTok{CliError \{}
    \NormalTok{Io(io::Error),}
    \NormalTok{Csv(csv::Error),}
    \NormalTok{NotFound,}
\NormalTok{\}}
\end{Highlighting}
\end{Shaded}

And now for impls on \texttt{Display} and \texttt{Error}:

\begin{Shaded}
\begin{Highlighting}[]
\KeywordTok{use} \NormalTok{std::fmt;}

\KeywordTok{impl} \NormalTok{fmt::}\BuiltInTok{Display} \KeywordTok{for} \NormalTok{CliError \{}
    \KeywordTok{fn} \NormalTok{fmt(&}\KeywordTok{self}\NormalTok{, f: &}\KeywordTok{mut} \NormalTok{fmt::Formatter) -> fmt::}\DataTypeTok{Result} \NormalTok{\{}
        \KeywordTok{match} \NormalTok{*}\KeywordTok{self} \NormalTok{\{}
            \NormalTok{CliError::Io(}\KeywordTok{ref} \NormalTok{err) => err.fmt(f),}
            \NormalTok{CliError::Csv(}\KeywordTok{ref} \NormalTok{err) => err.fmt(f),}
            \NormalTok{CliError::NotFound => }\PreprocessorTok{write!}\NormalTok{(f, }\StringTok{"No matching cities with a }\SpecialCharTok{\textbackslash{}}
\StringTok{                                             population were found."}\NormalTok{),}
        \NormalTok{\}}
    \NormalTok{\}}
\NormalTok{\}}

\KeywordTok{impl} \NormalTok{Error }\KeywordTok{for} \NormalTok{CliError \{}
    \KeywordTok{fn} \NormalTok{description(&}\KeywordTok{self}\NormalTok{) -> &}\DataTypeTok{str} \NormalTok{\{}
        \KeywordTok{match} \NormalTok{*}\KeywordTok{self} \NormalTok{\{}
            \NormalTok{CliError::Io(}\KeywordTok{ref} \NormalTok{err) => err.description(),}
            \NormalTok{CliError::Csv(}\KeywordTok{ref} \NormalTok{err) => err.description(),}
            \NormalTok{CliError::NotFound => }\StringTok{"not found"}\NormalTok{,}
        \NormalTok{\}}
    \NormalTok{\}}

    \KeywordTok{fn} \NormalTok{cause(&}\KeywordTok{self}\NormalTok{) -> }\DataTypeTok{Option}\NormalTok{<&Error> \{}
        \KeywordTok{match} \NormalTok{*}\KeywordTok{self} \NormalTok{\{}
            \NormalTok{CliError::Io(}\KeywordTok{ref} \NormalTok{err) => }\ConstantTok{Some}\NormalTok{(err),}
            \NormalTok{CliError::Csv(}\KeywordTok{ref} \NormalTok{err) => }\ConstantTok{Some}\NormalTok{(err),}
            \CommentTok{// Our custom error doesn't have an underlying cause,}
            \CommentTok{// but we could modify it so that it does.}
            \NormalTok{CliError::NotFound => }\ConstantTok{None}\NormalTok{,}
        \NormalTok{\}}
    \NormalTok{\}}
\NormalTok{\}}
\end{Highlighting}
\end{Shaded}

Before we can use our \texttt{CliError} type in our \texttt{search}
function, we need to provide a couple \texttt{From} impls. How do we
know which impls to provide? Well, we'll need to convert from both
\texttt{io::Error} and \texttt{csv::Error} to \texttt{CliError}. Those
are the only external errors, so we'll only need two \texttt{From} impls
for now:

\begin{Shaded}
\begin{Highlighting}[]
\KeywordTok{impl} \NormalTok{From<io::Error> }\KeywordTok{for} \NormalTok{CliError \{}
    \KeywordTok{fn} \NormalTok{from(err: io::Error) -> CliError \{}
        \NormalTok{CliError::Io(err)}
    \NormalTok{\}}
\NormalTok{\}}

\KeywordTok{impl} \NormalTok{From<csv::Error> }\KeywordTok{for} \NormalTok{CliError \{}
    \KeywordTok{fn} \NormalTok{from(err: csv::Error) -> CliError \{}
        \NormalTok{CliError::Csv(err)}
    \NormalTok{\}}
\NormalTok{\}}
\end{Highlighting}
\end{Shaded}

The \texttt{From} impls are important because of how
\protect\hyperlink{code-try-def}{\texttt{try!} is defined}. In
particular, if an error occurs, \texttt{From::from} is called on the
error, which in this case, will convert it to our own error type
\texttt{CliError}.

With the \texttt{From} impls done, we only need to make two small tweaks
to our \texttt{search} function: the return type and the ``not found''
error. Here it is in full:

\begin{Shaded}
\begin{Highlighting}[]
\KeywordTok{fn} \NormalTok{search<P: AsRef<Path>>}
         \NormalTok{(file_path: &}\DataTypeTok{Option}\NormalTok{<P>, city: &}\DataTypeTok{str}\NormalTok{)}
         \NormalTok{-> }\DataTypeTok{Result}\NormalTok{<}\DataTypeTok{Vec}\NormalTok{<PopulationCount>, CliError> \{}
    \KeywordTok{let} \KeywordTok{mut} \NormalTok{found = }\PreprocessorTok{vec!}\NormalTok{[];}
    \KeywordTok{let} \NormalTok{input: }\DataTypeTok{Box}\NormalTok{<io::Read> = }\KeywordTok{match} \NormalTok{*file_path \{}
        \ConstantTok{None} \NormalTok{=> }\DataTypeTok{Box}\NormalTok{::new(io::stdin()),}
        \ConstantTok{Some}\NormalTok{(}\KeywordTok{ref} \NormalTok{file_path) => }\DataTypeTok{Box}\NormalTok{::new(}\PreprocessorTok{try!}\NormalTok{(File::open(file_path))),}
    \NormalTok{\};}
    \KeywordTok{let} \KeywordTok{mut} \NormalTok{rdr = csv::Reader::from_reader(input);}
    \KeywordTok{for} \NormalTok{row }\KeywordTok{in} \NormalTok{rdr.decode::<Row>() \{}
        \KeywordTok{let} \NormalTok{row = }\PreprocessorTok{try!}\NormalTok{(row);}
        \KeywordTok{match} \NormalTok{row.population \{}
            \ConstantTok{None} \NormalTok{=> \{ \} }\CommentTok{// skip it}
            \ConstantTok{Some}\NormalTok{(count) => }\KeywordTok{if} \NormalTok{row.city == city \{}
                \NormalTok{found.push(PopulationCount \{}
                    \NormalTok{city: row.city,}
                    \NormalTok{country: row.country,}
                    \NormalTok{count: count,}
                \NormalTok{\});}
            \NormalTok{\},}
        \NormalTok{\}}
    \NormalTok{\}}
    \KeywordTok{if} \NormalTok{found.is_empty() \{}
        \ConstantTok{Err}\NormalTok{(CliError::NotFound)}
    \NormalTok{\} }\KeywordTok{else} \NormalTok{\{}
        \ConstantTok{Ok}\NormalTok{(found)}
    \NormalTok{\}}
\NormalTok{\}}
\end{Highlighting}
\end{Shaded}

No other changes are necessary.

\hypertarget{adding-functionality}{\subsubsection{Adding
functionality}\label{adding-functionality}}

Writing generic code is great, because generalizing stuff is cool, and
it can then be useful later. But sometimes, the juice isn't worth the
squeeze. Look at what we just did in the previous step:

\begin{enumerate}
\def\labelenumi{\arabic{enumi}.}
\tightlist
\item
  Defined a new error type.
\item
  Added impls for \texttt{Error}, \texttt{Display} and two for
  \texttt{From}.
\end{enumerate}

The big downside here is that our program didn't improve a whole lot.
There is quite a bit of overhead to representing errors with
\texttt{enum}s, especially in short programs like this.

\emph{One} useful aspect of using a custom error type like we've done
here is that the \texttt{main} function can now choose to handle errors
differently. Previously, with
\texttt{Box\textless{}Error\textgreater{}}, it didn't have much of a
choice: just print the message. We're still doing that here, but what if
we wanted to, say, add a \texttt{-\/-quiet} flag? The \texttt{-\/-quiet}
flag should silence any verbose output.

Right now, if the program doesn't find a match, it will output a message
saying so. This can be a little clumsy, especially if you intend for the
program to be used in shell scripts.

So let's start by adding the flags. Like before, we need to tweak the
usage string and add a flag to the Option variable. Once we've done
that, Getopts does the rest:

\begin{Shaded}
\begin{Highlighting}[]
\NormalTok{...}
    \KeywordTok{let} \KeywordTok{mut} \NormalTok{opts = Options::new();}
    \NormalTok{opts.optopt(}\StringTok{"f"}\NormalTok{, }\StringTok{"file"}\NormalTok{, }\StringTok{"Choose an input file, instead of using STDIN."}\NormalTok{, }\StringTok{"NAME"}\NormalTok{);}
    \NormalTok{opts.optflag(}\StringTok{"h"}\NormalTok{, }\StringTok{"help"}\NormalTok{, }\StringTok{"Show this usage message."}\NormalTok{);}
    \NormalTok{opts.optflag(}\StringTok{"q"}\NormalTok{, }\StringTok{"quiet"}\NormalTok{, }\StringTok{"Silences errors and warnings."}\NormalTok{);}
\NormalTok{...}
\end{Highlighting}
\end{Shaded}

Now we only need to implement our ``quiet'' functionality. This requires
us to tweak the case analysis in \texttt{main}:

\begin{Shaded}
\begin{Highlighting}[]
\KeywordTok{use} \NormalTok{std::process;}
\NormalTok{...}
    \KeywordTok{match} \NormalTok{search(&data_path, city) \{}
        \ConstantTok{Err}\NormalTok{(CliError::NotFound) }\KeywordTok{if} \NormalTok{matches.opt_present(}\StringTok{"q"}\NormalTok{) => process::exit(}\DecValTok{1}\NormalTok{),}
        \ConstantTok{Err}\NormalTok{(err) => }\PreprocessorTok{panic!}\NormalTok{(}\StringTok{"\{\}"}\NormalTok{, err),}
        \ConstantTok{Ok}\NormalTok{(pops) => }\KeywordTok{for} \NormalTok{pop }\KeywordTok{in} \NormalTok{pops \{}
            \PreprocessorTok{println!}\NormalTok{(}\StringTok{"\{\}, \{\}: \{:?\}"}\NormalTok{, pop.city, pop.country, pop.count);}
        \NormalTok{\}}
    \NormalTok{\}}
\NormalTok{...}
\end{Highlighting}
\end{Shaded}

Certainly, we don't want to be quiet if there was an IO error or if the
data failed to parse. Therefore, we use case analysis to check if the
error type is \texttt{NotFound} \emph{and} if \texttt{-\/-quiet} has
been enabled. If the search failed, we still quit with an exit code
(following \texttt{grep}'s convention).

If we had stuck with \texttt{Box\textless{}Error\textgreater{}}, then it
would be pretty tricky to implement the \texttt{-\/-quiet}
functionality.

This pretty much sums up our case study. From here, you should be ready
to go out into the world and write your own programs and libraries with
proper error handling.

\hypertarget{the-short-story}{\subsection{The Short
Story}\label{the-short-story}}

Since this section is long, it is useful to have a quick summary for
error handling in Rust. These are some good ``rules of thumb.'' They are
emphatically \emph{not} commandments. There are probably good reasons to
break every one of these heuristics!

\begin{itemize}
\tightlist
\item
  If you're writing short example code that would be overburdened by
  error handling, it's probably fine to use \texttt{unwrap} (whether
  that's
  \href{http://doc.rust-lang.org/std/result/enum.Result.html\#method.unwrap}{\texttt{Result::unwrap}},
  \href{http://doc.rust-lang.org/std/option/enum.Option.html\#method.unwrap}{\texttt{Option::unwrap}}
  or preferably
  \href{http://doc.rust-lang.org/std/option/enum.Option.html\#method.expect}{\texttt{Option::expect}}).
  Consumers of your code should know to use proper error handling. (If
  they don't, send them here!)
\item
  If you're writing a quick `n' dirty program, don't feel ashamed if you
  use \texttt{unwrap}. Be warned: if it winds up in someone else's
  hands, don't be surprised if they are agitated by poor error messages!
\item
  If you're writing a quick `n' dirty program and feel ashamed about
  panicking anyway, then use either a \texttt{String} or a
  \texttt{Box\textless{}Error\ +\ Send\ +\ Sync\textgreater{}} for your
  error type (the
  \texttt{Box\textless{}Error\ +\ Send\ +\ Sync\textgreater{}} type is
  because of the
  \href{http://doc.rust-lang.org/std/convert/trait.From.html}{available
  \texttt{From} impls}).
\item
  Otherwise, in a program, define your own error types with appropriate
  \href{http://doc.rust-lang.org/std/convert/trait.From.html}{\texttt{From}}
  and
  \href{http://doc.rust-lang.org/std/error/trait.Error.html}{\texttt{Error}}
  impls to make the
  \href{http://doc.rust-lang.org/std/macro.try!.html}{\texttt{try!}}
  macro more ergonomic.
\item
  If you're writing a library and your code can produce errors, define
  your own error type and implement the
  \href{http://doc.rust-lang.org/std/error/trait.Error.html}{\texttt{std::error::Error}}
  trait. Where appropriate, implement
  \href{http://doc.rust-lang.org/std/convert/trait.From.html}{\texttt{From}}
  to make both your library code and the caller's code easier to write.
  (Because of Rust's coherence rules, callers will not be able to impl
  \texttt{From} on your error type, so your library should do it.)
\item
  Learn the combinators defined on
  \href{http://doc.rust-lang.org/std/option/enum.Option.html}{\texttt{Option}}
  and
  \href{http://doc.rust-lang.org/std/result/enum.Result.html}{\texttt{Result}}.
  Using them exclusively can be a bit tiring at times, but I've
  personally found a healthy mix of \texttt{try!} and combinators to be
  quite appealing. \texttt{and\_then}, \texttt{map} and
  \texttt{unwrap\_or} are my favorites.
\end{itemize}

\hypertarget{sec--choosing-your-guarantees}{\section{Choosing your
Guarantees}\label{sec--choosing-your-guarantees}}

One important feature of Rust is that it lets us control the costs and
guarantees of a program.

There are various ``wrapper type'' abstractions in the Rust standard
library which embody a multitude of tradeoffs between cost, ergonomics,
and guarantees. Many let one choose between run time and compile time
enforcement. This section will explain a few selected abstractions in
detail.

Before proceeding, it is highly recommended that one reads about
\protect\hyperlink{sec--ownership}{ownership} and
\protect\hyperlink{sec--references-and-borrowing}{borrowing} in Rust.

\subsection{Basic pointer types}\label{basic-pointer-types}

\subsubsection{\texorpdfstring{\texttt{Box\textless{}T\textgreater{}}}{Box\textless{}T\textgreater{}}}\label{boxt}

\href{http://doc.rust-lang.org/std/boxed/struct.Box.html}{\texttt{Box\textless{}T\textgreater{}}}
is an ``owned'' pointer, or a ``box''. While it can hand out references
to the contained data, it is the only owner of the data. In particular,
consider the following:

\begin{Shaded}
\begin{Highlighting}[]
\KeywordTok{let} \NormalTok{x = }\DataTypeTok{Box}\NormalTok{::new(}\DecValTok{1}\NormalTok{);}
\KeywordTok{let} \NormalTok{y = x;}
\CommentTok{// x no longer accessible here}
\end{Highlighting}
\end{Shaded}

Here, the box was \emph{moved} into \texttt{y}. As \texttt{x} no longer
owns it, the compiler will no longer allow the programmer to use
\texttt{x} after this. A box can similarly be moved \emph{out} of a
function by returning it.

When a box (that hasn't been moved) goes out of scope, destructors are
run. These destructors take care of deallocating the inner data.

This is a zero-cost abstraction for dynamic allocation. If you want to
allocate some memory on the heap and safely pass around a pointer to
that memory, this is ideal. Note that you will only be allowed to share
references to this by the regular borrowing rules, checked at compile
time.

\subsubsection{\texorpdfstring{\texttt{\&T} and
\texttt{\&mut\ T}}{\&T and \&mut T}}\label{t-and-mut-t}

These are immutable and mutable references respectively. They follow the
``read-write lock'' pattern, such that one may either have only one
mutable reference to some data, or any number of immutable ones, but not
both. This guarantee is enforced at compile time, and has no visible
cost at runtime. In most cases these two pointer types suffice for
sharing cheap references between sections of code.

These pointers cannot be copied in such a way that they outlive the
lifetime associated with them.

\subsubsection{\texorpdfstring{\texttt{*const\ T} and
\texttt{*mut\ T}}{*const T and *mut T}}\label{const-t-and-mut-t}

These are C-like raw pointers with no lifetime or ownership attached to
them. They point to some location in memory with no other restrictions.
The only guarantee that these provide is that they cannot be
dereferenced except in code marked \texttt{unsafe}.

These are useful when building safe, low cost abstractions like
\texttt{Vec\textless{}T\textgreater{}}, but should be avoided in safe
code.

\subsubsection{\texorpdfstring{\texttt{Rc\textless{}T\textgreater{}}}{Rc\textless{}T\textgreater{}}}\label{rct}

This is the first wrapper we will cover that has a runtime cost.

\href{http://doc.rust-lang.org/std/rc/struct.Rc.html}{\texttt{Rc\textless{}T\textgreater{}}}
is a reference counted pointer. In other words, this lets us have
multiple ``owning'' pointers to the same data, and the data will be
dropped (destructors will be run) when all pointers are out of scope.

Internally, it contains a shared ``reference count'' (also called
``refcount''), which is incremented each time the \texttt{Rc} is cloned,
and decremented each time one of the \texttt{Rc}s goes out of scope. The
main responsibility of \texttt{Rc\textless{}T\textgreater{}} is to
ensure that destructors are called for shared data.

The internal data here is immutable, and if a cycle of references is
created, the data will be leaked. If we want data that doesn't leak when
there are cycles, we need a garbage collector.

\subparagraph{Guarantees}\label{guarantees}

The main guarantee provided here is that the data will not be destroyed
until all references to it are out of scope.

This should be used when we wish to dynamically allocate and share some
data (read-only) between various portions of your program, where it is
not certain which portion will finish using the pointer last. It's a
viable alternative to \texttt{\&T} when \texttt{\&T} is either
impossible to statically check for correctness, or creates extremely
unergonomic code where the programmer does not wish to spend the
development cost of working with.

This pointer is \emph{not} thread safe, and Rust will not let it be sent
or shared with other threads. This lets one avoid the cost of atomics in
situations where they are unnecessary.

There is a sister smart pointer to this one,
\texttt{Weak\textless{}T\textgreater{}}. This is a non-owning, but also
non-borrowed, smart pointer. It is also similar to \texttt{\&T}, but it
is not restricted in lifetime---a
\texttt{Weak\textless{}T\textgreater{}} can be held on to forever.
However, it is possible that an attempt to access the inner data may
fail and return \texttt{None}, since this can outlive the owned
\texttt{Rc}s. This is useful for cyclic data structures and other
things.

\subparagraph{Cost}\label{cost}

As far as memory goes, \texttt{Rc\textless{}T\textgreater{}} is a single
allocation, though it will allocate two extra words (i.e.~two
\texttt{usize} values) as compared to a regular
\texttt{Box\textless{}T\textgreater{}} (for ``strong'' and ``weak''
refcounts).

\texttt{Rc\textless{}T\textgreater{}} has the computational cost of
incrementing/decrementing the refcount whenever it is cloned or goes out
of scope respectively. Note that a clone will not do a deep copy, rather
it will simply increment the inner reference count and return a copy of
the \texttt{Rc\textless{}T\textgreater{}}.

\subsection{Cell types}\label{cell-types}

\texttt{Cell}s provide interior mutability. In other words, they contain
data which can be manipulated even if the type cannot be obtained in a
mutable form (for example, when it is behind an \texttt{\&}-ptr or
\texttt{Rc\textless{}T\textgreater{}}).

\href{http://doc.rust-lang.org/std/cell/}{The documentation for the
\texttt{cell} module has a pretty good explanation for these}.

These types are \emph{generally} found in struct fields, but they may be
found elsewhere too.

\subsubsection{\texorpdfstring{\texttt{Cell\textless{}T\textgreater{}}}{Cell\textless{}T\textgreater{}}}\label{cellt}

\href{http://doc.rust-lang.org/std/cell/struct.Cell.html}{\texttt{Cell\textless{}T\textgreater{}}}
is a type that provides zero-cost interior mutability, but only for
\texttt{Copy} types. Since the compiler knows that all the data owned by
the contained value is on the stack, there's no worry of leaking any
data behind references (or worse!) by simply replacing the data.

It is still possible to violate your own invariants using this wrapper,
so be careful when using it. If a field is wrapped in \texttt{Cell},
it's a nice indicator that the chunk of data is mutable and may not stay
the same between the time you first read it and when you intend to use
it.

\begin{Shaded}
\begin{Highlighting}[]
\KeywordTok{use} \NormalTok{std::cell::Cell;}

\KeywordTok{let} \NormalTok{x = Cell::new(}\DecValTok{1}\NormalTok{);}
\KeywordTok{let} \NormalTok{y = &x;}
\KeywordTok{let} \NormalTok{z = &x;}
\NormalTok{x.set(}\DecValTok{2}\NormalTok{);}
\NormalTok{y.set(}\DecValTok{3}\NormalTok{);}
\NormalTok{z.set(}\DecValTok{4}\NormalTok{);}
\PreprocessorTok{println!}\NormalTok{(}\StringTok{"\{\}"}\NormalTok{, x.get());}
\end{Highlighting}
\end{Shaded}

Note that here we were able to mutate the same value from various
immutable references.

This has the same runtime cost as the following:

\begin{Shaded}
\begin{Highlighting}[]
\KeywordTok{let} \KeywordTok{mut} \NormalTok{x = }\DecValTok{1}\NormalTok{;}
\KeywordTok{let} \NormalTok{y = &}\KeywordTok{mut} \NormalTok{x;}
\KeywordTok{let} \NormalTok{z = &}\KeywordTok{mut} \NormalTok{x;}
\NormalTok{x = }\DecValTok{2}\NormalTok{;}
\NormalTok{*y = }\DecValTok{3}\NormalTok{;}
\NormalTok{*z = }\DecValTok{4}\NormalTok{;}
\PreprocessorTok{println!}\NormalTok{(}\StringTok{"\{\}"}\NormalTok{, x);}
\end{Highlighting}
\end{Shaded}

but it has the added benefit of actually compiling successfully.

\subparagraph{Guarantees}\label{guarantees-1}

This relaxes the ``no aliasing with mutability'' restriction in places
where it's unnecessary. However, this also relaxes the guarantees that
the restriction provides; so if your invariants depend on data stored
within \texttt{Cell}, you should be careful.

This is useful for mutating primitives and other \texttt{Copy} types
when there is no easy way of doing it in line with the static rules of
\texttt{\&} and \texttt{\&mut}.

\texttt{Cell} does not let you obtain interior references to the data,
which makes it safe to freely mutate.

\subparagraph{Cost}\label{cost-1}

There is no runtime cost to using
\texttt{Cell\textless{}T\textgreater{}}, however if you are using it to
wrap larger (\texttt{Copy}) structs, it might be worthwhile to instead
wrap individual fields in \texttt{Cell\textless{}T\textgreater{}} since
each write is otherwise a full copy of the struct.

\hypertarget{refcellt}{\subsubsection{\texorpdfstring{\texttt{RefCell\textless{}T\textgreater{}}}{RefCell\textless{}T\textgreater{}}}\label{refcellt}}

\href{http://doc.rust-lang.org/std/cell/struct.RefCell.html}{\texttt{RefCell\textless{}T\textgreater{}}}
also provides interior mutability, but isn't restricted to \texttt{Copy}
types.

Instead, it has a runtime cost.
\texttt{RefCell\textless{}T\textgreater{}} enforces the read-write lock
pattern at runtime (it's like a single-threaded mutex), unlike
\texttt{\&T}/\texttt{\&mut\ T} which do so at compile time. This is done
by the \texttt{borrow()} and \texttt{borrow\_mut()} functions, which
modify an internal reference count and return smart pointers which can
be dereferenced immutably and mutably respectively. The refcount is
restored when the smart pointers go out of scope. With this system, we
can dynamically ensure that there are never any other borrows active
when a mutable borrow is active. If the programmer attempts to make such
a borrow, the thread will panic.

\begin{Shaded}
\begin{Highlighting}[]
\KeywordTok{use} \NormalTok{std::cell::RefCell;}

\KeywordTok{let} \NormalTok{x = RefCell::new(}\PreprocessorTok{vec!}\NormalTok{[}\DecValTok{1}\NormalTok{,}\DecValTok{2}\NormalTok{,}\DecValTok{3}\NormalTok{,}\DecValTok{4}\NormalTok{]);}
\NormalTok{\{}
    \PreprocessorTok{println!}\NormalTok{(}\StringTok{"\{:?\}"}\NormalTok{, *x.borrow())}
\NormalTok{\}}

\NormalTok{\{}
    \KeywordTok{let} \KeywordTok{mut} \NormalTok{my_ref = x.borrow_mut();}
    \NormalTok{my_ref.push(}\DecValTok{1}\NormalTok{);}
\NormalTok{\}}
\end{Highlighting}
\end{Shaded}

Similar to \texttt{Cell}, this is mainly useful for situations where
it's hard or impossible to satisfy the borrow checker. Generally we know
that such mutations won't happen in a nested form, but it's good to
check.

For large, complicated programs, it becomes useful to put some things in
\texttt{RefCell}s to make things simpler. For example, a lot of the maps
in the \texttt{ctxt} struct in the Rust compiler internals are inside
this wrapper. These are only modified once (during creation, which is
not right after initialization) or a couple of times in well-separated
places. However, since this struct is pervasively used everywhere,
juggling mutable and immutable pointers would be hard (perhaps
impossible) and probably form a soup of \texttt{\&}-ptrs which would be
hard to extend. On the other hand, the \texttt{RefCell} provides a cheap
(not zero-cost) way of safely accessing these. In the future, if someone
adds some code that attempts to modify the cell when it's already
borrowed, it will cause a (usually deterministic) panic which can be
traced back to the offending borrow.

Similarly, in Servo's DOM there is a lot of mutation, most of which is
local to a DOM type, but some of which crisscrosses the DOM and modifies
various things. Using \texttt{RefCell} and \texttt{Cell} to guard all
mutation lets us avoid worrying about mutability everywhere, and it
simultaneously highlights the places where mutation is \emph{actually}
happening.

Note that \texttt{RefCell} should be avoided if a mostly simple solution
is possible with \texttt{\&} pointers.

\subparagraph{Guarantees}\label{guarantees-2}

\texttt{RefCell} relaxes the \emph{static} restrictions preventing
aliased mutation, and replaces them with \emph{dynamic} ones. As such
the guarantees have not changed.

\subparagraph{Cost}\label{cost-2}

\texttt{RefCell} does not allocate, but it contains an additional
``borrow state'' indicator (one word in size) along with the data.

At runtime each borrow causes a modification/check of the refcount.

\subsection{Synchronous types}\label{synchronous-types}

Many of the types above cannot be used in a threadsafe manner.
Particularly, \texttt{Rc\textless{}T\textgreater{}} and
\texttt{RefCell\textless{}T\textgreater{}}, which both use non-atomic
reference counts (\emph{atomic} reference counts are those which can be
incremented from multiple threads without causing a data race), cannot
be used this way. This makes them cheaper to use, but we need thread
safe versions of these too. They exist, in the form of
\texttt{Arc\textless{}T\textgreater{}} and
\texttt{Mutex\textless{}T\textgreater{}}/\texttt{RwLock\textless{}T\textgreater{}}

Note that the non-threadsafe types \emph{cannot} be sent between
threads, and this is checked at compile time.

There are many useful wrappers for concurrent programming in the
\href{http://doc.rust-lang.org/std/sync/index.html}{sync} module, but
only the major ones will be covered below.

\subsubsection{\texorpdfstring{\texttt{Arc\textless{}T\textgreater{}}}{Arc\textless{}T\textgreater{}}}\label{arct}

\href{http://doc.rust-lang.org/std/sync/struct.Arc.html}{\texttt{Arc\textless{}T\textgreater{}}}
is a version of \texttt{Rc\textless{}T\textgreater{}} that uses an
atomic reference count (hence, ``Arc''). This can be sent freely between
threads.

C++'s \texttt{shared\_ptr} is similar to \texttt{Arc}, however in the
case of C++ the inner data is always mutable. For semantics similar to
that from C++, we should use
\texttt{Arc\textless{}Mutex\textless{}T\textgreater{}\textgreater{}},
\texttt{Arc\textless{}RwLock\textless{}T\textgreater{}\textgreater{}},
or
\texttt{Arc\textless{}UnsafeCell\textless{}T\textgreater{}\textgreater{}}\footnote{\texttt{Arc\textless{}UnsafeCell\textless{}T\textgreater{}\textgreater{}}
  actually won't compile since
  \texttt{UnsafeCell\textless{}T\textgreater{}} isn't \texttt{Send} or
  \texttt{Sync}, but we can wrap it in a type and implement
  \texttt{Send}/\texttt{Sync} for it manually to get
  \texttt{Arc\textless{}Wrapper\textless{}T\textgreater{}\textgreater{}}
  where \texttt{Wrapper} is
  \texttt{struct\ Wrapper\textless{}T\textgreater{}(UnsafeCell\textless{}T\textgreater{})}.}
(\texttt{UnsafeCell\textless{}T\textgreater{}} is a cell type that can
be used to hold any data and has no runtime cost, but accessing it
requires \texttt{unsafe} blocks). The last one should only be used if we
are certain that the usage won't cause any memory unsafety. Remember
that writing to a struct is not an atomic operation, and many functions
like \texttt{vec.push()} can reallocate internally and cause unsafe
behavior, so even monotonicity may not be enough to justify
\texttt{UnsafeCell}.

\subparagraph{Guarantees}\label{guarantees-3}

Like \texttt{Rc}, this provides the (thread safe) guarantee that the
destructor for the internal data will be run when the last \texttt{Arc}
goes out of scope (barring any cycles).

\subparagraph{Cost}\label{cost-3}

This has the added cost of using atomics for changing the refcount
(which will happen whenever it is cloned or goes out of scope). When
sharing data from an \texttt{Arc} in a single thread, it is preferable
to share \texttt{\&} pointers whenever possible.

\subsubsection{\texorpdfstring{\texttt{Mutex\textless{}T\textgreater{}}
and
\texttt{RwLock\textless{}T\textgreater{}}}{Mutex\textless{}T\textgreater{} and RwLock\textless{}T\textgreater{}}}\label{mutext-and-rwlockt}

\href{http://doc.rust-lang.org/std/sync/struct.Mutex.html}{\texttt{Mutex\textless{}T\textgreater{}}}
and
\href{http://doc.rust-lang.org/std/sync/struct.RwLock.html}{\texttt{RwLock\textless{}T\textgreater{}}}
provide mutual-exclusion via RAII guards (guards are objects which
maintain some state, like a lock, until their destructor is called). For
both of these, the mutex is opaque until we call \texttt{lock()} on it,
at which point the thread will block until a lock can be acquired, and
then a guard will be returned. This guard can be used to access the
inner data (mutably), and the lock will be released when the guard goes
out of scope.

\begin{Shaded}
\begin{Highlighting}[]
\NormalTok{\{}
    \KeywordTok{let} \NormalTok{guard = mutex.lock();}
    \CommentTok{// guard dereferences mutably to the inner type}
    \NormalTok{*guard += }\DecValTok{1}\NormalTok{;}
\NormalTok{\} }\CommentTok{// lock released when destructor runs}
\end{Highlighting}
\end{Shaded}

\texttt{RwLock} has the added benefit of being efficient for multiple
reads. It is always safe to have multiple readers to shared data as long
as there are no writers; and \texttt{RwLock} lets readers acquire a
``read lock''. Such locks can be acquired concurrently and are kept
track of via a reference count. Writers must obtain a ``write lock''
which can only be obtained when all readers have gone out of scope.

\subparagraph{Guarantees}\label{guarantees-4}

Both of these provide safe shared mutability across threads, however
they are prone to deadlocks. Some level of additional protocol safety
can be obtained via the type system.

\subparagraph{Costs}\label{costs}

These use internal atomic-like types to maintain the locks, which are
pretty costly (they can block all memory reads across processors till
they're done). Waiting on these locks can also be slow when there's a
lot of concurrent access happening.

\subsection{Composition}\label{composition}

A common gripe when reading Rust code is with types like
\texttt{Rc\textless{}RefCell\textless{}Vec\textless{}T\textgreater{}\textgreater{}\textgreater{}}
(or even more complicated compositions of such types). It's not always
clear what the composition does, or why the author chose one like this
(and when one should be using such a composition in one's own code)

Usually, it's a case of composing together the guarantees that you need,
without paying for stuff that is unnecessary.

For example,
\texttt{Rc\textless{}RefCell\textless{}T\textgreater{}\textgreater{}} is
one such composition. \texttt{Rc\textless{}T\textgreater{}} itself can't
be dereferenced mutably; because \texttt{Rc\textless{}T\textgreater{}}
provides sharing and shared mutability can lead to unsafe behavior, so
we put \texttt{RefCell\textless{}T\textgreater{}} inside to get
dynamically verified shared mutability. Now we have shared mutable data,
but it's shared in a way that there can only be one mutator (and no
readers) or multiple readers.

Now, we can take this a step further, and have
\texttt{Rc\textless{}RefCell\textless{}Vec\textless{}T\textgreater{}\textgreater{}\textgreater{}}
or
\texttt{Rc\textless{}Vec\textless{}RefCell\textless{}T\textgreater{}\textgreater{}\textgreater{}}.
These are both shareable, mutable vectors, but they're not the same.

With the former, the \texttt{RefCell\textless{}T\textgreater{}} is
wrapping the \texttt{Vec\textless{}T\textgreater{}}, so the
\texttt{Vec\textless{}T\textgreater{}} in its entirety is mutable. At
the same time, there can only be one mutable borrow of the whole
\texttt{Vec} at a given time. This means that your code cannot
simultaneously work on different elements of the vector from different
\texttt{Rc} handles. However, we are able to push and pop from the
\texttt{Vec\textless{}T\textgreater{}} at will. This is similar to a
\texttt{\&mut\ Vec\textless{}T\textgreater{}} with the borrow checking
done at runtime.

With the latter, the borrowing is of individual elements, but the
overall vector is immutable. Thus, we can independently borrow separate
elements, but we cannot push or pop from the vector. This is similar to
a \texttt{\&mut\ {[}T{]}}\footnote{\texttt{\&{[}T{]}} and
  \texttt{\&mut\ {[}T{]}} are \emph{slices}; they consist of a pointer
  and a length and can refer to a portion of a vector or array.
  \texttt{\&mut\ {[}T{]}} can have its elements mutated, however its
  length cannot be touched.}, but, again, the borrow checking is at
runtime.

In concurrent programs, we have a similar situation with
\texttt{Arc\textless{}Mutex\textless{}T\textgreater{}\textgreater{}},
which provides shared mutability and ownership.

When reading code that uses these, go in step by step and look at the
guarantees/costs provided.

When choosing a composed type, we must do the reverse; figure out which
guarantees we want, and at which point of the composition we need them.
For example, if there is a choice between
\texttt{Vec\textless{}RefCell\textless{}T\textgreater{}\textgreater{}}
and
\texttt{RefCell\textless{}Vec\textless{}T\textgreater{}\textgreater{}},
we should figure out the tradeoffs as done above and pick one.

\hypertarget{sec--ffi}{\section{FFI}\label{sec--ffi}}

\subsection{Introduction}\label{introduction-1}

This guide will use the \href{https://github.com/google/snappy}{snappy}
compression/decompression library as an introduction to writing bindings
for foreign code. Rust is currently unable to call directly into a C++
library, but snappy includes a C interface (documented in
\href{https://github.com/google/snappy/blob/master/snappy-c.h}{\texttt{snappy-c.h}}).

\subsubsection{A note about libc}\label{a-note-about-libc}

Many of these examples use \href{https://crates.io/crates/libc}{the
\texttt{libc} crate}, which provides various type definitions for C
types, among other things. If you're trying these examples yourself,
you'll need to add \texttt{libc} to your \texttt{Cargo.toml}:

\begin{verbatim}
[dependencies]
libc = "0.2.0"
\end{verbatim}

and add \texttt{extern\ crate\ libc;} to your crate root.

\subsubsection{Calling foreign
functions}\label{calling-foreign-functions}

The following is a minimal example of calling a foreign function which
will compile if snappy is installed:

\begin{Shaded}
\begin{Highlighting}[]
\KeywordTok{extern} \KeywordTok{crate} \NormalTok{libc;}
\KeywordTok{use} \NormalTok{libc::}\DataTypeTok{size_t}\NormalTok{;}

\AttributeTok{#[}\NormalTok{link}\AttributeTok{(}\NormalTok{name }\AttributeTok{=} \StringTok{"snappy"}\AttributeTok{)]}
\KeywordTok{extern} \NormalTok{\{}
    \KeywordTok{fn} \NormalTok{snappy_max_compressed_length(source_length: }\DataTypeTok{size_t}\NormalTok{) -> }\DataTypeTok{size_t}\NormalTok{;}
\NormalTok{\}}

\KeywordTok{fn} \NormalTok{main() \{}
    \KeywordTok{let} \NormalTok{x = }\KeywordTok{unsafe} \NormalTok{\{ snappy_max_compressed_length(}\DecValTok{100}\NormalTok{) \};}
    \PreprocessorTok{println!}\NormalTok{(}\StringTok{"max compressed length of a 100 byte buffer: \{\}"}\NormalTok{, x);}
\NormalTok{\}}
\end{Highlighting}
\end{Shaded}

The \texttt{extern} block is a list of function signatures in a foreign
library, in this case with the platform's C ABI. The
\texttt{\#{[}link(...){]}} attribute is used to instruct the linker to
link against the snappy library so the symbols are resolved.

Foreign functions are assumed to be unsafe so calls to them need to be
wrapped with \texttt{unsafe\ \{\}} as a promise to the compiler that
everything contained within truly is safe. C libraries often expose
interfaces that aren't thread-safe, and almost any function that takes a
pointer argument isn't valid for all possible inputs since the pointer
could be dangling, and raw pointers fall outside of Rust's safe memory
model.

When declaring the argument types to a foreign function, the Rust
compiler can not check if the declaration is correct, so specifying it
correctly is part of keeping the binding correct at runtime.

The \texttt{extern} block can be extended to cover the entire snappy
API:

\begin{Shaded}
\begin{Highlighting}[]
\KeywordTok{extern} \KeywordTok{crate} \NormalTok{libc;}
\KeywordTok{use} \NormalTok{libc::\{}\DataTypeTok{c_int}\NormalTok{, }\DataTypeTok{size_t}\NormalTok{\};}

\AttributeTok{#[}\NormalTok{link}\AttributeTok{(}\NormalTok{name }\AttributeTok{=} \StringTok{"snappy"}\AttributeTok{)]}
\KeywordTok{extern} \NormalTok{\{}
    \KeywordTok{fn} \NormalTok{snappy_compress(input: *}\KeywordTok{const} \DataTypeTok{u8}\NormalTok{,}
                       \NormalTok{input_length: }\DataTypeTok{size_t}\NormalTok{,}
                       \NormalTok{compressed: *}\KeywordTok{mut} \DataTypeTok{u8}\NormalTok{,}
                       \NormalTok{compressed_length: *}\KeywordTok{mut} \DataTypeTok{size_t}\NormalTok{) -> }\DataTypeTok{c_int}\NormalTok{;}
    \KeywordTok{fn} \NormalTok{snappy_uncompress(compressed: *}\KeywordTok{const} \DataTypeTok{u8}\NormalTok{,}
                         \NormalTok{compressed_length: }\DataTypeTok{size_t}\NormalTok{,}
                         \NormalTok{uncompressed: *}\KeywordTok{mut} \DataTypeTok{u8}\NormalTok{,}
                         \NormalTok{uncompressed_length: *}\KeywordTok{mut} \DataTypeTok{size_t}\NormalTok{) -> }\DataTypeTok{c_int}\NormalTok{;}
    \KeywordTok{fn} \NormalTok{snappy_max_compressed_length(source_length: }\DataTypeTok{size_t}\NormalTok{) -> }\DataTypeTok{size_t}\NormalTok{;}
    \KeywordTok{fn} \NormalTok{snappy_uncompressed_length(compressed: *}\KeywordTok{const} \DataTypeTok{u8}\NormalTok{,}
                                  \NormalTok{compressed_length: }\DataTypeTok{size_t}\NormalTok{,}
                                  \NormalTok{result: *}\KeywordTok{mut} \DataTypeTok{size_t}\NormalTok{) -> }\DataTypeTok{c_int}\NormalTok{;}
    \KeywordTok{fn} \NormalTok{snappy_validate_compressed_buffer(compressed: *}\KeywordTok{const} \DataTypeTok{u8}\NormalTok{,}
                                         \NormalTok{compressed_length: }\DataTypeTok{size_t}\NormalTok{) -> }\DataTypeTok{c_int}\NormalTok{;}
\NormalTok{\}}
\end{Highlighting}
\end{Shaded}

\subsection{Creating a safe interface}\label{creating-a-safe-interface}

The raw C API needs to be wrapped to provide memory safety and make use
of higher-level concepts like vectors. A library can choose to expose
only the safe, high-level interface and hide the unsafe internal
details.

Wrapping the functions which expect buffers involves using the
\texttt{slice::raw} module to manipulate Rust vectors as pointers to
memory. Rust's vectors are guaranteed to be a contiguous block of
memory. The length is number of elements currently contained, and the
capacity is the total size in elements of the allocated memory. The
length is less than or equal to the capacity.

\begin{Shaded}
\begin{Highlighting}[]
\KeywordTok{pub} \KeywordTok{fn} \NormalTok{validate_compressed_buffer(src: &[}\DataTypeTok{u8}\NormalTok{]) -> }\DataTypeTok{bool} \NormalTok{\{}
    \KeywordTok{unsafe} \NormalTok{\{}
        \NormalTok{snappy_validate_compressed_buffer(src.as_ptr(), src.len() }\KeywordTok{as} \DataTypeTok{size_t}\NormalTok{) == }\DecValTok{0}
    \NormalTok{\}}
\NormalTok{\}}
\end{Highlighting}
\end{Shaded}

The \texttt{validate\_compressed\_buffer} wrapper above makes use of an
\texttt{unsafe} block, but it makes the guarantee that calling it is
safe for all inputs by leaving off \texttt{unsafe} from the function
signature.

The \texttt{snappy\_compress} and \texttt{snappy\_uncompress} functions
are more complex, since a buffer has to be allocated to hold the output
too.

The \texttt{snappy\_max\_compressed\_length} function can be used to
allocate a vector with the maximum required capacity to hold the
compressed output. The vector can then be passed to the
\texttt{snappy\_compress} function as an output parameter. An output
parameter is also passed to retrieve the true length after compression
for setting the length.

\begin{Shaded}
\begin{Highlighting}[]
\KeywordTok{pub} \KeywordTok{fn} \NormalTok{compress(src: &[}\DataTypeTok{u8}\NormalTok{]) -> }\DataTypeTok{Vec}\NormalTok{<}\DataTypeTok{u8}\NormalTok{> \{}
    \KeywordTok{unsafe} \NormalTok{\{}
        \KeywordTok{let} \NormalTok{srclen = src.len() }\KeywordTok{as} \DataTypeTok{size_t}\NormalTok{;}
        \KeywordTok{let} \NormalTok{psrc = src.as_ptr();}

        \KeywordTok{let} \KeywordTok{mut} \NormalTok{dstlen = snappy_max_compressed_length(srclen);}
        \KeywordTok{let} \KeywordTok{mut} \NormalTok{dst = }\DataTypeTok{Vec}\NormalTok{::with_capacity(dstlen }\KeywordTok{as} \DataTypeTok{usize}\NormalTok{);}
        \KeywordTok{let} \NormalTok{pdst = dst.as_mut_ptr();}

        \NormalTok{snappy_compress(psrc, srclen, pdst, &}\KeywordTok{mut} \NormalTok{dstlen);}
        \NormalTok{dst.set_len(dstlen }\KeywordTok{as} \DataTypeTok{usize}\NormalTok{);}
        \NormalTok{dst}
    \NormalTok{\}}
\NormalTok{\}}
\end{Highlighting}
\end{Shaded}

Decompression is similar, because snappy stores the uncompressed size as
part of the compression format and \texttt{snappy\_uncompressed\_length}
will retrieve the exact buffer size required.

\begin{Shaded}
\begin{Highlighting}[]
\KeywordTok{pub} \KeywordTok{fn} \NormalTok{uncompress(src: &[}\DataTypeTok{u8}\NormalTok{]) -> }\DataTypeTok{Option}\NormalTok{<}\DataTypeTok{Vec}\NormalTok{<}\DataTypeTok{u8}\NormalTok{>> \{}
    \KeywordTok{unsafe} \NormalTok{\{}
        \KeywordTok{let} \NormalTok{srclen = src.len() }\KeywordTok{as} \DataTypeTok{size_t}\NormalTok{;}
        \KeywordTok{let} \NormalTok{psrc = src.as_ptr();}

        \KeywordTok{let} \KeywordTok{mut} \NormalTok{dstlen: }\DataTypeTok{size_t} \NormalTok{= }\DecValTok{0}\NormalTok{;}
        \NormalTok{snappy_uncompressed_length(psrc, srclen, &}\KeywordTok{mut} \NormalTok{dstlen);}

        \KeywordTok{let} \KeywordTok{mut} \NormalTok{dst = }\DataTypeTok{Vec}\NormalTok{::with_capacity(dstlen }\KeywordTok{as} \DataTypeTok{usize}\NormalTok{);}
        \KeywordTok{let} \NormalTok{pdst = dst.as_mut_ptr();}

        \KeywordTok{if} \NormalTok{snappy_uncompress(psrc, srclen, pdst, &}\KeywordTok{mut} \NormalTok{dstlen) == }\DecValTok{0} \NormalTok{\{}
            \NormalTok{dst.set_len(dstlen }\KeywordTok{as} \DataTypeTok{usize}\NormalTok{);}
            \ConstantTok{Some}\NormalTok{(dst)}
        \NormalTok{\} }\KeywordTok{else} \NormalTok{\{}
            \ConstantTok{None} \CommentTok{// SNAPPY_INVALID_INPUT}
        \NormalTok{\}}
    \NormalTok{\}}
\NormalTok{\}}
\end{Highlighting}
\end{Shaded}

For reference, the examples used here are also available as a
\href{https://github.com/thestinger/rust-snappy}{library on GitHub}.

\subsection{Destructors}\label{destructors}

Foreign libraries often hand off ownership of resources to the calling
code. When this occurs, we must use Rust's destructors to provide safety
and guarantee the release of these resources (especially in the case of
panic).

For more about destructors, see the
\href{http://doc.rust-lang.org/std/ops/trait.Drop.html}{Drop trait}.

\subsection{Callbacks from C code to Rust
functions}\label{callbacks-from-c-code-to-rust-functions}

Some external libraries require the usage of callbacks to report back
their current state or intermediate data to the caller. It is possible
to pass functions defined in Rust to an external library. The
requirement for this is that the callback function is marked as
\texttt{extern} with the correct calling convention to make it callable
from C code.

The callback function can then be sent through a registration call to
the C library and afterwards be invoked from there.

A basic example is:

Rust code:

\begin{Shaded}
\begin{Highlighting}[]
\KeywordTok{extern} \KeywordTok{fn} \NormalTok{callback(a: }\DataTypeTok{i32}\NormalTok{) \{}
    \PreprocessorTok{println!}\NormalTok{(}\StringTok{"I'm called from C with value \{0\}"}\NormalTok{, a);}
\NormalTok{\}}

\AttributeTok{#[}\NormalTok{link}\AttributeTok{(}\NormalTok{name }\AttributeTok{=} \StringTok{"extlib"}\AttributeTok{)]}
\KeywordTok{extern} \NormalTok{\{}
   \KeywordTok{fn} \NormalTok{register_callback(cb: }\KeywordTok{extern} \KeywordTok{fn}\NormalTok{(}\DataTypeTok{i32}\NormalTok{)) -> }\DataTypeTok{i32}\NormalTok{;}
   \KeywordTok{fn} \NormalTok{trigger_callback();}
\NormalTok{\}}

\KeywordTok{fn} \NormalTok{main() \{}
    \KeywordTok{unsafe} \NormalTok{\{}
        \NormalTok{register_callback(callback);}
        \NormalTok{trigger_callback(); }\CommentTok{// Triggers the callback}
    \NormalTok{\}}
\NormalTok{\}}
\end{Highlighting}
\end{Shaded}

C code:

\begin{Shaded}
\begin{Highlighting}[]
\KeywordTok{typedef} \DataTypeTok{void} \NormalTok{(*rust_callback)(}\DataTypeTok{int32_t}\NormalTok{);}
\NormalTok{rust_callback cb;}

\DataTypeTok{int32_t} \NormalTok{register_callback(rust_callback callback) \{}
    \NormalTok{cb = callback;}
    \KeywordTok{return} \DecValTok{1}\NormalTok{;}
\NormalTok{\}}

\DataTypeTok{void} \NormalTok{trigger_callback() \{}
  \NormalTok{cb(}\DecValTok{7}\NormalTok{); }\CommentTok{// Will call callback(7) in Rust}
\NormalTok{\}}
\end{Highlighting}
\end{Shaded}

In this example Rust's \texttt{main()} will call
\texttt{trigger\_callback()} in C, which would, in turn, call back to
\texttt{callback()} in Rust.

\subsubsection{Targeting callbacks to Rust
objects}\label{targeting-callbacks-to-rust-objects}

The former example showed how a global function can be called from C
code. However it is often desired that the callback is targeted to a
special Rust object. This could be the object that represents the
wrapper for the respective C object.

This can be achieved by passing an raw pointer to the object down to the
C library. The C library can then include the pointer to the Rust object
in the notification. This will allow the callback to unsafely access the
referenced Rust object.

Rust code:

\begin{Shaded}
\begin{Highlighting}[]
\AttributeTok{#[}\NormalTok{repr}\AttributeTok{(}\NormalTok{C}\AttributeTok{)]}
\KeywordTok{struct} \NormalTok{RustObject \{}
    \NormalTok{a: }\DataTypeTok{i32}\NormalTok{,}
    \CommentTok{// other members}
\NormalTok{\}}

\KeywordTok{extern} \StringTok{"C"} \KeywordTok{fn} \NormalTok{callback(target: *}\KeywordTok{mut} \NormalTok{RustObject, a: }\DataTypeTok{i32}\NormalTok{) \{}
    \PreprocessorTok{println!}\NormalTok{(}\StringTok{"I'm called from C with value \{0\}"}\NormalTok{, a);}
    \KeywordTok{unsafe} \NormalTok{\{}
        \CommentTok{// Update the value in RustObject with the value received from the callback}
        \NormalTok{(*target).a = a;}
    \NormalTok{\}}
\NormalTok{\}}

\AttributeTok{#[}\NormalTok{link}\AttributeTok{(}\NormalTok{name }\AttributeTok{=} \StringTok{"extlib"}\AttributeTok{)]}
\KeywordTok{extern} \NormalTok{\{}
   \KeywordTok{fn} \NormalTok{register_callback(target: *}\KeywordTok{mut} \NormalTok{RustObject,}
                        \NormalTok{cb: }\KeywordTok{extern} \KeywordTok{fn}\NormalTok{(*}\KeywordTok{mut} \NormalTok{RustObject, }\DataTypeTok{i32}\NormalTok{)) -> }\DataTypeTok{i32}\NormalTok{;}
   \KeywordTok{fn} \NormalTok{trigger_callback();}
\NormalTok{\}}

\KeywordTok{fn} \NormalTok{main() \{}
    \CommentTok{// Create the object that will be referenced in the callback}
    \KeywordTok{let} \KeywordTok{mut} \NormalTok{rust_object = }\DataTypeTok{Box}\NormalTok{::new(RustObject \{ a: }\DecValTok{5} \NormalTok{\});}

    \KeywordTok{unsafe} \NormalTok{\{}
        \NormalTok{register_callback(&}\KeywordTok{mut} \NormalTok{*rust_object, callback);}
        \NormalTok{trigger_callback();}
    \NormalTok{\}}
\NormalTok{\}}
\end{Highlighting}
\end{Shaded}

C code:

\begin{Shaded}
\begin{Highlighting}[]
\KeywordTok{typedef} \DataTypeTok{void} \NormalTok{(*rust_callback)(}\DataTypeTok{void}\NormalTok{*, }\DataTypeTok{int32_t}\NormalTok{);}
\DataTypeTok{void}\NormalTok{* cb_target;}
\NormalTok{rust_callback cb;}

\DataTypeTok{int32_t} \NormalTok{register_callback(}\DataTypeTok{void}\NormalTok{* callback_target, rust_callback callback) \{}
    \NormalTok{cb_target = callback_target;}
    \NormalTok{cb = callback;}
    \KeywordTok{return} \DecValTok{1}\NormalTok{;}
\NormalTok{\}}

\DataTypeTok{void} \NormalTok{trigger_callback() \{}
  \NormalTok{cb(cb_target, }\DecValTok{7}\NormalTok{); }\CommentTok{// Will call callback(&rustObject, 7) in Rust}
\NormalTok{\}}
\end{Highlighting}
\end{Shaded}

\subsubsection{Asynchronous callbacks}\label{asynchronous-callbacks}

In the previously given examples the callbacks are invoked as a direct
reaction to a function call to the external C library. The control over
the current thread is switched from Rust to C to Rust for the execution
of the callback, but in the end the callback is executed on the same
thread that called the function which triggered the callback.

Things get more complicated when the external library spawns its own
threads and invokes callbacks from there. In these cases access to Rust
data structures inside the callbacks is especially unsafe and proper
synchronization mechanisms must be used. Besides classical
synchronization mechanisms like mutexes, one possibility in Rust is to
use channels (in \texttt{std::sync::mpsc}) to forward data from the C
thread that invoked the callback into a Rust thread.

If an asynchronous callback targets a special object in the Rust address
space it is also absolutely necessary that no more callbacks are
performed by the C library after the respective Rust object gets
destroyed. This can be achieved by unregistering the callback in the
object's destructor and designing the library in a way that guarantees
that no callback will be performed after deregistration.

\subsection{Linking}\label{linking}

The \texttt{link} attribute on \texttt{extern} blocks provides the basic
building block for instructing rustc how it will link to native
libraries. There are two accepted forms of the link attribute today:

\begin{itemize}
\tightlist
\item
  \texttt{\#{[}link(name\ =\ "foo"){]}}
\item
  \texttt{\#{[}link(name\ =\ "foo",\ kind\ =\ "bar"){]}}
\end{itemize}

In both of these cases, \texttt{foo} is the name of the native library
that we're linking to, and in the second case \texttt{bar} is the type
of native library that the compiler is linking to. There are currently
three known types of native libraries:

\begin{itemize}
\tightlist
\item
  Dynamic - \texttt{\#{[}link(name\ =\ "readline"){]}}
\item
  Static -
  \texttt{\#{[}link(name\ =\ "my\_build\_dependency",\ kind\ =\ "static"){]}}
\item
  Frameworks -
  \texttt{\#{[}link(name\ =\ "CoreFoundation",\ kind\ =\ "framework"){]}}
\end{itemize}

Note that frameworks are only available on OSX targets.

The different \texttt{kind} values are meant to differentiate how the
native library participates in linkage. From a linkage perspective, the
Rust compiler creates two flavors of artifacts: partial (rlib/staticlib)
and final (dylib/binary). Native dynamic library and framework
dependencies are propagated to the final artifact boundary, while static
library dependencies are not propagated at all, because the static
libraries are integrated directly into the subsequent artifact.

A few examples of how this model can be used are:

\begin{itemize}
\tightlist
\item
  A native build dependency. Sometimes some C/C++ glue is needed when
  writing some Rust code, but distribution of the C/C++ code in a
  library format is a burden. In this case, the code will be archived
  into \texttt{libfoo.a} and then the Rust crate would declare a
  dependency via
  \texttt{\#{[}link(name\ =\ "foo",\ kind\ =\ \ \ "static"){]}}.
\end{itemize}

Regardless of the flavor of output for the crate, the native static
library will be included in the output, meaning that distribution of the
native static library is not necessary.

\begin{itemize}
\tightlist
\item
  A normal dynamic dependency. Common system libraries (like
  \texttt{readline}) are available on a large number of systems, and
  often a static copy of these libraries cannot be found. When this
  dependency is included in a Rust crate, partial targets (like rlibs)
  will not link to the library, but when the rlib is included in a final
  target (like a binary), the native library will be linked in.
\end{itemize}

On OSX, frameworks behave with the same semantics as a dynamic library.

\subsection{Unsafe blocks}\label{unsafe-blocks}

Some operations, like dereferencing raw pointers or calling functions
that have been marked unsafe are only allowed inside unsafe blocks.
Unsafe blocks isolate unsafety and are a promise to the compiler that
the unsafety does not leak out of the block.

Unsafe functions, on the other hand, advertise it to the world. An
unsafe function is written like this:

\begin{Shaded}
\begin{Highlighting}[]
\KeywordTok{unsafe} \KeywordTok{fn} \NormalTok{kaboom(ptr: *}\KeywordTok{const} \DataTypeTok{i32}\NormalTok{) -> }\DataTypeTok{i32} \NormalTok{\{ *ptr \}}
\end{Highlighting}
\end{Shaded}

This function can only be called from an \texttt{unsafe} block or
another \texttt{unsafe} function.

\subsection{Accessing foreign globals}\label{accessing-foreign-globals}

Foreign APIs often export a global variable which could do something
like track global state. In order to access these variables, you declare
them in \texttt{extern} blocks with the \texttt{static} keyword:

\begin{Shaded}
\begin{Highlighting}[]
\KeywordTok{extern} \KeywordTok{crate} \NormalTok{libc;}

\AttributeTok{#[}\NormalTok{link}\AttributeTok{(}\NormalTok{name }\AttributeTok{=} \StringTok{"readline"}\AttributeTok{)]}
\KeywordTok{extern} \NormalTok{\{}
    \KeywordTok{static} \NormalTok{rl_readline_version: libc::}\DataTypeTok{c_int}\NormalTok{;}
\NormalTok{\}}

\KeywordTok{fn} \NormalTok{main() \{}
    \PreprocessorTok{println!}\NormalTok{(}\StringTok{"You have readline version \{\} installed."}\NormalTok{,}
             \NormalTok{rl_readline_version }\KeywordTok{as} \DataTypeTok{i32}\NormalTok{);}
\NormalTok{\}}
\end{Highlighting}
\end{Shaded}

Alternatively, you may need to alter global state provided by a foreign
interface. To do this, statics can be declared with \texttt{mut} so we
can mutate them.

\begin{Shaded}
\begin{Highlighting}[]
\KeywordTok{extern} \KeywordTok{crate} \NormalTok{libc;}

\KeywordTok{use} \NormalTok{std::ffi::CString;}
\KeywordTok{use} \NormalTok{std::ptr;}

\AttributeTok{#[}\NormalTok{link}\AttributeTok{(}\NormalTok{name }\AttributeTok{=} \StringTok{"readline"}\AttributeTok{)]}
\KeywordTok{extern} \NormalTok{\{}
    \KeywordTok{static} \KeywordTok{mut} \NormalTok{rl_prompt: *}\KeywordTok{const} \NormalTok{libc::}\DataTypeTok{c_char}\NormalTok{;}
\NormalTok{\}}

\KeywordTok{fn} \NormalTok{main() \{}
    \KeywordTok{let} \NormalTok{prompt = CString::new(}\StringTok{"[my-awesome-shell] $"}\NormalTok{).unwrap();}
    \KeywordTok{unsafe} \NormalTok{\{}
        \NormalTok{rl_prompt = prompt.as_ptr();}

        \PreprocessorTok{println!}\NormalTok{(}\StringTok{"\{:?\}"}\NormalTok{, rl_prompt);}

        \NormalTok{rl_prompt = ptr::null();}
    \NormalTok{\}}
\NormalTok{\}}
\end{Highlighting}
\end{Shaded}

Note that all interaction with a \texttt{static\ mut} is unsafe, both
reading and writing. Dealing with global mutable state requires a great
deal of care.

\hypertarget{foreign-calling-conventions}{\subsection{Foreign calling
conventions}\label{foreign-calling-conventions}}

Most foreign code exposes a C ABI, and Rust uses the platform's C
calling convention by default when calling foreign functions. Some
foreign functions, most notably the Windows API, use other calling
conventions. Rust provides a way to tell the compiler which convention
to use:

\begin{Shaded}
\begin{Highlighting}[]
\KeywordTok{extern} \KeywordTok{crate} \NormalTok{libc;}

\AttributeTok{#[}\NormalTok{cfg}\AttributeTok{(}\NormalTok{all}\AttributeTok{(}\NormalTok{target_os }\AttributeTok{=} \StringTok{"win32"}\AttributeTok{,} \NormalTok{target_arch }\AttributeTok{=} \StringTok{"x86"}\AttributeTok{))]}
\AttributeTok{#[}\NormalTok{link}\AttributeTok{(}\NormalTok{name }\AttributeTok{=} \StringTok{"kernel32"}\AttributeTok{)]}
\AttributeTok{#[}\NormalTok{allow}\AttributeTok{(}\NormalTok{non_snake_case}\AttributeTok{)]}
\KeywordTok{extern} \StringTok{"stdcall"} \NormalTok{\{}
    \KeywordTok{fn} \NormalTok{SetEnvironmentVariableA(n: *}\KeywordTok{const} \DataTypeTok{u8}\NormalTok{, v: *}\KeywordTok{const} \DataTypeTok{u8}\NormalTok{) -> libc::}\DataTypeTok{c_int}\NormalTok{;}
\NormalTok{\}}
\end{Highlighting}
\end{Shaded}

This applies to the entire \texttt{extern} block. The list of supported
ABI constraints are:

\begin{itemize}
\tightlist
\item
  \texttt{stdcall}
\item
  \texttt{aapcs}
\item
  \texttt{cdecl}
\item
  \texttt{fastcall}
\item
  \texttt{vectorcall} This is currently hidden behind the
  \texttt{abi\_vectorcall} gate and is subject to change.
\item
  \texttt{Rust}
\item
  \texttt{rust-intrinsic}
\item
  \texttt{system}
\item
  \texttt{C}
\item
  \texttt{win64}
\end{itemize}

Most of the abis in this list are self-explanatory, but the
\texttt{system} abi may seem a little odd. This constraint selects
whatever the appropriate ABI is for interoperating with the target's
libraries. For example, on win32 with a x86 architecture, this means
that the abi used would be \texttt{stdcall}. On x86\_64, however,
windows uses the \texttt{C} calling convention, so \texttt{C} would be
used. This means that in our previous example, we could have used
\texttt{extern\ "system"\ \{\ ...\ \}} to define a block for all windows
systems, not only x86 ones.

\subsection{Interoperability with foreign
code}\label{interoperability-with-foreign-code}

Rust guarantees that the layout of a \texttt{struct} is compatible with
the platform's representation in C only if the \texttt{\#{[}repr(C){]}}
attribute is applied to it. \texttt{\#{[}repr(C,\ packed){]}} can be
used to lay out struct members without padding. \texttt{\#{[}repr(C){]}}
can also be applied to an enum.

Rust's owned boxes (\texttt{Box\textless{}T\textgreater{}}) use
non-nullable pointers as handles which point to the contained object.
However, they should not be manually created because they are managed by
internal allocators. References can safely be assumed to be non-nullable
pointers directly to the type. However, breaking the borrow checking or
mutability rules is not guaranteed to be safe, so prefer using raw
pointers (\texttt{*}) if that's needed because the compiler can't make
as many assumptions about them.

Vectors and strings share the same basic memory layout, and utilities
are available in the \texttt{vec} and \texttt{str} modules for working
with C APIs. However, strings are not terminated with
\texttt{\textbackslash{}0}. If you need a NUL-terminated string for
interoperability with C, you should use the \texttt{CString} type in the
\texttt{std::ffi} module.

The \href{https://crates.io/crates/libc}{\texttt{libc} crate on
crates.io} includes type aliases and function definitions for the C
standard library in the \texttt{libc} module, and Rust links against
\texttt{libc} and \texttt{libm} by default.

\subsection{\texorpdfstring{The ``nullable pointer
optimization''}{The nullable pointer optimization}}\label{the-nullable-pointer-optimization}

Certain types are defined to not be \texttt{null}. This includes
references (\texttt{\&T}, \texttt{\&mut\ T}), boxes
(\texttt{Box\textless{}T\textgreater{}}), and function pointers
(\texttt{extern\ "abi"\ fn()}). When interfacing with C, pointers that
might be null are often used. As a special case, a generic \texttt{enum}
that contains exactly two variants, one of which contains no data and
the other containing a single field, is eligible for the ``nullable
pointer optimization''. When such an enum is instantiated with one of
the non-nullable types, it is represented as a single pointer, and the
non-data variant is represented as the null pointer. So
\texttt{Option\textless{}extern\ "C"\ fn(c\_int)\ -\textgreater{}\ c\_int\textgreater{}}
is how one represents a nullable function pointer using the C ABI.

\subsection{Calling Rust code from C}\label{calling-rust-code-from-c}

You may wish to compile Rust code in a way so that it can be called from
C. This is fairly easy, but requires a few things:

\begin{Shaded}
\begin{Highlighting}[]
\AttributeTok{#[}\NormalTok{no_mangle}\AttributeTok{]}
\KeywordTok{pub} \KeywordTok{extern} \KeywordTok{fn} \NormalTok{hello_rust() -> *}\KeywordTok{const} \DataTypeTok{u8} \NormalTok{\{}
    \StringTok{"Hello, world!}\SpecialCharTok{\textbackslash{}}\ErrorTok{0}\StringTok{"}\NormalTok{.as_ptr()}
\NormalTok{\}}
\end{Highlighting}
\end{Shaded}

The \texttt{extern} makes this function adhere to the C calling
convention, as discussed above in
``\protect\hyperlink{foreign-calling-conventions}{Foreign Calling
Conventions}''. The \texttt{no\_mangle} attribute turns off Rust's name
mangling, so that it is easier to link to.

\subsection{FFI and panics}\label{ffi-and-panics}

It's important to be mindful of \texttt{panic!}s when working with FFI.
A \texttt{panic!} across an FFI boundary is undefined behavior. If
you're writing code that may panic, you should run it in another thread,
so that the panic doesn't bubble up to C:

\begin{Shaded}
\begin{Highlighting}[]
\KeywordTok{use} \NormalTok{std::thread;}

\AttributeTok{#[}\NormalTok{no_mangle}\AttributeTok{]}
\KeywordTok{pub} \KeywordTok{extern} \KeywordTok{fn} \NormalTok{oh_no() -> }\DataTypeTok{i32} \NormalTok{\{}
    \KeywordTok{let} \NormalTok{h = thread::spawn(|| \{}
        \PreprocessorTok{panic!}\NormalTok{(}\StringTok{"Oops!"}\NormalTok{);}
    \NormalTok{\});}

    \KeywordTok{match} \NormalTok{h.join() \{}
        \ConstantTok{Ok}\NormalTok{(_) => }\DecValTok{1}\NormalTok{,}
        \ConstantTok{Err}\NormalTok{(_) => }\DecValTok{0}\NormalTok{,}
    \NormalTok{\}}
\NormalTok{\}}
\end{Highlighting}
\end{Shaded}

\subsection{Representing opaque
structs}\label{representing-opaque-structs}

Sometimes, a C library wants to provide a pointer to something, but not
let you know the internal details of the thing it wants. The simplest
way is to use a \texttt{void\ *} argument:

\begin{Shaded}
\begin{Highlighting}[]
\DataTypeTok{void} \NormalTok{foo(}\DataTypeTok{void} \NormalTok{*arg);}
\DataTypeTok{void} \NormalTok{bar(}\DataTypeTok{void} \NormalTok{*arg);}
\end{Highlighting}
\end{Shaded}

We can represent this in Rust with the \texttt{c\_void} type:

\begin{Shaded}
\begin{Highlighting}[]
\KeywordTok{extern} \KeywordTok{crate} \NormalTok{libc;}

\KeywordTok{extern} \StringTok{"C"} \NormalTok{\{}
    \KeywordTok{pub} \KeywordTok{fn} \NormalTok{foo(arg: *}\KeywordTok{mut} \NormalTok{libc::}\DataTypeTok{c_void}\NormalTok{);}
    \KeywordTok{pub} \KeywordTok{fn} \NormalTok{bar(arg: *}\KeywordTok{mut} \NormalTok{libc::}\DataTypeTok{c_void}\NormalTok{);}
\NormalTok{\}}
\end{Highlighting}
\end{Shaded}

This is a perfectly valid way of handling the situation. However, we can
do a bit better. To solve this, some C libraries will instead create a
\texttt{struct}, where the details and memory layout of the struct are
private. This gives some amount of type safety. These structures are
called `opaque'. Here's an example, in C:

\begin{Shaded}
\begin{Highlighting}[]
\KeywordTok{struct} \NormalTok{Foo; }\CommentTok{/* Foo is a structure, but its contents are not part of the public interfa}
\CommentTok{↳ ce */}
\KeywordTok{struct} \NormalTok{Bar;}
\DataTypeTok{void} \NormalTok{foo(}\KeywordTok{struct} \NormalTok{Foo *arg);}
\DataTypeTok{void} \NormalTok{bar(}\KeywordTok{struct} \NormalTok{Bar *arg);}
\end{Highlighting}
\end{Shaded}

To do this in Rust, let's create our own opaque types with
\texttt{enum}:

\begin{Shaded}
\begin{Highlighting}[]
\KeywordTok{pub} \KeywordTok{enum} \NormalTok{Foo \{\}}
\KeywordTok{pub} \KeywordTok{enum} \NormalTok{Bar \{\}}

\KeywordTok{extern} \StringTok{"C"} \NormalTok{\{}
    \KeywordTok{pub} \KeywordTok{fn} \NormalTok{foo(arg: *}\KeywordTok{mut} \NormalTok{Foo);}
    \KeywordTok{pub} \KeywordTok{fn} \NormalTok{bar(arg: *}\KeywordTok{mut} \NormalTok{Bar);}
\NormalTok{\}}
\end{Highlighting}
\end{Shaded}

By using an \texttt{enum} with no variants, we create an opaque type
that we can't instantiate, as it has no variants. But because our
\texttt{Foo} and \texttt{Bar} types are different, we'll get type safety
between the two of them, so we cannot accidentally pass a pointer to
\texttt{Foo} to \texttt{bar()}.

\section{Borrow and AsRef}\label{sec--borrow-and-asref}

The
\href{http://doc.rust-lang.org/std/borrow/trait.Borrow.html}{\texttt{Borrow}}
and
\href{http://doc.rust-lang.org/std/convert/trait.AsRef.html}{\texttt{AsRef}}
traits are very similar, but different. Here's a quick refresher on what
these two traits mean.

\subsection{Borrow}\label{borrow}

The \texttt{Borrow} trait is used when you're writing a datastructure,
and you want to use either an owned or borrowed type as synonymous for
some purpose.

For example,
\href{http://doc.rust-lang.org/std/collections/struct.HashMap.html}{\texttt{HashMap}}
has a
\href{http://doc.rust-lang.org/std/collections/struct.HashMap.html\#method.get}{\texttt{get}
method} which uses \texttt{Borrow}:

\begin{Shaded}
\begin{Highlighting}[]
\KeywordTok{fn} \NormalTok{get<Q: ?}\BuiltInTok{Sized}\NormalTok{>(&}\KeywordTok{self}\NormalTok{, k: &Q) -> }\DataTypeTok{Option}\NormalTok{<&V>}
    \KeywordTok{where} \NormalTok{K: Borrow<Q>,}
          \NormalTok{Q: }\BuiltInTok{Hash} \NormalTok{+ }\BuiltInTok{Eq}
\end{Highlighting}
\end{Shaded}

This signature is pretty complicated. The \texttt{K} parameter is what
we're interested in here. It refers to a parameter of the
\texttt{HashMap} itself:

\begin{Shaded}
\begin{Highlighting}[]
\KeywordTok{struct} \NormalTok{HashMap<K, V, S = RandomState> \{}
\end{Highlighting}
\end{Shaded}

The \texttt{K} parameter is the type of \emph{key} the \texttt{HashMap}
uses. So, looking at the signature of \texttt{get()} again, we can use
\texttt{get()} when the key implements
\texttt{Borrow\textless{}Q\textgreater{}}. That way, we can make a
\texttt{HashMap} which uses \texttt{String} keys, but use
\texttt{\&str}s when we're searching:

\begin{Shaded}
\begin{Highlighting}[]
\KeywordTok{use} \NormalTok{std::collections::HashMap;}

\KeywordTok{let} \KeywordTok{mut} \NormalTok{map = HashMap::new();}
\NormalTok{map.insert(}\StringTok{"Foo"}\NormalTok{.to_string(), }\DecValTok{42}\NormalTok{);}

\PreprocessorTok{assert_eq!}\NormalTok{(map.get(}\StringTok{"Foo"}\NormalTok{), }\ConstantTok{Some}\NormalTok{(&}\DecValTok{42}\NormalTok{));}
\end{Highlighting}
\end{Shaded}

This is because the standard library has
\texttt{impl\ Borrow\textless{}str\textgreater{}\ for\ String}.

For most types, when you want to take an owned or borrowed type, a
\texttt{\&T} is enough. But one area where \texttt{Borrow} is effective
is when there's more than one kind of borrowed value. This is especially
true of references and slices: you can have both an \texttt{\&T} or a
\texttt{\&mut\ T}. If we wanted to accept both of these types,
\texttt{Borrow} is up for it:

\begin{Shaded}
\begin{Highlighting}[]
\KeywordTok{use} \NormalTok{std::borrow::Borrow;}
\KeywordTok{use} \NormalTok{std::fmt::}\BuiltInTok{Display}\NormalTok{;}

\KeywordTok{fn} \NormalTok{foo<T: Borrow<}\DataTypeTok{i32}\NormalTok{> + }\BuiltInTok{Display}\NormalTok{>(a: T) \{}
    \PreprocessorTok{println!}\NormalTok{(}\StringTok{"a is borrowed: \{\}"}\NormalTok{, a);}
\NormalTok{\}}

\KeywordTok{let} \KeywordTok{mut} \NormalTok{i = }\DecValTok{5}\NormalTok{;}

\NormalTok{foo(&i);}
\NormalTok{foo(&}\KeywordTok{mut} \NormalTok{i);}
\end{Highlighting}
\end{Shaded}

This will print out \texttt{a\ is\ borrowed:\ 5} twice.

\subsection{AsRef}\label{asref}

The \texttt{AsRef} trait is a conversion trait. It's used for converting
some value to a reference in generic code. Like this:

\begin{Shaded}
\begin{Highlighting}[]
\KeywordTok{let} \NormalTok{s = }\StringTok{"Hello"}\NormalTok{.to_string();}

\KeywordTok{fn} \NormalTok{foo<T: AsRef<}\DataTypeTok{str}\NormalTok{>>(s: T) \{}
    \KeywordTok{let} \NormalTok{slice = s.as_ref();}
\NormalTok{\}}
\end{Highlighting}
\end{Shaded}

\subsection{Which should I use?}\label{which-should-i-use}

We can see how they're kind of the same: they both deal with owned and
borrowed versions of some type. However, they're a bit different.

Choose \texttt{Borrow} when you want to abstract over different kinds of
borrowing, or when you're building a datastructure that treats owned and
borrowed values in equivalent ways, such as hashing and comparison.

Choose \texttt{AsRef} when you want to convert something to a reference
directly, and you're writing generic code.

\section{Release Channels}\label{sec--release-channels}

The Rust project uses a concept called `release channels' to manage
releases. It's important to understand this process to choose which
version of Rust your project should use.

\subsection{Overview}\label{overview}

There are three channels for Rust releases:

\begin{itemize}
\tightlist
\item
  Nightly
\item
  Beta
\item
  Stable
\end{itemize}

New nightly releases are created once a day. Every six weeks, the latest
nightly release is promoted to `Beta'. At that point, it will only
receive patches to fix serious errors. Six weeks later, the beta is
promoted to `Stable', and becomes the next release of \texttt{1.x}.

This process happens in parallel. So every six weeks, on the same day,
nightly goes to beta, beta goes to stable. When \texttt{1.x} is
released, at the same time, \texttt{1.(x\ +\ 1)-beta} is released, and
the nightly becomes the first version of \texttt{1.(x\ +\ 2)-nightly}.

\subsection{Choosing a version}\label{choosing-a-version}

Generally speaking, unless you have a specific reason, you should be
using the stable release channel. These releases are intended for a
general audience.

However, depending on your interest in Rust, you may choose to use
nightly instead. The basic tradeoff is this: in the nightly channel, you
can use unstable, new Rust features. However, unstable features are
subject to change, and so any new nightly release may break your code.
If you use the stable release, you cannot use experimental features, but
the next release of Rust will not cause significant issues through
breaking changes.

\subsection{Helping the ecosystem through
CI}\label{helping-the-ecosystem-through-ci}

What about beta? We encourage all Rust users who use the stable release
channel to also test against the beta channel in their continuous
integration systems. This will help alert the team in case there's an
accidental regression.

Additionally, testing against nightly can catch regressions even sooner,
and so if you don't mind a third build, we'd appreciate testing against
all channels.

As an example, many Rust programmers use
\href{https://travis-ci.org/}{Travis} to test their crates, which is
free for open source projects. Travis
\href{http://docs.travis-ci.com/user/languages/rust/}{supports Rust
directly}, and you can use a \texttt{.travis.yml} file like this to test
on all channels:

\begin{Shaded}
\begin{Highlighting}[]
\FunctionTok{language:} \NormalTok{rust}
\FunctionTok{rust:}
  \KeywordTok{-} \NormalTok{nightly}
  \KeywordTok{-} \NormalTok{beta}
  \KeywordTok{-} \NormalTok{stable}

\FunctionTok{matrix:}
  \FunctionTok{allow_failures:}
    \KeywordTok{-} \FunctionTok{rust:} \NormalTok{nightly}
\end{Highlighting}
\end{Shaded}

With this configuration, Travis will test all three channels, but if
something breaks on nightly, it won't fail your build. A similar
configuration is recommended for any CI system, check the documentation
of the one you're using for more details.

\hypertarget{sec--using-rust-without-the-standard-library}{\section{Using
Rust without the standard
library}\label{sec--using-rust-without-the-standard-library}}

Rust's standard library provides a lot of useful functionality, but
assumes support for various features of its host system: threads,
networking, heap allocation, and others. There are systems that do not
have these features, however, and Rust can work with those too! To do
so, we tell Rust that we don't want to use the standard library via an
attribute: \texttt{\#!{[}no\_std{]}}.

\begin{quote}
Note: This feature is technically stable, but there are some caveats.
For one, you can build a \texttt{\#!{[}no\_std{]}} \emph{library} on
stable, but not a \emph{binary}. For details on binaries without the
standard library, see \protect\hyperlink{sec--no-stdlib}{the nightly
chapter on \texttt{\#!{[}no\_std{]}}}
\end{quote}

To use \texttt{\#!{[}no\_std{]}}, add it to your crate root:

\begin{Shaded}
\begin{Highlighting}[]
\AttributeTok{#![}\NormalTok{no_std}\AttributeTok{]}

\KeywordTok{fn} \NormalTok{plus_one(x: }\DataTypeTok{i32}\NormalTok{) -> }\DataTypeTok{i32} \NormalTok{\{}
    \NormalTok{x + }\DecValTok{1}
\NormalTok{\}}
\end{Highlighting}
\end{Shaded}

Much of the functionality that's exposed in the standard library is also
available via the \href{http://doc.rust-lang.org/core/}{\texttt{core}
crate}. When we're using the standard library, Rust automatically brings
\texttt{std} into scope, allowing you to use its features without an
explicit import. By the same token, when using
\texttt{\#!{[}no\_std{]}}, Rust will bring \texttt{core} into scope for
you, as well as \href{http://doc.rust-lang.org/core/prelude/v1/}{its
prelude}. This means that a lot of code will Just Work:

\begin{Shaded}
\begin{Highlighting}[]
\AttributeTok{#![}\NormalTok{no_std}\AttributeTok{]}

\KeywordTok{fn} \NormalTok{may_fail(failure: }\DataTypeTok{bool}\NormalTok{) -> }\DataTypeTok{Result}\NormalTok{<(), &}\OtherTok{'static} \DataTypeTok{str}\NormalTok{> \{}
    \KeywordTok{if} \NormalTok{failure \{}
        \ConstantTok{Err}\NormalTok{(}\StringTok{"this didn’t work!"}\NormalTok{)}
    \NormalTok{\} }\KeywordTok{else} \NormalTok{\{}
        \ConstantTok{Ok}\NormalTok{(())}
    \NormalTok{\}}
\NormalTok{\}}
\end{Highlighting}
\end{Shaded}

\hypertarget{sec--nightly-rust}{\chapter{Nightly
Rust}\label{sec--nightly-rust}}

Rust provides three distribution channels for Rust: nightly, beta, and
stable. Unstable features are only available on nightly Rust. For more
details on this process, see
`\href{http://blog.rust-lang.org/2014/10/30/Stability.html}{Stability as
a deliverable}'.

To install nightly Rust, you can use \texttt{rustup.sh}:

\begin{Shaded}
\begin{Highlighting}[]
\NormalTok{$ }\KeywordTok{curl} \NormalTok{-s https://static.rust-lang.org/rustup.sh }\KeywordTok{|} \KeywordTok{sh} \NormalTok{-s -- --channel=nightly}
\end{Highlighting}
\end{Shaded}

If you're concerned about the
\href{http://curlpipesh.tumblr.com}{potential insecurity} of using
\texttt{curl\ \textbar{}\ sh}, please keep reading and see our
disclaimer below. And feel free to use a two-step version of the
installation and examine our installation script:

\begin{Shaded}
\begin{Highlighting}[]
\NormalTok{$ }\KeywordTok{curl} \NormalTok{-f -L https://static.rust-lang.org/rustup.sh -O}
\NormalTok{$ }\KeywordTok{sh} \NormalTok{rustup.sh --channel=nightly}
\end{Highlighting}
\end{Shaded}

If you're on Windows, please download either the
\href{https://static.rust-lang.org/dist/rust-nightly-i686-pc-windows-gnu.msi}{32-bit
installer} or the
\href{https://static.rust-lang.org/dist/rust-nightly-x86_64-pc-windows-gnu.msi}{64-bit
installer} and run it.

\subsubsection{Uninstalling}\label{uninstalling-1}

If you decide you don't want Rust anymore, we'll be a bit sad, but
that's okay. Not every programming language is great for everyone. Just
run the uninstall script:

\begin{Shaded}
\begin{Highlighting}[]
\NormalTok{$ }\KeywordTok{sudo} \NormalTok{/usr/local/lib/rustlib/uninstall.sh}
\end{Highlighting}
\end{Shaded}

If you used the Windows installer, re-run the \texttt{.msi} and it will
give you an uninstall option.

Some people, and somewhat rightfully so, get very upset when we tell you
to \texttt{curl\ \textbar{}\ sh}. Basically, when you do this, you are
trusting that the good people who maintain Rust aren't going to hack
your computer and do bad things. That's a good instinct! If you're one
of those people, please check out the documentation on
\href{https://github.com/rust-lang/rust\#building-from-source}{building
Rust from Source}, or \href{https://www.rust-lang.org/install.html}{the
official binary downloads}.

Oh, we should also mention the officially supported platforms:

\begin{itemize}
\tightlist
\item
  Windows (7, 8, Server 2008 R2)
\item
  Linux (2.6.18 or later, various distributions), x86 and x86-64
\item
  OSX 10.7 (Lion) or greater, x86 and x86-64
\end{itemize}

We extensively test Rust on these platforms, and a few others, too, like
Android. But these are the ones most likely to work, as they have the
most testing.

Finally, a comment about Windows. Rust considers Windows to be a
first-class platform upon release, but if we're honest, the Windows
experience isn't as integrated as the Linux/OS X experience is. We're
working on it! If anything does not work, it is a bug. Please let us
know if that happens. Each and every commit is tested against Windows
like any other platform.

If you've got Rust installed, you can open up a shell, and type this:

\begin{Shaded}
\begin{Highlighting}[]
\NormalTok{$ }\KeywordTok{rustc} \NormalTok{--version}
\end{Highlighting}
\end{Shaded}

You should see the version number, commit hash, commit date and build
date:

\begin{Shaded}
\begin{Highlighting}[]
\KeywordTok{rustc} \NormalTok{1.0.0-nightly (f11f3e7ba 2015-01-04) }\KeywordTok{(built} \NormalTok{2015-01-06}\KeywordTok{)}
\end{Highlighting}
\end{Shaded}

If you did, Rust has been installed successfully! Congrats!

This installer also installs a copy of the documentation locally, so you
can read it offline. On UNIX systems, \texttt{/usr/local/share/doc/rust}
is the location. On Windows, it's in a \texttt{share/doc} directory,
inside wherever you installed Rust to.

If not, there are a number of places where you can get help. The easiest
is \href{irc://irc.mozilla.org/\#rust}{the \#rust IRC channel on
irc.mozilla.org}, which you can access through
\href{http://chat.mibbit.com/?server=irc.mozilla.org\&channel=\%23rust}{Mibbit}.
Click that link, and you'll be chatting with other Rustaceans (a silly
nickname we call ourselves), and we can help you out. Other great
resources include \href{https://users.rust-lang.org/}{the user's forum},
and \href{http://stackoverflow.com/questions/tagged/rust}{Stack
Overflow}.

\hypertarget{sec--compiler-plugins}{\section{Compiler
Plugins}\label{sec--compiler-plugins}}

\subsection{Introduction}\label{introduction-2}

\texttt{rustc} can load compiler plugins, which are user-provided
libraries that extend the compiler's behavior with new syntax
extensions, lint checks, etc.

A plugin is a dynamic library crate with a designated \emph{registrar}
function that registers extensions with \texttt{rustc}. Other crates can
load these extensions using the crate attribute
\texttt{\#!{[}plugin(...){]}}. See the \texttt{rustc\_plugin}
documentation for more about the mechanics of defining and loading a
plugin.

If present, arguments passed as
\texttt{\#!{[}plugin(foo(...\ args\ ...)){]}} are not interpreted by
rustc itself. They are provided to the plugin through the
\texttt{Registry}'s \texttt{args} method.

In the vast majority of cases, a plugin should \emph{only} be used
through \texttt{\#!{[}plugin{]}} and not through an
\texttt{extern\ crate} item. Linking a plugin would pull in all of
libsyntax and librustc as dependencies of your crate. This is generally
unwanted unless you are building another plugin. The
\texttt{plugin\_as\_library} lint checks these guidelines.

The usual practice is to put compiler plugins in their own crate,
separate from any \texttt{macro\_rules!} macros or ordinary Rust code
meant to be used by consumers of a library.

\subsection{Syntax extensions}\label{syntax-extensions}

Plugins can extend Rust's syntax in various ways. One kind of syntax
extension is the procedural macro. These are invoked the same way as
\protect\hyperlink{sec--macros}{ordinary macros}, but the expansion is
performed by arbitrary Rust code that manipulates syntax trees at
compile time.

Let's write a plugin
\href{https://github.com/rust-lang/rust/tree/master/src/test/auxiliary/roman_numerals.rs}{\texttt{roman\_numerals.rs}}
that implements Roman numeral integer literals.

\begin{Shaded}
\begin{Highlighting}[]
\AttributeTok{#![}\NormalTok{crate_type}\AttributeTok{=}\StringTok{"dylib"}\AttributeTok{]}
\AttributeTok{#![}\NormalTok{feature}\AttributeTok{(}\NormalTok{plugin_registrar}\AttributeTok{,} \NormalTok{rustc_private}\AttributeTok{)]}

\KeywordTok{extern} \KeywordTok{crate} \NormalTok{syntax;}
\KeywordTok{extern} \KeywordTok{crate} \NormalTok{rustc;}
\KeywordTok{extern} \KeywordTok{crate} \NormalTok{rustc_plugin;}

\KeywordTok{use} \NormalTok{syntax::codemap::Span;}
\KeywordTok{use} \NormalTok{syntax::parse::token;}
\KeywordTok{use} \NormalTok{syntax::ast::TokenTree;}
\KeywordTok{use} \NormalTok{syntax::ext::base::\{ExtCtxt, MacResult, DummyResult, MacEager\};}
\KeywordTok{use} \NormalTok{syntax::ext::build::AstBuilder;  }\CommentTok{// trait for expr_usize}
\KeywordTok{use} \NormalTok{rustc_plugin::Registry;}

\KeywordTok{fn} \NormalTok{expand_rn(cx: &}\KeywordTok{mut} \NormalTok{ExtCtxt, sp: Span, args: &[TokenTree])}
        \NormalTok{-> }\DataTypeTok{Box}\NormalTok{<MacResult + }\OtherTok{'static}\NormalTok{> \{}

    \KeywordTok{static} \NormalTok{NUMERALS: &}\OtherTok{'static} \NormalTok{[(&}\OtherTok{'static} \DataTypeTok{str}\NormalTok{, }\DataTypeTok{usize}\NormalTok{)] = &[}
        \NormalTok{(}\StringTok{"M"}\NormalTok{, }\DecValTok{1000}\NormalTok{), (}\StringTok{"CM"}\NormalTok{, }\DecValTok{900}\NormalTok{), (}\StringTok{"D"}\NormalTok{, }\DecValTok{500}\NormalTok{), (}\StringTok{"CD"}\NormalTok{, }\DecValTok{400}\NormalTok{),}
        \NormalTok{(}\StringTok{"C"}\NormalTok{,  }\DecValTok{100}\NormalTok{), (}\StringTok{"XC"}\NormalTok{,  }\DecValTok{90}\NormalTok{), (}\StringTok{"L"}\NormalTok{,  }\DecValTok{50}\NormalTok{), (}\StringTok{"XL"}\NormalTok{,  }\DecValTok{40}\NormalTok{),}
        \NormalTok{(}\StringTok{"X"}\NormalTok{,   }\DecValTok{10}\NormalTok{), (}\StringTok{"IX"}\NormalTok{,   }\DecValTok{9}\NormalTok{), (}\StringTok{"V"}\NormalTok{,   }\DecValTok{5}\NormalTok{), (}\StringTok{"IV"}\NormalTok{,   }\DecValTok{4}\NormalTok{),}
        \NormalTok{(}\StringTok{"I"}\NormalTok{,    }\DecValTok{1}\NormalTok{)];}

    \KeywordTok{if} \NormalTok{args.len() != }\DecValTok{1} \NormalTok{\{}
        \NormalTok{cx.span_err(}
            \NormalTok{sp,}
            \NormalTok{&}\PreprocessorTok{format!}\NormalTok{(}\StringTok{"argument should be a single identifier, but got \{\} arguments"}\NormalTok{, a}
\NormalTok{↳ rgs.len()));}
        \KeywordTok{return} \NormalTok{DummyResult::any(sp);}
    \NormalTok{\}}

    \KeywordTok{let} \NormalTok{text = }\KeywordTok{match} \NormalTok{args[}\DecValTok{0}\NormalTok{] \{}
        \NormalTok{TokenTree::Token(_, token::Ident(s, _)) => s.to_string(),}
        \NormalTok{_ => \{}
            \NormalTok{cx.span_err(sp, }\StringTok{"argument should be a single identifier"}\NormalTok{);}
            \KeywordTok{return} \NormalTok{DummyResult::any(sp);}
        \NormalTok{\}}
    \NormalTok{\};}

    \KeywordTok{let} \KeywordTok{mut} \NormalTok{text = &*text;}
    \KeywordTok{let} \KeywordTok{mut} \NormalTok{total = }\DecValTok{0}\NormalTok{;}
    \KeywordTok{while} \NormalTok{!text.is_empty() \{}
        \KeywordTok{match} \NormalTok{NUMERALS.iter().find(|&&(rn, _)| text.starts_with(rn)) \{}
            \ConstantTok{Some}\NormalTok{(&(rn, val)) => \{}
                \NormalTok{total += val;}
                \NormalTok{text = &text[rn.len()..];}
            \NormalTok{\}}
            \ConstantTok{None} \NormalTok{=> \{}
                \NormalTok{cx.span_err(sp, }\StringTok{"invalid Roman numeral"}\NormalTok{);}
                \KeywordTok{return} \NormalTok{DummyResult::any(sp);}
            \NormalTok{\}}
        \NormalTok{\}}
    \NormalTok{\}}

    \NormalTok{MacEager::expr(cx.expr_usize(sp, total))}
\NormalTok{\}}

\AttributeTok{#[}\NormalTok{plugin_registrar}\AttributeTok{]}
\KeywordTok{pub} \KeywordTok{fn} \NormalTok{plugin_registrar(reg: &}\KeywordTok{mut} \NormalTok{Registry) \{}
    \NormalTok{reg.register_macro(}\StringTok{"rn"}\NormalTok{, expand_rn);}
\NormalTok{\}}
\end{Highlighting}
\end{Shaded}

Then we can use \texttt{rn!()} like any other macro:

\begin{Shaded}
\begin{Highlighting}[]
\AttributeTok{#![}\NormalTok{feature}\AttributeTok{(}\NormalTok{plugin}\AttributeTok{)]}
\AttributeTok{#![}\NormalTok{plugin}\AttributeTok{(}\NormalTok{roman_numerals}\AttributeTok{)]}

\KeywordTok{fn} \NormalTok{main() \{}
    \PreprocessorTok{assert_eq!}\NormalTok{(}\PreprocessorTok{rn!}\NormalTok{(MMXV), }\DecValTok{2015}\NormalTok{);}
\NormalTok{\}}
\end{Highlighting}
\end{Shaded}

The advantages over a simple \texttt{fn(\&str)\ -\textgreater{}\ u32}
are:

\begin{itemize}
\tightlist
\item
  The (arbitrarily complex) conversion is done at compile time.
\item
  Input validation is also performed at compile time.
\item
  It can be extended to allow use in patterns, which effectively gives a
  way to define new literal syntax for any data type.
\end{itemize}

In addition to procedural macros, you can define new
\href{http://doc.rust-lang.org/reference.html\#derive}{\texttt{derive}}-like
attributes and other kinds of extensions. See
\texttt{Registry::register\_syntax\_extension} and the
\texttt{SyntaxExtension} enum. For a more involved macro example, see
\href{https://github.com/rust-lang/regex/blob/master/regex_macros/src/lib.rs}{\texttt{regex\_macros}}.

\subsubsection{Tips and tricks}\label{tips-and-tricks}

Some of the \protect\hyperlink{debugging-macro-code}{macro debugging
tips} are applicable.

You can use \texttt{syntax::parse} to turn token trees into higher-level
syntax elements like expressions:

\begin{Shaded}
\begin{Highlighting}[]
\KeywordTok{fn} \NormalTok{expand_foo(cx: &}\KeywordTok{mut} \NormalTok{ExtCtxt, sp: Span, args: &[TokenTree])}
        \NormalTok{-> }\DataTypeTok{Box}\NormalTok{<MacResult+}\OtherTok{'static}\NormalTok{> \{}

    \KeywordTok{let} \KeywordTok{mut} \NormalTok{parser = cx.new_parser_from_tts(args);}

    \KeywordTok{let} \NormalTok{expr: P<Expr> = parser.parse_expr();}
\end{Highlighting}
\end{Shaded}

Looking through
\href{https://github.com/rust-lang/rust/blob/master/src/libsyntax/parse/parser.rs}{\texttt{libsyntax}
parser code} will give you a feel for how the parsing infrastructure
works.

Keep the \texttt{Span}s of everything you parse, for better error
reporting. You can wrap \texttt{Spanned} around your custom data
structures.

Calling \texttt{ExtCtxt::span\_fatal} will immediately abort
compilation. It's better to instead call \texttt{ExtCtxt::span\_err} and
return \texttt{DummyResult} so that the compiler can continue and find
further errors.

To print syntax fragments for debugging, you can use \texttt{span\_note}
together with \texttt{syntax::print::pprust::*\_to\_string}.

The example above produced an integer literal using
\texttt{AstBuilder::expr\_usize}. As an alternative to the
\texttt{AstBuilder} trait, \texttt{libsyntax} provides a set of
quasiquote macros. They are undocumented and very rough around the
edges. However, the implementation may be a good starting point for an
improved quasiquote as an ordinary plugin library.

\subsection{Lint plugins}\label{lint-plugins}

Plugins can extend
\href{http://doc.rust-lang.org/reference.html\#lint-check-attributes}{Rust's
lint infrastructure} with additional checks for code style, safety, etc.
Now let's write a plugin
\href{https://github.com/rust-lang/rust/blob/master/src/test/auxiliary/lint_plugin_test.rs}{\texttt{lint\_plugin\_test.rs}}
that warns about any item named \texttt{lintme}.

\begin{Shaded}
\begin{Highlighting}[]
\AttributeTok{#![}\NormalTok{feature}\AttributeTok{(}\NormalTok{plugin_registrar}\AttributeTok{)]}
\AttributeTok{#![}\NormalTok{feature}\AttributeTok{(}\NormalTok{box_syntax}\AttributeTok{,} \NormalTok{rustc_private}\AttributeTok{)]}

\KeywordTok{extern} \KeywordTok{crate} \NormalTok{syntax;}

\CommentTok{// Load rustc as a plugin to get macros}
\AttributeTok{#[}\NormalTok{macro_use}\AttributeTok{]}
\KeywordTok{extern} \KeywordTok{crate} \NormalTok{rustc;}
\KeywordTok{extern} \KeywordTok{crate} \NormalTok{rustc_plugin;}

\KeywordTok{use} \NormalTok{rustc::lint::\{EarlyContext, LintContext, LintPass, EarlyLintPass,}
                  \NormalTok{EarlyLintPassObject, LintArray\};}
\KeywordTok{use} \NormalTok{rustc_plugin::Registry;}
\KeywordTok{use} \NormalTok{syntax::ast;}

\PreprocessorTok{declare_lint!}\NormalTok{(TEST_LINT, Warn, }\StringTok{"Warn about items named 'lintme'"}\NormalTok{);}

\KeywordTok{struct} \NormalTok{Pass;}

\KeywordTok{impl} \NormalTok{LintPass }\KeywordTok{for} \NormalTok{Pass \{}
    \KeywordTok{fn} \NormalTok{get_lints(&}\KeywordTok{self}\NormalTok{) -> LintArray \{}
        \PreprocessorTok{lint_array!}\NormalTok{(TEST_LINT)}
    \NormalTok{\}}
\NormalTok{\}}

\KeywordTok{impl} \NormalTok{EarlyLintPass }\KeywordTok{for} \NormalTok{Pass \{}
    \KeywordTok{fn} \NormalTok{check_item(&}\KeywordTok{mut} \KeywordTok{self}\NormalTok{, cx: &EarlyContext, it: &ast::Item) \{}
        \KeywordTok{if} \NormalTok{it.ident.name.as_str() == }\StringTok{"lintme"} \NormalTok{\{}
            \NormalTok{cx.span_lint(TEST_LINT, it.span, }\StringTok{"item is named 'lintme'"}\NormalTok{);}
        \NormalTok{\}}
    \NormalTok{\}}
\NormalTok{\}}

\AttributeTok{#[}\NormalTok{plugin_registrar}\AttributeTok{]}
\KeywordTok{pub} \KeywordTok{fn} \NormalTok{plugin_registrar(reg: &}\KeywordTok{mut} \NormalTok{Registry) \{}
    \NormalTok{reg.register_early_lint_pass(}\KeywordTok{box} \NormalTok{Pass }\KeywordTok{as} \NormalTok{EarlyLintPassObject);}
\NormalTok{\}}
\end{Highlighting}
\end{Shaded}

Then code like

\begin{Shaded}
\begin{Highlighting}[]
\AttributeTok{#![}\NormalTok{plugin}\AttributeTok{(}\NormalTok{lint_plugin_test}\AttributeTok{)]}

\KeywordTok{fn} \NormalTok{lintme() \{ \}}
\end{Highlighting}
\end{Shaded}

will produce a compiler warning:

\begin{verbatim}
foo.rs:4:1: 4:16 warning: item is named 'lintme', #[warn(test_lint)] on by default
foo.rs:4 fn lintme() { }
         ^~~~~~~~~~~~~~~
\end{verbatim}

The components of a lint plugin are:

\begin{itemize}
\item
  one or more \texttt{declare\_lint!} invocations, which define static
  \texttt{Lint} structs;
\item
  a struct holding any state needed by the lint pass (here, none);
\item
  a \texttt{LintPass} implementation defining how to check each syntax
  element. A single \texttt{LintPass} may call \texttt{span\_lint} for
  several different \texttt{Lint}s, but should register them all through
  the \texttt{get\_lints} method.
\end{itemize}

Lint passes are syntax traversals, but they run at a late stage of
compilation where type information is available. \texttt{rustc}'s
\href{https://github.com/rust-lang/rust/blob/master/src/librustc/lint/builtin.rs}{built-in
lints} mostly use the same infrastructure as lint plugins, and provide
examples of how to access type information.

Lints defined by plugins are controlled by the usual
\href{http://doc.rust-lang.org/reference.html\#lint-check-attributes}{attributes
and compiler flags}, e.g. \texttt{\#{[}allow(test\_lint){]}} or
\texttt{-A\ test-lint}. These identifiers are derived from the first
argument to \texttt{declare\_lint!}, with appropriate case and
punctuation conversion.

You can run \texttt{rustc\ -W\ help\ foo.rs} to see a list of lints
known to \texttt{rustc}, including those provided by plugins loaded by
\texttt{foo.rs}.

\section{Inline Assembly}\label{sec--inline-assembly}

For extremely low-level manipulations and performance reasons, one might
wish to control the CPU directly. Rust supports using inline assembly to
do this via the \texttt{asm!} macro.

\begin{Shaded}
\begin{Highlighting}[]
\PreprocessorTok{asm!}\NormalTok{(assembly template}
   \NormalTok{: output operands}
   \NormalTok{: input operands}
   \NormalTok{: clobbers}
   \NormalTok{: options}
   \NormalTok{);}
\end{Highlighting}
\end{Shaded}

Any use of \texttt{asm} is feature gated (requires
\texttt{\#!{[}feature(asm){]}} on the crate to allow) and of course
requires an \texttt{unsafe} block.

\begin{quote}
\textbf{Note}: the examples here are given in x86/x86-64 assembly, but
all platforms are supported.
\end{quote}

\subsubsection{Assembly template}\label{assembly-template}

The \texttt{assembly\ template} is the only required parameter and must
be a literal string (i.e. \texttt{""})

\begin{Shaded}
\begin{Highlighting}[]
\AttributeTok{#![}\NormalTok{feature}\AttributeTok{(}\NormalTok{asm}\AttributeTok{)]}

\AttributeTok{#[}\NormalTok{cfg}\AttributeTok{(}\NormalTok{any}\AttributeTok{(}\NormalTok{target_arch }\AttributeTok{=} \StringTok{"x86"}\AttributeTok{,} \NormalTok{target_arch }\AttributeTok{=} \StringTok{"x86_64"}\AttributeTok{))]}
\KeywordTok{fn} \NormalTok{foo() \{}
    \KeywordTok{unsafe} \NormalTok{\{}
        \PreprocessorTok{asm!}\NormalTok{(}\StringTok{"NOP"}\NormalTok{);}
    \NormalTok{\}}
\NormalTok{\}}

\CommentTok{// other platforms}
\AttributeTok{#[}\NormalTok{cfg}\AttributeTok{(}\NormalTok{not}\AttributeTok{(}\NormalTok{any}\AttributeTok{(}\NormalTok{target_arch }\AttributeTok{=} \StringTok{"x86"}\AttributeTok{,} \NormalTok{target_arch }\AttributeTok{=} \StringTok{"x86_64"}\AttributeTok{)))]}
\KeywordTok{fn} \NormalTok{foo() \{ }\CommentTok{/* ... */} \NormalTok{\}}

\KeywordTok{fn} \NormalTok{main() \{}
    \CommentTok{// ...}
    \NormalTok{foo();}
    \CommentTok{// ...}
\NormalTok{\}}
\end{Highlighting}
\end{Shaded}

(The \texttt{feature(asm)} and \texttt{\#{[}cfg{]}}s are omitted from
now on.)

Output operands, input operands, clobbers and options are all optional
but you must add the right number of \texttt{:} if you skip them:

\begin{Shaded}
\begin{Highlighting}[]
\PreprocessorTok{asm!}\NormalTok{(}\StringTok{"xor %eax, %eax"}
    \NormalTok{:}
    \NormalTok{:}
    \NormalTok{: }\StringTok{"\{eax\}"}
   \NormalTok{);}
\end{Highlighting}
\end{Shaded}

Whitespace also doesn't matter:

\begin{Shaded}
\begin{Highlighting}[]
\PreprocessorTok{asm!}\NormalTok{(}\StringTok{"xor %eax, %eax"} \NormalTok{::: }\StringTok{"\{eax\}"}\NormalTok{);}
\end{Highlighting}
\end{Shaded}

\subsubsection{Operands}\label{operands}

Input and output operands follow the same format:
\texttt{:\ "constraints1"(expr1),\ "constraints2"(expr2),\ ..."}. Output
operand expressions must be mutable lvalues, or not yet assigned:

\begin{Shaded}
\begin{Highlighting}[]
\KeywordTok{fn} \NormalTok{add(a: }\DataTypeTok{i32}\NormalTok{, b: }\DataTypeTok{i32}\NormalTok{) -> }\DataTypeTok{i32} \NormalTok{\{}
    \KeywordTok{let} \NormalTok{c: }\DataTypeTok{i32}\NormalTok{;}
    \KeywordTok{unsafe} \NormalTok{\{}
        \PreprocessorTok{asm!}\NormalTok{(}\StringTok{"add $2, $0"}
             \NormalTok{: }\StringTok{"=r"}\NormalTok{(c)}
             \NormalTok{: }\StringTok{"0"}\NormalTok{(a), }\StringTok{"r"}\NormalTok{(b)}
             \NormalTok{);}
    \NormalTok{\}}
    \NormalTok{c}
\NormalTok{\}}

\KeywordTok{fn} \NormalTok{main() \{}
    \PreprocessorTok{assert_eq!}\NormalTok{(add(}\DecValTok{3}\NormalTok{, }\DecValTok{14159}\NormalTok{), }\DecValTok{14162}\NormalTok{)}
\NormalTok{\}}
\end{Highlighting}
\end{Shaded}

If you would like to use real operands in this position, however, you
are required to put curly braces \texttt{\{\}} around the register that
you want, and you are required to put the specific size of the operand.
This is useful for very low level programming, where which register you
use is important:

\begin{Shaded}
\begin{Highlighting}[]
\KeywordTok{let} \NormalTok{result: }\DataTypeTok{u8}\NormalTok{;}
\PreprocessorTok{asm!}\NormalTok{(}\StringTok{"in %dx, %al"} \NormalTok{: }\StringTok{"=\{al\}"}\NormalTok{(result) : }\StringTok{"\{dx\}"}\NormalTok{(port));}
\NormalTok{result}
\end{Highlighting}
\end{Shaded}

\subsubsection{Clobbers}\label{clobbers}

Some instructions modify registers which might otherwise have held
different values so we use the clobbers list to indicate to the compiler
not to assume any values loaded into those registers will stay valid.

\begin{Shaded}
\begin{Highlighting}[]
\CommentTok{// Put the value 0x200 in eax}
\PreprocessorTok{asm!}\NormalTok{(}\StringTok{"mov $$0x200, %eax"} \NormalTok{: }\CommentTok{/* no outputs */} \NormalTok{: }\CommentTok{/* no inputs */} \NormalTok{: }\StringTok{"\{eax\}"}\NormalTok{);}
\end{Highlighting}
\end{Shaded}

Input and output registers need not be listed since that information is
already communicated by the given constraints. Otherwise, any other
registers used either implicitly or explicitly should be listed.

If the assembly changes the condition code register \texttt{cc} should
be specified as one of the clobbers. Similarly, if the assembly modifies
memory, \texttt{memory} should also be specified.

\subsubsection{Options}\label{options}

The last section, \texttt{options} is specific to Rust. The format is
comma separated literal strings (i.e. \texttt{:"foo",\ "bar",\ "baz"}).
It's used to specify some extra info about the inline assembly:

Current valid options are:

\begin{enumerate}
\def\labelenumi{\arabic{enumi}.}
\tightlist
\item
  \emph{volatile} - specifying this is analogous to
  \texttt{\_\_asm\_\_\ \_\_volatile\_\_\ (...)} in gcc/clang.
\item
  \emph{alignstack} - certain instructions expect the stack to be
  aligned a certain way (i.e.~SSE) and specifying this indicates to the
  compiler to insert its usual stack alignment code
\item
  \emph{intel} - use intel syntax instead of the default AT\&T.
\end{enumerate}

\begin{Shaded}
\begin{Highlighting}[]
\KeywordTok{let} \NormalTok{result: }\DataTypeTok{i32}\NormalTok{;}
\KeywordTok{unsafe} \NormalTok{\{}
   \PreprocessorTok{asm!}\NormalTok{(}\StringTok{"mov eax, 2"} \NormalTok{: }\StringTok{"=\{eax\}"}\NormalTok{(result) : : : }\StringTok{"intel"}\NormalTok{)}
\NormalTok{\}}
\PreprocessorTok{println!}\NormalTok{(}\StringTok{"eax is currently \{\}"}\NormalTok{, result);}
\end{Highlighting}
\end{Shaded}

\subsubsection{More Information}\label{more-information}

The current implementation of the \texttt{asm!} macro is a direct
binding to
\href{http://llvm.org/docs/LangRef.html\#inline-assembler-expressions}{LLVM's
inline assembler expressions}, so be sure to check out
\href{http://llvm.org/docs/LangRef.html\#inline-assembler-expressions}{their
documentation as well} for more information about clobbers, constraints,
etc.

\hypertarget{sec--no-stdlib}{\section{No stdlib}\label{sec--no-stdlib}}

Rust's standard library provides a lot of useful functionality, but
assumes support for various features of its host system: threads,
networking, heap allocation, and others. There are systems that do not
have these features, however, and Rust can work with those too! To do
so, we tell Rust that we don't want to use the standard library via an
attribute: \texttt{\#!{[}no\_std{]}}.

\begin{quote}
Note: This feature is technically stable, but there are some caveats.
For one, you can build a \texttt{\#!{[}no\_std{]}} \emph{library} on
stable, but not a \emph{binary}. For details on libraries without the
standard library, see
\protect\hyperlink{sec--using-rust-without-the-standard-library}{the
chapter on \texttt{\#!{[}no\_std{]}}}
\end{quote}

Obviously there's more to life than just libraries: one can use
\texttt{\#{[}no\_std{]}} with an executable, controlling the entry point
is possible in two ways: the \texttt{\#{[}start{]}} attribute, or
overriding the default shim for the C \texttt{main} function with your
own.

The function marked \texttt{\#{[}start{]}} is passed the command line
parameters in the same format as C:

\begin{Shaded}
\begin{Highlighting}[]
\AttributeTok{#![}\NormalTok{feature}\AttributeTok{(}\NormalTok{lang_items}\AttributeTok{)]}
\AttributeTok{#![}\NormalTok{feature}\AttributeTok{(}\NormalTok{start}\AttributeTok{)]}
\AttributeTok{#![}\NormalTok{no_std}\AttributeTok{]}

\CommentTok{// Pull in the system libc library for what crt0.o likely requires}
\KeywordTok{extern} \KeywordTok{crate} \NormalTok{libc;}

\CommentTok{// Entry point for this program}
\AttributeTok{#[}\NormalTok{start}\AttributeTok{]}
\KeywordTok{fn} \NormalTok{start(_argc: }\DataTypeTok{isize}\NormalTok{, _argv: *}\KeywordTok{const} \NormalTok{*}\KeywordTok{const} \DataTypeTok{u8}\NormalTok{) -> }\DataTypeTok{isize} \NormalTok{\{}
    \DecValTok{0}
\NormalTok{\}}

\CommentTok{// These functions and traits are used by the compiler, but not}
\CommentTok{// for a bare-bones hello world. These are normally}
\CommentTok{// provided by libstd.}
\AttributeTok{#[}\NormalTok{lang }\AttributeTok{=} \StringTok{"eh_personality"}\AttributeTok{]} \KeywordTok{extern} \KeywordTok{fn} \NormalTok{eh_personality() \{\}}
\AttributeTok{#[}\NormalTok{lang }\AttributeTok{=} \StringTok{"panic_fmt"}\AttributeTok{]} \KeywordTok{extern} \KeywordTok{fn} \NormalTok{panic_fmt() -> ! \{ }\KeywordTok{loop} \NormalTok{\{\} \}}
\end{Highlighting}
\end{Shaded}

To override the compiler-inserted \texttt{main} shim, one has to disable
it with \texttt{\#!{[}no\_main{]}} and then create the appropriate
symbol with the correct ABI and the correct name, which requires
overriding the compiler's name mangling too:

\begin{Shaded}
\begin{Highlighting}[]
\AttributeTok{#![}\NormalTok{feature}\AttributeTok{(}\NormalTok{lang_items}\AttributeTok{)]}
\AttributeTok{#![}\NormalTok{feature}\AttributeTok{(}\NormalTok{start}\AttributeTok{)]}
\AttributeTok{#![}\NormalTok{no_std}\AttributeTok{]}
\AttributeTok{#![}\NormalTok{no_main}\AttributeTok{]}

\KeywordTok{extern} \KeywordTok{crate} \NormalTok{libc;}

\AttributeTok{#[}\NormalTok{no_mangle}\AttributeTok{]} \CommentTok{// ensure that this symbol is called `main` in the output}
\KeywordTok{pub} \KeywordTok{extern} \KeywordTok{fn} \NormalTok{main(argc: }\DataTypeTok{i32}\NormalTok{, argv: *}\KeywordTok{const} \NormalTok{*}\KeywordTok{const} \DataTypeTok{u8}\NormalTok{) -> }\DataTypeTok{i32} \NormalTok{\{}
    \DecValTok{0}
\NormalTok{\}}

\AttributeTok{#[}\NormalTok{lang }\AttributeTok{=} \StringTok{"eh_personality"}\AttributeTok{]} \KeywordTok{extern} \KeywordTok{fn} \NormalTok{eh_personality() \{\}}
\AttributeTok{#[}\NormalTok{lang }\AttributeTok{=} \StringTok{"panic_fmt"}\AttributeTok{]} \KeywordTok{extern} \KeywordTok{fn} \NormalTok{panic_fmt() -> ! \{ }\KeywordTok{loop} \NormalTok{\{\} \}}
\end{Highlighting}
\end{Shaded}

The compiler currently makes a few assumptions about symbols which are
available in the executable to call. Normally these functions are
provided by the standard library, but without it you must define your
own.

The first of these two functions, \texttt{eh\_personality}, is used by
the failure mechanisms of the compiler. This is often mapped to GCC's
personality function (see the
\href{https://github.com/rust-lang/rust/blob/master/src/libstd/sys/common/unwind/gcc.rs}{libstd
implementation} for more information), but crates which do not trigger a
panic can be assured that this function is never called. The second
function, \texttt{panic\_fmt}, is also used by the failure mechanisms of
the compiler.

\hypertarget{sec--intrinsics}{\section{Intrinsics}\label{sec--intrinsics}}

\begin{quote}
\textbf{Note}: intrinsics will forever have an unstable interface, it is
recommended to use the stable interfaces of libcore rather than
intrinsics directly.
\end{quote}

These are imported as if they were FFI functions, with the special
\texttt{rust-intrinsic} ABI. For example, if one was in a freestanding
context, but wished to be able to \texttt{transmute} between types, and
perform efficient pointer arithmetic, one would import those functions
via a declaration like

\begin{Shaded}
\begin{Highlighting}[]
\AttributeTok{#![}\NormalTok{feature}\AttributeTok{(}\NormalTok{intrinsics}\AttributeTok{)]}

\KeywordTok{extern} \StringTok{"rust-intrinsic"} \NormalTok{\{}
    \KeywordTok{fn} \NormalTok{transmute<T, U>(x: T) -> U;}

    \KeywordTok{fn} \NormalTok{offset<T>(dst: *}\KeywordTok{const} \NormalTok{T, offset: }\DataTypeTok{isize}\NormalTok{) -> *}\KeywordTok{const} \NormalTok{T;}
\NormalTok{\}}
\end{Highlighting}
\end{Shaded}

As with any other FFI functions, these are always \texttt{unsafe} to
call.

\section{Lang items}\label{sec--lang-items}

\begin{quote}
\textbf{Note}: lang items are often provided by crates in the Rust
distribution, and lang items themselves have an unstable interface. It
is recommended to use officially distributed crates instead of defining
your own lang items.
\end{quote}

The \texttt{rustc} compiler has certain pluggable operations, that is,
functionality that isn't hard-coded into the language, but is
implemented in libraries, with a special marker to tell the compiler it
exists. The marker is the attribute \texttt{\#{[}lang\ =\ "..."{]}} and
there are various different values of \texttt{...}, i.e.~various
different `lang items'.

For example, \texttt{Box} pointers require two lang items, one for
allocation and one for deallocation. A freestanding program that uses
the \texttt{Box} sugar for dynamic allocations via \texttt{malloc} and
\texttt{free}:

\begin{Shaded}
\begin{Highlighting}[]
\AttributeTok{#![}\NormalTok{feature}\AttributeTok{(}\NormalTok{lang_items}\AttributeTok{,} \NormalTok{box_syntax}\AttributeTok{,} \NormalTok{start}\AttributeTok{,} \NormalTok{libc}\AttributeTok{)]}
\AttributeTok{#![}\NormalTok{no_std}\AttributeTok{]}

\KeywordTok{extern} \KeywordTok{crate} \NormalTok{libc;}

\KeywordTok{extern} \NormalTok{\{}
    \KeywordTok{fn} \NormalTok{abort() -> !;}
\NormalTok{\}}

\AttributeTok{#[}\NormalTok{lang }\AttributeTok{=} \StringTok{"owned_box"}\AttributeTok{]}
\KeywordTok{pub} \KeywordTok{struct} \DataTypeTok{Box}\NormalTok{<T>(*}\KeywordTok{mut} \NormalTok{T);}

\AttributeTok{#[}\NormalTok{lang }\AttributeTok{=} \StringTok{"exchange_malloc"}\AttributeTok{]}
\KeywordTok{unsafe} \KeywordTok{fn} \NormalTok{allocate(size: }\DataTypeTok{usize}\NormalTok{, _align: }\DataTypeTok{usize}\NormalTok{) -> *}\KeywordTok{mut} \DataTypeTok{u8} \NormalTok{\{}
    \KeywordTok{let} \NormalTok{p = libc::malloc(size }\KeywordTok{as} \NormalTok{libc::}\DataTypeTok{size_t}\NormalTok{) }\KeywordTok{as} \NormalTok{*}\KeywordTok{mut} \DataTypeTok{u8}\NormalTok{;}

    \CommentTok{// malloc failed}
    \KeywordTok{if} \NormalTok{p }\KeywordTok{as} \DataTypeTok{usize} \NormalTok{== }\DecValTok{0} \NormalTok{\{}
        \NormalTok{abort();}
    \NormalTok{\}}

    \NormalTok{p}
\NormalTok{\}}

\AttributeTok{#[}\NormalTok{lang }\AttributeTok{=} \StringTok{"exchange_free"}\AttributeTok{]}
\KeywordTok{unsafe} \KeywordTok{fn} \NormalTok{deallocate(ptr: *}\KeywordTok{mut} \DataTypeTok{u8}\NormalTok{, _size: }\DataTypeTok{usize}\NormalTok{, _align: }\DataTypeTok{usize}\NormalTok{) \{}
    \NormalTok{libc::free(ptr }\KeywordTok{as} \NormalTok{*}\KeywordTok{mut} \NormalTok{libc::}\DataTypeTok{c_void}\NormalTok{)}
\NormalTok{\}}

\AttributeTok{#[}\NormalTok{lang }\AttributeTok{=} \StringTok{"box_free"}\AttributeTok{]}
\KeywordTok{unsafe} \KeywordTok{fn} \NormalTok{box_free<T>(ptr: *}\KeywordTok{mut} \NormalTok{T) \{}
    \NormalTok{deallocate(ptr }\KeywordTok{as} \NormalTok{*}\KeywordTok{mut} \DataTypeTok{u8}\NormalTok{, ::core::mem::size_of::<T>(), ::core::mem::align_of::<T>}
\NormalTok{↳ ());}
\NormalTok{\}}

\AttributeTok{#[}\NormalTok{start}\AttributeTok{]}
\KeywordTok{fn} \NormalTok{main(argc: }\DataTypeTok{isize}\NormalTok{, argv: *}\KeywordTok{const} \NormalTok{*}\KeywordTok{const} \DataTypeTok{u8}\NormalTok{) -> }\DataTypeTok{isize} \NormalTok{\{}
    \KeywordTok{let} \NormalTok{x = }\KeywordTok{box} \DecValTok{1}\NormalTok{;}

    \DecValTok{0}
\NormalTok{\}}

\AttributeTok{#[}\NormalTok{lang }\AttributeTok{=} \StringTok{"eh_personality"}\AttributeTok{]} \KeywordTok{extern} \KeywordTok{fn} \NormalTok{eh_personality() \{\}}
\AttributeTok{#[}\NormalTok{lang }\AttributeTok{=} \StringTok{"panic_fmt"}\AttributeTok{]} \KeywordTok{fn} \NormalTok{panic_fmt() -> ! \{ }\KeywordTok{loop} \NormalTok{\{\} \}}
\end{Highlighting}
\end{Shaded}

Note the use of \texttt{abort}: the \texttt{exchange\_malloc} lang item
is assumed to return a valid pointer, and so needs to do the check
internally.

Other features provided by lang items include:

\begin{itemize}
\tightlist
\item
  overloadable operators via traits: the traits corresponding to the
  \texttt{==}, \texttt{\textless{}}, dereferencing (\texttt{*}) and
  \texttt{+} (etc.) operators are all marked with lang items; those
  specific four are \texttt{eq}, \texttt{ord}, \texttt{deref}, and
  \texttt{add} respectively.
\item
  stack unwinding and general failure; the \texttt{eh\_personality},
  \texttt{fail} and \texttt{fail\_bounds\_checks} lang items.
\item
  the traits in \texttt{std::marker} used to indicate types of various
  kinds; lang items \texttt{send}, \texttt{sync} and \texttt{copy}.
\item
  the marker types and variance indicators found in
  \texttt{std::marker}; lang items \texttt{covariant\_type},
  \texttt{contravariant\_lifetime}, etc.
\end{itemize}

Lang items are loaded lazily by the compiler; e.g.~if one never uses
\texttt{Box} then there is no need to define functions for
\texttt{exchange\_malloc} and \texttt{exchange\_free}. \texttt{rustc}
will emit an error when an item is needed but not found in the current
crate or any that it depends on.

\section{Advanced linking}\label{sec--advanced-linking}

The common cases of linking with Rust have been covered earlier in this
book, but supporting the range of linking possibilities made available
by other languages is important for Rust to achieve seamless interaction
with native libraries.

\subsection{Link args}\label{link-args}

There is one other way to tell \texttt{rustc} how to customize linking,
and that is via the \texttt{link\_args} attribute. This attribute is
applied to \texttt{extern} blocks and specifies raw flags which need to
get passed to the linker when producing an artifact. An example usage
would be:

\begin{Shaded}
\begin{Highlighting}[]
\AttributeTok{#![}\NormalTok{feature}\AttributeTok{(}\NormalTok{link_args}\AttributeTok{)]}

\AttributeTok{#[}\NormalTok{link_args }\AttributeTok{=} \StringTok{"-foo -bar -baz"}\AttributeTok{]}
\KeywordTok{extern} \NormalTok{\{\}}
\end{Highlighting}
\end{Shaded}

Note that this feature is currently hidden behind the
\texttt{feature(link\_args)} gate because this is not a sanctioned way
of performing linking. Right now \texttt{rustc} shells out to the system
linker (\texttt{gcc} on most systems, \texttt{link.exe} on MSVC), so it
makes sense to provide extra command line arguments, but this will not
always be the case. In the future \texttt{rustc} may use LLVM directly
to link native libraries, in which case \texttt{link\_args} will have no
meaning. You can achieve the same effect as the \texttt{link\_args}
attribute with the \texttt{-C\ link-args} argument to \texttt{rustc}.

It is highly recommended to \emph{not} use this attribute, and rather
use the more formal \texttt{\#{[}link(...){]}} attribute on
\texttt{extern} blocks instead.

\subsection{Static linking}\label{static-linking}

Static linking refers to the process of creating output that contains
all required libraries and so doesn't need libraries installed on every
system where you want to use your compiled project. Pure-Rust
dependencies are statically linked by default so you can use created
binaries and libraries without installing Rust everywhere. By contrast,
native libraries (e.g. \texttt{libc} and \texttt{libm}) are usually
dynamically linked, but it is possible to change this and statically
link them as well.

Linking is a very platform-dependent topic, and static linking may not
even be possible on some platforms! This section assumes some basic
familiarity with linking on your platform of choice.

\subsubsection{Linux}\label{linux}

By default, all Rust programs on Linux will link to the system
\texttt{libc} along with a number of other libraries. Let's look at an
example on a 64-bit Linux machine with GCC and \texttt{glibc} (by far
the most common \texttt{libc} on Linux):

\begin{verbatim}
$ cat example.rs
fn main() {}
$ rustc example.rs
$ ldd example
        linux-vdso.so.1 =>  (0x00007ffd565fd000)
        libdl.so.2 => /lib/x86_64-linux-gnu/libdl.so.2 (0x00007fa81889c000)
        libpthread.so.0 => /lib/x86_64-linux-gnu/libpthread.so.0 (0x00007fa81867e000)
        librt.so.1 => /lib/x86_64-linux-gnu/librt.so.1 (0x00007fa818475000)
        libgcc_s.so.1 => /lib/x86_64-linux-gnu/libgcc_s.so.1 (0x00007fa81825f000)
        libc.so.6 => /lib/x86_64-linux-gnu/libc.so.6 (0x00007fa817e9a000)
        /lib64/ld-linux-x86-64.so.2 (0x00007fa818cf9000)
        libm.so.6 => /lib/x86_64-linux-gnu/libm.so.6 (0x00007fa817b93000)
\end{verbatim}

Dynamic linking on Linux can be undesirable if you wish to use new
library features on old systems or target systems which do not have the
required dependencies for your program to run.

Static linking is supported via an alternative \texttt{libc},
\href{http://www.musl-libc.org}{\texttt{musl}}. You can compile your own
version of Rust with \texttt{musl} enabled and install it into a custom
directory with the instructions below:

\begin{verbatim}
$ mkdir musldist
$ PREFIX=$(pwd)/musldist
$
$ # Build musl
$ curl -O http://www.musl-libc.org/releases/musl-1.1.10.tar.gz
$ tar xf musl-1.1.10.tar.gz
$ cd musl-1.1.10/
musl-1.1.10 $ ./configure --disable-shared --prefix=$PREFIX
musl-1.1.10 $ make
musl-1.1.10 $ make install
musl-1.1.10 $ cd ..
$ du -h musldist/lib/libc.a
2.2M    musldist/lib/libc.a
$
$ # Build libunwind.a
$ curl -O http://llvm.org/releases/3.7.0/llvm-3.7.0.src.tar.xz
$ tar xf llvm-3.7.0.src.tar.xz
$ cd llvm-3.7.0.src/projects/
llvm-3.7.0.src/projects $ curl http://llvm.org/releases/3.7.0/libunwind-3.7.0.src.tar.
↳ xz | tar xJf -
llvm-3.7.0.src/projects $ mv libunwind-3.7.0.src libunwind
llvm-3.7.0.src/projects $ mkdir libunwind/build
llvm-3.7.0.src/projects $ cd libunwind/build
llvm-3.7.0.src/projects/libunwind/build $ cmake -DLLVM_PATH=../../.. -DLIBUNWIND_ENABL
↳ E_SHARED=0 ..
llvm-3.7.0.src/projects/libunwind/build $ make
llvm-3.7.0.src/projects/libunwind/build $ cp lib/libunwind.a $PREFIX/lib/
llvm-3.7.0.src/projects/libunwind/build $ cd ../../../../
$ du -h musldist/lib/libunwind.a
164K    musldist/lib/libunwind.a
$
$ # Build musl-enabled rust
$ git clone https://github.com/rust-lang/rust.git muslrust
$ cd muslrust
muslrust $ ./configure --target=x86_64-unknown-linux-musl --musl-root=$PREFIX --prefix
↳ =$PREFIX
muslrust $ make
muslrust $ make install
muslrust $ cd ..
$ du -h musldist/bin/rustc
12K     musldist/bin/rustc
\end{verbatim}

You now have a build of a \texttt{musl}-enabled Rust! Because we've
installed it to a custom prefix we need to make sure our system can find
the binaries and appropriate libraries when we try and run it:

\begin{verbatim}
$ export PATH=$PREFIX/bin:$PATH
$ export LD_LIBRARY_PATH=$PREFIX/lib:$LD_LIBRARY_PATH
\end{verbatim}

Let's try it out!

\begin{verbatim}
$ echo 'fn main() { println!("hi!"); panic!("failed"); }' > example.rs
$ rustc --target=x86_64-unknown-linux-musl example.rs
$ ldd example
        not a dynamic executable
$ ./example
hi!
thread '<main>' panicked at 'failed', example.rs:1
\end{verbatim}

Success! This binary can be copied to almost any Linux machine with the
same machine architecture and run without issues.

\texttt{cargo\ build} also permits the \texttt{-\/-target} option so you
should be able to build your crates as normal. However, you may need to
recompile your native libraries against \texttt{musl} before they can be
linked against.

\section{Benchmark Tests}\label{sec--benchmark-tests}

Rust supports benchmark tests, which can test the performance of your
code. Let's make our \texttt{src/lib.rs} look like this (comments
elided):

\begin{Shaded}
\begin{Highlighting}[]
\AttributeTok{#![}\NormalTok{feature}\AttributeTok{(}\NormalTok{test}\AttributeTok{)]}

\KeywordTok{extern} \KeywordTok{crate} \NormalTok{test;}

\KeywordTok{pub} \KeywordTok{fn} \NormalTok{add_two(a: }\DataTypeTok{i32}\NormalTok{) -> }\DataTypeTok{i32} \NormalTok{\{}
    \NormalTok{a + }\DecValTok{2}
\NormalTok{\}}

\AttributeTok{#[}\NormalTok{cfg}\AttributeTok{(}\NormalTok{test}\AttributeTok{)]}
\KeywordTok{mod} \NormalTok{tests \{}
    \KeywordTok{use} \KeywordTok{super}\NormalTok{::*;}
    \KeywordTok{use} \NormalTok{test::Bencher;}

    \AttributeTok{#[}\NormalTok{test}\AttributeTok{]}
    \KeywordTok{fn} \NormalTok{it_works() \{}
        \PreprocessorTok{assert_eq!}\NormalTok{(}\DecValTok{4}\NormalTok{, add_two(}\DecValTok{2}\NormalTok{));}
    \NormalTok{\}}

    \AttributeTok{#[}\NormalTok{bench}\AttributeTok{]}
    \KeywordTok{fn} \NormalTok{bench_add_two(b: &}\KeywordTok{mut} \NormalTok{Bencher) \{}
        \NormalTok{b.iter(|| add_two(}\DecValTok{2}\NormalTok{));}
    \NormalTok{\}}
\NormalTok{\}}
\end{Highlighting}
\end{Shaded}

Note the \texttt{test} feature gate, which enables this unstable
feature.

We've imported the \texttt{test} crate, which contains our benchmarking
support. We have a new function as well, with the \texttt{bench}
attribute. Unlike regular tests, which take no arguments, benchmark
tests take a \texttt{\&mut\ Bencher}. This \texttt{Bencher} provides an
\texttt{iter} method, which takes a closure. This closure contains the
code we'd like to benchmark.

We can run benchmark tests with \texttt{cargo\ bench}:

\begin{Shaded}
\begin{Highlighting}[]
\NormalTok{$ }\KeywordTok{cargo} \NormalTok{bench}
   \KeywordTok{Compiling} \NormalTok{adder v0.0.1 (file:///home/steve/tmp/adder)}
     \KeywordTok{Running} \NormalTok{target/release/adder-91b3e234d4ed382a}

\KeywordTok{running} \NormalTok{2 tests}
\KeywordTok{test} \NormalTok{tests::it_works ... ignored}
\KeywordTok{test} \NormalTok{tests::bench_add_two ... bench:         1 ns/iter (+/- 0)}

\KeywordTok{test} \NormalTok{result: ok. 0 passed}\KeywordTok{;} \KeywordTok{0} \NormalTok{failed}\KeywordTok{;} \KeywordTok{1} \NormalTok{ignored}\KeywordTok{;} \KeywordTok{1} \NormalTok{measured}
\end{Highlighting}
\end{Shaded}

Our non-benchmark test was ignored. You may have noticed that
\texttt{cargo\ bench} takes a bit longer than \texttt{cargo\ test}. This
is because Rust runs our benchmark a number of times, and then takes the
average. Because we're doing so little work in this example, we have a
\texttt{1\ ns/iter\ (+/-\ 0)}, but this would show the variance if there
was one.

Advice on writing benchmarks:

\begin{itemize}
\tightlist
\item
  Move setup code outside the \texttt{iter} loop; only put the part you
  want to measure inside
\item
  Make the code do ``the same thing'' on each iteration; do not
  accumulate or change state
\item
  Make the outer function idempotent too; the benchmark runner is likely
  to run it many times
\item
  Make the inner \texttt{iter} loop short and fast so benchmark runs are
  fast and the calibrator can adjust the run-length at fine resolution
\item
  Make the code in the \texttt{iter} loop do something simple, to assist
  in pinpointing performance improvements (or regressions)
\end{itemize}

\subsubsection{Gotcha: optimizations}\label{gotcha-optimizations}

There's another tricky part to writing benchmarks: benchmarks compiled
with optimizations activated can be dramatically changed by the
optimizer so that the benchmark is no longer benchmarking what one
expects. For example, the compiler might recognize that some calculation
has no external effects and remove it entirely.

\begin{Shaded}
\begin{Highlighting}[]
\AttributeTok{#![}\NormalTok{feature}\AttributeTok{(}\NormalTok{test}\AttributeTok{)]}

\KeywordTok{extern} \KeywordTok{crate} \NormalTok{test;}
\KeywordTok{use} \NormalTok{test::Bencher;}

\AttributeTok{#[}\NormalTok{bench}\AttributeTok{]}
\KeywordTok{fn} \NormalTok{bench_xor_1000_ints(b: &}\KeywordTok{mut} \NormalTok{Bencher) \{}
    \NormalTok{b.iter(|| \{}
        \NormalTok{(}\DecValTok{0.}\NormalTok{.}\DecValTok{1000}\NormalTok{).fold(}\DecValTok{0}\NormalTok{, |old, new| old ^ new);}
    \NormalTok{\});}
\NormalTok{\}}
\end{Highlighting}
\end{Shaded}

gives the following results

\begin{verbatim}
running 1 test
test bench_xor_1000_ints ... bench:         0 ns/iter (+/- 0)

test result: ok. 0 passed; 0 failed; 0 ignored; 1 measured
\end{verbatim}

The benchmarking runner offers two ways to avoid this. Either, the
closure that the \texttt{iter} method receives can return an arbitrary
value which forces the optimizer to consider the result used and ensures
it cannot remove the computation entirely. This could be done for the
example above by adjusting the \texttt{b.iter} call to

\begin{Shaded}
\begin{Highlighting}[]
\NormalTok{b.iter(|| \{}
    \CommentTok{// note lack of `;` (could also use an explicit `return`).}
    \NormalTok{(}\DecValTok{0.}\NormalTok{.}\DecValTok{1000}\NormalTok{).fold(}\DecValTok{0}\NormalTok{, |old, new| old ^ new)}
\NormalTok{\});}
\end{Highlighting}
\end{Shaded}

Or, the other option is to call the generic \texttt{test::black\_box}
function, which is an opaque ``black box'' to the optimizer and so
forces it to consider any argument as used.

\begin{Shaded}
\begin{Highlighting}[]
\AttributeTok{#![}\NormalTok{feature}\AttributeTok{(}\NormalTok{test}\AttributeTok{)]}

\KeywordTok{extern} \KeywordTok{crate} \NormalTok{test;}

\NormalTok{b.iter(|| \{}
    \KeywordTok{let} \NormalTok{n = test::black_box(}\DecValTok{1000}\NormalTok{);}

    \NormalTok{(}\DecValTok{0.}\NormalTok{.n).fold(}\DecValTok{0}\NormalTok{, |a, b| a ^ b)}
\NormalTok{\})}
\end{Highlighting}
\end{Shaded}

Neither of these read or modify the value, and are very cheap for small
values. Larger values can be passed indirectly to reduce overhead (e.g.
\texttt{black\_box(\&huge\_struct)}).

Performing either of the above changes gives the following benchmarking
results

\begin{verbatim}
running 1 test
test bench_xor_1000_ints ... bench:       131 ns/iter (+/- 3)

test result: ok. 0 passed; 0 failed; 0 ignored; 1 measured
\end{verbatim}

However, the optimizer can still modify a testcase in an undesirable
manner even when using either of the above.

\section{Box Syntax and Patterns}\label{sec--box-syntax-and-patterns}

Currently the only stable way to create a \texttt{Box} is via the
\texttt{Box::new} method. Also it is not possible in stable Rust to
destructure a \texttt{Box} in a match pattern. The unstable \texttt{box}
keyword can be used to both create and destructure a \texttt{Box}. An
example usage would be:

\begin{Shaded}
\begin{Highlighting}[]
\AttributeTok{#![}\NormalTok{feature}\AttributeTok{(}\NormalTok{box_syntax}\AttributeTok{,} \NormalTok{box_patterns}\AttributeTok{)]}

\KeywordTok{fn} \NormalTok{main() \{}
    \KeywordTok{let} \NormalTok{b = }\ConstantTok{Some}\NormalTok{(}\KeywordTok{box} \DecValTok{5}\NormalTok{);}
    \KeywordTok{match} \NormalTok{b \{}
        \ConstantTok{Some}\NormalTok{(}\KeywordTok{box} \NormalTok{n) }\KeywordTok{if} \NormalTok{n < }\DecValTok{0} \NormalTok{=> \{}
            \PreprocessorTok{println!}\NormalTok{(}\StringTok{"Box contains negative number \{\}"}\NormalTok{, n);}
        \NormalTok{\},}
        \ConstantTok{Some}\NormalTok{(}\KeywordTok{box} \NormalTok{n) }\KeywordTok{if} \NormalTok{n >= }\DecValTok{0} \NormalTok{=> \{}
            \PreprocessorTok{println!}\NormalTok{(}\StringTok{"Box contains non-negative number \{\}"}\NormalTok{, n);}
        \NormalTok{\},}
        \ConstantTok{None} \NormalTok{=> \{}
            \PreprocessorTok{println!}\NormalTok{(}\StringTok{"No box"}\NormalTok{);}
        \NormalTok{\},}
        \NormalTok{_ => }\PreprocessorTok{unreachable!}\NormalTok{()}
    \NormalTok{\}}
\NormalTok{\}}
\end{Highlighting}
\end{Shaded}

Note that these features are currently hidden behind the
\texttt{box\_syntax} (box creation) and \texttt{box\_patterns}
(destructuring and pattern matching) gates because the syntax may still
change in the future.

\subsection{Returning Pointers}\label{returning-pointers}

In many languages with pointers, you'd return a pointer from a function
so as to avoid copying a large data structure. For example:

\begin{Shaded}
\begin{Highlighting}[]
\KeywordTok{struct} \NormalTok{BigStruct \{}
    \NormalTok{one: }\DataTypeTok{i32}\NormalTok{,}
    \NormalTok{two: }\DataTypeTok{i32}\NormalTok{,}
    \CommentTok{// etc}
    \NormalTok{one_hundred: }\DataTypeTok{i32}\NormalTok{,}
\NormalTok{\}}

\KeywordTok{fn} \NormalTok{foo(x: }\DataTypeTok{Box}\NormalTok{<BigStruct>) -> }\DataTypeTok{Box}\NormalTok{<BigStruct> \{}
    \DataTypeTok{Box}\NormalTok{::new(*x)}
\NormalTok{\}}

\KeywordTok{fn} \NormalTok{main() \{}
    \KeywordTok{let} \NormalTok{x = }\DataTypeTok{Box}\NormalTok{::new(BigStruct \{}
        \NormalTok{one: }\DecValTok{1}\NormalTok{,}
        \NormalTok{two: }\DecValTok{2}\NormalTok{,}
        \NormalTok{one_hundred: }\DecValTok{100}\NormalTok{,}
    \NormalTok{\});}

    \KeywordTok{let} \NormalTok{y = foo(x);}
\NormalTok{\}}
\end{Highlighting}
\end{Shaded}

The idea is that by passing around a box, you're only copying a pointer,
rather than the hundred \texttt{i32}s that make up the
\texttt{BigStruct}.

This is an antipattern in Rust. Instead, write this:

\begin{Shaded}
\begin{Highlighting}[]
\AttributeTok{#![}\NormalTok{feature}\AttributeTok{(}\NormalTok{box_syntax}\AttributeTok{)]}

\KeywordTok{struct} \NormalTok{BigStruct \{}
    \NormalTok{one: }\DataTypeTok{i32}\NormalTok{,}
    \NormalTok{two: }\DataTypeTok{i32}\NormalTok{,}
    \CommentTok{// etc}
    \NormalTok{one_hundred: }\DataTypeTok{i32}\NormalTok{,}
\NormalTok{\}}

\KeywordTok{fn} \NormalTok{foo(x: }\DataTypeTok{Box}\NormalTok{<BigStruct>) -> BigStruct \{}
    \NormalTok{*x}
\NormalTok{\}}

\KeywordTok{fn} \NormalTok{main() \{}
    \KeywordTok{let} \NormalTok{x = }\DataTypeTok{Box}\NormalTok{::new(BigStruct \{}
        \NormalTok{one: }\DecValTok{1}\NormalTok{,}
        \NormalTok{two: }\DecValTok{2}\NormalTok{,}
        \NormalTok{one_hundred: }\DecValTok{100}\NormalTok{,}
    \NormalTok{\});}

    \KeywordTok{let} \NormalTok{y: }\DataTypeTok{Box}\NormalTok{<BigStruct> = }\KeywordTok{box} \NormalTok{foo(x);}
\NormalTok{\}}
\end{Highlighting}
\end{Shaded}

This gives you flexibility without sacrificing performance.

You may think that this gives us terrible performance: return a value
and then immediately box it up ?! Isn't this pattern the worst of both
worlds? Rust is smarter than that. There is no copy in this code.
\texttt{main} allocates enough room for the \texttt{box}, passes a
pointer to that memory into \texttt{foo} as \texttt{x}, and then
\texttt{foo} writes the value straight into the
\texttt{Box\textless{}T\textgreater{}}.

This is important enough that it bears repeating: pointers are not for
optimizing returning values from your code. Allow the caller to choose
how they want to use your output.

\section{Slice Patterns}\label{sec--slice-patterns}

If you want to match against a slice or array, you can use \texttt{\&}
with the \texttt{slice\_patterns} feature:

\begin{Shaded}
\begin{Highlighting}[]
\AttributeTok{#![}\NormalTok{feature}\AttributeTok{(}\NormalTok{slice_patterns}\AttributeTok{)]}

\KeywordTok{fn} \NormalTok{main() \{}
    \KeywordTok{let} \NormalTok{v = }\PreprocessorTok{vec!}\NormalTok{[}\StringTok{"match_this"}\NormalTok{, }\StringTok{"1"}\NormalTok{];}

    \KeywordTok{match} \NormalTok{&v[..] \{}
        \NormalTok{[}\StringTok{"match_this"}\NormalTok{, second] => }\PreprocessorTok{println!}\NormalTok{(}\StringTok{"The second element is \{\}"}\NormalTok{, second),}
        \NormalTok{_ => \{\},}
    \NormalTok{\}}
\NormalTok{\}}
\end{Highlighting}
\end{Shaded}

The \texttt{advanced\_slice\_patterns} gate lets you use \texttt{..} to
indicate any number of elements inside a pattern matching a slice. This
wildcard can only be used once for a given array. If there's an
identifier before the \texttt{..}, the result of the slice will be bound
to that name. For example:

\begin{Shaded}
\begin{Highlighting}[]
\AttributeTok{#![}\NormalTok{feature}\AttributeTok{(}\NormalTok{advanced_slice_patterns}\AttributeTok{,} \NormalTok{slice_patterns}\AttributeTok{)]}

\KeywordTok{fn} \NormalTok{is_symmetric(list: &[}\DataTypeTok{u32}\NormalTok{]) -> }\DataTypeTok{bool} \NormalTok{\{}
    \KeywordTok{match} \NormalTok{list \{}
        \NormalTok{[] | [_] => }\ConstantTok{true}\NormalTok{,}
        \NormalTok{[x, inside.., y] }\KeywordTok{if} \NormalTok{x == y => is_symmetric(inside),}
        \NormalTok{_ => }\ConstantTok{false}
    \NormalTok{\}}
\NormalTok{\}}

\KeywordTok{fn} \NormalTok{main() \{}
    \KeywordTok{let} \NormalTok{sym = &[}\DecValTok{0}\NormalTok{, }\DecValTok{1}\NormalTok{, }\DecValTok{4}\NormalTok{, }\DecValTok{2}\NormalTok{, }\DecValTok{4}\NormalTok{, }\DecValTok{1}\NormalTok{, }\DecValTok{0}\NormalTok{];}
    \PreprocessorTok{assert!}\NormalTok{(is_symmetric(sym));}

    \KeywordTok{let} \NormalTok{not_sym = &[}\DecValTok{0}\NormalTok{, }\DecValTok{1}\NormalTok{, }\DecValTok{7}\NormalTok{, }\DecValTok{2}\NormalTok{, }\DecValTok{4}\NormalTok{, }\DecValTok{1}\NormalTok{, }\DecValTok{0}\NormalTok{];}
    \PreprocessorTok{assert!}\NormalTok{(!is_symmetric(not_sym));}
\NormalTok{\}}
\end{Highlighting}
\end{Shaded}

\section{Associated Constants}\label{sec--associated-constants}

With the \texttt{associated\_consts} feature, you can define constants
like this:

\begin{Shaded}
\begin{Highlighting}[]
\AttributeTok{#![}\NormalTok{feature}\AttributeTok{(}\NormalTok{associated_consts}\AttributeTok{)]}

\KeywordTok{trait} \NormalTok{Foo \{}
    \KeywordTok{const} \NormalTok{ID: }\DataTypeTok{i32}\NormalTok{;}
\NormalTok{\}}

\KeywordTok{impl} \NormalTok{Foo }\KeywordTok{for} \DataTypeTok{i32} \NormalTok{\{}
    \KeywordTok{const} \NormalTok{ID: }\DataTypeTok{i32} \NormalTok{= }\DecValTok{1}\NormalTok{;}
\NormalTok{\}}

\KeywordTok{fn} \NormalTok{main() \{}
    \PreprocessorTok{assert_eq!}\NormalTok{(}\DecValTok{1}\NormalTok{, }\DataTypeTok{i32}\NormalTok{::ID);}
\NormalTok{\}}
\end{Highlighting}
\end{Shaded}

Any implementor of \texttt{Foo} will have to define \texttt{ID}. Without
the definition:

\begin{Shaded}
\begin{Highlighting}[]
\AttributeTok{#![}\NormalTok{feature}\AttributeTok{(}\NormalTok{associated_consts}\AttributeTok{)]}

\KeywordTok{trait} \NormalTok{Foo \{}
    \KeywordTok{const} \NormalTok{ID: }\DataTypeTok{i32}\NormalTok{;}
\NormalTok{\}}

\KeywordTok{impl} \NormalTok{Foo }\KeywordTok{for} \DataTypeTok{i32} \NormalTok{\{}
\NormalTok{\}}
\end{Highlighting}
\end{Shaded}

gives

\begin{verbatim}
error: not all trait items implemented, missing: `ID` [E0046]
     impl Foo for i32 {
     }
\end{verbatim}

A default value can be implemented as well:

\begin{Shaded}
\begin{Highlighting}[]
\AttributeTok{#![}\NormalTok{feature}\AttributeTok{(}\NormalTok{associated_consts}\AttributeTok{)]}

\KeywordTok{trait} \NormalTok{Foo \{}
    \KeywordTok{const} \NormalTok{ID: }\DataTypeTok{i32} \NormalTok{= }\DecValTok{1}\NormalTok{;}
\NormalTok{\}}

\KeywordTok{impl} \NormalTok{Foo }\KeywordTok{for} \DataTypeTok{i32} \NormalTok{\{}
\NormalTok{\}}

\KeywordTok{impl} \NormalTok{Foo }\KeywordTok{for} \DataTypeTok{i64} \NormalTok{\{}
    \KeywordTok{const} \NormalTok{ID: }\DataTypeTok{i32} \NormalTok{= }\DecValTok{5}\NormalTok{;}
\NormalTok{\}}

\KeywordTok{fn} \NormalTok{main() \{}
    \PreprocessorTok{assert_eq!}\NormalTok{(}\DecValTok{1}\NormalTok{, }\DataTypeTok{i32}\NormalTok{::ID);}
    \PreprocessorTok{assert_eq!}\NormalTok{(}\DecValTok{5}\NormalTok{, }\DataTypeTok{i64}\NormalTok{::ID);}
\NormalTok{\}}
\end{Highlighting}
\end{Shaded}

As you can see, when implementing \texttt{Foo}, you can leave it
unimplemented, as with \texttt{i32}. It will then use the default value.
But, as in \texttt{i64}, we can also add our own definition.

Associated constants don't have to be associated with a trait. An
\texttt{impl} block for a \texttt{struct} or an \texttt{enum} works fine
too:

\begin{Shaded}
\begin{Highlighting}[]
\AttributeTok{#![}\NormalTok{feature}\AttributeTok{(}\NormalTok{associated_consts}\AttributeTok{)]}

\KeywordTok{struct} \NormalTok{Foo;}

\KeywordTok{impl} \NormalTok{Foo \{}
    \KeywordTok{const} \NormalTok{FOO: }\DataTypeTok{u32} \NormalTok{= }\DecValTok{3}\NormalTok{;}
\NormalTok{\}}
\end{Highlighting}
\end{Shaded}

\section{Custom Allocators}\label{sec--custom-allocators}

Allocating memory isn't always the easiest thing to do, and while Rust
generally takes care of this by default it often becomes necessary to
customize how allocation occurs. The compiler and standard library
currently allow switching out the default global allocator in use at
compile time. The design is currently spelled out in
\href{https://github.com/rust-lang/rfcs/blob/master/text/1183-swap-out-jemalloc.md}{RFC
1183} but this will walk you through how to get your own allocator up
and running.

\subsection{Default Allocator}\label{default-allocator}

The compiler currently ships two default allocators:
\texttt{alloc\_system} and \texttt{alloc\_jemalloc} (some targets don't
have jemalloc, however). These allocators are normal Rust crates and
contain an implementation of the routines to allocate and deallocate
memory. The standard library is not compiled assuming either one, and
the compiler will decide which allocator is in use at compile-time
depending on the type of output artifact being produced.

Binaries generated by the compiler will use \texttt{alloc\_jemalloc} by
default (where available). In this situation the compiler ``controls the
world'' in the sense of it has power over the final link. Primarily this
means that the allocator decision can be left up the compiler.

Dynamic and static libraries, however, will use \texttt{alloc\_system}
by default. Here Rust is typically a `guest' in another application or
another world where it cannot authoritatively decide what allocator is
in use. As a result it resorts back to the standard APIs (e.g.
\texttt{malloc} and \texttt{free}) for acquiring and releasing memory.

\subsection{Switching Allocators}\label{switching-allocators}

Although the compiler's default choices may work most of the time, it's
often necessary to tweak certain aspects. Overriding the compiler's
decision about which allocator is in use is done simply by linking to
the desired allocator:

\begin{Shaded}
\begin{Highlighting}[]
\AttributeTok{#![}\NormalTok{feature}\AttributeTok{(}\NormalTok{alloc_system}\AttributeTok{)]}

\KeywordTok{extern} \KeywordTok{crate} \NormalTok{alloc_system;}

\KeywordTok{fn} \NormalTok{main() \{}
    \KeywordTok{let} \NormalTok{a = }\DataTypeTok{Box}\NormalTok{::new(}\DecValTok{4}\NormalTok{); }\CommentTok{// allocates from the system allocator}
    \PreprocessorTok{println!}\NormalTok{(}\StringTok{"\{\}"}\NormalTok{, a);}
\NormalTok{\}}
\end{Highlighting}
\end{Shaded}

In this example the binary generated will not link to jemalloc by
default but instead use the system allocator. Conversely to generate a
dynamic library which uses jemalloc by default one would write:

\begin{Shaded}
\begin{Highlighting}[]
\AttributeTok{#![}\NormalTok{feature}\AttributeTok{(}\NormalTok{alloc_jemalloc}\AttributeTok{)]}
\AttributeTok{#![}\NormalTok{crate_type }\AttributeTok{=} \StringTok{"dylib"}\AttributeTok{]}

\KeywordTok{extern} \KeywordTok{crate} \NormalTok{alloc_jemalloc;}

\KeywordTok{pub} \KeywordTok{fn} \NormalTok{foo() \{}
    \KeywordTok{let} \NormalTok{a = }\DataTypeTok{Box}\NormalTok{::new(}\DecValTok{4}\NormalTok{); }\CommentTok{// allocates from jemalloc}
    \PreprocessorTok{println!}\NormalTok{(}\StringTok{"\{\}"}\NormalTok{, a);}
\NormalTok{\}}
\end{Highlighting}
\end{Shaded}

\subsection{Writing a custom
allocator}\label{writing-a-custom-allocator}

Sometimes even the choices of jemalloc vs the system allocator aren't
enough and an entirely new custom allocator is required. In this you'll
write your own crate which implements the allocator API (e.g.~the same
as \texttt{alloc\_system} or \texttt{alloc\_jemalloc}). As an example,
let's take a look at a simplified and annotated version of
\texttt{alloc\_system}

\begin{Shaded}
\begin{Highlighting}[]
\CommentTok{// The compiler needs to be instructed that this crate is an allocator in order}
\CommentTok{// to realize that when this is linked in another allocator like jemalloc should}
\CommentTok{// not be linked in}
\AttributeTok{#![}\NormalTok{feature}\AttributeTok{(}\NormalTok{allocator}\AttributeTok{)]}
\AttributeTok{#![}\NormalTok{allocator}\AttributeTok{]}

\CommentTok{// Allocators are not allowed to depend on the standard library which in turn}
\CommentTok{// requires an allocator in order to avoid circular dependencies. This crate,}
\CommentTok{// however, can use all of libcore.}
\AttributeTok{#![}\NormalTok{no_std}\AttributeTok{]}

\CommentTok{// Let's give a unique name to our custom allocator}
\AttributeTok{#![}\NormalTok{crate_name }\AttributeTok{=} \StringTok{"my_allocator"}\AttributeTok{]}
\AttributeTok{#![}\NormalTok{crate_type }\AttributeTok{=} \StringTok{"rlib"}\AttributeTok{]}

\CommentTok{// Our system allocator will use the in-tree libc crate for FFI bindings. Note}
\CommentTok{// that currently the external (crates.io) libc cannot be used because it links}
\CommentTok{// to the standard library (e.g. `#![no_std]` isn't stable yet), so that's why}
\CommentTok{// this specifically requires the in-tree version.}
\AttributeTok{#![}\NormalTok{feature}\AttributeTok{(}\NormalTok{libc}\AttributeTok{)]}
\KeywordTok{extern} \KeywordTok{crate} \NormalTok{libc;}

\CommentTok{// Listed below are the five allocation functions currently required by custom}
\CommentTok{// allocators. Their signatures and symbol names are not currently typechecked}
\CommentTok{// by the compiler, but this is a future extension and are required to match}
\CommentTok{// what is found below.}
\CommentTok{//}
\CommentTok{// Note that the standard `malloc` and `realloc` functions do not provide a way}
\CommentTok{// to communicate alignment so this implementation would need to be improved}
\CommentTok{// with respect to alignment in that aspect.}

\AttributeTok{#[}\NormalTok{no_mangle}\AttributeTok{]}
\KeywordTok{pub} \KeywordTok{extern} \KeywordTok{fn} \NormalTok{__rust_allocate(size: }\DataTypeTok{usize}\NormalTok{, _align: }\DataTypeTok{usize}\NormalTok{) -> *}\KeywordTok{mut} \DataTypeTok{u8} \NormalTok{\{}
    \KeywordTok{unsafe} \NormalTok{\{ libc::malloc(size }\KeywordTok{as} \NormalTok{libc::}\DataTypeTok{size_t}\NormalTok{) }\KeywordTok{as} \NormalTok{*}\KeywordTok{mut} \DataTypeTok{u8} \NormalTok{\}}
\NormalTok{\}}

\AttributeTok{#[}\NormalTok{no_mangle}\AttributeTok{]}
\KeywordTok{pub} \KeywordTok{extern} \KeywordTok{fn} \NormalTok{__rust_deallocate(ptr: *}\KeywordTok{mut} \DataTypeTok{u8}\NormalTok{, _old_size: }\DataTypeTok{usize}\NormalTok{, _align: }\DataTypeTok{usize}\NormalTok{) \{}
    \KeywordTok{unsafe} \NormalTok{\{ libc::free(ptr }\KeywordTok{as} \NormalTok{*}\KeywordTok{mut} \NormalTok{libc::}\DataTypeTok{c_void}\NormalTok{) \}}
\NormalTok{\}}

\AttributeTok{#[}\NormalTok{no_mangle}\AttributeTok{]}
\KeywordTok{pub} \KeywordTok{extern} \KeywordTok{fn} \NormalTok{__rust_reallocate(ptr: *}\KeywordTok{mut} \DataTypeTok{u8}\NormalTok{, _old_size: }\DataTypeTok{usize}\NormalTok{, size: }\DataTypeTok{usize}\NormalTok{,}
                                \NormalTok{_align: }\DataTypeTok{usize}\NormalTok{) -> *}\KeywordTok{mut} \DataTypeTok{u8} \NormalTok{\{}
    \KeywordTok{unsafe} \NormalTok{\{}
        \NormalTok{libc::realloc(ptr }\KeywordTok{as} \NormalTok{*}\KeywordTok{mut} \NormalTok{libc::}\DataTypeTok{c_void}\NormalTok{, size }\KeywordTok{as} \NormalTok{libc::}\DataTypeTok{size_t}\NormalTok{) }\KeywordTok{as} \NormalTok{*}\KeywordTok{mut} \DataTypeTok{u8}
    \NormalTok{\}}
\NormalTok{\}}

\AttributeTok{#[}\NormalTok{no_mangle}\AttributeTok{]}
\KeywordTok{pub} \KeywordTok{extern} \KeywordTok{fn} \NormalTok{__rust_reallocate_inplace(_ptr: *}\KeywordTok{mut} \DataTypeTok{u8}\NormalTok{, old_size: }\DataTypeTok{usize}\NormalTok{,}
                                        \NormalTok{_size: }\DataTypeTok{usize}\NormalTok{, _align: }\DataTypeTok{usize}\NormalTok{) -> }\DataTypeTok{usize} \NormalTok{\{}
    \NormalTok{old_size }\CommentTok{// this api is not supported by libc}
\NormalTok{\}}

\AttributeTok{#[}\NormalTok{no_mangle}\AttributeTok{]}
\KeywordTok{pub} \KeywordTok{extern} \KeywordTok{fn} \NormalTok{__rust_usable_size(size: }\DataTypeTok{usize}\NormalTok{, _align: }\DataTypeTok{usize}\NormalTok{) -> }\DataTypeTok{usize} \NormalTok{\{}
    \NormalTok{size}
\NormalTok{\}}
\end{Highlighting}
\end{Shaded}

After we compile this crate, it can be used as follows:

\begin{Shaded}
\begin{Highlighting}[]
\KeywordTok{extern} \KeywordTok{crate} \NormalTok{my_allocator;}

\KeywordTok{fn} \NormalTok{main() \{}
    \KeywordTok{let} \NormalTok{a = }\DataTypeTok{Box}\NormalTok{::new(}\DecValTok{8}\NormalTok{); }\CommentTok{// allocates memory via our custom allocator crate}
    \PreprocessorTok{println!}\NormalTok{(}\StringTok{"\{\}"}\NormalTok{, a);}
\NormalTok{\}}
\end{Highlighting}
\end{Shaded}

\subsection{Custom allocator
limitations}\label{custom-allocator-limitations}

There are a few restrictions when working with custom allocators which
may cause compiler errors:

\begin{itemize}
\item
  Any one artifact may only be linked to at most one allocator.
  Binaries, dylibs, and staticlibs must link to exactly one allocator,
  and if none have been explicitly chosen the compiler will choose one.
  On the other hand rlibs do not need to link to an allocator (but still
  can).
\item
  A consumer of an allocator is tagged with
  \texttt{\#!{[}needs\_allocator{]}} (e.g.~the \texttt{liballoc} crate
  currently) and an \texttt{\#{[}allocator{]}} crate cannot transitively
  depend on a crate which needs an allocator (e.g.~circular dependencies
  are not allowed). This basically means that allocators must restrict
  themselves to libcore currently.
\end{itemize}

\hypertarget{sec--glossary}{\chapter{Glossary}\label{sec--glossary}}

Not every Rustacean has a background in systems programming, nor in
computer science, so we've added explanations of terms that might be
unfamiliar.

\hypertarget{abstract-syntax-tree}{\paragraph{Abstract Syntax
Tree}\label{abstract-syntax-tree}}

When a compiler is compiling your program, it does a number of different
things. One of the things that it does is turn the text of your program
into an `abstract syntax tree', or `AST'. This tree is a representation
of the structure of your program. For example, \texttt{2\ +\ 3} can be
turned into a tree:

\begin{verbatim}
  +
 / \
2   3
\end{verbatim}

And \texttt{2\ +\ (3\ *\ 4)} would look like this:

\begin{verbatim}
  +
 / \
2   *
   / \
  3   4
\end{verbatim}

\hypertarget{arity}{\paragraph{Arity}\label{arity}}

Arity refers to the number of arguments a function or operation takes.

\begin{Shaded}
\begin{Highlighting}[]
\KeywordTok{let} \NormalTok{x = (}\DecValTok{2}\NormalTok{, }\DecValTok{3}\NormalTok{);}
\KeywordTok{let} \NormalTok{y = (}\DecValTok{4}\NormalTok{, }\DecValTok{6}\NormalTok{);}
\KeywordTok{let} \NormalTok{z = (}\DecValTok{8}\NormalTok{, }\DecValTok{2}\NormalTok{, }\DecValTok{6}\NormalTok{);}
\end{Highlighting}
\end{Shaded}

In the example above \texttt{x} and \texttt{y} have arity 2. \texttt{z}
has arity 3.

\hypertarget{bounds}{\paragraph{Bounds}\label{bounds}}

Bounds are constraints on a type or
\protect\hyperlink{sec--traits}{trait}. For example, if a bound is
placed on the argument a function takes, types passed to that function
must abide by that constraint.

\paragraph{DST (Dynamically Sized
Type)}\label{dst-dynamically-sized-type}

A type without a statically known size or alignment.
(\href{../nomicon/exotic-sizes.html\#dynamically-sized-types-dsts}{more
info})

\hypertarget{expression}{\paragraph{Expression}\label{expression}}

In computer programming, an expression is a combination of values,
constants, variables, operators and functions that evaluate to a single
value. For example, \texttt{2\ +\ (3\ *\ 4)} is an expression that
returns the value 14. It is worth noting that expressions can have
side-effects. For example, a function included in an expression might
perform actions other than simply returning a value.

\hypertarget{expression-oriented-language}{\paragraph{Expression-Oriented
Language}\label{expression-oriented-language}}

In early programming languages,
\protect\hyperlink{expression}{expressions} and
\protect\hyperlink{statement}{statements} were two separate syntactic
categories: expressions had a value and statements did things. However,
later languages blurred this distinction, allowing expressions to do
things and statements to have a value. In an expression-oriented
language, (nearly) every statement is an expression and therefore
returns a value. Consequently, these expression statements can
themselves form part of larger expressions.

\hypertarget{statement}{\paragraph{Statement}\label{statement}}

In computer programming, a statement is the smallest standalone element
of a programming language that commands a computer to perform an action.

\chapter{Syntax Index}\label{sec--syntax-index}

\subsubsection{Keywords}\label{keywords}

\begin{itemize}
\tightlist
\item
  \texttt{as}: primitive casting, or disambiguating the specific trait
  containing an item. See {[}Casting Between Types (\texttt{as}){]},
  {[}Universal Function Call Syntax (Angle-bracket Form){]},
  \protect\hyperlink{sec--associated-types}{Associated Types}.
\item
  \texttt{break}: break out of loop. See {[}Loops (Ending Iteration
  Early){]}.
\item
  \texttt{const}: constant items and constant raw pointers. See
  \protect\hyperlink{sec--const-and-static}{\texttt{const} and
  \texttt{static}}, \protect\hyperlink{sec--raw-pointers}{Raw Pointers}.
\item
  \texttt{continue}: continue to next loop iteration. See {[}Loops
  (Ending Iteration Early){]}.
\item
  \texttt{crate}: external crate linkage. See {[}Crates and Modules
  (Importing External Crates){]}.
\item
  \texttt{else}: fallback for \texttt{if} and \texttt{if\ let}
  constructs. See {[}\texttt{if}{]}, {[}\texttt{if\ let}{]}.
\item
  \texttt{enum}: defining enumeration. See
  \protect\hyperlink{sec--enums}{Enums}.
\item
  \texttt{extern}: external crate, function, and variable linkage. See
  {[}Crates and Modules (Importing External Crates){]}, {[}Foreign
  Function Interface{]}.
\item
  \texttt{false}: boolean false literal. See {[}Primitive Types
  (Booleans){]}.
\item
  \texttt{fn}: function definition and function pointer types. See
  \protect\hyperlink{functions}{Functions}.
\item
  \texttt{for}: iterator loop, part of trait \texttt{impl} syntax, and
  higher-ranked lifetime syntax. See {[}Loops (\texttt{for}){]},
  \protect\hyperlink{sec--method-syntax}{Method Syntax}.
\item
  \texttt{if}: conditional branching. See {[}\texttt{if}{]},
  {[}\texttt{if\ let}{]}.
\item
  \texttt{impl}: inherent and trait implementation blocks. See
  \protect\hyperlink{sec--method-syntax}{Method Syntax}.
\item
  \texttt{in}: part of \texttt{for} loop syntax. See {[}Loops
  (\texttt{for}){]}.
\item
  \texttt{let}: variable binding. See
  \protect\hyperlink{sec--variable-bindings}{Variable Bindings}.
\item
  \texttt{loop}: unconditional, infinite loop. See {[}Loops
  (\texttt{loop}){]}.
\item
  \texttt{match}: pattern matching. See
  \protect\hyperlink{sec--match}{Match}.
\item
  \texttt{mod}: module declaration. See {[}Crates and Modules (Defining
  Modules){]}.
\item
  \texttt{move}: part of closure syntax. See {[}Closures (\texttt{move}
  closures){]}.
\item
  \texttt{mut}: denotes mutability in pointer types and pattern
  bindings. See \protect\hyperlink{mutability-1}{Mutability}.
\item
  \texttt{pub}: denotes public visibility in \texttt{struct} fields,
  \texttt{impl} blocks, and modules. See {[}Crates and Modules
  (Exporting a Public Interface){]}.
\item
  \texttt{ref}: by-reference binding. See {[}Patterns (\texttt{ref} and
  \texttt{ref\ mut}){]}.
\item
  \texttt{return}: return from function. See {[}Functions (Early
  Returns){]}.
\item
  \texttt{Self}: implementor type alias. See
  \protect\hyperlink{sec--traits}{Traits}.
\item
  \texttt{self}: method subject. See {[}Method Syntax (Method Calls){]}.
\item
  \texttt{static}: global variable. See {[}\texttt{const} and
  \texttt{static} (\texttt{static}){]}.
\item
  \texttt{struct}: structure definition. See
  \protect\hyperlink{sec--structs}{Structs}.
\item
  \texttt{trait}: trait definition. See
  \protect\hyperlink{sec--traits}{Traits}.
\item
  \texttt{true}: boolean true literal. See {[}Primitive Types
  (Booleans){]}.
\item
  \texttt{type}: type alias, and associated type definition. See
  \protect\hyperlink{sec--type-aliases}{\texttt{type} Aliases},
  \protect\hyperlink{sec--associated-types}{Associated Types}.
\item
  \texttt{unsafe}: denotes unsafe code, functions, traits, and
  implementations. See {[}Unsafe{]}.
\item
  \texttt{use}: import symbols into scope. See {[}Crates and Modules
  (Importing Modules with \texttt{use}){]}.
\item
  \texttt{where}: type constraint clauses. See {[}Traits (\texttt{where}
  clause){]}.
\item
  \texttt{while}: conditional loop. See {[}Loops (\texttt{while}){]}.
\end{itemize}

\subsubsection{Operators and Symbols}\label{operators-and-symbols}

\begin{itemize}
\tightlist
\item
  \texttt{!} (\texttt{ident!(\ldots{})}, \texttt{ident!\{\ldots{}\}},
  \texttt{ident!{[}\ldots{}{]}}): denotes macro expansion. See
  \protect\hyperlink{sec--macros}{Macros}.
\item
  \texttt{!} (\texttt{!expr}): bitwise or logical complement.
  Overloadable (\texttt{Not}).
\item
  \texttt{!=} (\texttt{var\ !=\ expr}): nonequality comparison.
  Overloadable (\texttt{PartialEq}).
\item
  \texttt{\%} (\texttt{expr\ \%\ expr}): arithmetic remainder.
  Overloadable (\texttt{Rem}).
\item
  \texttt{\%=} (\texttt{var\ \%=\ expr}): arithmetic remainder \&
  assignment. Overloadable (\texttt{RemAssign}).
\item
  \texttt{\&} (\texttt{expr\ \&\ expr}): bitwise and. Overloadable
  (\texttt{BitAnd}).
\item
  \texttt{\&} (\texttt{\&expr}): borrow. See
  \protect\hyperlink{sec--references-and-borrowing}{References and
  Borrowing}.
\item
  \texttt{\&} (\texttt{\&type}, \texttt{\&mut\ type},
  \texttt{\&\textquotesingle{}a\ type},
  \texttt{\&\textquotesingle{}a\ mut\ type}): borrowed pointer type. See
  \protect\hyperlink{sec--references-and-borrowing}{References and
  Borrowing}.
\item
  \texttt{\&=} (\texttt{var\ \&=\ expr}): bitwise and \& assignment.
  Overloadable (\texttt{BitAndAssign}).
\item
  \texttt{\&\&} (\texttt{expr\ \&\&\ expr}): logical and.
\item
  \texttt{*} (\texttt{expr\ *\ expr}): arithmetic multiplication.
  Overloadable (\texttt{Mul}).
\item
  \texttt{*} (\texttt{*expr}): dereference.
\item
  \texttt{*} (\texttt{*const\ type}, \texttt{*mut\ type}): raw pointer.
  See \protect\hyperlink{sec--raw-pointers}{Raw Pointers}.
\item
  \texttt{*=} (\texttt{var\ *=\ expr}): arithmetic multiplication \&
  assignment. Overloadable (\texttt{MulAssign}).
\item
  \texttt{+} (\texttt{expr\ +\ expr}): arithmetic addition. Overloadable
  (\texttt{Add}).
\item
  \texttt{+} (\texttt{trait\ +\ trait},
  \texttt{\textquotesingle{}a\ +\ trait}): compound type constraint. See
  {[}Traits (Multiple Trait Bounds){]}.
\item
  \texttt{+=} (\texttt{var\ +=\ expr}): arithmetic addition \&
  assignment. Overloadable (\texttt{AddAssign}).
\item
  \texttt{,}: argument and element separator. See
  \protect\hyperlink{sec--attributes}{Attributes},
  \protect\hyperlink{functions}{Functions},
  \protect\hyperlink{sec--structs}{Structs},
  \protect\hyperlink{sec--generics}{Generics},
  \protect\hyperlink{sec--match}{Match},
  \protect\hyperlink{sec--closures}{Closures}, {[}Crates and Modules
  (Importing Modules with \texttt{use}){]}.
\item
  \texttt{-} (\texttt{expr\ -\ expr}): arithmetic subtraction.
  Overloadable (\texttt{Sub}).
\item
  \texttt{-} (\texttt{-\ expr}): arithmetic negation. Overloadable
  (\texttt{Neg}).
\item
  \texttt{-=} (\texttt{var\ -=\ expr}): arithmetic subtraction \&
  assignment. Overloadable (\texttt{SubAssign}).
\item
  \texttt{-\textgreater{}}
  (\texttt{fn(\ldots{})\ -\textgreater{}\ type},
  \texttt{\textbar{}\ldots{}\textbar{}\ -\textgreater{}\ type}):
  function and closure return type. See
  \protect\hyperlink{functions}{Functions},
  \protect\hyperlink{sec--closures}{Closures}.
\item
  \texttt{-\textgreater{}\ !}
  (\texttt{fn(\ldots{})\ -\textgreater{}\ !},
  \texttt{\textbar{}\ldots{}\textbar{}\ -\textgreater{}\ !}): diverging
  function or closure. See
  \protect\hyperlink{diverging-functions}{Diverging Functions}.
\item
  \texttt{.} (\texttt{expr.ident}): member access. See
  \protect\hyperlink{sec--structs}{Structs},
  \protect\hyperlink{sec--method-syntax}{Method Syntax}.
\item
  \texttt{..} (\texttt{..}, \texttt{expr..}, \texttt{..expr},
  \texttt{expr..expr}): right-exclusive range literal.
\item
  \texttt{..} (\texttt{..expr}): struct literal update syntax. See
  {[}Structs (Update syntax){]}.
\item
  \texttt{..} (\texttt{variant(x,\ ..)},
  \texttt{struct\_type\ \{\ x,\ ..\ \}}): ``and the rest'' pattern
  binding. See {[}Patterns (Ignoring bindings){]}.
\item
  \texttt{...} (\texttt{...expr}, \texttt{expr...expr}) \emph{in an
  expression}: inclusive range expression. See
  \protect\hyperlink{iterators}{Iterators}.
\item
  \texttt{...} (\texttt{expr...expr}) \emph{in a pattern}: inclusive
  range pattern. See {[}Patterns (Ranges){]}.
\item
  \texttt{/} (\texttt{expr\ /\ expr}): arithmetic division. Overloadable
  (\texttt{Div}).
\item
  \texttt{/=} (\texttt{var\ /=\ expr}): arithmetic division \&
  assignment. Overloadable (\texttt{DivAssign}).
\item
  \texttt{:} (\texttt{pat:\ type}, \texttt{ident:\ type}): constraints.
  See \protect\hyperlink{sec--variable-bindings}{Variable Bindings},
  \protect\hyperlink{functions}{Functions},
  \protect\hyperlink{sec--structs}{Structs},
  \protect\hyperlink{sec--traits}{Traits}.
\item
  \texttt{:} (\texttt{ident:\ expr}): struct field initializer. See
  \protect\hyperlink{sec--structs}{Structs}.
\item
  \texttt{:} (\texttt{\textquotesingle{}a:\ loop\ \{\ldots{}\}}): loop
  label. See {[}Loops (Loops Labels){]}.
\item
  \texttt{;}: statement and item terminator.
\item
  \texttt{;} (\texttt{{[}\ldots{};\ len{]}}): part of fixed-size array
  syntax. See {[}Primitive Types (Arrays){]}.
\item
  \texttt{\textless{}\textless{}}
  (\texttt{expr\ \textless{}\textless{}\ expr}): left-shift.
  Overloadable (\texttt{Shl}).
\item
  \texttt{\textless{}\textless{}=}
  (\texttt{var\ \textless{}\textless{}=\ expr}): left-shift \&
  assignment. Overloadable (\texttt{ShlAssign}).
\item
  \texttt{\textless{}} (\texttt{expr\ \textless{}\ expr}): less-than
  comparison. Overloadable (\texttt{PartialOrd}).
\item
  \texttt{\textless{}=} (\texttt{var\ \textless{}=\ expr}): less-than or
  equal-to comparison. Overloadable (\texttt{PartialOrd}).
\item
  \texttt{=} (\texttt{var\ =\ expr}, \texttt{ident\ =\ type}):
  assignment/equivalence. See
  \protect\hyperlink{sec--variable-bindings}{Variable Bindings},
  \protect\hyperlink{sec--type-aliases}{\texttt{type} Aliases}, generic
  parameter defaults.
\item
  \texttt{==} (\texttt{var\ ==\ expr}): equality comparison.
  Overloadable (\texttt{PartialEq}).
\item
  \texttt{=\textgreater{}} (\texttt{pat\ =\textgreater{}\ expr}): part
  of match arm syntax. See \protect\hyperlink{sec--match}{Match}.
\item
  \texttt{\textgreater{}} (\texttt{expr\ \textgreater{}\ expr}):
  greater-than comparison. Overloadable (\texttt{PartialOrd}).
\item
  \texttt{\textgreater{}=} (\texttt{var\ \textgreater{}=\ expr}):
  greater-than or equal-to comparison. Overloadable
  (\texttt{PartialOrd}).
\item
  \texttt{\textgreater{}\textgreater{}}
  (\texttt{expr\ \textgreater{}\textgreater{}\ expr}): right-shift.
  Overloadable (\texttt{Shr}).
\item
  \texttt{\textgreater{}\textgreater{}=}
  (\texttt{var\ \textgreater{}\textgreater{}=\ expr}): right-shift \&
  assignment. Overloadable (\texttt{ShrAssign}).
\item
  \texttt{@} (\texttt{ident\ @\ pat}): pattern binding. See {[}Patterns
  (Bindings){]}.
\item
  \texttt{\^{}} (\texttt{expr\ \^{}\ expr}): bitwise exclusive or.
  Overloadable (\texttt{BitXor}).
\item
  \texttt{\^{}=} (\texttt{var\ \^{}=\ expr}): bitwise exclusive or \&
  assignment. Overloadable (\texttt{BitXorAssign}).
\item
  \texttt{\textbar{}} (\texttt{expr\ \textbar{}\ expr}): bitwise or.
  Overloadable (\texttt{BitOr}).
\item
  \texttt{\textbar{}} (\texttt{pat\ \textbar{}\ pat}): pattern
  alternatives. See {[}Patterns (Multiple patterns){]}.
\item
  \texttt{\textbar{}} (\texttt{\textbar{}\ldots{}\textbar{}\ expr}):
  closures. See \protect\hyperlink{sec--closures}{Closures}.
\item
  \texttt{\textbar{}=} (\texttt{var\ \textbar{}=\ expr}): bitwise or \&
  assignment. Overloadable (\texttt{BitOrAssign}).
\item
  \texttt{\textbar{}\textbar{}}
  (\texttt{expr\ \textbar{}\textbar{}\ expr}): logical or.
\item
  \texttt{\_}: ``ignored'' pattern binding. See {[}Patterns (Ignoring
  bindings){]}.
\end{itemize}

\subsubsection{Other Syntax}\label{other-syntax}

\begin{itemize}
\tightlist
\item
  \texttt{\textquotesingle{}ident}: named lifetime or loop label. See
  \protect\hyperlink{lifetimes}{Lifetimes}, {[}Loops (Loops Labels){]}.
\item
  \texttt{\ldots{}u8}, \texttt{\ldots{}i32}, \texttt{\ldots{}f64},
  \texttt{\ldots{}usize}, \ldots{}: numeric literal of specific type.
\item
  \texttt{"\ldots{}"}: string literal. See
  \protect\hyperlink{sec--strings}{Strings}.
\item
  \texttt{r"\ldots{}"}, \texttt{r\#"\ldots{}"\#},
  \texttt{r\#\#"\ldots{}"\#\#}, \ldots{}: raw string literal, escape
  characters are not processed. See {[}Reference (Raw String
  Literals){]}.
\item
  \texttt{b"\ldots{}"}: byte string literal, constructs a
  \texttt{{[}u8{]}} instead of a string. See {[}Reference (Byte String
  Literals){]}.
\item
  \texttt{br"\ldots{}"}, \texttt{br\#"\ldots{}"\#},
  \texttt{br\#\#"\ldots{}"\#\#}, \ldots{}: raw byte string literal,
  combination of raw and byte string literal. See {[}Reference (Raw Byte
  String Literals){]}.
\item
  \texttt{\textquotesingle{}\ldots{}\textquotesingle{}}: character
  literal. See {[}Primitive Types (\texttt{char}){]}.
\item
  \texttt{b\textquotesingle{}\ldots{}\textquotesingle{}}: ASCII byte
  literal.
\item
  \texttt{\textbar{}\ldots{}\textbar{}\ expr}: closure. See
  \protect\hyperlink{sec--closures}{Closures}.
\end{itemize}

\begin{itemize}
\tightlist
\item
  \texttt{ident::ident}: path. See {[}Crates and Modules (Defining
  Modules){]}.
\item
  \texttt{::path}: path relative to the crate root (\emph{i.e.} an
  explicitly absolute path). See {[}Crates and Modules (Re-exporting
  with \texttt{pub\ use}){]}.
\item
  \texttt{self::path}: path relative to the current module (\emph{i.e.}
  an explicitly relative path). See {[}Crates and Modules (Re-exporting
  with \texttt{pub\ use}){]}.
\item
  \texttt{super::path}: path relative to the parent of the current
  module. See {[}Crates and Modules (Re-exporting with
  \texttt{pub\ use}){]}.
\item
  \texttt{type::ident},
  \texttt{\textless{}type\ as\ trait\textgreater{}::ident}: associated
  constants, functions, and types. See
  \protect\hyperlink{sec--associated-types}{Associated Types}.
\item
  \texttt{\textless{}type\textgreater{}::\ldots{}}: associated item for
  a type which cannot be directly named (\emph{e.g.}
  \texttt{\textless{}\&T\textgreater{}::\ldots{}},
  \texttt{\textless{}{[}T{]}\textgreater{}::\ldots{}}, \emph{etc.}). See
  \protect\hyperlink{sec--associated-types}{Associated Types}.
\item
  \texttt{trait::method(\ldots{})}: disambiguating a method call by
  naming the trait which defines it. See
  \protect\hyperlink{sec--ufcs}{Universal Function Call Syntax}.
\item
  \texttt{type::method(\ldots{})}: disambiguating a method call by
  naming the type for which it's defined. See
  \protect\hyperlink{sec--ufcs}{Universal Function Call Syntax}.
\item
  \texttt{\textless{}type\ as\ trait\textgreater{}::method(\ldots{})}:
  disambiguating a method call by naming the trait \emph{and} type. See
  {[}Universal Function Call Syntax (Angle-bracket Form){]}.
\end{itemize}

\begin{itemize}
\tightlist
\item
  \texttt{path\textless{}\ldots{}\textgreater{}} (\emph{e.g.}
  \texttt{Vec\textless{}u8\textgreater{}}): specifies parameters to
  generic type \emph{in a type}. See
  \protect\hyperlink{sec--generics}{Generics}.
\item
  \texttt{path::\textless{}\ldots{}\textgreater{}},
  \texttt{method::\textless{}\ldots{}\textgreater{}} (\emph{e.g.}
  \texttt{"42".parse::\textless{}i32\textgreater{}()}): specifies
  parameters to generic type, function, or method \emph{in an
  expression}.
\item
  \texttt{fn\ ident\textless{}\ldots{}\textgreater{}\ \ldots{}}: define
  generic function. See \protect\hyperlink{sec--generics}{Generics}.
\item
  \texttt{struct\ ident\textless{}\ldots{}\textgreater{}\ \ldots{}}:
  define generic structure. See
  \protect\hyperlink{sec--generics}{Generics}.
\item
  \texttt{enum\ ident\textless{}\ldots{}\textgreater{}\ \ldots{}}:
  define generic enumeration. See
  \protect\hyperlink{sec--generics}{Generics}.
\item
  \texttt{impl\textless{}\ldots{}\textgreater{}\ \ldots{}}: define
  generic implementation.
\item
  \texttt{for\textless{}\ldots{}\textgreater{}\ type}: higher-ranked
  lifetime bounds.
\item
  \texttt{type\textless{}ident=type\textgreater{}} (\emph{e.g.}
  \texttt{Iterator\textless{}Item=T\textgreater{}}): a generic type
  where one or more associated types have specific assignments. See
  \protect\hyperlink{sec--associated-types}{Associated Types}.
\end{itemize}

\begin{itemize}
\tightlist
\item
  \texttt{T:\ U}: generic parameter \texttt{T} constrained to types that
  implement \texttt{U}. See \protect\hyperlink{sec--traits}{Traits}.
\item
  \texttt{T:\ \textquotesingle{}a}: generic type \texttt{T} must outlive
  lifetime \texttt{\textquotesingle{}a}. When we say that a type
  `outlives' the lifetime, we mean that it cannot transitively contain
  any references with lifetimes shorter than
  \texttt{\textquotesingle{}a}.
\item
  \texttt{T\ :\ \textquotesingle{}static}: The generic type \texttt{T}
  contains no borrowed references other than
  \texttt{\textquotesingle{}static} ones.
\item
  \texttt{\textquotesingle{}b:\ \textquotesingle{}a}: generic lifetime
  \texttt{\textquotesingle{}b} must outlive lifetime
  \texttt{\textquotesingle{}a}.
\item
  \texttt{T:\ ?Sized}: allow generic type parameter to be a
  dynamically-sized type. See {[}Unsized Types (\texttt{?Sized}){]}.
\item
  \texttt{\textquotesingle{}a\ +\ trait}, \texttt{trait\ +\ trait}:
  compound type constraint. See {[}Traits (Multiple Trait Bounds){]}.
\end{itemize}

\begin{itemize}
\tightlist
\item
  \texttt{\#{[}meta{]}}: outer attribute. See
  \protect\hyperlink{sec--attributes}{Attributes}.
\item
  \texttt{\#!{[}meta{]}}: inner attribute. See
  \protect\hyperlink{sec--attributes}{Attributes}.
\item
  \texttt{\$ident}: macro substitution. See
  \protect\hyperlink{sec--macros}{Macros}.
\item
  \texttt{\$ident:kind}: macro capture. See
  \protect\hyperlink{sec--macros}{Macros}.
\item
  \texttt{\$(\ldots{})\ldots{}}: macro repetition. See
  \protect\hyperlink{sec--macros}{Macros}.
\end{itemize}

\begin{itemize}
\tightlist
\item
  \texttt{//}: line comment. See
  \protect\hyperlink{sec--comments}{Comments}.
\item
  \texttt{//!}: inner line doc comment. See
  \protect\hyperlink{sec--comments}{Comments}.
\item
  \texttt{///}: outer line doc comment. See
  \protect\hyperlink{sec--comments}{Comments}.
\item
  \texttt{/*\ldots{}*/}: block comment. See
  \protect\hyperlink{sec--comments}{Comments}.
\item
  \texttt{/*!\ldots{}*/}: inner block doc comment. See
  \protect\hyperlink{sec--comments}{Comments}.
\item
  \texttt{/**\ldots{}*/}: outer block doc comment. See
  \protect\hyperlink{sec--comments}{Comments}.
\end{itemize}

\begin{itemize}
\tightlist
\item
  \texttt{()}: empty tuple (\emph{a.k.a.} unit), both literal and type.
\item
  \texttt{(expr)}: parenthesized expression.
\item
  \texttt{(expr,)}: single-element tuple expression. See {[}Primitive
  Types (Tuples){]}.
\item
  \texttt{(type,)}: single-element tuple type. See {[}Primitive Types
  (Tuples){]}.
\item
  \texttt{(expr,\ \ldots{})}: tuple expression. See {[}Primitive Types
  (Tuples){]}.
\item
  \texttt{(type,\ \ldots{})}: tuple type. See {[}Primitive Types
  (Tuples){]}.
\item
  \texttt{expr(expr,\ \ldots{})}: function call expression. Also used to
  initialize tuple \texttt{struct}s and tuple \texttt{enum} variants.
  See \protect\hyperlink{functions}{Functions}.
\item
  \texttt{ident!(\ldots{})}, \texttt{ident!\{\ldots{}\}},
  \texttt{ident!{[}\ldots{}{]}}: macro invocation. See
  \protect\hyperlink{sec--macros}{Macros}.
\item
  \texttt{expr.0}, \texttt{expr.1}, \ldots{}: tuple indexing. See
  {[}Primitive Types (Tuple Indexing){]}.
\end{itemize}

\begin{itemize}
\tightlist
\item
  \texttt{\{\ldots{}\}}: block expression.
\item
  \texttt{Type\ \{\ldots{}\}}: \texttt{struct} literal. See
  \protect\hyperlink{sec--structs}{Structs}.
\end{itemize}

\begin{itemize}
\tightlist
\item
  \texttt{{[}\ldots{}{]}}: array literal. See {[}Primitive Types
  (Arrays){]}.
\item
  \texttt{{[}expr;\ len{]}}: array literal containing \texttt{len}
  copies of \texttt{expr}. See {[}Primitive Types (Arrays){]}.
\item
  \texttt{{[}type;\ len{]}}: array type containing \texttt{len}
  instances of \texttt{type}. See {[}Primitive Types (Arrays){]}.
\item
  \texttt{expr{[}expr{]}}: collection indexing. Overloadable
  (\texttt{Index}, \texttt{IndexMut}).
\item
  \texttt{expr{[}..{]}}, \texttt{expr{[}a..{]}}, \texttt{expr{[}..b{]}},
  \texttt{expr{[}a..b{]}}: collection indexing pretending to be
  collection slicing, using \texttt{Range}, \texttt{RangeFrom},
  \texttt{RangeTo}, \texttt{RangeFull} as the ``index''.
\end{itemize}

\hypertarget{sec--bibliography}{\chapter{Bibliography}\label{sec--bibliography}}

This is a reading list of material relevant to Rust. It includes prior
research that has - at one time or another - influenced the design of
Rust, as well as publications about Rust.

\paragraph{Type system}\label{type-system}

\begin{itemize}
\tightlist
\item
  \href{http://209.68.42.137/ucsd-pages/Courses/cse227.w03/handouts/cyclone-regions.pdf}{Region
  based memory management in Cyclone}
\item
  \href{http://www.cs.umd.edu/projects/PL/cyclone/scp.pdf}{Safe manual
  memory management in Cyclone}
\item
  \href{http://www.ps.uni-sb.de/courses/typen-ws99/class.ps.gz}{Typeclasses:
  making ad-hoc polymorphism less ad hoc}
\item
  \href{https://www.cs.utah.edu/plt/publications/jfp12-draft-fcdf.pdf}{Macros
  that work together}
\item
  \href{http://scg.unibe.ch/archive/papers/Scha03aTraits.pdf}{Traits:
  composable units of behavior}
\item
  \href{http://www.cs.uwm.edu/faculty/boyland/papers/unique-preprint.ps}{Alias
  burying} - We tried something similar and abandoned it.
\item
  \href{http://www.cs.uu.nl/research/techreps/UU-CS-2002-048.html}{External
  uniqueness is unique enough}
\item
  \href{https://research.microsoft.com/pubs/170528/msr-tr-2012-79.pdf}{Uniqueness
  and Reference Immutability for Safe Parallelism}
\item
  \href{http://www.cs.ucla.edu/~palsberg/tba/papers/tofte-talpin-iandc97.pdf}{Region
  Based Memory Management}
\end{itemize}

\paragraph{Concurrency}\label{concurrency}

\begin{itemize}
\tightlist
\item
  \href{https://research.microsoft.com/pubs/69431/osr2007_rethinkingsoftwarestack.pdf}{Singularity:
  rethinking the software stack}
\item
  \href{https://research.microsoft.com/pubs/67482/singsharp.pdf}{Language
  support for fast and reliable message passing in singularity OS}
\item
  \href{http://supertech.csail.mit.edu/papers/steal.pdf}{Scheduling
  multithreaded computations by work stealing}
\item
  \href{http://www.eecis.udel.edu/\%7Ecavazos/cisc879-spring2008/papers/arora98thread.pdf}{Thread
  scheduling for multiprogramming multiprocessors}
\item
  \href{http://www.aladdin.cs.cmu.edu/papers/pdfs/y2000/locality_spaa00.pdf}{The
  data locality of work stealing}
\item
  \href{http://citeseerx.ist.psu.edu/viewdoc/download?doi=10.1.1.170.1097\&rep=rep1\&type=pdf}{Dynamic
  circular work stealing deque} - The Chase/Lev deque
\item
  \href{http://www.cs.rice.edu/\%7Eyguo/pubs/PID824943.pdf}{Work-first
  and help-first scheduling policies for async-finish task parallelism}
  - More general than fully-strict work stealing
\item
  \href{http://www.coopsoft.com/ar/CalamityArticle.html}{A Java
  fork/join calamity} - critique of Java's fork/join library,
  particularly its application of work stealing to non-strict
  computation
\item
  \href{http://www.stanford.edu/~ouster/cgi-bin/papers/coscheduling.pdf}{Scheduling
  techniques for concurrent systems}
\item
  \href{http://www.blagodurov.net/files/a8-blagodurov.pdf}{Contention
  aware scheduling}
\item
  \href{http://www.cse.ohio-state.edu/hpcs/WWW/HTML/publications/papers/TR-12-1.pdf}{Balanced
  work stealing for time-sharing multicores}
\item
  \href{http://dl.acm.org/citation.cfm?id=1953616\&dl=ACM\&coll=DL\&CFID=524387192\&CFTOKEN=44362705}{Three
  layer cake for shared-memory programming}
\item
  \href{http://www.cs.bgu.ac.il/\%7Ehendlerd/papers/p280-hendler.pdf}{Non-blocking
  steal-half work queues}
\item
  \href{http://www.mpi-sws.org/~turon/reagents.pdf}{Reagents: expressing
  and composing fine-grained concurrency}
\item
  \href{https://www.cs.rochester.edu/u/scott/papers/1991_TOCS_synch.pdf}{Algorithms
  for scalable synchronization of shared-memory multiprocessors}
\item
  \href{https://www.cl.cam.ac.uk/techreports/UCAM-CL-TR-579.pdf}{Epoch-based
  reclamation}.
\end{itemize}

\paragraph{Others}\label{others}

\begin{itemize}
\tightlist
\item
  \href{https://www.usenix.org/legacy/events/hotos03/tech/full_papers/candea/candea.pdf}{Crash-only
  software}
\item
  \href{http://people.cs.umass.edu/~emery/pubs/berger-pldi2001.pdf}{Composing
  High-Performance Memory Allocators}
\item
  \href{http://people.cs.umass.edu/~emery/pubs/berger-oopsla2002.pdf}{Reconsidering
  Custom Memory Allocation}
\end{itemize}

\paragraph{\texorpdfstring{Papers \emph{about}
Rust}{Papers about Rust}}\label{papers-about-rust}

\begin{itemize}
\tightlist
\item
  \href{http://www.cs.indiana.edu/~eholk/papers/hips2013.pdf}{GPU
  Programming in Rust: Implementing High Level Abstractions in a Systems
  Level Language}. Early GPU work by Eric Holk.
\item
  \href{https://www.usenix.org/conference/hotpar12/parallel-closures-new-twist-old-idea}{Parallel
  closures: a new twist on an old idea}
\item
  not exactly about Rust, but by nmatsakis
\item
  \href{ftp://ftp.cs.washington.edu/tr/2015/03/UW-CSE-15-03-02.pdf}{Patina:
  A Formalization of the Rust Programming Language}. Early formalization
  of a subset of the type system, by Eric Reed.
\item
  \href{http://arxiv.org/abs/1505.07383}{Experience Report: Developing
  the Servo Web Browser Engine using Rust}. By Lars Bergstrom.
\item
  \href{https://michaelsproul.github.io/rust_radix_paper/rust-radix-sproul.pdf}{Implementing
  a Generic Radix Trie in Rust}. Undergrad paper by Michael Sproul.
\item
  \href{http://scialex.github.io/reenix.pdf}{Reenix: Implementing a
  Unix-Like Operating System in Rust}. Undergrad paper by Alex Light.
\item
  {[}Evaluation of performance and productivity metrics of potential
  programming languages in the HPC environment{]}
  (http://octarineparrot.com/assets/mrfloya-thesis-ba.pdf). Bachelor's
  thesis by Florian Wilkens. Compares C, Go and Rust.
\item
  \href{http://spw15.langsec.org/papers/couprie-nom.pdf}{Nom, a byte
  oriented, streaming, zero copy, parser combinators library in Rust}.
  By Geoffroy Couprie, research for VLC.
\item
  \href{http://compilers.cs.uni-saarland.de/papers/lkh15_cgo.pdf}{Graph-Based
  Higher-Order Intermediate Representation}. An experimental IR
  implemented in Impala, a Rust-like language.
\item
  \href{http://compilers.cs.uni-saarland.de/papers/ppl14_web.pdf}{Code
  Refinement of Stencil Codes}. Another paper using Impala.
\item
  \href{http://publications.lib.chalmers.se/records/fulltext/219016/219016.pdf}{Parallelization
  in Rust with fork-join and friends}. Linus Farnstrand's master's
  thesis.
\item
  \href{http://munksgaard.me/papers/laumann-munksgaard-larsen.pdf}{Session
  Types for Rust}. Philip Munksgaard's master's thesis. Research for
  Servo.
\item
  \href{http://amitlevy.com/papers/tock-plos2015.pdf}{Ownership is
  Theft: Experiences Building an Embedded OS in Rust - Amit Levy, et.
  al.}
\item
  \href{https://raw.githubusercontent.com/Gankro/thesis/master/thesis.pdf}{You
  can't spell trust without Rust}. Alexis Beingessner's master's thesis.
\end{itemize}



\end{document}
